\chapter{Structure du graphene}

Le graphite est constitué d'un empilement de feuillets faiblement connectés
(liaisons de van der Waals). Les propriétés d'un seul feuillet (graphène) sont
très différentes de celles du graphite.

Il existe plusieurs techniques pour la synthèse du graphène : le dépot sur
membrane, la suspension, etc.
Du point de vue thermique, il semble très difficile d'imaginer un matériau
purement 2D. Lorsqu'il est suspendu, c'est la modulation vibrationnelle qui le
stabilise.

Du point de vue électronique, on pourra montrer que c'est un conducteur et
calculer sa structure de bande. On montrera que les électrons ont des propriétés
particulières dans ce type de matériau 2D. Les électrons relativistes -> vitesse
de fermi.
relation de dispersion linéaire. énergie proportionnelle à k et non plus à $k^2$
(fermion de dirac)

effet Hall quantique à température ambiante dans le graphène.
protection des états quantiques.

il s'agit d'un matériau topologique, on peut faire des isolants topologiques,
c'est à dire isolants dans le bulk, conducteurs en surface. exemple : fermion de
majorana : particule prédite qui est sa propre antiparticule. il est observé dans
des isolants topologiques, couplé à des semi-conducteurs.
Approche des électrons fortement liés. On part d'orbitales en considérant le
recouvrement, structure de bande.

Orbitales $sp^2$, forme triangulaire à 120\textdegree.

\begin{figure}
    \TODO
    \caption{figure des $sp^2$ et $2p_z$}
    \label{fig:orbitales}
\end{figure}

La maille élémentaire du graphène contient deux sites différents. Le carbone
possède 4 électrons de valence. Trois de ces électrons sont engagés dans des
liaisons $sp^2$ à 120\textdegree les unes des autres, et qui pointent vers les
plus proches voisins. Ces liaisons étant très fortes (bandes profondes et
étroites) ; cela nous permet de négliger ces électrons. Il reste ainsi un
électron qui occupe une orbitale $2p_z$ qui se trouve dans les parages, sans
vraiment savoir quoi faire.

La maille élémentaire est définie par les vecteurs $\mathbf{a}_i$ et
$\mathbf{a}_2$, et on définit $a$ tel que $a=a_1=a_2 = \sqrt{3}a_0$ est la
distance entre deux atomes dans le plan. On définit le vecteur $\mathbf{b}$ comme
celui qui relie deux atomes de carbone les plus proces entre eux. Les angles
$\alpha$ et $\gamma$ valent respectivement 30\textdegree et 120\textdegree.


\begin{figure}
    \TODO
    \caption{Mailles du graphène dans l'espace direct et l'espace réciproque}
    \label{fig:maillegraphene}
\end{figure}

Les vecteurs du réseau direct sont :

\begin{equation}
    \mathbf{a}_1 =
    (\frac{1}{2}\mathbf{\hat{x}}+\frac{\sqrt{3}}{2}\hat{\mathbf{y}}),\quad
    \mathbf{a}_2 = a \mathbf{\hat{x}},\quad
    \mathbf{b} = a_0
    (\frac{\sqrt{3}}{2}\hat{\mathbf{x}}+\frac{1}{2}\hat{\mathbf{y}}) =
    \frac{1}{3}(\mathbf{a}_1 + \mathbf{a}_2)
\end{equation}

Ceux du réseau réciproque peuvent s'écrire :

\begin{equation}
    \mathbf{a}_1^* = \frac{2\pi}{a_0} \cdot \frac{2}{3} \mathbf{\hat{y}} =
    \frac{4\pi}{3a_0}\mathbf{\hat{y}},\quad
    \mathbf{a}_2^* = \frac{2\pi}{3a_0}(\sqrt{3}\mathbf{\hat{x}}+\mathbf{\hat{y}})
\end{equation}

Posons $V(\mathbf{r}-\mathbf{R})$ le potentiel d'un site atomique arbitraire.
Nous pouvons écrire l'hamiltonien d'un élecron dans le feuillet de graphène comme
:

\begin{equation}
    \mathcal{H} = \frac{p^2}{2m} + V(\mathbf{r}-\mathbf{R}) + (V(\mathbf{r-R-b}))
\end{equation}

Nous allons déterminer les énergies $\mathcal{E}(\mathbf{k})$ à partir d'une
approche variationnelle.

Prenons la fonction d'onde :

\begin{equation}
    |\psi_k> = \frac{1}{\sqrt{N}} \sum_{\mathbf{R}} \exp (i
    \mathbf{k}\cdot\mathbf{R}) \left( \mathbf{a}_k |\phi_{R_0}> + \mathbf{b}_k
    |\phi_{R_b}> \right)
\end{equation}

Ici, $<r|\psi_{R_0}> = \phi(\mathbf{r}-\mathbf{R})$ et représente l'orbitale
$2p_z$ sur le site 0.

$<r|\psi_{R_b}> = \phi(\mathbf{r}-\mathbf{R}-\mathbf{b})$ erprésente l'orbitale
$2p_z$ au site 1.

L'énergie peut s'écrire comme :
\begin{equation}
    \mathcal{E}(\mathbf{k}) = \frac{<\psi_k|\mathcal{H}|\psi_k>}{<\psi_k|\psi_k>}
\label{eq:grapheneenergie}
\end{equation}

On veut minimiser $\mathcal{E}(\mathbf{k})$. Par conséquent, il faut trouver
$\mathbf{a}_k$ et $\mathbf{b}_k$.

On peut alors réécrire l'équation \ref{eq:grapheneenergie}
\begin{equation}
    <\psi_k|\mathcal{H}|\psi_k> = \mathcal{E}(\mathbf{k}) <\psi_k|\psi_k>
\end{equation}

En dérivant, on obtient :
\begin{eqnarray}
    \frac{\partial}{\partial \mathbf{a}_k} <\psi_k|\mathcal{H}|\psi_k> & = &
    \frac{\partial \mathcal{E}_k}{\partial \mathbf{a}_k} <\psi_k|\psi_k> +
    \mathcal{E}_k \frac{\partial}{\partial\mathbf{a}_k} <\psi_k | \psi_k>\\
    \frac{\partial}{\partial \mathbf{b}_k} <\psi_k|\mathcal{H}|\psi_k> & = &
    \frac{\partial \mathcal{E}_k}{\partial \mathbf{b}_k} <\psi_k|\psi_k> +
    \mathcal{E}_k \frac{\partial}{\partial\mathbf{b}_k} <\psi_k | \psi_k>
\end{eqnarray}

Dans cette formulation, les énergies $\mathcal{E}_k$ des électrons sont
indépendantes de leurs positions, donc des $\mathbf{a}_k$ et des $\mathbf{b}_k$.
Par conséquent, les dérivées sont nulles, et on peut réécrire :

\begin{eqnarray}
    \frac{\partial}{\partial \mathbf{a}_k} <\psi_k|\mathcal{H}|\psi_k> & = &
    \mathcal{E}_k \frac{\partial}{\partial\mathbf{a}_k} <\psi_k | \psi_k>\\
    \frac{\partial}{\partial \mathbf{b}_k} <\psi_k|\mathcal{H}|\psi_k> & = &
    \mathcal{E}_k \frac{\partial}{\partial\mathbf{b}_k} <\psi_k | \psi_k>
\end{eqnarray}

\begin{equation}
    <\psi_k|\psi_k> = \int dk (...)
\end{equation}

Soit finalement,

\begin{eqnarray*}
    <\psi_k | \psi_k> & = & \frac{1}{N} \sum_{\mathbf{R}'}
    e^{i\mathbf{k}\cdot\mathbf{R}'} (\mathbf{a}_k^* <\psi_{\mathbf{R}_0'}| +
    \mathbf{b}_k^* <\psi_{\mathbf{R}_b'}|) \cdot \sum_\mathbf{R}
    e^{i\mathbf{k}\cdot \mathbf{R}} (\mathbf{a}_k |\psi_{R_0}> + \mathbf{b}_k |
    \psi_{\mathbf{R}_b}>) \\
    . & = & \frac{1}{N} \sum_{\mathbf{R,R}'}
    e^{i(\mathbf{k}\cdot\mathbf{R}-\mathbf{k}'\cdot\mathbf{R}')}
    (\mathbf{a}_k^*\cdot\mathbf{a}_k <\psi_{\mathbf{R}_0'}|\psi_{\mathbf{R}_0}> +
    \mathbf{b}_k^* \cdot\mathbf{b}_k <\psi_{\mathbf{R}_b'}|\psi_{\mathbf{R}_b}>)
\end{eqnarray*}

D'où on tire :

\begin{align}
    <\psi_k|\psi_k> & = &\frac{1}{N} \sum_R (a_k^2 <\phi_{\mathbf{R}_0'} |
    \phi_{\mathbf{R}_0}> + b_k^2 <\phi_{R_b'}|\phi_{R_b}>)\\
    & = & a_k^2 + b_k^2
    & = & \mathbf{a}_k\cdot\mathbf{a}_k^* + \mathbf{b}_k\cdot\mathbf{b}_k^*
\end{align}

On peut donc retrouver :

\begin{eqnarray}
    \frac{\partial}{\partial \mathbf{a}_k} <\psi_k|\psi_k> & = & \mathbf{a}_k^* \\
    \frac{\partial}{\partial \mathbf{b}_k} <\psi_k|\psi_k> & = & \mathbf{b}_k^*
\end{eqnarray}

\TODO
