\chapter{Équations du mouvement}

Nous démontrons les équations du mouvement d'un électron dans une bande
d'énergie. Nous regardons le mouvement d'un paquet d'ondes dans un champ
électrique donné. Supposons que le paquet d'ondes est fait de fonctions d'ondes
assemblées au voisinage d'un vecteur d'onde $k$. La vitesse de groupe, est par
définition $v_g = \frac{d\omega}{dk}$. La fréquence associée à cette fonction
d'onde d'énergie $\epsilon$, dans la théorie quantique est $\omega =
\frac{\epsilon}{\hbar}$, et ainsi :

\begin{equation}
    v_g = \frac{1}{\hbar} \frac{d\epsilon}{dk}
\end{equation}

Les effets du cristal dues au mouvement d'électrons sont contenus dans la
relation de dispersion $\epsilon(k)$.

Le travail $\delta \epsilon$ fourin par le champ électrique E sur l'électron en
un temps $\delta t$ est :
\begin{equation}
    \delta \epsilon = - e E v_g \delta t
\end{equation}
On peut observer que :
\begin{equation}
    \delta \epsilon = (\frac{d\epsilon}{dk})\delta k = \hbar v_g \delta k
\end{equation}

En combinant ces trois  équations, on obtient :
\begin{equation}
    \delta k = -(eE/\hbar)\delta t
\end{equation}
d'où $\hbar \frac{dk}{dt} = -eE$
On peut réécrire cette équation en faisant intervenir la force $\mathbf{F}$ : 
\begin{equation}
    \hbar \frac{d\mathbf{k}}{dt} = \mathbf{F}
\end{equation}

C'est une relation importante : dans un cristal, $\hbar d\mathbf{k}/dt$ est égal
à la force subie par l'électron. Dans l'espace libre, $d(m\mathbf{v})/dt$ est
égal à la force. Mais attention, on n'est pas en contradiction avec la seconde
loi de Newton : l'électron dans le cristal est sujet à des forces du réseau
cristallin autant que de sources externes.

Le terme de force dans cette équation inclut aussi un champ électrique et une
force de Lorentz sur un élecrton dans un champ magnétique, sous certaines
conditions, où le champ magnétique est assez faible pour ne pas casser la
structure de bandes. Ainsi, l'équiation du mouvement d'un élecrton, d'une vitesse
de groupe $\mathbf{v}$ dans un champ magnétique constant est :
\begin{equation}
    \hbar \frac{d\mathbf{k}}{dt} = -e\mathbf{v}\times{B}
\end{equation}

où le terme de droite est la force de Lorentw sur l'électron. Avec la vitesse de
groupe $\mathbf{v} = \frac{1}{\hbar} \grad_\mathbf{k}\mathbf{\epsilon}$, le taux
de variation du vecteur d'onde est
\begin{equation}
    \frac{d\mathbf{k}}{dt} = -\frac{e}{\hbar} \Delta_\mathbf{k}\mathbf{\epsilon}
    \times \mathbf{B}
\end{equation}
où chaque côté de l'équation se réfère aux coordonnées de l'espace réciproque.

Nous voyons que le produit vectoriel dans cette dernière équation, que dans un
champ magnétique, un électron se déplace dans l'espace réciproque dans la
direction normale à la direction du gradient d'énergie $\epsilon$, de telle sorte
à ce qu'\emph{un électron se déplace sur les surfaces isoénergétiques}. La valeur
de la profection $k_\mathbf{B}$ de $\mathbf{k}$ sur $\mathbf{B}$ est constante
durant le mouvement. Le mouvement dans l'espace réciproque est sur un plan normal
à la direction de $\mathbf{B}$, et l'orbite est définie par l'intersection de ce
plan avec la surface isoénergétique.

\section{démonstration physique de $\hbar \mathbf{k} = \mathbf{F}$}
Considérons la fonction propre de Bloch $\psi_k$, associée à l'énergie propre
$\mathbf{\epsilon_k}$ et au vecteur d'onde $\mathbf{k}$ :
\begin{equation}
    \psi_\mathbf{k} = \sum_\mathbf{G} C(\mathbf{k+G})\exp [i(\mathbf{k+G})\cdot
    \mathbf{r}]
\end{equation}
La valeur attendue du moment d'un électron dans l'état $\mathbf{k}$ de Bloch est
:
\begin{equation}
    \mathbf{P_el} = <\mathbf{k}|-i\hbar\Delta|\mathbf{k}> = \sum_\mathbf{G} \hbar
    (\mathbf{k+G}) |C(\mathbf{k+G})|^2 = \hbar (\mathbf{k} + \sum_\mathbf{G}
    \mathbf{G}|C(\mathbf{k+G})|^2)
\end{equation}
en utilisant l'égalité $\sum |C(\mathbf{k+G})|^2 = 1$

Nousévaluons le transfert de moments entre l'électron et le réseau, lorsque
l'électron pass de $\mathbf{k}$ à $\mathbf{k}+\Delta\mathbf{k}$, à cause d'une
force extérieure. Nous imaginons un cristal isolant, neutre électrostatiquement,
excepté pour un seul électron dans l'espace réciproque d'une autre bnade vide.
Supposons qu'une force externe faible soit appliquée pour un intervalle de temps
tel que l'implusion totale donnée au système cristallin entier soit $\mathbf{J} =
\int \mathbf{F} dt$. Si l'électron de conduction est libre ($m*=m$), alors le
moment total transim au système cristallin par l'impulsion apparaîtrait comme la
différentiell edu moment de l'électron de conduction :
\begin{equation}
    \mathbf{J} = \Delta\mathbf{p_{tot}} = \Delta\mathbf{p_{el}} = \hbar \Delta
    \mathbf{k}
\end{equation}
Le cristal neutre ne subut aucune interaction avec le champ électrique, que ce
soit directement ou indirectement à travers les électrons libres.
Si l'électron de conduction interagit avec le potentiel périodique du réseau
cristallin, nous devons alors avoir :
\begin{equation}
    \mathbf{J} = \Delta{p_{tot}} = \Delta \mathbf{p_{rés}} +
    \Delta\mathbf{p_{el}}
\end{equation}
Ce qui donne :
\begin{equation}
    \Delta \mathbf{p_{el}} = \hbar \Delta \mathbf{k} + \sum_\mathbf{G}
    \hbar\mathbf{G}[(\nabla_k|C(\mathbf{k+G})|^2)\cdot\Delta\mathbf{k}]
\end{equation}
La variation $\Delta \mathbf{p_{rés}}$ dans le moment du réseau résultant de la
variation de l'état de l'électron peut être démontrée par des considérations
physiques élémentaires. Un élecrton, réfléchi par le réseau, transfeère un moment
au réseau. Si un électron indicdent avec une composante du moment sous forme
d'onde plane $\hbar \mathbf{k}$ est réfléchi avec un moment
$\hbar(\mathbf{k+G})$, le réseau acquierre un moment $-\hbar \mathbf{G}$, c'est
la conservation du moment. Le transfert de moments au réseau lorsque l'état varie
de $\psi_\mathbf{k}$ à $\psi_{\mathbf{k+\Delta k}}$ est :
\begin{equation}
    \Delta \mathbf{p_{rés}} = -\hbar \sum_\mathbf{G}
    \mathbf{G}[(\nabla_k|C(\mathbf{k+G})|^2)\cdot\Delta\mathbf{k}]
\end{equation}
car la partie $(\nabla_k|C(\mathbf{k+G})|^2)\cdot\Delta\mathbf{k}$ de chaque
composante individuelle de l'état initial est réfléchie pendat le chamgement
d'état $\Delta \mathbf{k}$. La différentielle du moment devient alors :
\begin{equation}
    \Delta \mathbf{p_{el}} + \Delta \mathbf{p_{rés}} = \mathbf{J} = \hbar \Delta
    \mathbf{k}
\end{equation}
ce qui est exactement la forme pour des électrons libres. Par conséquent, à
partir de la définition de $\mathbf{J}$, on obtient :
\begin{equation}
    \hbar \frac{d\mathbf{k}}{dt} = \mathbf{F}
\end{equation}
démontré plus tôt par une méthode différente. Une démonstration riguoureuse est
page 655 de Kittel. à regarder. peut-être à ajouter. à voir.

\section{Trous}

Les propriétés des orbitales vacantes dans une bande remplie sont importantes
dans la physique des semiconductors et dans l'électronique de l'état solide. Les
orbitales vacantes dans une bande sont communément appelés trous, et sans trous,
il n'y aurait pas de transistors. Un trou agit dans un champ électrique et
magnétique, comme s'il avait une charge positive $+e$. Cela peut être démontré de
la façon suivante :

page 194 à 196 kittel.

\section{Masse effective}
Lorsque l'on regarde à la relation d'onde en énergie $\epsilon =
(\frac{\hbar^2}{2m})k^2$ pour des électrons libres, on voit que le coefficient de
$k^2$ détermine la courbure de $\epsilon$ en fonction de $k$. Ainsi, on peut dire
que $1/m$, la masse réciproque, détermine la courbure. Pour des électrons dans
une bande, il y a des réguions de courbure anormalement hautes, près de la bande
interdite, et aux frontières, comme on a pu le voir précédemment, à partir des
solutions de l'équation d'onde aux limites de zone de Brillouin. Si l'énergie du
gap est petite devant l'anérgie $\lambda$ de l'électron libre à la limite de la
zone, la courbure est augmentée d'un facteur $\lambda/E_g$.

Dans les semiconducteurs, la largeur de bande, qui est telle que l'énergie de
l'électron libre est de l'ordre de 20eV, tandis que le bandgap est de l'ordre de
0.2 à 2eV. Ainsi, la masse réciproque est augmentée d'un facteur 10 ou 100, et la
masse effective est réduite au dixième ou au centième de la masse de l'électron
libre. Ces valeurs s'appliquent près du bandgap, lorque l'on s'en éloigne, la
courbure et les masses se rapprochent de celles de l'électron libre.

Pour résumer les solutions que nous avons vu précedemment pour U positif, un
électron proche de la partie la plus basse de la seconde bande a une énergie qui
peut être écrite comme :
\begin{equation}
    \epsilon(K) = \epsilon_c + (\hbar^2/2m_e)K^2 ; \hfill m_e/m =
    1/[(2\lambda/U)-1]
\end{equation}
Ici, K est le vecteur d'onde mesuré de la bordure de zone, et $m_e$ représente la
masse effective d e l'électon proche de la frontière de la seconde bande. Un
électron proche du haut de la première bande a l'énergie :
\begin{equation}
    \epsilon(K) = \epsilon_v - (\hbar^2/2m_h)K^2 ; \hfill m_h/m =
    1/[(2\lambda/U)+1]
\end{equation}
cf tut pour ajout d'informations.

La courbure et par conséquent la masse seront négatives proches du haut de la
première bande, mais nous avons introduit un signe moins dans la dernière
équation, de sorte à ce que le symbole $m_h$ pour la masse du trou ait une valeur
positive.

Le cristal ne pèse pas moins si la masse effective d'un porteur est mois que la
masse d'un électron libre, et de même, la seconde loi de Newton est respecté pour
le cristal \emph{pris dans son ensemble}, avec les ions plus les porteurs. Le
point important est qu'un électron dans un potentiel périodique est accéléré
relativement au résaeu dans un champ électrique ou magéntique, comme si la masse
de l'électron était égale à la masse éffective que nous venons de définir.

On dérive le résultat pour la vitess ede groupe et on obtient alors :

\begin{equation}
    \frac{dv_g}{dt} = \frac{1}{hbar} \frac{d^2\epsilon}{dk dt} = \frac{1}{\hbar}
    \left( \frac{d^2e}{dk^2} \frac{dk}{dt}\right)
\end{equation}
On sait depuis la section précédete que $dk/dt = F/\hbar$, d'où on tire :

\begin{equation}
    \frac{dv_g}{dt} = \left( \frac{1}{\hbar^2} \frac{d^2\epsilon}{dk^2} \right) F
    ; \text{ soit } F = \frac{\hbar^2}{d^2 \epsilon/dk^2} \frac{dv_g}{dt}
\end{equation}
En identifiant $\hbar^2/(d^2\epsilon/dk^2)$ comme une masse, alors la précédente
équation peut se voir comme une formulation de la seconde loi de Newton. On
définit la masse effective $m*$ par :

\begin{equation}
    \frac{1}{m*} = \frac{1}{\hbar^2} \frac{d^2\epsilon}{dk^2}
\end{equation}
Il devient facile de gééraliser ceci pour prendre en compte la surface d'énergie
d'un électron anisotropique, comme pour les électrons du silicium ou du
germanium. On introduit alors les composantes du tenseur de masse éffective
réciproque :

\begin{equation}
    \left( \frac{1}{m*} \right)_{\mu \nu} = \frac{1}{\hbar}
    \frac{d^2\epsilon_k}{dk_\mu dk_\nu} ; \hfill \frac{dv_\mu}{dt} =
    \left(\frac{1}{m*}\right)_{\mu \nu} F_\nu
\end{equation}
où $\mu, \nu$ sont les coordonnées cartésiennes

\section{interprétation physique de la masse effective}
Comment un électron de masse $m$ mis dans un crital répond-il à un champ appliqué
comme si sa masse était $m*$ ? Il est utile de penser au processus de réflection
de Bragg d'ondes délectrons dans un réseau. Considérons l'approximation de
l'interaction faible, traitée précédemment. Près du bas de la bande faible,
l'orbitale est représentée de façon plutôt juste par une onde plane $\exp(ikx)$
avec un moment $\hbar k$. La composant ede l'onde $\exp(i(k-G)x)$ avec le moment
$\hbar(k-G)$ est petite est augmente faiblement avec k, et dans cette région
$m*\simeq$. Une augemntation de la composante réfléchie $\exp(i(k-G)x)$ avec k
représente un transfert du moment du réseau sur l'électron.

Près des limites de zone, la composante réfléchie est importante ; à la limite,
elle devient égale en amplitude avec la composante directe, et en ce point les
fonctions propres sont des ondes stationnaires, plutôt que propagatives. Ici, la
composante du moment $\hbar(-1/2 G)$ annule la composante du moment $\hbar(1/2
G)$.

Un électron seul dans une bande d'énergie peut avoir une masse effective positive
ou négative : les états de masse effective positive apparaissent près du bas de
la bande, à cause du fait que la masse effective positive implique que la bande a
une courbure vers le haut ($d^2\epsilon/dk^2$ positif). Les états de masse
effective négative apparaissent près du haut de la bnade. Une masse effective
négative signifique qu'en passant d'un état $k$ à un état $k+\Delta k$, le
transfert de moment de l'électron au réseau est plus grand que le transfer de
moment de la force appliquée à l'électron. Même si $k$ est augmenté de $\Delta k
$ par le champ électrique appliqué, l'approche de la réflexion de Bragg peut
donner une décroissance globale du moment direct des électrons, quand cela
arrive, la masse effective est négative.

Lorsque nous travaillons sur la seconde bande, loin des bords de zone,
l'amplitude de $\exp[i(k-G)x]$ décroit rapidement et $m*$ a une valeur positive
faible. Ici, l'augmentation de la vitesse des électrons, résultant d'un stimulus
extérieur donné est plus grande que celle qu'aurait donné l'expérience d'un
électro libre. Le réseau construit une différence à travers le léger recul qu'il
subit lorsque l'amplitude de $\exp[i(k-G)x]$ diminue.

Si l'énergie dans une bande dépend seulement faiblement de $k$, alors la masse
effective sera très importante. Cela étant, $m*/m \gg 1$ lorsque $d^2\epsilon /
dk^2 \ll 1$. Le modèle des liaisons fortes, discuté précedemment, donne un aperçu
de la formation de bandes étroites. Si les fonctions d'onde centrées aux
voisinages d'atomes se superposent un peu, alors l'intégrale de recouvrement sera
petite ; la largeur de bande étroite, et la masse effective importantee. La
superposition des fonctions d'ode centrée sur un atome voisin est petite pour les
électrons internes ou de cœur. Les électrons 4f des métaux rares, par exemple, se
superposent faiblement.

\section{Masses effectives dans les semiconducteurs}

Dans beaucoup de semiconducteurs, il a été possible de déterminer, par résonnance
cyclotron, la masse effective des porteurs des bandes de valence et de conduction
près des bords de bandes. La détermination de la surface d'énergie est
équivalente à la détermination du tenser deu masse effective. La résonnance
cyclotron dans un semiconducteur est effectuée avec des ondes d'une longueur
d'onde du milli ou du centimètre, à de faibles concentrations en porteurs.

Les porteurs de courrant sont accélérés sur des horbites hélicoïdales sur l'axe
d'un champ magnétique statique. La fréquence de rotation angulaire $\omega_c$ est
donnée par :
\begin{equation}
\omega_c = \frac{eB}{m*}
\end{equation}
où $m*$ est la masse effective du cyclotron approprié. L'absorption de résonnance
de l'énergie pour un champ électrique rf perpendiculaire au champ magnétiqu
statique apparaît lorsque la fréquence rf est égale à la fréquence cyclotron. Les
trous et les électrons tournent dans des sens opposés dans un champ magnétique.

On considère l'expérience pour $m*/m=0.1$. À $f_c=24GHz$ ou $\omega_c = 1.5
\times 10^{11} s^{-1}$, nous avons $ B = 860 G$ à la résonnance. La largeur de
ligne est déterminée par le temps de relaxation de collision $\tau$, et pour
obtenir une résonnance distinctive, il est nécessaire que $\omega_c \tau \geq 1$.
Le libre parcours moyen doit être suffisemmet long pour permettre aux porteurs
moyens de parcourir un radian autour d'un cercle avant d'entrer en collision. Il
est nécessaire d'utiliser une fréquence importante et un champ magnétique
importante, avec des cristaux purs dans l'hélium liquide.

Dans des semiconducteurs à gap direct, avec des bords de bandes aux centre de la
zone de Brillouin, les bandes ont la structure donnée par la figure suivante
\TODO figure 13kittel. Le bord de la bande de coduction est sphérique avec la
masse effective $m_0$ :
\begin{equation}
    \epsilon_c = E_g + \hbar^2 k^2 / 2m_e
\end{equation}

La bande de valence est généralement repliée près du bord, avec les bandes des
trous lourds hh et des trous légers lh dégénérées au centre, et une bande soh
splittée par le couplage spin-orbite $\Delta$ :
\begin{eqnarray}
    \epsilon_v(hh) \simeq -\hbar^2 k^2/2m_{hh}\\
    \epsilon_v(lh) \simeq -\hbar^2 k^2/2m_{lh}\\
    \epsilon_v(soh) \simeq - \Delta - \hbar^2 k^2/2m_{soh}
\end{eqnarray}

Les valeurs des paramètres de masse sont données dans le tableau suivant
\TODO tableau (page 201 kittel)

Les formes données ici ne sont que des approximations, parce que même proche de
$k = 0$, les bandes des trous lourds et légers ne sont pas sphériques, comme nous
allons le montrer dans le cas du Silicium et du Gemanium.

La théorie des perturbations aux bords des bandes usggère que la masse effective
de l'électron doit être proportionelle au bandgap, approximativement, pour un
cristal à gap direct. Les deux tables précédentent montrent que ces valeurs sont
cnastantes pour la série InSb, InAs et InP, conformément à cette hypothèse

\section{Silicium et Germanium}

Les bandes de conduction et de valence du germanium sont montrées sur la figure
suivante, basés sur une combinaisons de résultats théoriques et expérimentaux. Le
bord de la bande de valence, à la fois pour le silicium et le germanium, est à
$\mathbf{k}=0$ et cela peut être démontré à partir des états $p_{3/2}$ et
$p_{1/2}$ des atomes libres, clairement dans le modèle des liaisons fortes pour
les fonctions d'onde.

Le niveau $p_{3/2}$ est quadruplement dégénéré comme pour l'atome ; les quatre
états correspondent à des valeurs de $m_J$ de $\pm \frac{3}{2}$ et $\pm
\frac{1}{2}$. Le niveau $p_{1/2}$ est doublement dégénéré, avec $m_J = \pm
\frac{1}{2}$. Les niveau $p_{3/2}$ sont plus haut en énergie que les niveaux
$p_{1/2}$ ; la différence d'énergie $\Delta$ est une mesure de l'interaction
spin-orbite.

Les bords de la bande de valence ne sont pas simples. Les trous près du bord de
la bande sont caractérisés par deux masses effectives, légère et lourde.
Celles-ci proviennent de deux bandes formées des niveaux $p_{3/2}$ des atomes. Il
y a également une bande formée depuis le niveau $p_{1/2}$, séparé du niveau
$p_{3/2}$ par le couplage spin-orbite. Les surfaces d'énergie ne sont pas
sphériques, mais déformées :
\begin{equation}
    \epsilon(\mathbf{k}) = Ak^2 \pm [B^2 k^4 + C^2(k_x^2k_y^2 + k_y^2k_z^2 +
    k_z^2k_x^2)]^{1/2}
\end{equation}
Le choix du signe distingue les deux masses. La bande séparée correspond à
l'équation $\epsilon(k) = -\Delta + Ak^2$. L'expérience donne, en unités
$\hbar^2/2m$ :

\begin{tabular}{lllll}
Si : & A = -4.29 ; & |B| = 0.68 ; & |C|=4.87 ; & $\Delta$=0.044eV \\
Ge : & A = -13.38; & |B| = 8.48; & |C| = 13.15 ; & $\Delta$=0.29eV
\end{tabular}

Grosso modo, les trous légiers et lourds du germanium ont une masse de 0.043m et
0.34m. Dans le silicium, 0.16m et 0.52m. Dans le diamant, 0.7m et 2.12m.

Les bords de la bande de conduction dans Ge sont à des points équivalents L de la
zone de Brillouin. Chaque bord de bande a une surface d'énergie d'orientation
sphéroïdale, suivant l'axe <111>, avec une masse longitudinale $m_l = 1.59m$ et
une masse transverse $m_t = 0.082m$. Pour un champ magnétique statique à un angle
$\theta$ avec l'axe longitudinal d'une sphéroide, la masse effective
cyclotronique $m_c$ est :
\begin{equation}
    \frac{1}{m_c} = \frac{\cos^2\theta}{m_t^2} + \frac{\sin^2 \theta}{m_t m_l}
\end{equation}

Les résultas pour le germanium sont motrées sur la figure ... (masse effective
cyclotronique)
Dans le silicium, las bords de la bnade de conduction sont des sphéroides
orietées selon les directions équivalentes <100> dans la zone de Brillouin, avec
des paramètres de masse $m_l = 0.92m$ et $m_t = 0.19m$, motnré sur la figure ...
Les bords de bande se placent selon les lignes marqées $\Delta$ dans la zone, un
peu déplacées des bords X.

Dans GaAs, on a A = -6.98, B = -4.5, |C| = 6.2, $\Delta$= 0.341eV. la structure
de bande est montrée ... il y a un gap direct avec une masse des électron de
conduction isotropique de 0.067m.

\TODO : points classiques pour réseaux bcc, fcc, etc. gamma, delta, sigma, G, N,
P, etc. et rajouter les autres en annexes



\begin{marginfigure}
\TODO
\caption{bande du germanium}
\label{bandeGe}
\end{marginfigure}
