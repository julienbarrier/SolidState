\chapter{Quantification : les phonons}

L'énergie des vibrations de réseau peut être quantifiée. La quantité d'énergie
est appelée \emph{phonon}, par analogie avec le photon d'une onde
électromagnétique. L'énergie d'un mode élastique de fréquence angulaire $\omega$
est :
\begin{equation}
\mathcal{E} = \left( n + \frac{1}{2} \right) \hbar \omega
\end{equation}
Dans cette expression, le mode est excité au nombre quantique n. C'est le cas
lorsque le mode est occupé par $n$ photons. Le terme $\frac{1}{2}\hbar\omega$
correspond au niveau d'énergie fondamental. Il existe à la fois pour les photons
et les phonons en conséquence de leur équivalence à un oscillateur harmonique quantique de pulsation $\omega$, pour lequel les valeurs prpores de
l'énergie sont également $(n+\frac{1}{2})\hbar\omega$. On peut quantifier le
carré moyen de l'amplitude pour le mode stationnaire :
\begin{equation}
u = u_0 \cos kx \cos \omega t
\end{equation}

Où u est le déplacement d'un élément de volume de sa position d'équilibre à x
dans le cristal. L'énergie du mode, comme dans un oscillateur harmonique, est
composée pour moitié d'énergie cinétique, et pour moitié d'énergie potentielle
lorsque moyenné sur le temps. La densité d'énergie cinétique est :
\begin{equation}
\frac{1}{2}\rho \left( \frac{\partial u}{\partial t} \right)^2
\end{equation}

où $\rho$ est la densité de masse.

Dans un cristal de volume V, l'intégrale volumique de l'énergie cinétique
peut s'écrire :
\begin{equation}
\frac{1}{4}\rho V \omega^2 u_0^2 \sin^2 \omega t
\end{equation}
a moyenne au cours du temps s'écrit : 
\begin{equation}
    \frac{1}{8} \rho V \omega^2 u_0^2 = \frac{1}{2} \left( n+\frac{1}{2} \right) \hbar \omega
\end{equation}
Le carré de l'amplitude du mode s'écrit :
\begin{equation}
u_0^2 = \frac{4\left( n+\frac{1}{2} \right) \hbar}{\rho V \omega}
\end{equation}

Cela décrit le déplacement de l'occupation n d'un phonon dans un mode donné.

La question à se poser est sur le signe de $\omega$. L'énergie d'un phonon doit
être, en principe, positive. Par conséquent, $\omega > 0$. Si la structure
critalline est instable, en revanche, $\omega^2 < 0$ et alors $\omega \in i\mathcal{R}$.

Un phonon de vecteur d'onde $k$ peut interagir avec des particules comme des
photons, des neutrons et des électrons, comme s'il avait une quantité de
mouvement $\hbar k$. Cependant, un phonon ne transporte physiquement pas 
d'impulsion.

La raison pour laquelle un phonon dans unréseau n'a pas de quantité de mouvement
est que a coordonnée d'un phonon (sauf pour $k=0$) implique le problème soit
traité à partir des coordonnées relatives des atomes. Par conséquent, dans une
olécule comme \ch{H2}, la vibration internucléaire $\mathbf{r_1}-\mathbf{r_2}$
est en coordonnées relative et n'a pas d'impulsion linéaire . Le centre de masse
$\frac{1}{2}(\mathbf{r_1}-\mathbf{r_2}$ correspond au mode $k=0$ et peut avoir
une quantité de mouvement linéaire.

Dans les cristaux, il existe des règles de sélection des vecteurs d'ondes pour
les transitions autorisées (entre les états quantifiés).
Par exemple, pour la diffusion élastique en diffraction par les rayons X, la
règle de sélection est : 
\begin{equation}
\mathbf{k'} = \mathbf{k} + \mathbf{G}
\end{equation}




\section{Dynamique, modes normaux, propriétés}

\section{Capacité calorifique des phonons}

La capacité calorifique à volume constant peut s'écrire
$C_v = \left( \frac{\partial U}{\partial T}\right)_V$. La contribution des
phonons à la capacité calorifique est appelée chaleur latente du réseau et est
notée $C_{res}$.

L'énergie totale des phonons dans un cristal peut s'écrire comme la somme des
énergies sur tous les modes de phonons, ici indexés par le vecteur d'onde
$k$ et l'indice de polarisation $p$.

\begin{equation}
    U_{res} = \sum_k \sum_p U_{k,p} = \sum_k \sum_p <n_{k,p}> \hbar \omega_{k,p}
    \label{Ures}
\end{equation}
Dans cette formule, $<n_{k,p}>$ représente l'occuption des phonons de vecteur
d'onde $k$ et de polarisation $p$ à l'équilibre thermique. Il est donné par la
distribution de Planck :

\begin{equation}
    <n> = \frac{1}{\exp \left( \frac{\hbar \omega}{\tau} \right) - 1}
    \label{distrplanck}
\end{equation}

L'énergie d'un assemblage d'osscillateurs harmoniques de fréquence
$\omega_{k,p}$ à l'équilibre thermique est donné par l'équation \ref{Ures}.

Généralement, on préfère assimiler la somme sur $k$ à une intégrale. Pour cela,
supposons que le cristal a $D_p(\omega) d\omega$ modes d'une polarisation $p$
pour les fréquences comprises entre $\omega$ et $\omega + d\omega$.

On obtient alors :

\begin{equation}
    U = \sum_p \int d\omega D_p(\omega) \frac{\hbar \omega}{\exp \left( \frac{\hbar\omega}{\tau} \right) -1}
    \label{Uresint}
\end{equation}

La capacité calorifique du réseau est trouvée par différentiation de la fonction
de température $x = \frac{\hbar \omega}{\tau} = \frac{\hbar\omega}{k_B T}$ :

\begin{equation}
    C_{res} = k_B \sum_p \int d\omega D_p(\omega) \frac{x^2 \exp x}{(\exp x -1)^2}
    \label{Cres}
\end{equation}

À ce stade, tout le problème réside dans cette fonction $D(\omega)$ : le nombre
de modes pour une fréqence donnée. Cette fonction est appelée densité de modes,
ou plus généralement densité d'états.

\section{Densité d'états à une dimension}

Considérons le problème de vibrations unidimensionnelle : sur un segment de
longueur L, on transporte N+1 particules, chacune d'entre elles étant espacée
d'une distance a.

Supposons que les particules indexées par $s=0$ et $s=N$ sont fixées.
Chaque mode de vibration de polarisation p a la forme d'une onde stationnaire,
où $u_s$ est le déplacement de la particule $s$.

\begin{equation}
    u_s = u(0) \exp (-i \omega_{k,p} t) \sim (ska)
\end{equation}

Dans cette formule, $\omega_{k,p}$ est relié à $k$ par la relation de dispersion
appropriée.

Le vecteur d'onde $k$ est lui même restreint par les conditions limites :
$k = \frac{\pi}{L} ; \frac{2 \pi}{L} ; \frac{3\pi}{L} ; \cdots ; \frac{(N-1)\pi}{L} $.

La solution pour $k=\frac{\pi}{L}$ admet comme solution $u_s \propto \sin \left(\frac{s\pi a}{L} \right)$, et s'annule pour $s=0$ ou $s=N$.

La solution pour $k=\frac{N\pi}{L} = \frac{\pi}{a} = k_{frontière}$ est
$u_s \propto \sin (s\pi)$. Cela ne permet pas aux atomes de se déplacer car
$\sin s\pi$ est nul à chaque atome. Par conséquent, il y a $(N-1)$ valeurs de k
indépendantes autorisées. Ce nombre est égal au nobmre de particules qui peuvent
se mouvoir. Chaque valeur de $k$ est asociée à une onde stationnaire. Pour le
problème à une dimension, il y a un mode pour chaque intervalle
$\Delta k = \frac{\pi}{L}$, de sorte à ce que le nombre de modes par échelle
d'unité $k$ est $\frac{L}{\pi}$ pour $k \in ]-\frac{\pi}{a} ; \frac{\pi}{a} ]$.

Il y a trois polarisations $p$ possibles pour chaque $k$ :

À une dimension, deux d'entre elles sont transverses et une longitudinale.
À trois dimensions, la polarisation n'est simple que pour des vecteurs d'onde
dans les directions caractéristiques du crital. On peut utiliser les conditions
de Born von Karman : on considère le milieu infini, mais en supposant que les
solutions sont périodiques sur une distance $L$ grande, de sorte que
$u(sa) = u(sa + L)$.

\begin{marginfigure}
    \TODO
    \caption{BVK}
\end{marginfigure}

Considérons N particules contraintes sur un anneau circulaire. Les particules euvent oscillées si elles sont connectées par des ressorts.

Les modes normaux correspondent à des déplacements $u_s$ d'un atome $s$ de la
forme $\sin ska$ ou $\cos ska$ : ce sont des modes indépendants. La périodicité
impose les conditions aux limites : $u_{N+s} = u_s,\quad \forall s$. Par
conséquent, $Nka$ est un entier multiple de $2\pi$.

Pour une solution sous forme d'onde propagative, $u_s = u_0 \exp i (ska - \omega_k t)$, et donc les valeurs possibles pour k sont :

\begin{equation}
    k\in \{ 0,\pm\frac{2\pi}{L}, \pm\frac{4\pi}{L},\cdots, \frac{N\pi}{L} \} 
\end{equation}

Cette méthode de dénombrement donne le même nombre de modes (1 par atome mobile)
, mais on a maintenant des valeurs positives et négatives pour $k$, avec un
intervalle $\Delta k$ entre deux valeurs successives de $k$ égal à
$\frac{2\pi}{L}$. Pour des conditions limites périodiques (Born von Karman), le
nombre de modes par échelle de $k$ est $\frac{L}{2\pi}$ pour
$k \in ]\frac{-\pi}{a},\frac{\pi}{a} ]$ et 0 ailleurs.

Alors, on a besoin de connaître le nombre $D(\omega)$ de modes par unité de
fréquence pour une polarisation donnée. Le nombre de modes $D(\omega)d\omega$
entre $\omega$ et $\omega + d\omega$ est donné à une dimension par :

\begin{equation}
    D_1(\omega) d\omega = \frac{L}{\pi} \frac{dk}{d\omega} d\omega = \frac{L}{\pi} \cdot \frac{d\omega}{\frac{d\omega}{dk}}
\end{equation}

On peut obtenir la vitesse de groupe $\frac{d\omega}{dk}$ à partir de la
relation de dispersion $\omega = f(k)$.

Il y a une singularité dans $D_1(\omega)$ dès que $\omega(k)$ est horizontale,
c'est à dire lorsque la vitesse de groupe est nulle.


\section{Densité d'états à trois dimensions}

On applique les conditions périodiques sur $N^3$ mailles primitives. Dans un
cube de côté L, de sorte que $k$ soit déterminé par la condition :

\begin{equation}
    \exp \left[ i(k_x x + k_y y + k_z z) \right] = \exp i \left[ k_x (x+L) + k_y (y+L) + k_z (z+L) \right]
\end{equation}
Avec :

\begin{equation}
    k_x,k_y,k_z \in \left\{ 0, \pm\frac{2\pi}{L}, \pm \frac{4\pi}{L}, \cdots,\frac{N\pi}{L} \right\}
\end{equation}

Par conséquent, il n'y a qu'une valeur de $\mathbf{k}$ permise par volume de
$\left( \frac{2\pi}{L} \right)^3$ dans l'espace réciproque, c'est à dire
$\left( \frac{L}{2\pi} \right)^3 = \frac{V}{8\pi^3}$ valeurs de $\mathbf{k}$
permises par unité de volume dans l'espace réciproque, pour chaque branche ou
chaque polarisation. Le volume de l'échantillon est $V = L^3$.

Le nombre total des modes avec un vecteur d'onde plus petit que $k$ est le
produit du volume d'une sphère de rayon k par $\left( \frac{L}{2\pi} \right)^3$

Ainsi, pour chaque type de polarisation,

\begin{equation}
    N = \left( \frac{L}{2\pi} \right)^3 \left( \frac{4\pi k^3}{3} \right)
\end{equation}

La densité d'états de chaque polarisation est :

\begin{equation}
    D(\omega) = \frac{dN}{d\omega} = \left( \frac{Vk^2}{2\pi^2} \right) \frac{dk}{d\omega}
\end{equation}

\subsection{Modèle de Debye pour la densité d'états}

L'approximation de Debye consiste à dire que la vitess du son est constante pour
chaque type de polarisation, comme c'est le cas dans un milieu élastique continu
. La relation de dispersion s'écrit :

\begin{equation}
    \omega = v k
\end{equation}

Avec $v$ est la vitesse du son, qui est prise constante. Dans ce cas, la densité
d'états s'écrit :

\begin{equation}
    D(\omega) = \frac{V\omega^2}{2\pi^2v^3}
\end{equation}

Il y a N mailles primitives dans l'échantillon. Par conséquent, le nombre total 
de modes acoustiques de phonons est N.

La fréquence de coupure est $\omega_D$.
On peut écrire :

\begin{equation}
    \omega_D^3 = \frac{6\pi^2v^3 N}{V}
\end{equation}

À cette fréquence, on associe un vecteur d'onde de coupure dans l'espace
réciproque :

\begin{equation}
    k_D = \frac{\omega_D}{v} = \left( \frac{6\pi^2N}{V} \right)^{\frac{1}{3}}
\end{equation}

Le modèle de Debye n'autorise pas les modes de vecteurs d'ondes plus grands que
$k_D$. Le nombre de modes $k\leq k_D$ réduit le nombre de degrés de liberté
pour un réseau monoatomique.
L'énergie thermique est, pour chaque type de polarisation :

\begin{eqnarray}
U & = & \int d\omega D\omega <n(\omega)> \hbar \omega \\
 & = & \int_0^{\omega_D} d\omega \left( \frac{V\omega^2}{2\pi^2v^3} \right) \left( \frac{\hbar \omega}{e^{\frac{\hbar \omega}{\tau}} - 1} \right)
\end{eqnarray}

On admet que la vitesse des phonons est indépendante de la polarisation, donc on
obtient :

\begin{eqnarray}
    U &= & \frac{3V\hbar}{2\pi^2v^3} \int_0^{\omega_D} d\omega \frac{\omega^3}{\exp ( \frac{\hbar\omega}{\tau}) -1}\\
    \label{energieU}
    & = & \frac{3Vk_B^4 T^4}{2\pi^2v^3\hbar^3} \int_0^{x_D} dx \frac{x^3}{e^x -1}
\end{eqnarray}
Où on a effectué le changement de variables $x=\frac{\hbar\omega}{\tau}$. Posons
également la tepérature de Debye $\theta$, définie par la relation :
\begin{equation}
    x_D = \frac{\hbar \omega_D}{k_B T} = \frac{\theta}{T}
\end{equation}
Que l'on peut développer :

\begin{equation}
    \theta = \frac{\hbar v}{k_B} \left( \frac{6\pi^2 N}{V} \right)^3
\end{equation}

L'énergie totale de phonons est alors :

\begin{equation}
    U = 9 N k_B T \left( \frac{T}{\theta} \right)^3 \int_0^{x_D} dx \frac{x^3}{e^x -1}
\end{equation}

La capacité calorifique est trouvée facilement en différentiant l'expression
\ref{energieU}. Alors :

\begin{eqnarray}
    C_v &=& \frac{3V\hbar^2}{2\pi^2v^2k_B T^2} \int_0^{\omega_D} d\omega \frac{\omega^4 e^{\frac{\hbar \omega}{\tau}}}{\left(
        e^{\frac{\hbar\omega}{\tau}} -1 \right)^2}\\
        & = & 9 N k_B \left( \frac{T}{\theta} \right)^3 \int_0^{x_D} dx \frac{x^4 e^x}{(e^x -1)^2}
\end{eqnarray}

La capacité calorifique de Debye peut être alors tracée :

\begin{figure}
    \TODO
    \label{debye}
    \caption{capacité calorifique selon le modèle de Debye}
\end{figure}

Pour $T>>\theta$, la chaleur latente est approximativement celle obtenue par la
valeur classique de $3Nk_B$



\subsection{Einstein}

Prenons N oscillateurs de même fréquence $\omega_0$ et un à une dimension.
La densité d'état dans le modèle de Einstein est
$D(\omega) = N \delta(\omega - \omega_0)$. L'énergie thermique du système est:

\begin{equation}
    U = N <x> \hbar \omega = \frac{N \hbar \omega}{e^{\frac{\hbar \omega}{\tau}}-1}
\end{equation}

(Pour simplifier, j'écris $\omega$ à la place de $\omega_0$. On peut dériver
cette expression pour trouver la capacité calorifique :

\begin{equation}
    C_v = N k_B \left(\frac{\hbar \omega}{\tau} \right)^2 
    \frac{e^{\frac{\hbar\omega}{\tau}}}{(e^{\frac{\hbar \omega}{\tau}} - 1)^2}
\end{equation}

Le résultat pour la contribution de N oscillateurs identiques à la capacité
calorifique d'un solide. À 3 dimensions, N devient 3N.
La limite haute température de $C_v$ devient $3Nk_B$, ce qui correspond à la
valeur de Dulong et Petit.

À basse température, $C_v$ décroit de façon exponentielle :
$\exp \left( - \frac{\hbar\omega}{\tau} \right)$, alors qu'expérimenalement, 
c'est plus une décroissance en $T^3$ avec le modèle de Debye.

En revanche, le modèle d'Einstein est plus utilisé pour approximer la partie
phonon optique du spectre de phonons.

\subsection{haute température}

\subsection{$T^3$}

Pour $T<<\theta$, on a $x_D \rightarrow \infty$.

\begin{eqnarray}
    \int_0^\infty dx \frac{x^3}{e^x-1} & = & \int_0^\infty dx x^3 \sum_{s=1}^{\infty} \exp -sx\\
    & = & 6 \sum_1^\infty \frac{1}{s^4}\\
    & = & \frac{\pi^4}{15}
\end{eqnarray}

Par conséquent, pour $T<<\theta$,

\begin{equation}
    U ~= \frac{3 \pi^4 N k_B T^4}{5 \theta^3}
\end{equation}

Ainsi, on peut écrire la capacité calorifique :
\begin{eqnarray}
    C_v & = & \frac{12\pi^4}{5} N k_B \left( \frac{T}{\theta}\right)^3 \\
    & = & 234 N k_B \left( \frac{T}{\theta} \right) ^3
\end{eqnarray}

Pour de faibles températures, cette approximation est pas mauvaise. En fait,
ceci est valable lorsque seulement les modes acoustiques des grandes longueurs
d'onde sont excités therimquement. Ce sont juste les modes qui peuvent être
traités en tant que milieu élastique continu avec les constantes élastiques
macroscopiques.

Seuls les modes pour lesquels $\hbar \omega < k_B T$ seront excités de façon
suffisante à faible température. L'excitation de ces modes sera à peu près
classique, chacun d'entre eux avec une énergie $k_BT$.

Dans le volume autorisé dans l'espace réciproque, la fraction occupée par les
modes excités est de l'ordre de
$\left( \frac{\omega_T}{\omega_D} \right)^3$, soit
$\left( \frac{k_T}{k_D} \right)^3$ où $k_T$ est la longueur d'onde thermique,
définie par $\hbar v k_T = k_B T$ et $k_D$ est la longueur d'onde de coupure
du modèle de Debye.
La fraction occupée du volume total dans l'espace réciproque est
$\left( \frac{T}{\theta} \right)^3$. En outre, il y a de l'ordre de
$3N \left( \frac{T}{\theta} \right)^3$ modes excités, qui ont chacun une énergie
$k_B T$. L'énergie est approximativement
$3Nk_B T \left( \frac{T}{\theta} \right)^3$ et la capacité calorifique
$12Nk_B \left( \frac{T}{\theta} \right)^3$.

\begin{marginfigure}
    \TODO
    \caption{représentation des modes excités sur l'enseble des modes}
\end{marginfigure}

En pratique, pour que l'approximation soit correcte, il faut
$T=\frac{\theta}{50}$. Par exemple, pour le sicilium,
$\theta = \SI{645}{\kelvin}$ ; pour le germanium,
$\theta = \SI{374}{\kelvin}$, donc il faudrait travailler à des températures de 
l'ordre du kelvin pour que l'approximation soit correcte.

On peut noter que plus un atome est lourd, plus $\theta$ est petit. En fait,
la vitesse du son diminue lorsque la densité augmente, c'est ce qui provoque
cette variation en terme de vecteurs d'onde.

\section{Corps noir}
ppp
