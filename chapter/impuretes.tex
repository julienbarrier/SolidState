\chapter{Conductivité due aux impuretés}

Certaines impuretés et imperfections affectent dramatiqument les propriétés des
semiconducteurs. L'addition de bore dans le silicium dans des proportions de 1
atomes de bore pour $10^5$ de silicium augmente la conductivité du silicium pur à
température ambiante d'un facteur $10^3$. Dans un composé semiconducteur, une
déficience stoechiométrique en un constituent peut agir comme une impureté ; de
tels semiconducteurs sont appelés \emph{semiconducteurs déficitaires}. L'addition
délibérée d'impuretés à un semiconducteur est appelé \emph{dopage}.

On considère l'effet des impuretés dans le silicium et le germanium. Ces éléments
cristallisent dans la structure diamant. Chaque atome forme quatre liaisons
covalentes, une avec chacun de ses plus proches voisins, correspondant à la
valence de l'élément qui est de 4. Si une impureté, par exemple un atome de
valence 5 comm ele phosphore, l'arsenic au l'antimoine, est substituée dans le
réseau à la place d'un atome normal, il y aura un électron de valence de l'atome
de l'impureté laissé vide après que les 4 laisons covalentes se soient faites
avec les plus proches voisins.. C'est, après que l'atome d'impureté a été
accomodé dans la structure avec une légèe perturbation, possible. Les atomes
d'impureté qui peuvent donner des électrons sont appelés \emph{donneurs}.

\section{État donneur}
La structure présentée sur la figure suivante a une charge positive portée par
l'impureté (qui a perdu un électron). L'étude du réseau a vérifié que les
impuretés pentavalentes entrent dans le réseau par substitution d'atomes normaux,
et non pas aux positions interstitielles. Le cristal dans son ensemble demeure
neutre car les électrons restent dans le cristal.

\begin{marginfigure}
    \TODO
    \caption{charges associées avec une impureté d'arsenic dans le silicium. avec
    niveau d'énergie de la perturbation}
\end{marginfigure}

L'électron supplémentaire se déplace dans el potentiel de Coulomb $e/\epsilon r$
de l'impureté ionique, où $\epsilon$ est la constante diélectrique statique du
milieu dans un cristal covalent. Le facteur $1/\epsilon$ prend en compte la
réduction des forces de Coulomb enntre les charges causées par la polarisation
électronique du milieu. Ce traitement est valide pour les orbites larges en
comparaison avec la distance entr eles atomes, et pour des mouvements lents des
électrons, de telle sorte à ce que la fréquence d'orbite soit faible en
comparaison avec la fréquence $\omega_g$ correspondant au gap d'énergie. Ces
conditions sont vérifiées, et plutôt bien, dans le germanium et le silicium, pour
des électrons donneurs de P, As ou Sb.

On peut estimer l'énergie d'ionisation de l'impureté donneuse. La théorie de Bohr
de l'atome d'hydrogène peut être modifiée pour prendre en compte la constante
diélectrique du milieu et la masse effective d'un électron dans le potentiel
périodique du cristal. L'énergie d'ionisation d'hydrogène atomique est
$-e^4m/2(4\pi\epsilon_0\hbar)^2$.

Dans les semiconducteurs avec une constante diélectrique $\epsilon$, on remplace
$e^2$ par $e^2/\epsilon$ et $m$ par la masse effective $m_e$, et on obtient
l'énergie d'ionisation du doneur dans le semiconducteur :
\begin{equation}
    E_d = \frac{e^4 m_e}{2(4\pi\epsilon\epsilon_0 \hbar)^2}
\end{equation}

Le rayon de Bohr du bas de la bande de l'hydrogène est $4 \pi \epsilon_0\hbar^2 /
me^2$. Ainsi, le rayon de Bohr du donneur est :
\begin{equation}
    a_d = \frac{4 \pi \epsilon \epsilon_0 \hbar^2}{m_ee^2}
\end{equation}

L'ajout d'un état d'impureté au germanium et au silicium est compliquée par la
masse effective anisotropique des électrons de conduction. En revanche, la
constante diélectrique a un effet plus important sur l'énergie du donneur parce
qu'elle est au carré, alors que la masse effective apparaît directement sans
puissance.

Pour obtenir une impression générale des niveaux d'impureté, on doit utiliser
$m_e \approx 0.1m$ pour les électrons du germanium et $m_e \approx 0.2m$ dans le
silicium. Les constantes diélectriques statiques sont données dans le tableau
suivant L'énergie d'ionisation d'un atome d'hydrogène libre est 13.6eV. Pour l
germanium, l'énergie d'ionisation du donneur $E_d$ dans ce modèle est de $5meV$,
réduite avec l'hydrogène d'un facteur $m_e/m \epsilon^2 = 4\cdot 10^{-4}$. Le
résultat correspondant pour le silicium est de $20meV$. Les calculs utilisant le
tenseur de masse anisotropique approprié prédisent $9.05meV$ pour le germanium et
$29.8meV$ pour le silicum. Les valeurs observées des énergies d'ionisations des
doneurs dans le silicium et le germanium sot données dans la table suivante. Pour
l'arsenure de gallium, on a $E_d\approx \SI{6}{\milli\electronvolt}$.

\begin{table}[ht]
    \begin{tabularx}{\textwidth}{lRlRlR}
        \toprule
        cristal & $\epsilon$ & cristal & $\epsilon$ & cristal & $\epsilon$ \\
        \midrule
        Diamant & 5.5 & InAs & 14.55 & AlAs & 10.1 \\
        Si & 11.7 & InP & 12.37 & AlSb & 10.3 \\
        Ge & 15.8 & GaSb & 15.69 & SiC & 10.2 \\
        InSb & 17.88 & GaAs & 13.13 & $Cu_2O$ & 7.1\\
        \bottomrule
    \end{tabularx}
    \label{}
    \caption{constantes diélectriques statiques relatives des semiconducteurs}
\end{table}

\begin{table}[ht]
    \begin{tabularx}{\textwidth}{XRRR}
        \toprule
        & P & As & Sb \\
        \midrule
        Si & 45.0 & 49.0 & 39.0 \\
        Ge & 12.0 & 12.7 & 9.6\\
        \bottomrule
    \end{tabularx}
    \label{}
    \caption{Énergies d'ionisation de donneurs $E_d$ pour des impretés
    pentavalentes (\si{\milli\electronvolt})}
\end{table}
 Le rayon de la première orbite de Bohr est augmentée de $\epsilon m/m_e$ de
 \SI{0.53}{\angstrom} pour l'atome d'hydrogène libre. La rayon correspondant est
 $(160)(0.53) \simeq \SI{80}{\angstrom}$ pour le germanium et $(60)(0.53) \simeq
 \SI{30}{\angstrom}$ pour le silicium. Ce ssont de grands rayons, et par conséquent
 l'orbite du donneur se recouvre à des concentrations en donneurs relativements
 basses, comparés au nombre d'atome hôtes. Avec un recouvrement orbitaluire
 suffisnant, une \emph{bande d'impuretés} est formée à partir d'état donneur.
 
 Le semiconducteur peut conduire dans la bande d'impuretés grâce au
 \emph{hopping} des électrons de donneurs en donneurs. Le processus de conduction
 de bande d'impureté établit à une concentration en donneur faible un niveau s'il
 y a assez d'atomes acepteurs présents, de telle sorte à ce que les donneurs sont
 toujours ionisés. Il est plus facile pour un électrodonneur de sauter sur un
 doneur ionisé (inoccupé) que  sur un atome d'oneur occupé, pour que deux
 électrons n'occupent pas le même site pendant le transport de charge.
 
 \section{États accepteurs}
 
 Un trou peut être lié à une impureté trivalente dans le germunium ou le
 silicium, de la même façon qu'un électron est lié à une impureté pentavalente.
 Les impuretés trivalentes, comme B, Al, Ga ou In sont appelées des
 \emph{accepteurs} car ils acceptent des électros de la bande de valence pour
 compléter des liaisons covalentes avec les atomes voisins, laissant des trous
 dans la bande.
 
 
\begin{table}[ht]
    \begin{tabularx}{\textwidth}{XRRRR}
        \toprule
        & B & Al & Ga & In \\
        \midrule
        Si & 45.0 & 57.0 & 65.0 & 157.0 \\
        Ge & 10.4 & 10.2 & 10.8 & 11.2 \\
        \bottomrule
    \end{tabularx}
    \label{}
    \caption{Énergies d'ionisation d'accepteurs $E_a$ pour des impretés
    trivalentes (\si{\milli\electronvolt})}
\end{table}

Quand un accepteur est ionisé, un trou est libéré, ce qui requière une source
d'énergie. Dans le diagramme de bande usuel, un électron monte lorsqu'il gagne de
l'énergie, alors qu'un trou descend s'il en gagne.

L'ionisation expérimentale des énergies d'accepteurs dans le germanium et le
sicilium sont données sur la table ci-dessus. Le modèle de Bohr s'applique
qualitativement pour les trous de la même façon que poru les électrons, mais la
dégénerescence en haut de la bande de valence complique le problème de la masse
effective.

Les tables montrent que les énergies d'ionisation des doneurs et accepteurs dans
le silicium sont comparables avec $k_BT$ à température ambiante (26meV), de telle
sorte à ce que l'ionisation thermique des doneurs et accepteurs est importante
dans la conductivité électrique du silicium à température ambiante. Si les atomes
doneurs sont présents en nombre considérablement supérieur que les accepteurs,
l'ionisation thermique des doneurs libérera des électrons dans la bande de
conduction. La conductivité des espèces sera contrôlée par les électrons (charges
négatives) , et le matériau sera dit de type $n$.

Si les accepteurs sont dominants, les trous seront relachés dans la bande de
valence en même temps que la conductivité sera contrôlée par les trous (charges
positives) : le matériau est de typee $p$. Une autre technique pratique est le
signe du potentiel thermoélectrique, dont nous ne traiteront pas ici. peut être
trouvé chez Kittel, chap 8.

Le nombre de trous e t d'électrons est égal dans le régime intrinsèque. La
concentration intrinsèque en électrons $n_i$ à 300K est
\SI{1.7e13}{\per\cubic\centi\metre} dans le germanium et
\SI{4.6e9}{\per\cubic\centi\metre} dans le silicium. La résistivité électrique de
matériaux intrinsèques est \SI{43}{\ohm\centi\metre} pour le germanium et
\SI{2.6e5}{\ohm\centi\metre} pour le silicium.

Le germanium a \SI{4.42e22}{} atomes par centimètre cube. La purification du
germanium a été réalisée plus loin que pour d'autres éléments. La concentration
en impuretés communément électriquement actives - les impuretés donneuses ou
accepteuses superficielles - ont été réduites en dessous de une impureté pour
$10^11$ atomes de germanium. Par exemple, la concentration en potassium dans le
germanium peut être réduite sous \SI{4e10}{\per\cubic\centi\metre}. Il y a des
impuretés (H,O,Si,C) dont les concentrations dans le germanium ne peuvent
généralement pas être réduite sous \SI{e12}{} - \SI{e14}{\per\cubic\centi\metre},
mais cela n'affecte pas les mesures électriques et par conséquent est difficile à
détecter.

\section{Ionisation thermique des donneurs et accepteurs}

Le calcul des concentration d'équilibre des électrons de conduction depuis des
donneurs ionisés est identique avec le calcul standard de la mécanique
statistique de l'ionisation thermique des atomes d'hydrogène. S'il n'y a pas
d'accepteurs présents, le résultat à la limite basse température $k_BT \ll E_d$
donne :
\begin{equation}
    n \cong (n_0N_d)^{1/2} \exp (-E_d / 2k_BT)
\end{equation}
avec $n_0 \equiv 2(m_ek_BT/2\pi\hbar^2)^{3/2}$. Ici, $N_d$ est la concentration
en donneurs. Pour obtenir l'équation ci-dessus, on applique les lois de
l'équilibre chimique au ratio $[e][N_d^+]/[N_d]$ et on établit $[N_d^+]=[e]=n$.
Des résultats identiques sont valables pour les accepeteurs, avec les
considération des atomes non donneurs.

Si les conctentrations en donneurs et accepteurs sont comparables, les choses se
compliquent et les équations peuvent être résolues avec des méthodes numériques.
Ainsi, la loi d'action de masse requière que le produit $np$ doit être constant à
une température donnée. Un excès de donneurs augmentera la concentration en
électrons et une diminution de la concentration en trous. La somme $n+p$
augmentera. La conductivité augmente en $n+p$ si les mobilités sont égales.
%\ref{trucmuche}

\begin{marginfigure}
    \TODO
    \caption{à propos du dernier paragraphe}
    \label{trucmuche}
\end{marginfigure}

\section{Effets thermoélectriques}

Considérons un semiconducteur maintenu à une température constante, parcouru par
un champ électrique de densité de courrant $j_q$. Si le courrant est porté
uniquement par les électrons, le flux de charge est :
\begin{equation}
    j_q = n(-e)(-\mu_e)E = ne\mu_e E
\end{equation}
où $\mu_e$ est la mobilité électronique.. L'énergie moyenne transportée par un
électron dépend du niveau de Fermi $\mu$ :
\begin{equation*}
    (E_c - \mu) + \frac{3}{2}k_BT
\end{equation*}
où $E_c$ est l'énergie du bord de la bande de conduction. On se rapporte à
l'énergie du niveau de Fermi parce que différents conducteurs en contact ont un
même niveau de Fermi, comme nous allons le voir dans le chapitre suivant. Le flux
d'énergie qui accompagne le flux de charge est :
\begin{equation}
    j_U = n(E_c - \mu +\frac{3}{2}k_BT)(-\mu_e)E
\end{equation}
Le \emph{coefficient de Peltier} $\Pi$ est défini par $j_U = \Pi j_q$ ou
l'énergie transportée par charge unitaire. Pour des électrons,
\begin{equation}
    \Pi_e = -(E_c - \mu + \frac{3}{2}k_BT)/e
\end{equation}
qui est négatif car le flux d'énergie est opposé au flux de charge. Pour des
trous,
\begin{equation}
    j_q = pe\mu_hE; \hfill j_U = p(\mu - E_v + \frac{3}{2}k_BT)\mu_h E
\end{equation}
où $E_v$ est l'énergie au bord de la bande de valence. Par conséquent,
\begin{equation}
    \Pi_h = (\mu - E_v + \frac{3}{2}k_BT)/e
\end{equation}
et est positif. Ces deux équations sont le résultat de notre simple théorie de la
vitesse de dérive. Un traitement par l'équation du transport de Bortzmann donne
des valeurs numériques différentes.

Le \emph{pouvoir thermoélectrique absolu Q} est défini comme le champ électrique
en circuit ouvert, créé par un gradient de température :
\begin{equation}
    E = Q \grad T
\end{equation}
Le coefficient de Peltier $\Pi$ est relié au pouvoir thermoélectrique Q par la
relation :
\begin{equation}
    \Pi = QT
\end{equation}

C'est la fameuse relation de Kelvin de l'irréversibilité thermodynamique. Une
mesure du signe du voltage le long d'un échantillon semiconducteur, à une
extrémité chauffée, est un moyen efficace et brut de dire si l'échantillon est
dopé n ou p.

\begin{marginfigure}
    \TODO
    \caption{coefficient s de peltier pour du silicium n et p en fonction de la
    température}
    \label{peltier}
\end{marginfigure}
