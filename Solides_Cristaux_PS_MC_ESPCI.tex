\documentclass[a4paper,justified,twoside,nobib]{tufte-book}

\usepackage[utf8]{inputenc}
\usepackage[frenchb]{babel}

\hypersetup{colorlinks}% uncomment this line if you prefer colored hyperlinks (e.g., for onscreen viewing)


%%
% Book metadata
\title{Physique du solide et des cristaux}
\author[Julien Barrier]{Julien Barrier}
\publisher{ESPCI Paris}
\newcommand{\thetitle}{Physique du solide et des cristaux}
\newcommand{\theauthor}{Julien Barrier --- 133\ieme promotion}
\newcommand{\thepublisher}{ESPCI Paris}
\newcommand{\thesubtitle}{Propagation d'ondes dans la matière condensée}


%%
% If they're installed, use Bergamo and Chantilly from www.fontsite.com.
% They're clones of Bembo and Gill Sans, respectively.
%\IfFileExists{bergamo.sty}{\usepackage[osf]{bergamo}}{}% Bembo
%\IfFileExists{chantill.sty}{\usepackage{chantill}}{}% Gill Sans

\usepackage{microtype}
\usepackage{textcase}

%%
% Just some sample text
\usepackage{lipsum}

%%
% For nicely typeset tabular material
\usepackage{booktabs}
\usepackage{tabularx}
\newcolumntype{R}{>{\raggedleft\arraybackslash}X}
\newcolumntype{C}{>{\centering\arraybackslash}X}

%%
% For graphics / images
\usepackage{graphicx}
\setkeys{Gin}{width=\linewidth,totalheight=\textheight,keepaspectratio}
\graphicspath{{graphics/}}
\usepackage{subfig}

% symboles cristallo
\DeclareFontFamily{U}{cry}{\hyphenchar\font=-1}
\DeclareFontShape{U}{cry}{m}{n}{ <-> cryst}{}
\newcommand{\cry}[1]{{\usefont{U}{cry}{m}{n} \symbol{#1}}}
\renewcommand{\Pr}{\mbox{$\mathsf P$}}
\newcommand{\E}{\mbox{$\mathsf E$}}
\newcommand{\Var}{\mbox{$\mathcal V$}}
\newcommand{\Cov}{\mbox{$\mathcal C$}}
\newcommand {\half}{\mbox{$\frac{1}{2}$}}
\newcommand{\est}{\widehat}

%SIUNITX
\usepackage{siunitx}
\DeclareSIUnit{\calorie}{cal}

% The fancyvrb package lets us customize the formatting of verbatim
% environments.  We use a slightly smaller font.
\usepackage{fancyvrb}
\fvset{fontsize=\normalsize}

%%
% Prints argument within hanging parentheses (i.e., parentheses that take
% up no horizontal space).  Useful in tabular environments.
\newcommand{\hangp}[1]{\makebox[0pt][r]{(}#1\makebox[0pt][l]{)}}

%%
% Prints an asterisk that takes up no horizontal space.
% Useful in tabular environments.
\newcommand{\hangstar}{\makebox[0pt][l]{*}}

%%
% Prints a trailing space in a smart way.
\usepackage{xspace}

\newcommand{\pc}{l'\textit{ESPCI Paris}\xspace}


% Prints an epigraph and speaker in sans serif, all-caps type.
\newcommand{\openepigraph}[2]{%
  %\sffamily\fontsize{14}{16}\selectfont
  \begin{fullwidth}
  \sffamily\large
  \begin{doublespace}
  \noindent\allcaps{#1}\\% epigraph
  \noindent\allcaps{#2}% author
  \end{doublespace}
  \end{fullwidth}
}

% Inserts a blank page
\newcommand{\blankpage}{\newpage\hbox{}\thispagestyle{empty}\newpage}

\usepackage{stmaryrd}
\usepackage{amsmath,amsfonts,amssymb}
\usepackage{chemformula}

% Typesets the font size, leading, and measure in the form of 10/12x26 pc.
\newcommand{\measure}[3]{#1/#2$\times$\unit[#3]{pc}}

% Macros for typesetting the documentation
\newcommand{\hlred}[1]{\textcolor{Maroon}{#1}}% prints in red
\newcommand{\hangleft}[1]{\makebox[0pt][r]{#1}}
\newcommand{\hairsp}{\hspace{1pt}}% hair space
\newcommand{\hquad}{\hskip0.5em\relax}% half quad space
\newcommand{\TODO}{\textcolor{red}{\bf À FAIRE !}\xspace}
\newcommand{\ie}{\textit{i.\hairsp{}e.}\xspace}
\newcommand{\eg}{\textit{e.\hairsp{}g.}\xspace}
\newcommand{\na}{\quad--}% used in tables for N/A cells
\providecommand{\XeLaTeX}{X\lower.5ex\hbox{\kern-0.15em\reflectbox{E}}\kern-0.1em\LaTeX}
\newcommand{\tXeLaTeX}{\XeLaTeX\index{XeLaTeX@\protect\XeLaTeX}}
% \index{\texttt{\textbackslash xyz}@\hangleft{\texttt{\textbackslash}}\texttt{xyz}}
\newcommand{\tuftebs}{\symbol{'134}}% a backslash in tt type in OT1/T1
\newcommand{\doccmdnoindex}[2][]{\texttt{\tuftebs#2}}% command name -- adds backslash automatically (and doesn't add cmd to the index)
\newcommand{\doccmddef}[2][]{%
  \hlred{\texttt{\tuftebs#2}}\label{cmd:#2}%
  \ifthenelse{\isempty{#1}}%
    {% add the command to the index
      \index{#2 command@\protect\hangleft{\texttt{\tuftebs}}\texttt{#2}}% command name
    }%
    {% add the command and package to the index
      \index{#2 command@\protect\hangleft{\texttt{\tuftebs}}\texttt{#2} (\texttt{#1} package)}% command name
      \index{#1 package@\texttt{#1} package}\index{packages!#1@\texttt{#1}}% package name
    }%
}% command name -- adds backslash automatically
\newcommand{\doccmd}[2][]{%
  \texttt{\tuftebs#2}%
  \ifthenelse{\isempty{#1}}%
    {% add the command to the index
      \index{#2 command@\protect\hangleft{\texttt{\tuftebs}}\texttt{#2}}% command name
    }%
    {% add the command and package to the index
      \index{#2 command@\protect\hangleft{\texttt{\tuftebs}}\texttt{#2} (\texttt{#1} package)}% command name
      \index{#1 package@\texttt{#1} package}\index{packages!#1@\texttt{#1}}% package name
    }%
}% command name -- adds backslash automatically
\newcommand{\docopt}[1]{\ensuremath{\langle}\textrm{\textit{#1}}\ensuremath{\rangle}}% optional command argument
\newcommand{\docarg}[1]{\textrm{\textit{#1}}}% (required) command argument
\newenvironment{docspec}{\begin{quotation}\ttfamily\parskip0pt\parindent0pt\ignorespaces}{\end{quotation}}% command specification environment
\newcommand{\docenv}[1]{\texttt{#1}\index{#1 environment@\texttt{#1} environment}\index{environments!#1@\texttt{#1}}}% environment name
\newcommand{\docenvdef}[1]{\hlred{\texttt{#1}}\label{env:#1}\index{#1 environment@\texttt{#1} environment}\index{environments!#1@\texttt{#1}}}% environment name
\newcommand{\docpkg}[1]{\texttt{#1}\index{#1 package@\texttt{#1} package}\index{packages!#1@\texttt{#1}}}% package name
\newcommand{\doccls}[1]{\texttt{#1}}% document class name
\newcommand{\docclsopt}[1]{\texttt{#1}\index{#1 class option@\texttt{#1} class option}\index{class options!#1@\texttt{#1}}}% document class option name
\newcommand{\docclsoptdef}[1]{\hlred{\texttt{#1}}\label{clsopt:#1}\index{#1 class option@\texttt{#1} class option}\index{class options!#1@\texttt{#1}}}% document class option name defined
\newcommand{\docmsg}[2]{\bigskip\begin{fullwidth}\noindent\ttfamily#1\end{fullwidth}\medskip\par\noindent#2}
\newcommand{\docfilehook}[2]{\texttt{#1}\index{file hooks!#2}\index{#1@\texttt{#1}}}
\newcommand{\doccounter}[1]{\texttt{#1}\index{#1 counter@\texttt{#1} counter}}

% \DeclareMathOperator{\Sample}{Sample}
\let\vaccent=\v % rename builtin command \v{} to \vaccent{}
\renewcommand{\v}[1]{\ensuremath{\mathbf{#1}}} % for vectors
\newcommand{\gv}[1]{\ensuremath{\mbox{\boldmath$ #1 $}}} 
% for vectors of Greek letters
\newcommand{\uv}[1]{\ensuremath{\mathbf{\hat{#1}}}} % for unit vector
\newcommand{\abs}[1]{\left| #1 \right|} % for absolute value
\newcommand{\avg}[1]{\left< #1 \right>} % for average
\let\underdot=\d % rename builtin command \d{} to \underdot{}
\renewcommand{\d}[2]{\frac{d #1}{d #2}} % for derivatives
\newcommand{\dd}[2]{\frac{d^2 #1}{d #2^2}} % for double derivatives
\newcommand{\pd}[2]{\frac{\partial #1}{\partial #2}} 
% for partial derivatives
\newcommand{\pdd}[2]{\frac{\partial^2 #1}{\partial #2^2}} 
% for double partial derivatives
\newcommand{\pdc}[3]{\left( \frac{\partial #1}{\partial #2}
 \right)_{#3}} % for thermodynamic partial derivatives
\newcommand{\ket}[1]{\left| #1 \right>} % for Dirac bras
\newcommand{\bra}[1]{\left< #1 \right|} % for Dirac kets
\newcommand{\braket}[2]{\left< #1 \vphantom{#2} \right|
 \left. #2 \vphantom{#1} \right>} % for Dirac brackets
\newcommand{\matrixel}[3]{\left< #1 \vphantom{#2#3} \right|
 #2 \left| #3 \vphantom{#1#2} \right>} % for Dirac matrix elements
\newcommand{\grad}[1]{\gv{\nabla} #1} % for gradient
\let\divsymb=\div % rename builtin command \div to \divsymb
\renewcommand{\div}[1]{\gv{\nabla} \cdot #1} % for divergence
\newcommand{\curl}[1]{\gv{\nabla} \times #1} % for curl
\let\baraccent=\= % rename builtin command \= to \baraccent
\renewcommand{\=}[1]{\stackrel{#1}{=}} % for putting numbers above =

% Generates the index
\usepackage{makeidx}
\makeindex

\usepackage{titlesec,titletoc}
\usepackage{multirow}

\begin{document}
\frontmatter

\thispagestyle{empty}
\begin{fullwidth}
\setlength{\parindent}{0pt}

\begin{center}\fontsize{24}{24}\selectfont\textit{\includegraphics*[width=2.6in]{ESPCI_baseline_couleur}}
\end{center}

\vspace{3in}\fontsize{32.1}{54}\selectfont\thetitle

\vspace{0.125in}\fontsize{18}{18}\selectfont\thesubtitle

\vfill\fontsize{14}{14}\selectfont\textit{\theauthor}
\end{fullwidth}

\newpage


\cleardoublepage
\chapter*{Introduction}
%    En deuxième année à \pc, les solides sont abordés dans trois matières 
%    différentes : \emph{Matériaux Cristallisés} au premier semestre, 
%    \emph{Chimie et Matériaux Inorganiques} et \emph{Physique du Solide} au 
%    deuxième. Je ne pense pas qu'il faille distinguer les trois disciplines de cette
%    façon : beaucoup d'articles scientifiques traitent de problèmes de matière 
%    condensée présentent d'abord les structures cristallines que les auteurs ont 
%    synthétisé avant d'en étudier les propriétés physiques. Je me suis donc efforcé,
%    dans ce document, de relier les notions et de créer des passerelles entre 
%    chacune d'entre elles.

%    Attention, il a fallu faire un choix éditorial. J'ai choisi de me concentrer
%    sur les liens entre les cristaux et les propriétés physiques que l'on pouvait
%    mesurer. Ainsi, il y a de nombreuses parties du cours de matériaux cristallisés
%    sur lesquels j'ai choisi de faire l'impasse, en particulier toute la
%    cristallochimie, les propriétés de symétrie des cristaux (projection
%    stéréographique, canevas de Wulff, goniométrie, indexation des faces), la 
%    diffusion diffuse, les cristaux apériodiques et les solutions solides. Je n'ai
%    pas non plus détaillé les symboles car je considère que le poly et le cours 
%    proposés par N. Lequeux sont suffisants de ce point de vue. J'ai également choisi
%    de ne pas évoquer le formalisme mathématique lié aux opérations de symétrie.
%    Les calculs fait dans le poly de N. Lequeux permettent une assez bonne compréhension
%    des algèbres utilisées.

%    Je n'ai pas souhaité faire un poly qui ne sert qu'à réviser le cours d'une 
%    matière ou d'une autre. Je l'ai fait car j'ai senti au cours de mes stages que
%    certains points du cours étaient difficiles à appréhender. J'ai essayé non pas de
%    faire une liste d'exemples qui permettent de réussir un examen, mais de
%    détailler ce qui me semblait intéressant pour aborder une carrière 
%    scientifique dans ce domaine.

%    J'espère que ce document vous sera utile pour mieux appréhender chacune des 
%    matières, pour que cela vous serve dans vos futures réflexions scientifiques.
%    En particulier, j'ai essayé, dans la mesure du possible, d'inclure des exemples à 
%    chaque fois que je le pouvais. Cela n'a pas pour objectif d'être exhaustif, mais 
%    plutôt de fournir un ordre d'idée sur une gamme de sujets que j'ai voulu la plus 
%    vaste possible.

%    Ce document est composé de trois parties. La première présente les bases de
%    cristallographie et les propriétés de diffraction liées à la nature cristalline
%    de la matière.
%    La seconde partie présente les vibrations dans les réseaux et la quantification 
%    possible.
%    La troisième traite du sujet des différents modèles de conduction électronique.
%    Enfin, à partir des modèles, on peut étudier différents types de solides : 
%    métaux, semi-conducteurs et isolants. C'est l'objet des parties en construction.
    %la quatrième partie.
    %toute la première partie me permet de définir les cristaux, les liaisons, la
    %façon dont on peut les aborder. essentiel pour mieux comprendre les modèles 
    %électroniques, qui sont basés sur une définition propre des réseaux.

%    Les sources qui m'ont permis de rédiger ce document ne sont pas détaillées au fur
%    et à mesure. En revanche, j'ai été inspiré par les ouvrages suivants, dont je
%    recommande la lecture à ceux qui souhaitent en apprendre plus sur ce sujet 
%    passionnant :
Sources :
    \begin{itemize}
        \item \emph{Elements of X-Ray Diffraction}, B. Cullity ;
        \item \emph{Solid State Physics}, N. Ashcroft, B. Meermin ;
        %\item \emph{Electronic Structures of Strongly correlated Materials}, V.
            Anisimov, Y. Izyumov ;
        \item \emph{Introduction to Solid State Physics}, C. Kittel ;
        \item \emph{Electrons and Phonons}, J.M. Ziman. 
    \end{itemize}

    \vspace{2cm}
    
En construction pour l'instant :
\begin{description}
    \item[Partie 1 (Cristallographie + RX)] 90\%
    \item[Partie 2 (Vibrations, phonons)] 60\%
    \item[Partie 3 (Modèles électroniques)] 30\%
    \item[Partie 4 (Défauts)] 10\%
    \item[Partie 5 (Transport d'électrons)] 40\%
\end{description}

%    La cristallographie est une science un peu trop souvent négligée. Pourtant,
%    elle est la base de toute la physique et la chimie de la matière condensée, qui
%    est un domaine scientifique en plein essort.
%    Les métaux et les semiconducteurs sont très souvents des solides cristallins.
%    Si cela n'est pas évident au premier coup d'œil sur des matériaux façonnés,
%    nous allons explorer la matière pour en découvrir la structure périodique.
%    La comprendre nous permettra d'appréhender les propriétés physiques qui en
%    découlent. J'espère que ce document saura être suffisamment clair.



\tableofcontents\thispagestyle{empty}

%\listoffigures

%\listoftables

\cleardoublepage
~\vfill
\begin{doublespace}
\noindent\fontsize{18
}{22}\selectfont\itshape
\nohyphenation
«~Personnellement, si j'en étais resté aux impressions éprouvées à la suite des 
premières leçons de sciences de mes professeurs [...] si je n'avais pris un
contact ultérieur ou différent avec la réalité, j'aurais pu penser que la 
science était faite, qu'il ne restait plus rien à découvrir, alors que nous en 
sommes à peine aux premiers balbutiements dans la connaissance du monde 
extérieur.~»\\

\flushright{Paul Langevin, 7\ieme promotion}
\end{doublespace}
\vfill
\vfill

\mainmatter

\part[LE RÉSEAU CRISTALLIN]{Le réseau cristallin}

\chapter{Réseau cristallin}
\label{ch:reseaucrist}

En cristallographie, les cristaux sont définis par l'organisation des ions, qui
est périodique au niveau microscopique. Longtemps dans l'histoire des sciences,
l'idée selon laquelle l'organisation microscopique d'un matériau est similaire à
l'organisation macroscopique est réstée un postulat. Ce n'est qu'au début du
20\ieme~ siècle que sir William Henry Brag et son fils sir William Lawrence
Bragg utilisent des rayons X pour observer l'organisation des atomes dans les
solides.

Avant de pouvoir étudier les différentes structures par diffraction des rayons X
, il nous faut commencer par comprendre comment ceux-ci sont construits et les
différentes notions qui y sont associées. Cette partie a pour objet de
comprendre ces concepts et d'établir les différentes symétries.

\section{Postulat de la cristallographie}

En 1866, Bravais formule la loi des indices rationnels sous la forme suivante :
\begin{description}
    \item[Postulat de Bravais]
État donné un point P, quelconque dans un cristal, il existe dans le milieu, une
infinité discrète, illimitée dans les trois directions de l'espace, de points 
autour desquels l'arrangement des atomes est le même qu'autour du point P, et ce
avec la même orientation.
\end{description}

Ce postulat a été complété à la fin du XIX\ieme siècle, simultanément et de
manière indépendante par Schönflies et Fedorov :
\begin{description}
 \item[Postulat de Schönflies-Fedorov]
Étant donné un point quelconque P dans un cristal, il existe dans le milieu une
infinité discrète, illimitée dans les trois directions de l'espace, de points
autour desquels l'arrangement des atomes est le même qu'autour du point P, ou
est une image de cet arrangement.
\end{description}

La différence par rapport au postulat de Bravais est que dans cette définition,
il n'y a plus aucune exigence d'identité d'orientation autour des points 
équivalents. En outre, la notion d'image (symétrie par rapport à un point) y est
introduite.

\section{Systèmes de coordonnées}

Puisque l'on étudie les réseaux cristallins, il est nécessaire de définir des
vecteurs de base. Soient $\v{a}$, $\v{b}$ et $\v{c}$ des vecteurs
formant une base. Celle-ci n'est pas nécesairement orthogonale, ni même normée.
En revanche, on les choisit de sorte à former un trièdre direct.

Dans ce réseau, on peut définir la position d'un point $A$ par son vecteur 
position $\v{r}_A$ :

\begin{equation}
    \v{r}_A = x_A \v{a} + y_A \v{b} + z_A \v{c}
\end{equation}


\subsection{Vecteur primitif et rangées réticulaires}

Dans un réseau, on appelle rangée réticulaire (ou direction d'un réseau), 
l'ensemble des doites parallèles qui passent par au moins deux nœuds du réseau.

Ces rangées réticulaires sont définies par un vecteur primitif $\v{t}$ tel que :

\begin{equation}
    \v{t} = u \v{a} + v \v{b} + w \v{c}
\end{equation}

Dans cette équation, les indices de la rangée $u$, $v$ et $w$ sont des entiers
premiers entre eux. Comme une rangée contient toujours au moins deux nœuds, le
vecteur primitif $t$ de cette rangée est toujours un vecteur du réseau. Ce
vecteur ne définit pas qu'une seule droite, mais une infinité de droites, toutes
parallèles et équivalentes par translation du réseau.

\begin{figure}
    \includegraphics{./images/part1/cullity43.eps}
    \caption{Différentes rangées réticullaires possibles dans un réseau 2D}
    \label{fig:rangees2D}
\end{figure}

On peut écrire une rangée réticulaire avec ses indices entre crochets :
$[uvw]$. Si une des composantes est négative, elle est notée avec "$\bar{\cdot}$",
comme par exemple $[1\bar{2}0]$.

\begin{marginfigure}
    \includegraphics{./images/part1/reticulaire-02}
    \caption{Représentation de différentes rangées réticulaires dans une maille 3D}
    \label{fig:rangees3D}
\end{marginfigure}

Il vient alors, de façon évidente que les propriétés des vecteurs peuvent 
s'appliquer sur les vecteurs primitifs associés aux rangées réticulaires. Par
exemple, si $[u_1v_1w_1]$ et $[u_2v_2w_2]$ sont orthogonales, le produit 
scalaire de leurs vecteurs primitifs est nul :
$\v{t}_1 \cdot \v{t}_2 = 0$. De plus, $[uvw]$ et
$[\bar{u}\bar{v}\bar{w}]$ désignent la même rangée.

\section{Réseau de Bravais}

La notion de \emph{réseau de Bravais} est un concept assez fondamental dans la
description d'un solide cristallin. Il définit une structure cristalline, dans
laquelle les unités répétées du cristal s'arrangent. Les unités en elles-mêmes
peuvent être de simples atomes, mais aussi des groupes d'atomes, des molécules, des ions, etc.

Le réseau de Bravais ne définit que la géométrie de la structure périodique,
peu importe l'échelle d'observation, et peu importe la taille de la structure.

On peut en trouver deux définitions équivalentes :
\begin{enumerate}
    \item \label{bravaisa} c'est l'ensemble des points R tels que
    $R = m_1 a_1 + m_2 a_2 + m_3 a_3$ (en 3D) où $a_1$, $a_2$, $a_3$ sont les
    vecteurs élémentaires du cristal ;
\item c'est un réseau infini de points discrets avec un arrangement et une
    orientation qui sont exactement les mêmes, peu importe le point duquel elles
     sont vues.
\end{enumerate}

Par conséquent, tout nœud que l'on translate d'un certain vecteur 
$\v{R}$, se retrouve être aussi un nœud.

\begin{marginfigure}
    \includegraphics{./images/part1/bravaiswigner-01}
    \caption{Réseau 2D en structure alvéolaire : il ne forme pas un réseau de Bravais. En effet, si le réseau a la même apparence lorsqu'il est vu du point P ou Q, son orientation subit une rotation de \SI{180}{\degree} du point R.}
    \label{fig:alveolaire}
\end{marginfigure}

Les vecteurs $a_i$ qui apparaissent dans la définition \ref{bravaisa} d'un
réseau de Bravais sont appelés des vecteurs primitifs. On dit qu'ils sont 
générateurs du réseau.

Attention, dans un réseau de Bravais, il ne doit pas y avoir que l'arrangement
des atomes qui doit être conservé, mais aussi l'orientation qui doit rester, en
chaque point du réseau de Bravais, identique. Par exemple le motif alvéolaire 2D
(figure \ref{fig:alveolaire}) ne forme pas un réseau de Bravais. En effet,
l'orientation n'est pas la même si on se place en un point et en un autre.

Par définition, comme tous les points sont équivalents, un résau de Bravais est
infini. Les vrais cristaux sont, bien entendu, finis, mais on les considère 
suffisemment grands pour dire que tous les points sont tellement loins de la
surface qu'ils ne sont pas affectés par l'existence de bords.

\section{Exemples de réseaux simples, assemblage de sphères dures}

\begin{marginfigure}
    \includegraphics{./images/part1/bravais}
    \caption{plusieurs choix possibles de vecteurs primitifs pour un réseau de Bravais 2D}
    \label{fig:choixbravais2D}
\end{marginfigure}

Des deux définitions d'un réseau de Bravais, la première (\ref{bravaisa}) est 
mathématiquement plus précise et est le point de départ évident pour tout
travail analytique. Cependant, elle implique plusieurs propriétés. En
particulier, pour tout réseau de Bravais, le choix de vecteurs primitifs n'est
jamais unique ; il y a en fait une infinité de vecteurs de Bravais qui ne sont
pas équivalents. Cette section donne quelques exemples simples de réseaux, basés
sur le modèle dit des \emph{sphères dures}, qui consiste à considérer chacun des
atomes comme s'il s'agissait de boules de billard : pas d'interaction 
électrostatique entre eux, un potentiel nul à une distance supérieure au rayon
d'une boule, infini pour les distances inférieures aux rayon.

\begin{marginfigure}
    \includegraphics{./images/part1/cullity60-03}
    \caption{empilement de sphères dures : réseau hexagonal compact}
    \label{fig:spheresdures}
\end{marginfigure}

\subsection{Réseau hexagonal compact}

%\begin{table}[ht]
%\begin{tabularx}{\textwidth}{lRRRlRRR}
%\toprule
%Élément & a(\angstrom) & c & c/a & Élément & a(\angstrom) & c & c/a\\
%\midrule
%Be & 2.29 & 3.58 & 1.56 & Os & 2.74 & 4.32 & 1.58 \\
%Cd & 2.98 & 5.62 & 1.89 & Pr & 3.67 & 5.29 & 1.61 \\
%Ce & 3.65 & 5.96 & 1.63 & Re & 2.76 & 4.46 & 1.62\\
%$\alpha$-Co & 2.51 & 4.07 & 1.52 & Ru & 2.70 & 4.28 & 1.59\\
%Dy  & 3.59 & 5.65 & 1.57 & Sc & 3.31 & 5.27 & 1.59\\
%Er & 3.56 & 5.59 & 1.57 & Tb & 3.60 & 5.69 & 1.58\\
%Gd & 3.64 & 5.78 & 1.59 & Ti & 2.95 & 4.69 & 1.59\\
%He(2K) & 3.57 & 5.83 & 1.63 & Tl & 3.46 & 5.53 & 1.60\\
%Hf & 3.20 & 5.06 & 1.58 & Tm & 3.54 & 5.55 & 1.57\\
%Ho & 3.58 & 5.62 & 1.57 & Y & 3.65 & 5.73 & 1.57\\
%La & 3.75 & 6.07 & 1.62 & Zn & 2.66 & 4.95 & 1.86\\
%Lu & 3.50 & 5.55 & 1.59 & Zr & 3.23 & 5.15 & 1.59\\
%Mg & 3.21 & 5.21 & 1.62 &  &  &  & \\
%Nd & 3.66 & 5.90 & 1.61 & \emph{idéal} &  &  & 1.63\\
%\bottomrule
%\end{tabularx}
%\caption[Éléments formant une structure hexagonale compacte]{Éléments formant une structure hexagonale compacte\cite{wyckoff1960crystal}(lorsque ce n'est pas précisé, les valeurs sont données dans les conditionsnormales de températures et de pression)(lorsque ce n'est pas précisé, les valeurs sont données dans les conditionsnormales de températures et de pression)}
%\label{hcp}
%\end{table}


Si elle ne forme pas un réseau de Bravais, la structure hexagonale compacte
est une des plus importantes. Une  trentaine d'éléments %(tableau \ref{tab:hcp})
cristallisent dans cette structure parce qu'elle minimise l'énergie en étant
la plus compacte possible.

 \begin{figure}
     \subfloat[Représentation du réseau hexagonal compact
     (hcp)]{\includegraphics[width=0.45\textwidth]{./images/part1/cullity60-02}}
     \hfill
     \subfloat[Maille primitive du réseau
     hcp]{\includegraphics[width=.45\textwidth]{./images/part1/cullity60-01}}
    \caption{Construction d'un réseau hexagonal compact}
    \label{fig:hcp}
\end{figure}

Le réseau de Bravais de cette structure est hexagonal simple, qui est donné en
superposant deux réseaux de triangles l'un au dessus de l'autre. L'empilement est
réalisé suivant la direction$\v{c}$. Les trois vecteurs primitifs sont :

\begin{equation}
    \v{a}_1 = a \uv{x},\quad \v{a}_2 = \frac{a}{2}\uv{x}+\frac{\sqrt{3}a}{2}\uv{},\quad \v{c}= c\uv{z}
\end{equation}

Les deux premiers vecteurs génèrent un réseau triangulaire dans le plan $(x,y)$. Le troisième vecteur créé l'empilement de ces deux réseaux l'un au dessus de l'autre.

Cette structure hexagonale compacte est la plus compacte dans la considération de sphères dures. Par exemple, si on empile des boules de billard, on va former spontanément une structure hexagonale compacte, dont le paramètre $c$ sera égal à :

\begin{equation}
    c = \sqrt{\frac{8}{3}}a = 1.63299 a
\end{equation}

La densité est alors :
\begin{equation}
d = \frac{\sqrt{2}\pi}{6} = 0.74
\end{equation}

Dans certains cas, la structure électronique des molécules ne permet pas de se placer dans le modèle des sphères dures. Dans ce cas, l'arrangement peut prendre différentes formes.

\subsection{Réseau cubique simple}

\begin{marginfigure}
    \includegraphics{./images/part1/cubic}
    \caption{Réseau cubique simple le système de vecteurs primitifs}
    \label{fig:sc}
\end{marginfigure}

Le réseau cubique simple (on verra plus tard qu'il s'agit du système cubique primitif $P$) se forme en fait assez rarement : parmi les 118 éléments, seule la phase $\alpha$ du polonium est connue pour cristalliser en réseau cubique simple dans des conditions normales de température et de pression. Ce réseau est pourtant assez simple à comprendre et permet de générer les autres réseaux cubiques.

Dans le système cubique, le réseau est généré par des vecteurs $a\uv{x}$,$a\uv{y}$ et $a\uv{z}$. C'est la forme la plus simple d'un réseau de Bravais.
 
\subsection{Réseau cubique centré}

%\begin{table}[ht]
%    \begin{tabularx}{\textwidth}{lRlRlR}
%        \toprule
%        Élément & a(\angstrom) & Élément & a(\angstrom) & Élément & a(\angstrom) \\
%        \midrule
%        Ba & 5.02 & Li(78K) & 3.49 & Ta & 3.31 \\
%        Cr & 2.88 & Mo & 3.15 & Tl & 3.88\\
%        Cs(78K) & 6.05 & Na(5K) & 4.23 & V & 3.02 \\
%        Fe & 2.87 & Nb & 3.30 & W & 3.16 \\
%        K(5K) & 5.23 & Rb(5K) & 5.59 & & \\
%        \bottomrule
%    \end{tabularx}
%    \caption[Éléments formant une structure cubique centrée monoatomique]{Éléments formant une structure cubique centrée monoatomique \cite{wyckoff1960crystal}}
%\end{table}

\begin{marginfigure}
    \includegraphics{./images/part1/cullity59-01}
    \caption{Réseau cubique centré avec le système de vecteurs primitifs}
    \label{fig:bcc}
\end{marginfigure}


Ajoutons maintenant un point supplémentaire au centre de ce réseau cubique simple. Ce point peut être vu à la fois comme le centre d'une maille cubique, ou comme le sommet d'une autre maille cubique, dans lequel les sommets de la première maille deviennent des centres. On vient de former un réseau cubique centré.

L'ensemble des vecteurs primitifs devient ici :
\begin{equation}
\v{a}_1 = a\uv{x},\quad\v{a}_2 = a\uv{y},\quad \v{a}_3 = \frac{a}{2}(\uv{x}+\uv{y}+\uv{z})
\end{equation}

On considère généralement un ensemble de vecteurs primitifs moins intuitifs, mais plus utiles de façon analytique :

\begin{equation}
    \v{a}_1 = \frac{a}{2}(-\uv{x}+\uv{y}+\uv{z}),
    \quad
    \v{a}_2 = \frac{a}{2}(\uv{x}-\uv{y}+\uv{z}),
    \quad
    \v{a}_3 = \frac{a}{2}(\uv{x}+\uv{y}-\uv{z})
\end{equation}

Ce système est très important parce qu'un très grand nombre d'éléments crystallisent dans cette forme.

La densité d'un réseau cubique centré est :

\begin{equation}
d = \frac{\sqrt{3}\pi}{8} = 0.68
\end{equation}

\subsection{Réseau cubique faces-centrées} 

%\begin{table}[ht]
%    \begin{tabularx}{\textwidth}{lRlRlR}
%        \toprule
%        Élément & a(\angstrom) & Élément & a(\angstrom) & Élément & a(\angstrom) \\
%        \midrule
%        Ar(4.2 K) & 5.26 & Ir & 3.84 & Pt & 3.92 \\
%        Ag & 4.09 & Kr(58K) & 5.72 & $\delta-Pu$ & 4.64\\
%        Al & 4.05 & La & 5.30 & Rh & 3.80\\
%        Au & 4.08 & Ne(4.2K) & 4.43 & Sc & 4.54\\
%        Ca & 5.58 & Ni & 3.52 & Sr & 6.08 \\
%        Ce & 5.16 & Pb & 4.95 & Th & 5.08 \\
%        $\beta-Co$ & 3.55 & Pd & 3.89 & Xe(58K) & 6.20 \\
%        Cu & 3.61 & Pr & 5.16 & Yb & 5.49 \\
%        \bottomrule
%        \caption[Éléments formant une structure cubique face-centrée monoatomique]{Éléments formant une structure cubique face-centrée monoatomique \cite{wyckoff1960crystal}}
%    \end{tabularx}
%\end{table}

\begin{marginfigure}
    \includegraphics{./images/part1/cullity59-02}
    \caption{Réseau cubique faces-centrées avec le système de vecteurs primitifs commun}
    \label{fig:fcc}
\end{marginfigure}

Considérons maintenant l'ajout, au centre de chaque chaque face du réseau cubique, d'un nouveau point. On peut penser ici que l'ajout de ces six nouveaux points par maille cubique les rendent tous non équivalents. En fait, si on se place dans le cube formé par chacun des points aux centres des faces, on retrouve encore une fois la même structure : le réseau cubique faces-centrées est un exemple de réseau de Bravais. Il s'agit du réseau cubique faces-centrées.

Un exemple de vecteurs primitifs d'un réseau cubique faces-centrées peut-être défini par :

\begin{equation}
\v{a}_1 = \frac{a}{2}(\uv{y}+\uv{z}),
\quad
\v{a}_2 = \frac{a}{2}(\uv{x}+\uv{z}),
\quad
\v{a}_3 = \frac{a}{2}(\uv{x}+\uv{y})
\end{equation}

La densité d'un réseau cubique faces-centrées est :

\begin{equation}
d = \frac{\sqrt{2}\pi}{6} = 0.74
\end{equation}

Cette densité est la même que pour l'hexagonal compact. Pour cette raison, on appelle parfois le cubique faces-centrées \emph{cubic compact}. Une trentaine d'éléments cristallisent naturellement en structure cubique à faces centrées.

\section{Mailles usuelles}
\subsection{Coordinence}

Dans un réseau de Bravais, on appelle les plus proches voisins les points qui sont les plus proches d'un point donné. Comme, par définition, un réseau de Bravais est périodique, chaque point du réseau a le même nombre de plus proches voisins. Ce nombre devient alors une propriété du réseau, appelé \emph{nombre de coordination} ou \emph{coordinence}.

Par exemple, un réseau cubique simple a une coordinence de 6, un cubique centré, de 8 et un cubique faces-centrés de 12. Cette notion de coordinence peut également être étendue à tout réseau qui n'est pas un réseau de Bravais, à condition que les points du réseau aient tous le même nombre de plus proches voisins.

\subsection{Maille primitive}

\begin{figure}
    \includegraphics{./images/part1/cullity39-02}
    \caption{Dans ce réseau, la maille primitive peut être délimitée par 8
    sphères noires. Elle peut également être translatée; on a toujours une maille
    primitive.}
    \label{fig:choixmaillespriv}
\end{figure}

Prenons un volume d'espace du réseau. S'il peut être translaté par chacun des vecteurs du réseau de Bravais et compléter l'espace tout entier sans se superposer avec lui-même ni laisser de vide, alors on appelle ce volume une \emph{maille primitive}.
L'espace cristallin peut alors être considéré comme un ensemble de mailles primitives analogues qui pavent l'espace sans créer de lacune.

Il est intéressant de remarquer qu'une maille primitive n'est jamais unique : il y a toujours plusieurs façon de la choisir, comme présenté sur la figure \ref{fig:choixmaillespriv}.

Une maille primitive doit contenir exactement un nœud du réseau. En conséquence, si $n$ est la densité de points dans le réseau et $v$ le volume de la maille primitve, il vient que $nv = 1$, soit $v = \frac{1}{n}$. On vient de montrer que peu importe le choix de maille primitive que l'on fait, celle-ci aura toujours le même volume.


\subsection{Cellule de Wigner-Seitz}

\begin{marginfigure}
    \includegraphics{./images/part1/bravaiswigner-02}
    \caption{Construction de la cellule de Wigner-Seitz pour un réseau quelconque 2D}
    \label{fig:ws2d}
\end{marginfigure}

Dans un réseau de Bravais, on peut toujours trouver une maille unitaire qui possède la symétrie totale du réseau. Un de ces choix a été \emph{normalisé} ; il s'agit de la \emph{cellule de Wigner-Seitz}.

La cellule de Wigner-Seitz sur un nœud du réseau est la région de l'espace qui est plus proche de ce nœud que de n'importe quel autre nœud du réseau. La figure \ref{fig:ws2d} montre la construction d'une cellule de Wigner-Seitz. La figure \ref{fig:ws} présente deux exemples de cellules de Wigner-Seitz pour un réseau cubique centré et un cubique-faces-centrées.

\begin{figure}
\includegraphics{./images/part1/wignerseitz}
\caption{Cellules de Wigner-Seitz pour un cubique centré (a) et un cubique faces-centrées (b). Pour le réseau cubique faces-centrées, la maille représentée n'est pas la maille conventionnelle.}
\label{fig:ws}
\end{figure}

Certaines propriétés découlent de cette définition. En particulier, la cellule de Wigner-Seitz a toujours les mêmes symétries que le réseau de Bravais.

On note que pour la construire, on peut retracer les lignes qui relient les points entre eux, et les bisectrices forment les limites de la cellule de Wigner-Seitz.

\subsection{Maille conventionnelle}

Dans certain cas, utiliser des mailles primitives n'est pas toujours pertinent. Par exemple, dans le cas du réseau cubique faces-centrées que l'on verra plus tard (figure \ref{fig:fcctrigonal}), le cube est une maille conventionnnelle qui n'est pas primitive.

Une maille conventionnelle est une région qui, translatée des sous-vecteurs du réseau de Bravais, peut remplir l'espace sans se recouvrir avec elle-même. Celle-ci est généralement choisie plus grande que la maille primitive, ce qui permet de retrouver visuellement les bonnes symétries.

\chapter{Réseau réciproque}
\label{ch:reseaurec}


Il est assez facile de s'imaginer le réseau direct présenté dans la section \ref{ch:reseaucrist} pour un cristal classique, car il correspond à l'agencement des atomes dans celui-ci. En revanche, dès que l'on souhaite pousser un peu l'étude de structures périodiques, on a besoin d'une autre représentation du cristal : dans l'\emph{espace réciproque}. Ce \emph{réseau réciproque} va jouer un rôle fondamental dans les études analytiques, que ce soit pour la structure électronique ou pour la résolution d'une structure par diffraction des rayons X.

L'objectif de cette section est donc de décrire la géométrie de l'espace réciproque et de présenter quelques implications élémentaires liées à cette définition. Il n'est pas question ici de donner une vision exhaustive de ce qu'est l'espace réciproque, mais plutôt d'introduire les concepts qui permettent de mieux comprendre la diffraction des rayons X ou l'étude des structures électroniques des cristaux.

\section{Définition du réseau réciproque}

\subsection{Définition}

Soit un réseau de Bravais constitué d'un ensemble de points $\mathbf{R}$, et une
onde plane $e^{i\mathbf{k}\cdot\mathbf{r}}$. Dans le cas général ($\mathbf{k}$ quelconque), l'onde plane que nous venons de définir n'a pas la périodicité du réseau de Bravais. Il existe cependant un choix de $\mathbf{k}$ qui aura cette périodicité : celui-ci définit le réseau réciproque :

\begin{description}
\item[Définition] L'ensemble des vecteurs d'ondes $\mathbf{K}$ qui résultent en une onde plane avec la périodicité d'un réseau de Bravais donné est appelé réseau réciproque.
\end{description}

Cette définition est équivalente à la propriété suivante : $\mathbf{K}$ appartient au réseau réciproque d'un réseau
de Bravais de points $\mathbf{R}$ si et seulement si :

\begin{equation}
\exp(i \mathbf{K}\cdot \mathbf{R}) = 1
\label{eq:defresreciproque}
\end{equation}

Il vient alors que :

\begin{equation}
    \mathbf{K} \cdot \mathbf{R} = 2 \pi n
    \label{eq:fondreseaurec}
\end{equation}

Cette relation est fondamentale. On peut alors donner les descriptions suivantes, basées sur le volume de la
maille :

\begin{equation}
\mathbf{b}_1 = 2\pi \frac{\mathbf{a}_2 \times \mathbf{a}_3}{V},\quad
\mathbf{b}_2 = 2\pi \frac{\mathbf{a}_3 \times \mathbf{a}_1}{V},\quad
\mathbf{b}_3 = 2\pi \frac{\mathbf{a}_1 \times \mathbf{a}_2}{V}
\end{equation}

dans lesquelles $\mathbf{a}_i$ est un vecteur du réseau direct et $\mathbf{b}_i$ du réseau réciproque. le volume de la maille $V$ peut être écrit avec le produit mixte $\mathbf{a}_1 \cdot (\mathbf{a}_2 \times \mathbf{a}_3)$.

On utilise également une formulation équivalente, basée sur le produit scalaire :

\begin{equation}
\mathbf{b}_i \cdot \mathbf{a}_j = 2\pi \delta_i^j
\label{eq:reseaureciproque}
\end{equation}

Cette équation est absolument fondamentale et nous permet, dans la plupart des cas, de calculer le réseau de Bravais sans avoir à faire le calcul fastidieux du volume puis du produit tensoriel.

Puisque l'espace de Bravais de départ est l'espace des longueurs réelles, l'espace réciproque est de dimension 1/L. En outre, si $v$ est le volume d'une maille primitive dans le réseau direct, alors la maille primitive du réseau réciproque aura un volume égal à $\frac{(2\pi)^3}{v}$.

\begin{figure}
    \includegraphics{./images/part1/cullity45}
    \caption{Illustration de réseaux cristallins (gauche) et des réseaux
        réciproques correspondants (droite), pour un système cubique (en haut) et
    un système hexagonal (en bas)}
    \label{fig:reciprocal}
\end{figure}

\subsection{Réseau de Bravais}

Dans cette partie, nous allons montrer que le réseau réciproque d'un réseau de Bravais est également un réseau de Bravais. Pour cela, considérons $\mathbf{a}_1$, $\mathbf{a}_2$ et $\mathbf{a}_3$ un ensemble de vecteurs primitifs du réseau direct.

Soit $\mathbf{k}$ un vecteur quelcoque du réseau réciproque. Écrivons le comme une combinaison linéaire des $\mathbf{b}_i$, et de la même manière, $\mathbf{R}$ comme la combinaison linéaire des $\mathbf{a}_j$ :

\begin{eqnarray}
    \mathbf{k} & = & k_1 \mathbf{b}_1 + k_2 \mathbf{b}_2 + k_3 \mathbf{b}_3\\
    \mathbf{R} & = & n_1 \mathbf{a}_1 + n_2 \mathbf{a}_2 + n_3 \mathbf{a}_3
\end{eqnarray}

En effectuant le produit scalaire, on trouve :

\begin{equation}
    \mathbf{k}\cdot\mathbf{R} = 2\pi (k_1 n_1 + k_2n_2 + k_3n_3)
\end{equation}

Si on veut vérifier la définition (équation \ref{eq:defresreciproque}), il est nécessaire que $\mathbf{k}\cdot\mathbf{R}$ soit égal à $2\pi$ fois un entier, pour tout choix d'entiers $n_i$. Dès lors, il faut que les coefficients $k_i$ soient eux-même des entiers. Par conséquent, le réseau réciproque est un réseau de Bravais et les $\mathbf{b}_i$ peuvent être considérés comme des vecteurs
primitifs du réseau réciproque.

\section{Plans réticulaires}

Les vecteurs du réseau réciproque et les plans passant par les nœuds du réseau direct sont reliés par la notion de plans réticulaires. Cela deviendra très important dans la théorie de la diffraction. Nous décrivons ici cette relation par la géométrie.

\subsection{Définition}

Considérons un réseau de Bravais. Un plan réticulaire est défini comme un plan qui contient au moins trois nœuds du réseau non alignés. Comme le réseau de Bravais est invariant par translation, n'importe quel plan qui correspond à cette définition contient lui même un infinité de nœuds. Ceux-ci forment un réseau de Bravais bi-dimensionnel dans ce plan. Ces plans sont définis pour n'importe quel réseau de Bravais, qu'il soit dans l'espace réel ou réciproque.

On peut alors définir des familles de plans réticulaires, qui forment un ensemble de plans parallèles, équidistants, et qui contiennent à eux tous les points d'un réseau de Bravais. Chaque plan réticulaire est un élément de cette famille. Le réseau réciproque apporte un moyen facile de classifier toutes les familles possibles de plans réticulaires, qui sont inclus dans le théorème suivant :

\begin{description}
    \item[Théorème] Pour une famille donnée de plans réticulaires, séparés d'une distance $d$, il existe au moins un vecteur du réseau réciproque perpendiculaire à ce plan. Le plus court d'entre eux a une longueur de $2\pi/d$.
    \item[Réciproque] Pour tout vecteur du réseau réciproque $\mathbf{K}$, il y a une famille de plans réticulaires normaux à $\mathbf{K}$ et séparés d'une distance $d$, où $2\pi/d$ est la longueur du plus petit vecteur du réseau réciproque parallèle à $\mathbf{K}$.
\end{description}

\begin{figure}
    \includegraphics{./images/part1/reticulaire-01}
    \caption{Plans réticulaires indexés par les indices de Miller correspondants. (a) $x_P = 1/h$, $y_p = 1/k$, $z_P = 1/l$. (b) $x_P = 1/4$, $y_P = 1/2$, $z_P = 1$}
    \label{fig:retic}
\end{figure}

Un plan réticulaire P est défini par son intersection avec les axes du système de coordonnées, comme présenté sur la figure \ref{fig:retic}. Les coordonnées des points d'intersection sont $(x_P,0,0)$, $(0,y_P,0)$ et $(0,0,z_P)$. Comme le plan P contient des nœuds du réseau, les coordonnées $x_P$,$y_P$ et $z_P$ sont des nombres rationnels. Si un plan est parralèle à un axe du système de coordonnées, son intersection avec cet axe a lieu à l'infini : la coordonnée correspondante est notée $\infty$.

Un plan réticulaire est indexé par les indices $h$,$k$ et $l$ entre parenthèses : $(hkl)$. Il ne s'agit pas directement des coordonnées $x_P$,$y_P$ et $z_P$ des points d'intersectin du plan avec les axes : $h$,$k$ et $l$ sont des nombres entiers que nous allons définir plus en détail. Comme pour les rangées réticulaires, si un indice est négatif, il est symbolisé avec un trait au dessus.

\subsection{Indices de Miller}

Pour assurer la correspondance entre les vecteurs du réseau réciproque et les familles de plans réticulaires, nous avons défini une indexation $h$,$k$,$l$. Ces indices sont appelés \emph{indices de Miller} et sont définis par la réciproque de l'intersection des plans avec les axes cristallographiques. Si les indices de Miller d'un plan sont $(hkl)$ (écrits entre parenthèses), alors le plan intersecte les axes en $1/h$, $1/k$ et $1/l$. Si la maille a des côtés de longueur $a$, $b$ et $c$, alors le plan intersecte celle-ci en $a/h$, $b/k$, $c/l$, comme présenté dans la figure \ref{fig:retic}. Comme il y a une infinité de plans réticulaires parallèles entre eux, on choisit générallement les indices de Miller les plus petits possibles, et ils définissent cette famille complète de plans réticulaires parallèles.

Comme nous l'avons dit précédemment, pour un plan parallèle à un axe, la coordonnée de l'intersection est infinie. L'indice de Miller correspondant est $0$. Si un plan intersecte un axe en une coordonnée négative, on note cette coordonnée, encore une fois avec une barre au dessus.
En plus de cela, les plans $(nh\,nk\,nl)$ sont parallèles aux plans $(hkl)$ et en sont séparés d'une distance $d = \frac{1}{n}$

En outre, un plan réticulaire défini par les indices de Miller $h$,$k$ et $l$ est normal au vecteur du réseau réciproque $h\mathbf{b}_1 + k \mathbf{b}_2 + l\mathbf{b}_3$. 

\begin{figure}
    \includegraphics{./images/part1/cullity42-01}
    \caption{Indices de Miller de plans du réseau. La distance $d_{hkl}$
    correspond à l'espacement entre chacun de ces plans}
    \label{fig:miller}
\end{figure}

Par définition, comme chaque vecteur du réseau réciproque est une combinaison linéaire des trois vecteurs primitifs avec des coefficients intégraux, les indices de Miller sont toujours des entiers.



\section{Zones de Brillouin}

\begin{marginfigure}
\includegraphics{./images/part1/brillouin-01}
\caption{Première zone de Brillouin pour un réseau cubique centré}
\label{fig:brillouinbcc}
\end{marginfigure}

Nous avons introduit précedemment le concept de \emph{cellule de Wigner-Seitz}. Dans le réseau réciproque, on appelle \emph{première zone de Brillouin} la cellule de Wigner-Seitz.
Même si la première zone de Brillouin et la cellule de Wigner-Seitz du réseau réciproque correspondent aux mêmes concepts, la première n'existe que dans le réseau réciproque.

Les zones de Brillouin donnent une interprétation géométrique des conditions de la diffraction que l'on étudiera plus tard.

La première zone de Brillouin pour un cristal cubique centré (figure \ref{fig:brillouinbcc}) a la même forme que la cellule de Wigner-Seitz d'un cristal cubique à faces centrées, car le réseau réciproque d'un cristal cubique centré est un cristal cubique à faces centrées. Sur la figure \ref{fig:brillouinbcc}, les points de symétrie élevée sont représentés par les lettres $K$,$L$,$\Gamma$, $X$, etc. L'espace réciproque du réseau cubique centré est défini par :

\begin{eqnarray}
    b_1 & = & \frac{4\pi}{a} \half (\hat{y} + \hat{z})\\
    b_2 & = & \frac{4\pi}{a} \half (\hat{z} + \hat{x})\\
    b_3 & = & \frac{4\pi}{a} \half (\hat{x} + \hat{y})
\end{eqnarray}
    
\begin{marginfigure}
\includegraphics{./images/part1/brillouin-02}
\caption{Première zone de Brillouin pour un réseau cubique faces-centrées}
\label{fig:brillouinfcc}
\end{marginfigure}

De la même façon, la première zone de Brillouin d'un réseau cubique à faces centrées (figure \ref{fig:brillouinfcc}) a la même forme que la cellule de Wigner-Seitz d'un cristal cubique centré. Le réseau réciproque est défini par :

\begin{eqnarray}
    b_1 & = & \frac{4\pi}{a} \half (-\hat{x} + \hat{y} + \hat{z})\\
    b_2 & = & \frac{4\pi}{a} \half (\hat{x} - \hat{y} + \hat{z})\\
    b_3 & = & \frac{4\pi}{a} \half (\hat{x} + \hat{y} - \hat{z})
\end{eqnarray}

On peut généraliser la notion de zone de Brillouin à $n$. Remarquons que la première zone de Brillouin délimite l'ensemble des points de l'espace réciproque qui peuvent être atteints depuis l'origine sans traverser de plan bissecteur (également appelés plans de Bragg).

\begin{marginfigure}
    \includegraphics{./images/part1/zonesbrillouin-02}
    \caption{Illustration des 3 premières zones de Brillouin, contenues dans les plans de Bragg représentés pour un carré de côté $2b$ ($b=2\pi/a$ pour un réseau carré 2D).}
    \label{fig:constructionbrillouin}
\end{marginfigure}
Le seconde zone de Brillouin correspond à l'ensemble des points qui peuvent être atteints à partir de l'origine en traversant un plan de Bragg. Ainsi, on peut généraliser cela :
la n\ieme zone de Brillouin est l'ensemble des points qui peuvent être atteints en traversant (n-1) plans de Bragg.

Une zone de Brillouin est une maille primitive du réseau réciproque. Par conséquent, le volume de la n\ieme zone de Brillouin est égal au volume de la première zone. Pour le voir, on peut représenter en schéma de zone réduite les zones de Brillouin. Il faut donc découper les parties de la n\ieme zone de Brillouin qui sortent de la maille primitive usuelle, et les replacer à l'intérieur, comme présenté sur la figure \ref{fig:brillouinreduite}.

\begin{figure}
    \includegraphics{./images/part1/zonesbrillouin-01}
    \caption{Représentation des 3 premières zones de Brillouin dans un schéma de zone réduite (les parties sont translatées d'un vecteur $\mathbf{G}$ du réseau réciproque. Les surfaces des zones sont identiques.}
    \label{fig:brillouinreduite}
\end{figure}

\section{Exemple : réseau réciproque à deux dimensions}

L'exercice consiste à considérer un réseau oblique à deux dimensions, dont les vecteurs de base rapportés à un repère orthonormé $(\hat{x},\hat{y})$ sont :
\begin{equation}
\mathbf{a} = 2\mathbf{\hat{x}},\quad \mathbf{b} = \mathbf{\hat{x}}+2\mathbf{\hat{y}}
\end{equation}.

Soient $A$ et $B$ les vecteurs de base du réseau réciproque.

On peut écrire, à partir de la relation \ref{eq:reseaureciproque} :

\begin{equation}
\mathbf{a}_i \cdot \mathbf{A}_j = 2\pi \delta_i^j
\end{equation}

Ce qui, en réécrivant les vecteurs, se traduit par :

\begin{equation*}
\mathbf{a} \cdot \mathbf{A} = 2\pi = \begin{pmatrix} 2 \\0 \end{pmatrix}\cdot\begin{pmatrix} A_x \\ A_y \end{pmatrix} = 2A_x
\end{equation*}

\begin{equation*}
\mathbf{b} \cdot \mathbf{A} = 0 = \begin{pmatrix} 1 \\2 \end{pmatrix}\cdot\begin{pmatrix} A_x \\ A_y \end{pmatrix} = A_x + 2A_y
\end{equation*}

Soit

\begin{equation*}
A_y = -\frac{A_x}{2} = -\frac{\pi}{2}
\end{equation*}

et

\begin{equation*}
\mathbf{a} \cdot \mathbf{B} = 0 = \begin{pmatrix} 2 \\0 \end{pmatrix}\cdot\begin{pmatrix} B_x \\ B_y \end{pmatrix} = 2B_x
\end{equation*}

\begin{equation*}
\mathbf{b} \cdot \mathbf{B} = 2\pi = \begin{pmatrix} 1 \\2 \end{pmatrix}\cdot\begin{pmatrix} B_x \\ B_y \end{pmatrix} = B_x + 2B_y
\end{equation*}


Soit

\begin{equation*}
B_y = \pi
\end{equation*}

On obtient alors les vecteurs primitifs $\mathbf{A}$ et $\mathbf{B}$ du réseau réciproque :

\begin{equation}
\mathbf{A} = \pi\mathbf{\hat{x}}-\frac{\pi}{2}\mathbf{\hat{y}},\quad \mathbf{B} = \pi \mathbf{\hat{y}}
\end{equation}.

On peut donc tracer :

\begin{figure}
\TODO
\caption{Exemple : réseau réciproque, zones de Brillouin}
\label{fig:exemplebrillouin}
\end{figure}
\chapter{Classification des réseaux cristallins}
\label{ch:classification}

Mon but ici n'est pas de rentrer en détails dans les détails de la classification
mais d'en donner un apperçu pratique et utile. On trouvera plus de détails à ce
sujet dans les ouvrages suivants :
\begin{description}
    \item[Diffraction from Materials] L.H. Schwartz et J.B: Cohen,
        Springer-Verlag.
    \item[Essentials of Crystallography] D. et C. Mc Kie, Blackwell Scientific
        Publications, 1986.
    \item[The Basics of Crystallography and Diffraction] C. Hammond,
        International Union of Crystallography Text on Crystallography, Oxford
        University Press, 1997.
\end{description}

Le polycopié de N. Lequeux de \pc est largement plus complet que ce document de ce
point de vue. En particulier, j'ai choisi de ne pas détailler les aspects
mathématiques liées aux symétrie et de me concentrer sur la géométrie. Je ne fais ni
mention des 32 groupes ponctuels, ni de la notation de Schonflies ou des 230 groupes
d'espace.

\section{Opérations de symétrie}

La périodicité d'un réseau est due aux translations de réseau. Celles-ci sont
définies par le postulat de Bravais, complété par celui de Schönflies-Fedorov.

L'ensemble des points d'un réseau, appelés \emph{nœuds du réseau}, constitue un
réseau spatial périodique. On construit celui-ci en appliquant à chaque nœud
l'ensemble des translations :
\begin{equation}
\mathbf{t}_n = u \mathbf{t}_1 + v \mathbf{t}_2 + w \mathbf{t}_3
\end{equation}
où $u$, $v$ et $w$ sont des entiers et $\mathbf{t}_1$, $\mathbf{t}_2$, $\mathbf{t}_3$
forment une base primitive.

La périodicité du réseau est une contrainte forte qui limite le nombre et la nature des opérations de symétrie assurant l'invariance du réseau.

Avant de considérer comment les symétries sont incorporées dans les réseau, il
est nécessaire de comprendre comment les éléments de symétrie opèrent sur leur
environnement. Si un certain objet est à une position donnée d'un élément de
symétrie, le type d'élément de symétrie impose l'emplacement et l'orientation
d'un objet identique. Alternativement, un corps ou une structure est dite
symétrique lorsque ses composants sont arrangés de sorte à ce que certaines
opérations de symétries peuvent être effectuées en son sein, en le recouvrant
totalement.
Par exemple, si un corps est symétrique par rapport à un plan passant à son
travers, alors la réflexion de chaque moitié de ce corps par le miroir plan
produira un corps qui coincidera avec l'autre moitié. Par conséquent, un cube a
plusieurs plans de symétrie, dont l'un d'entre eux est représenté sur la figure
\ref{cullity8}. Les ponits $A_1$ et $A_2$ sur cette figure doivent être
identiques en raison du miroir plan passant par le centre du cube ; ils sont
reliés par une réflexion.

\begin{figure}
    \includegraphics{./images/part1/cullity46.eps}
    \caption{Certains éléments de symétrie sur un cube. (a) réflexion plane :
    $A_1$ devient $A_2$. (b): rotation d'ordre 4: $A_1$ devient $A_2$ ; rotation
    d'ordre 3 : $A_1$ devient $A_3$ ; rotation d'ordre 2 : $A_1$ devient $A_4$.
    (c) centre d'inversion. (d) rotation d'ordre 4 suvie d'une inversion : $A_1$
    devient $A'_1$ par la rotation d'ordre 4 puis $A_2$ par
l'inversion.}
    \label{cullity8}
\end{figure}

Il y a quatre opérations de symétrie macroscopiques : rélfexion, rotation,
inversion et roto-inversion. Un corps a une symmétrie de rotation d'ordre n selon
un axe si une rotation de 360°/n coincide avec le cristal. Par conséquenct, un
cube a un axe de rotation d'ordre 4 normal à chaque face, un axe de rotation
d'ordre 3 le long de chaque grande diagonale, et un axe d'ordre 2 liant le centre
de chaque côté opposé. Certains de ceux-ci sont représentés sur la figure
\ref{cullity8}. En général, les axes de rotation peuvent être d'ordre 1, 2, 3, 4
ou 6. Les axes de multiplicité 1 sont présents dans tous les ojbets, et ne sont
normalement pas représenté. En revanche, les axes de rotation d'orde 5 ou d'ordre
supérieur à 6 sont impossibles, parce qu'une maille primitive qui aurait une
telle symétrie ne pourrait pas paver tout l'espace sans laisser de lacunes.

Un corps a un centre d'inversion si les points correspondants de son corps sont
situés à des distances égales du centre d'une ligne tracée à travers le centre.
Un corps possédant un centre d'inversion se superposera parfaitement avec lui
même en chaque point du corps s'il est inversé, ou réfléchi par le centre
d'inversion. Un cube a un centre d'inversion à l'intersection de ses grandes
diagonales.
Finalement, un corps peut avoir un axe de rotation-inevrsion, d'ordre 1, 2, 3, 4
ou 6. S'il a un axe de rotation-inversion d'ordre n, alors il peut être ramené à
lui-même par une rotation de 360°/n par l'axe, suivie d'une inversion par le
centre, qui est situé sur l'axe.

Considérons a présent toutes lesp ositions et orientations qu'un objet ou un
motif peut prendre suite aux opérations de symétrie de différent types (figure
\ref{cullity9}). Le motif doit apparaître encore plus fréquemment que si, par
exemple, deux opérations de symétrie passent par le même point. L'opération
combinée d'un axe d'ordre 2 situé sur un miroir plan produit un second miroir
plan, perpendiculaire au premier, et contenant également un axe d'ordre 2.
Lorsqu'un axe d'ordre 4 est situé sur un miroir plan, la symétrie requiert qu'un
total de 8 motifs identiques (dans des orientations diverses) et 4 miroirs plans
soient présents.

\begin{figure}
    \includegraphics{./images/part1/cullity48.eps}
    \caption{Opérations de symétrie et symboles associées pour des rotations
        d'ordre 1 (a), d'ordre 2 (b), d'ordre 3 (c), d'ordre 4 (d), d'ordre 6
        (e). (f) représente un miroir plan, (g) un miroir plan et un axe d'ordre
    2 et (h) un miroir plan et un axe d'ordre 4.}
    \label{cullity9}
\end{figure}

Les différentes opérations de symétrie agissant sur un point sont appelées
\emph{groupe ponctuel}. À deux dimensions, il y a dix groupe ponctuels qui
peuvent être inclus dans des réseaux. À trois dimensions, le nombre de groupes
ponctuels est de trente-deux : contrairement aux réseaux bi-dimensionnels, les
centres d'inversions ne sont plus équivalents à un axe de rotation d'ordre 2, et
les combinaisons comme celles des miroirs perpendiculaires à des axes de rotation
sont possibles.
Il est important d'insister sur le fait que les éléments de symétries agissent
sur l'ensemble de l'espace. La discusison jusqu'à présent s'est concentrée sur
l'espace réel, mais tous les principes ici présents s'appliquent également dans
l'espace réciproque.

\section{Systèmes cristallins}

Lorsque l'on définit un réseau avec trois vecteurs de réseau non coplanaires, les
mailles primitives de diverses formes peuvent résulter, selon la longueur et
l'orientation des vecteurs. Par exemple, si les vecteurs
$\mathbf{a}$,$\mathbf{b}$,$\mathbf{c}$ sont de longueur égale et à des angles
droits les uns des autres, \ie $\mathbf{a = b = c}$ et $\alpha = \beta = \gamma
= \SI{90}{\degree}$, alors la maille primitive est cubique. Donner des valeurs
différentes aux longueurs axiales et aux angles produira des cellules de
différentes formes, et par conséquent de différent types de points du réseau, car
les points du réseau sont situés sur les coins de la maille primitive. Il vient
alors qu'il n'y a que sept sorte de mailles qui sont nécessairement inclues dans
tous les points du réseau possibles. Celles-ci correspondent aux sept
\emph{systèmes cristallins} parmi lesquels les cristaux peuvent être classifiés.
Ces systèmes sont listés sur le tableau \ref{tab:syscrist}. \footnote{Le système
trigonal est parfois appelé rhomboédrique.}

\begin{table}
    \resizebox{\textwidth}{!}{
    \begin{tabular}{lclr}
    \toprule
    Système & axes et angles & réseaux de Bravais & Symbole\\
    \midrule
    \multirow{3}{*}{Cubique} & \multirow{3}{*}{ $a=b=c$, $\alpha = \beta = \gamma
    = \SI{90}{\degree}$} & Simple & P \\
    & & Centré & I \\
    & & Faces-centrées & F \\
    & & & \\
    \multirow{2}{*}{Tétragonal} & \multirow{2}{*}{$a=b\neq c$, $\alpha = \beta =
    \gamma = \SI{90}{\degree}$} & Simple & P \\
    & & Centré & I \\
    & & & \\
    \multirow{4}{*}{Orthorhombique} & \multirow{4}{*}{$a \neq b \neq c$, $\alpha
    = \beta = \gamma = \SI{90}{\degree}$} & Simple & P \\
    & & Centré & I \\
    & & Base-centré & C \\
    & & Faces-centrées & F \\
    & & & \\
    Trigonal & $a = b = c$,
    $\alpha = \beta = \gamma \neq \SI{90}{\degree}$ & Simple & R \\
    & & & \\
    Hexagonal & $a = b \neq c$, $\alpha = \beta = \SI{90}{\degree}$, $\gamma =
    \SI{120}{\degree}$ & Simple & P \\
    & & & \\
    \multirow{2}{*}{Monoclinique} & \multirow{2}{*}{$a \neq b \neq c$, $\alpha =
    \gamma = \SI{90}{\degree} \neq \beta$} & Simple & P\\
    & & Base-centré & C\\
    & & & \\
    Triclinique & $a \neq b \neq c$, $\alpha \neq \beta \neq \gamma \neq
    \SI{90}{\degree}$ & Simple & P\\
    \bottomrule
\end{tabular}}
\caption{Systèmes cristallins et résaux de Bravais}
\label{tab:syscrist}
\end{table}
    
\TODO attention réseau ponctuel parfois mal traduit avant en point du réseau !

Sept réseaux ponctuels différents peuvent être obtenus simplement en plaçant les
points aux côtés des mailles primitives des sept systèmes cristallins. Cependant,
il y a d'autres arrangements de points qui peuvent respecter les conditions d'un
réseau ponctuel, à savoir que chaque point du réseau a un environnement
identique. Le cristallographe français Bravais a travaillé sur ce problème et a
démontré en 1848 qu'il y a 14 réseaux ponctuels possibles, et pas plus. Ce
résultat est très important ; en hommage, le terme de \emph{réseau de Bravais}
est devenu synonyme de \emph{réseau ponctuel}. Par exemple, si un point est placé
au centre de chaque maille d'un réseau ponctuel cubique, le nouvel arrangement de
points forme également un réseau de Bravais. De façon similaier, un autre réseau
ponctuel peut être basé sur une maille cubique n'ayant des nœuds du réseau qu'à
chaque sommet, et au centre de chaque face.

Les 14 réseaux de Bravais sont décrits dans la table \ref{tab:syscrist}.
Certaines mailles sont simples (ou primitives) (symbole P ou R), et certaines
sont non-primitives (les autres symboles). Les mailles primitives n'ont qu'un
nœud du réseau par maille, alors que les non primitives en ont plus qu'une. Un
nœud du réseau à l'intérieur d'une maille appartient à cette maille, alors qu'un
point sur une face ou sur un sommet ne sera pas à compter plusieurs fois.

\begin{figure}
    \includegraphics{./images/part1/cullity50.eps}
    \caption{les 14 réseaux de Bravais}
    \label{fig:bravaisschema}
\end{figure}

Chaque maille contenant des points du réseau sur ses sommets est primitive, alors
qu'une maille contenant des points en son centre ou sur ses faces est
non-primitive. Les symboles $F$ et $I$ se réfèrent respectivement aux mailles à
faces centrées et centrées, alors que $A$, $B$ et $C$ se réfèrent aux mailles
base-centrée, avec un atome au centre de deux faces $A$, $B$ ou $C$
opposées\footnote{la face $A$ est définie par les axes $b$ et $c$, la face $B$
par les axes $a$ et $c$ et la face $C$ par les axes $a$ et $b$}. Le symbole $R$
est utilisé principalement pour le système trigonal (ou rhomboédrique). Sur la
figure \ref{fig:bravaisschema}, les axes de longueur égale dans un système
particulier ont le même symbole, par exemple les axes du système cubique sont
tous marqués $a$, dans le système tétragonal (dans lequel $a = b \neq c$), deux
axes sont marqués $a$ et un $c$.

À première vue, la liste des réseaux de Bravais dans le tableau
\ref{tab:syscrist} est incomplète : pourquoi, par exemple, on n'a pas de réseau
tétragonal à base centrée ? En fait, si l'on trace un réseau tétragonal C de
paramètre de maille $a$, on se rend compte que celui-ci peut se réduire à un
réseau tétragonal P de paramètre de maille $\frac{a}{\sqrt{2}}$.

Les ponits du réseau d'une maille non primitive peuvent être étendus dans tous
l'espace par des translations des vecteurs unitaires $\mathbf{a}$, $\mathbf{b}$
et $\mathbf{c}$. Les points du réseau associés à ces mailles unitaires peuvent
être translatés un a un comme un groupe. Dans chacun des cas, les points
équivalents du réseau dans les mailles unitaires sont séparés par un des vecteurs
primitifs, peu importe où ces points sont localisés dans la maille.

À présent, les systèmes cristallins sont définis par la possession d'un certain
nombre d'éléments de symétrie. Chaque système se différencie d'un autre à partir
du nombre d'opérations de symétrie dont il dispose et par les valeurs des
longueurs axiales et des angles. En fait, ceux-ci sont interdépendants. Par
exemple, l'existence d'un axe de rotation d'ordre 4, normal aux faces d'une
maille cubique requiert que les bords de la cellule soient de même longueur et à
\SI{90}{\degree} les uns des autres. D'un autre côté, une maille tétragonale n'a
qu'un axe de rotation d'ordre 4, et cette symmétrie requiert qu'il n'y a que deux
bords de maille qui doivent être égau, \ie les deux qui sont normaux à l'axe.

Le nombre minimal d'opération de symétrie que possède chaque système cristallin
est listé dans le tableau \ref{tab:minsym}. Certains cristaux peuvent posséder
plus de symmétrie que ce nombre minimal requis par le système cristallin auquel
ils appartiennent, mais aucun n'en a moins. L'existence d'une certaine opération
de symétrie implique généralement l'existence d'autres. Par exemple, un cristal
qui possèdent trois axes de rotation d'ordre 4 a nécessairement, quatre axes de
rotation d'ordre 3 et appartient aux système cubique. La réciproque n'est pas
forcément vraie : il y a des systèmes cubques qui n'ont pas forcément trois axes
de rotation d'ordre 4.

\begin{table}
    \begin{tabularx}{\textwidth}{lX}
        \toprule
        Système & Nombre minimal d'éléments de symétrie \\
        \midrule
        Cubique & 4 axes de rotation d'ordre 4 \\
        Tétragonal & Un axe de rotation (ou rotation-inversion) d'ordre 4\\
        Orthorhombique & Trois axes de rotation (ou rotation-inversion) d'ordre
        2, orthogonaux \\
        Trigonal & Un axe de rotation (ou rotation-inversion) d'ordre 3 \\
        Hexagonal & Un axe de rotation (ou rotation-inversion) d'ordre 6 \\
        Monoclinique & Un axe de rotation (ou rotation-inversion) d'ordre 2 \\
        Triclinique & Aucun\\
        \bottomrule
    \end{tabularx}
    \label{tab:minsym}
    \caption{Éléments de symétrie minimums retrouvés dans chacun des systèmes
    cristallins}
\end{table}

\section{Mailles primitives et non-primitives}

Dans chacun des réseau ponctuels, une maille unitaire peut être choisie d'une
infinité de façons différentes et peut contenir un ou plusieurs nœuds du réseau.
Il est important de remarquer qu'une maille primitive n'\emph{existe} pas
forcément dans un réseau : il s'agit d'une construction mentale et est choisie
pour son utilité. Les mailles conventionnelles présentées en figure
\ref{fig:bravaisschema} sont pratiques et conformes avec les éléments de symétrie
du réseau. Dans certains cas, on pourra en choisir d'autres.

Chacun des 14 réseaux de Bravais peut être réduit à une maille primitive. Par
exemple, le réceau cubique faces-centrées, présenté en figure \ref{fig:fcc} peut
être considéré dans le système trigonal (figure \ref{fig:fcctrigonal}). Chaque
maille cubique a 4 nœuds qui y sont associés; une maille trigonale n'en a qu'un :
le réseau cubique faces-centrées (avec une maille cubique) a donc un
volume quatre fois supérieur à la maille primitive (dans le système trigonal).
Cependant, il est souvent plus pratique de considérer une maille cubique plutôt
que trigonale parce que sa forme suggère immédiatement la symétrie cubique que le
réseau possède. De façon similaire, les autres mailles non-primitives listées
dans le tableau \ref{tab:syscrist} sont souvent préférées aux mailles primitives.

\begin{marginfigure}
    \includegraphics{./images/part1/cullity53.eps}
    \caption{Le réseau cubique faces-centrées appartient au système trigonal : la
    maille en pointillés est la maille primitive}
    \label{fig:fcctrigonal}
\end{marginfigure}

Dès lors, pourquoi les réseaux centrais apparraissent dans la liste des 14
réseaux de Bravais ? Si deux mailles peuvent décrire le même ensemble de nœuds du
réseau, alors, pourquoi ne pas éliminer la maille cubique et n'utiliser que la
maille trigonale ? La réponse est que cette maille est une maille particulière du
réseau trigonale, avec un angle $\alpha = \SI{60}{\degree}$. Dans le réseau
trigonal classique, aucune restriction n'est faite sur l'angle $\alpha$ ; le
résultat est un réseau de points avec un axe de symétrie d'orde 3. Lorsque
$\alpha = \SI{60}{\degree}$, alors le réseau a 4 axes de rotation d'ordre 3, et
cette symétrie le place dans le système cubique.

Si des mailles non primitives sont utilisées, le vecteur de l'origine de
n'importe quel nœud du réseau aura des composentes qui seront des multiples non
entiers des  vecteurs du réseau $\mathbf{a}$, $\mathbf{b}$ et $\mathbf{c}$. La
position de n'importe quel point du réseau dans la maille sera donnée en terme de
ses coordonnées ; si le vecteur de l'origine de la maille unitaire à un nœud
donné a des composantes $x\mathbf{a}, y\mathbf{b}, z\mathbf{c}$ où $x,y,z$ sont
des nombres rationels, alors les coordonnées des points sont $x\,y\,z$. Par
conséquent, le point $A$ sur la figure \ref{fig:fcctrigonal}, pris comme
l'origine, a comme coordonnées $0\,0\,0$, alors que les points $B, C$ et $D$,
dans le système cubique, ont des coordonnées respectives $0\,\half\,\half$,
$\half\,0\,\half$ et $\half\,\half\,0$. Le point $E$ a pour coordonnées
$\half\,\half\,1$ et est équivalent au point $D$, sépéré du vecteur $\mathbf{c}$.
Les coordonnées des points équivalents dans différentes mailles peuvent être
rendues identiques par l'addition ou la soustraction par un ensemble de
coordonnées entières : dans ce cas, la soustraction de $\half\,\half\,1$ par
$0\,0\,1$ (la coordonnée de $E$) donne $\half\,\half\,0$ (la coordonnée de D).

Notons que la coordonnée d'un nœud d'un réseau centré ($I$), par exemple, est
toujours $\half\,\half\,\half$, peu importe que la maille unitaire soit cubique,
tétragonale, orthorhombique ou peu importe sa taille. La coordonnée d'une
position ponctuelle, comme $\half\,\half\,\half$, peut également être vue comme
un opérateur qui, lorsqu'il est appliqué à un point à l'origine, le translatera à
la position $\half\,\half\,\half$, la position finale obtenue par simple addition
de l'opérateur $\half\half\half$ et la position originale $0\,0\,0$. Dans ce cas,
le veucteur entre $0\,0\,0$ et toutes les positions du centre dans la maille
cubique centrée, \ie $<\half\half\half>$ sont appelées \emph{translation de
réseau I}, car elles produisent les deux nœuds ponctuels caractéristiques
du réseau en étant appliquées à un point à l'origine. De façon similaire, les
quatre positions ponctuelles caractéristiques du système cubique faces-centrées
($F$), \ie $0\,0\,0$, $0\,\half\,\half$, $\half\,0\,\half$ et $\half\,\half\,0$,
sont reliées par la \emph{translation de réseau F} $<\half\half 0>$. Les
translations de réseau $A$, $B$ ou $C$  dépendent de la paire de faces opposées
sur laquelle elles s'appliquent. Si la maille est centrée sur la face $C$ par
exemple, alors les positions équivalentes sont $0\,0\,0$, $\half\,\half\,0$ et
les translations sont donc $[\half\half 0]$. Ainsi, on peut résumer ainsi :

\begin{itemize}
    \item translation de réseau $I$ : $<\half\half\half>$
    \item translation de réseau $F$ : $<\half \half 0>$
    \item translation de réseau $A$ : $[0 \half \half]$
    \item translation de réseau $B$ : $[\half 0 \half]$
    \item translation de réseau $C$ : $[\half \half 0]$
\end{itemize}

Pour les cristaux qui ne possèdent qu'un atome par nœud (\ie Nb, Ni, Cu, etc.),
on peut généralement écrire les positions comme \emph{$0\,0\,0$ + translation de
réseau $I$} par exemple. Si les mailles primitives ont plus d'un atome par nœud,
comme le silicium par exemple (qui a un réseau de Bravais cubique à
faces-centrées en $0\,0\,0$ et $\frac{1}{4}\,\frac{1}{4}\,\frac{1}{4}$ en plus
des translations de réseau $F$), ce qui fait un total de 8 atomes par maille. Des
cristaux moléculaires plus complexes, comme ceux trouvés dans les systèmes
biologiques, peuvent avoir un plus grand nombre d'atomes de différents types à
chaque nœud du réseau.

Il est important de noter que les incides d'un plan ou d'une direction n'ont
aucun sens si on ne définit pas préalablement l'orientation de la maille. Cela
signifie que les indices d'un plan réticulaire dépendent le la maille choisie.

Dans chaque système cristallin, il y a des ensembles de plans du réseau
équivalents, reliés par des symétries. Ceux-ci sont appelés \emph{famille de
plans}, et les indices de chacun de ces plans sont notés entre accolades
($\{hkl\}$) pour signifier la famille complète. En général, les plans d'une même
famille ont le même espacement mais des indices de Miller différents. Par
exemple, les faces d'un cube ($100$),($010$),($001$), ($\bar{1}00$),
($0\bar{1}0$) et ($00\bar{1}$) sont des plans de la famille $\{100\}$, car chacun
d'entre eux peut être généré par les autres, par l'opération de l'axe de rotation
d'ordre 4 perpendiculaire à la face du cube. Dans le système tétragonal,
cependant, seuls les plans ($100$),($010$), ($\bar{1}00$) et ($0\bar{1}0$) sont
équivalents (appartiennent à la même famille $\{100\}$), car l'axe $c$ a une
longueu différente. Les deux autres plans ($001$) et ($00\bar{1}$) appartiennent
à la famille $\{001\}$. Il est facile de voir que les quatre premiers sont reliés
entre eux par un axe de rotation d'ordre 4, et le troisième par un axe d'ordre 2.

Les plans d'une famille sont \emph{en zone} (ou aussi \emph{tautozonaux}) s'ils
sont tous parallèles à une même rangée, dite \emph{axe de zone}. L'ensemble des
plans est spécifié en donnant les indices de la rangée. De tels plans peuvent
avoir des indices et des espacements différents, la seule contrainte est qu'ils
soient parallèles à l'axe de zone.

\begin{marginfigure}
    \includegraphics{./images/part1/cullity56}
    \caption{Les plans grisésdu réseau cubique sont les plans en zone $\{001\}$}
    \label{fig:tautozonaux}
\end{marginfigure}

Prenons par exemple un axe de zone $[uvw]$. Alors chaque plan $(hkl)$ qui
appartient à cette zone vérifie la relation :

\begin{equation}
    hu + kv + lw = 0
\end{equation}

Chaque couple de plans non parallèles sont des plans de zone car ils sont
parallèles à la droite définie par leur intersection. Soient leurs indices :
$(h_1k_1l_1)$ et $(h_2k_2l_2)$, alors les indices de leur axe de zone $[uvw]$
sont définis par le produit tensoriel $[h_1k_1l_1] \times [h_2k_2l_2]$ :
\begin{eqnarray}
    u & = & k_1 l_2 - k_2 l_1\\
    v & = & l_1 h_2 - l_2 h_1\\
    w & = & h_1 k_2 - h_2 k_1
\end{eqnarray}

\subsection{distance interréticulaire}

La distance interréticulaire $d_{hkl}$ de la famille de plans $\{hkl\}$ dépend du
système cristallin dans lequel on se place. Le système cubique a la forme la plus
simple :

\begin{equation}
    \text{(cubique)}\quad d_{hkl} = \frac{a}{\sqrt{h^2 + k^2 + l^2}}
\end{equation}

Dans le système tétragonal, l'équation fait intervenir à la fois a et c, qui ne
sont généralement pas égaux :

\begin{equation}
    \text{(tétragonal)}\quad d_{hkl} = \frac{a}{\sqrt{h^2 + k^2 + l^2
    \left(\frac{a^2}{c^2}\right) }}
\end{equation}

Dans le système cubique, il est important de se rappeler que $[hkl]$ est
orthogonal à $(hkl)$. Pour tous les autres systèmes cristallins, cela est
généralement faux.

%    \section{autres}
%
%
%    \TODO parler des opérations de symétrie microscopiques, au niveau de
%    l'arrangement des atomes, pas seulement visibles sur les cristaux.
%
%
%
%    \subsection{Opérations de symétrie compatibles avec la nature du réseau
%    cristallin}
%
%    La démonstration des propriétés évoquées ici est fournie dans le poly de N.
%    Lequeux. Pour cette raison, cela n'est pas présenté dans ce document.
%
%    \begin{enumerate}
%    \item Seuls les ordres de rotation 1 (identité), 2, 3, 4 et 6 sont permis. Les
%        ordres 5 et >6 sont interdits ;
%    \item tout axe de rotation est parallèle à une translation de réseau ;
%    \item tout axe de rotation est perpendiculaire à un plan réticulaire ;
%    \item une opération de symétrie directe est toujours une opération hélicoïdale.
%    \end{enumerate}
%
%\begin{table}[ht]
%\begin{tabularx}{\textwidth}{lRRR}
%\toprule
%ordre de rotation & translation $\omega_{//}$ & notation & symbole graphique \\
%\midrule
%1 & 0 & 1 & \\
%2 & 0 & $2$ & \cry{2}\\
%& $\frac{1}{2}\mathbf{a}$ & $2_1$ & \cry{21}\\
%3 & 0 & $3$ & \cry{3} \\
%& $\frac{1}{3}\mathbf{a}$ & $3_1$ & \cry{31} \\
%& $\frac{2}{3}\mathbf{a}$ & $3_2$ & \cry{32} \\
%4 & 0 & $4$ & \cry{4} \\
%& $\frac{1}{4}\mathbf{a}$ & $4_1$ & \cry{41} \\
%& $\frac{2}{4}\mathbf{a}$ & $4_2$ & \cry{42} \\
%& $\frac{3}{4}\mathbf{a}$ & $4_3$ & \cry{43} \\
%6 & 0 & $6$ & \cry{6} \\
%& $\frac{1}{6}\mathbf{a}$ & $6_1$ & \cry{61} \\
%& $\frac{2}{6}\mathbf{a}$ & $6_2$ & \cry{62} \\
%& $\frac{3}{6}\mathbf{a}$ & $6_3$ & \cry{63} \\
%& $\frac{4}{6}\mathbf{a}$ & $6_4$ & \cry{64} \\
%& $\frac{5}{6}\mathbf{a}$ & $6_5$ & \cry{65} \\
%\bottomrule
%\end{tabularx}
%\label{}
%\caption{Liste des symétries directes et symboles associés}
%\end{table}

%    \subsection{opérations de symétries inverses}
%    les miroirs purs se notent m
%
%    on note $\mathbf{a}$,$\mathbf{b}$,$\mathbf{c}$ les miroirs de glissement de
%    translation $\frac{\mathbf{a}}{2}$, $\frac{\mathbf{b}}{2}$,
%    $\frac{\mathbf{c}}{2}$.
%
%    les miroirs de translation $\frac{\mathbf{a\pm b}}{2}$, $\frac{\mathbf{a\pm
%    c}}{2}$ et $\frac{\mathbf{b \pm c}}{2}$ se notent $\mathbf{n}$.
%
%    il existe des miroirs de glissement \emph{diamant} notés $\mathbf{d}$ de
%    translations $\frac{\mathbf{a\pm b}}{4}$ et $\frac{\mathbf{a\pm b \pm c}}{4}$
%    autorisés dans certains réseaux de Bravais F et I, qui sont décrits par des
%    mailles non primitives et qui présentent des translations non entières.
%
%    \begin{table}[ht]
%        \begin{tabularx}{\textwidth}{lRR}
%            \toprule
%            ordre de rotation & notation & symbole graphique \\
%            \midrule
%            1 & $\bar{1}$ ou $I$ & \cry{10} \\
%            3 & $\bar{3}$ & \cry{30} \\
%            4 & $\bar{4}$ & \cry{24} \\
%            5 & $\bar{5}$ & \cry{60} \\
%            \bottomrule
%        \end{tabularx}
%        \label{}
%        \caption{Liste des symétries inverses et symboles associés}
%    \end{table}
%
%    Donc :
%    dans un cristal, les opérations de symétrie sont du type $(W,\mathbf{\omega})$ où
%    $W$ est une rotation ou une rotation-inversion d'ordre 1,2,3,4 ou 6.
%
%    $(E,\mathbf{t}_n)$ sont les translatinos de réseau, notées $\mathbf{t}_n$.
%
%    Toute opération directe d'ordre n supérieur à 1 est équivalente à une opération
%    hélicoïdale par rapport à un axe $(W, \omega_{//})$, où $\omega_{//}$ est une
%    translation de (m/n) fois la plus petite translation parallèle à l'axe de
%    rotation. On les note $\mathbf{n}$ pour les opérations pures, et $\mathbf{n}_m$
%    pour les opérations avec glissement.
%
%    Toute opération inverse d'ordre $n\neq 2$ est une opération inverse pure par
%    rapport à un point. On les note $\mathbf{\bar{n}} = \bar{1}, \bar{3}, \bar{4},
%    \bar{6}$
%
%    Les opérations inverses d'ordre $n=2$ sont des miroirs avec glissement $(\bar{2},
%    \omega_{\perp})$ où $\omega_\perp$ est soit nul (opération miroir pur), soit égal
%    à \half fois le plus petit vecteur de translation dans une rangée parallèle au
%    plan miroir. On les note $\mathbf{m}$ pour des miroirs purs et
%    $\mathbf{a}$,$\mathbf{b}$,$\mathbf{c}$,$\mathbf{n}$ et $\mathbf{d}$ pour des
%    miroirs avec glissement.
%
%    Lors du dénombrement des groupes ponctuels cristallographique, on peut alors
%    trouver :
%    \begin{itemize}
%    \item 11 groupes propres ;
%    \item 11 groupes impropres contenant l'inversion ;
%    \item 10 groupes impropres ne contenant pas l'inversion.
%    \end{itemize}
%    Sout au total 32 groupes ponctuels.
%
%
%    \section{32 groupes de symétrie}
%    On pourait redémontrer qu'il existe 32 groupes de symétrie d'orientation
%    cristalline à 3D qui résultent des combinaisons des rotations directes ou
%    inverses ponctuelles W d'ordre 1,2, 3, 4 et 6.
%
%    Chacun des 32 groupes ponctuels forme ce qu'on peut appeler une \emph{classe
%    cristalline}.
%
%
%    \section{Notations de Schonflies}
%
%    \section{7 systèmes cristallins}
%    Les 32 groupes de symétrie d'orientation sont regroupés en 7 systèmes
%    cristallins :
%    \begin{itemize}
%        \item triclinique
%        \item monoclinique
%        \item orthorhombique
%        \item quadratique
%        \item trigonal
%        \item hexagonal
%        \item cubique
%    \end{itemize}
%
%    Le système cubique est associé nécessairement à la présence dans un cristal de 4
%    axes ternaires orientés suivant les 4 diagonales du cube (angle de 
%    \SI{70.53}{\degree}).
%
%    Le système hexagonal est associé à la présence d'un axe sénaire (ordre 6).
%
%    Le système trigonal est associé à la présence d'un seul axe ternaire.
%
%    Le système quadratique est associé à la présence d'un axe quaternaire mais pas
%    d'axes ternaires.
%    \section{14 réseaux de bravais}
%    \section{230 groupes d'espace}
%    notation de Hermann-Mauguin

\section{Récapitulatif des notations utilisées}

\begin{table}[ht]
\begin{tabularx}{.8\textwidth}{ll}
\toprule
notation & signification \\
\midrule
$\mathbf{R}$ & points du réseau de Bravais \\
$\mathbf{a}_i$ & vecteurs du réseau de Bravais \\
$\mathbf{K}$ & points du résau réciproque (vecteur d'onde) \\
$\mathbf{b}_i$ & vecteurs du réseau réciproque \\
$[u v w]$ & rangée du réseau direct\\
$(hkl)$ & plan du réseau direct \\
$[hkl]^*$ & rangée du réseau réciproque \\
$(uvw)^*$ & plan du réseau réciproque \\
$<hkl>$ & famille de rangées directes \\
$\{hkl\}$ & hamille de plans équivalents\\
\bottomrule
\end{tabularx}
\caption[Notations utilisées en cristallographie]{Rappel des notations utilisées en cristallographie}
\label{tab:rappel}
\end{table}


\section{Exemples de structures ioniques simples}
\subsection{NaCl}
\subsection{CsCl}
\subsection{Carbone diamant}
\subsection{Zinc blende}

%\include{chapter/04-cristaux-ioniques}
%\include{chapter/05-energie-reticulaire} % en fait il s'agit des cristaux moléculaires
%\include{chapter/07-liaisons-atomiques}
\chapter{Résolution de structure}

Dans les solides, les distances typiques entre atomes sont de l'ordre de 
l'angstrom (\SI{e-10}{\metre}). Une sonde électromagnétique de la structure
microscopique d'un solide doit par conséquent avoir une longueur d'onde au
moins aussi petite, ce qui correspond à une énergie de l'ordre de :

\begin{equation}
\hbar \omega \sim \frac{hc}{\lambda} = \frac{\SI{e-34}{}\SI{e8}{}}{\SI{e-10}{}} = \SI{e-16}{\joule} = \SI{e3}{\electronvolt}
\end{equation}

Les énergies de cet ordre de grandeur sont caractéristiques des rayons X. Pour 
cette raison, on va étudier l'influence des rayons X sur les structures
cristallines, pour permettre de sonder la matière à l'échelle de l'atome.


\section{Diffusion des rayons X}

Les rayons X ont été découverts par Röntgen en 1895. Leur nature ondulatoire a
été mise en valeur en 1913 avec la réalisation des premières expériences de
diffraction suggérées par von Laue. Plus tard, Barkla a montré le caractère
transversal de ces ondes, établissant ainsi qu'il s'agissait d'ondes
électromagnétiques.

Le domaine de longueur d'onde des rayons X va de 0.1 (limite des rayons
$\gamma$) à \SI{100}{\angstrom} (limite de l'UV lointain). En terme d'énergies,
cela correspond à la gamme 0.1 à \SI{100}{\kilo\electronvolt}. En
cristallographie, on utilise généralement des rayons X dont la longueur d'onde 
varie entre 0.5 et \SI{2.5}{\angstrom}.

Dans les solides, ce sont les électrons, plus que les particules du noyau, qui
interagissent avec les radiations électromagnétique. L'amplitude de diffusion
$\epsilon(\mathbf{q})$ d'une densité de charge distribuée $\rho(\mathbf{r})$ peut
s'écrire :

\begin{equation}
    \epsilon(\mathbf{q}) = \frac{E_0 r_e}{R} \exp [i(\omega t - k R)] f(\mathbf{q})
\end{equation}
où $f$ est le facteurs de diffusion et représente l'intensité de la diffusion,
relativement à un électron libre. Le facteur de diffusion d'un atome est défini
par :
\begin{equation}
    f(\mathbf{q}) = \int e^{i\mathbf{q}\cdot\mathbf{r}} \rho(\mathbf{r}) dV
\end{equation}
où $\mathbf{q = k' - k}$ est le vecteur d'onde de l'espace réciproque entre
l'onde diffusée et l'onde incidente. Il s'agit simplement de la transformée de
Fourier de la densité de distribution électronique $\rho(\mathbf{r})$.

\subsection{Diffusion d'un groupe d'atomes}

\TODO f s'appelle le facteur de forme atomique. traduire mieux.


On peut considérer à présent la diffusion à partir d'un arrangement périodique
d'atomes. On montrera par là que cela mène à la diffraction, qui est, dans
certaines conditions, de la diffusion d'atomes qui s'ajoutent de façon cohérente
pour produire des pics indenses de la radiation diffusée.
Historiquement, la diffarction des rayons X par les cristaux a été la première
confiration de deux concepts importants en physique : le fait que les rayons X
peuvent être décrits comme des ondes, et que beaucoup de solides sont constitués
d'arrangements périodiques d'atomes. La diffraction des rayons X est encore
aujourd'hui très largement utilisée dans de très larges domaines en
physico-chimie, pour de l'identification de phase, de la détermination de
structure, des mesures de contraintes, pour la détermination de la concentration
en défauts, pour des transformations ordre-désordre et pour la détermination de
paramètres structuraux de super-réseaux. Récemment, la diffraction des rayons X
en incidence rasante (grazing incidence XRD) a été utilisée pour mesurer avec
précision les constantes de réseaux de couches monoatomiques.

À partir de ce point, il y a deux chemins équivalents que nous pouvons suivre
pour dédiuire la diffusion d'un arrangement d'atomes. Le premier serait de
représenter l'arangement des atomes par une distribution de densité électronique
$\rho_c(\mathbf{r})$ pour l'arrangement complet. Par conséquent,
$\rho_c(\mathbf{r})$ représente la densité totale électronique de tous les
électrons de tous les atomes du cristal. L'amplitude diffusée est simplement la
transformée de Fourier de cette distribution de densité électronique :
\begin{equation}
    \epsilon(\mathbf{q}) = \frac{E_0 r_e}{R} \exp [i(\omega t - k R)] \int e^{i \mathbf{q \cdot r}} \rho_c (\mathbf{r}) dV
    \label{fftdistrelec}
\end{equation}
Un moyen équivalent et totalement différent de trouver l'amplitude diffusée est
d'assigner chacun des électrons du solide à un atome. On calcule ensuite la
distribution de densité électronique pour chaque type d'atome. Par exemple, les
électrons assignés au p\ieme type d'atome seraient représentés par
$\rho_p(\mathbf{r})$ où $\mathbf{r}$ est la position relative au centre atomique.
On peut ainsi écrire la densité électronique du cristal comme :
\begin{equation}
    \rho_c(\mathbf{r}) = \sum_p \rho_p(\mathbf{r - R_p})
    \label{densiteelectroniquecristal}
\end{equation}

où la somme sur $p$ est effectuée sur tous les atomes du cristal. L'équation
\ref{densiteelectroniquecristal} place un atome $p$ en un site du cristal, au
bout du vecteur $\mathbf{R_p}$. Replacer cela dans l'équation \ref{fftdistrelec}
et échanger l'ondre de l'intégration et de la somme nous ramène à :

\begin{equation}
   \epsilon(\mathbf{q}) = \frac{E_0 r_e}{R} \exp [i(\omega t - k R)] 
   \sum_p \int e^{i \mathbf{q \cdot r}} \rho_p (\mathbf{r - R_p}) dV
    \label{fftdistrelecsum}
\end{equation}

En posant le changement de variables $\mathbf{r' = r - R_p}$, on se ramène alors
à :

\begin{equation}
    \epsilon(\mathbf{q}) = \frac{E_0 r_e}{R} \exp [i(\omega t - k R)] 
    \sum_p e^{i\mathbf{q \cdot R_p}} \int  \rho_p (\mathbf{r'}) dV'
    \label{fftdistrelecsumred}
\end{equation}

L'intégrale de l'équation \ref{fftdistrelecsumred} corresponsd au facteur de
diffusion $f$ de l'atome $p$ dans le solide. On obtient alors :


\begin{equation}
   \epsilon(\mathbf{q}) = \frac{E_0 r_e}{R} \exp [i(\omega t - k R)] \sum_p  f_p e^{i \mathbf{q \cdot R_p}}
    \label{fftdistrelecsumf}
\end{equation}

Dans cette équation, $f_p$ est le facteur de diffusion de l'atome $p$ ; $\mathbf{R_p}$ est la position relative à une position de référence dans le
cristal, et la somme sur $p$ est effectuée sur toutes les positions atomiques du 
cristal. Avec ce formalisme, on simplifie la tâche -- ardue -- de construire une
fonction de densité électronique pour le cristal, le calcul de $f_p$ pour chaque type d'atome jouant ce rôle de simplification.
Comme seuls les électrons périphériques entrent en jeu dans la cohésion d'un
cristal, $f$ ne dépend que faiblement de l'environnement dans lequel l'atome est
placé, et en pratique, les facteurs de diffusions des atomes libres sont
couremment utilisés \footnote{À l'exception des solides où les laisons ont un
    large caractère ionique, comme NaCl, où ce sont les facteurs de diffusion
des ions qui seront privilégiés}.

\subsection{Diffusion à partir d'un arrangement périodique d'atomes}

Pour les solides cristallins, on peut simplifier l'équation
\ref{fftdistrelecsumf} en écrivant les positions atomiques comme la somme des
positions $\mathbf{R}_m$ de la maille primitive dans laquelle l'atome est fixé
et d'une position $\mathbf{r}_n$ de l'atome dans la maille. Cela correspond à
écrire :
\begin{equation}
    \mathbf{R}_p = \mathbf{R}_m^n = \mathbf{R}_m + \mathbf{r}_n =
    \sum_{j=1}^3 m_j \mathbf{a}_j + \mathbf{r}_n
\end{equation}

La somme sur tous les atomes du solide se réduit ainsi à une somme sur tous
les atomes de la maille primitive et à la somme sur toutes les mailles primitives
du cristal :

\begin{eqnarray}
    \sum_p f_p e^{i\mathbf{q \cdot R_p}} & = & \sum_m^{N_c} \sum_n^{N_b}
    f_n e^{i\mathbf{q\cdot(R_m + r_n)}}\\
    & = & \left( \sum_m^{N_c} e^{i\mathbf{q\cdot R_m}} \right)
    \left( \sum_n^{N_b} f_n e^{i\mathbf{q\cdot r_n}} \right)
\end{eqnarray}

où $N_c$ est le nombre de mailles primitives dans le cristal et $N_b$ le nombre
d'atomes ans la maille primitive.

On définit le facteur de structure $F(\mathbf{q})$ comme la somme sur tous les
atomes de la maille :

\begin{equation}
    F(\mathbf{q}) = \sum_n^{N_b} f_n e^{i \mathbf{q \cdot r_n}}
\end{equation}

Ce facteur de structure contient toute l'information sur les positions atomiques
de la maille, et s'affranchit de toutes les complications liées à la distribution
élecrtonique des atomes, cachée dans les facteurs de diffusions.

En combinant toutes ces équations, on se ramène à l'amplitude diffractée par le
cristal :

\begin{equation}
    \epsilon = \frac{E_0 r_e}{R} \exp[i(\omega t - k R)] F(\mathbf{q}) \sum_m^{N_c} e^{i\mathbf{q\cdot R_m}}
    \label{ampdiffract}
\end{equation}

\section{Diffraction des rayons X}

\subsection{Réseau réciproque}

L'équation \ref{ampdiffract} donne le formalisme pour calculer l'amplitude de
diffusion élastique pour un arrangement périodique d'atomes. Concentrons-nous,
dans un premier temps, sur la somme sur une maille primitive:

\begin{equation}
    \sum_m^{N_c} e^{i\mathbf{q\cdot R}_m}
\end{equation}

On considère un cristal avec les vecterus de translation de réseau $\mathbf{a}_1$
, $\mathbf{a}_2$ et $\mathbf{a}_3$, de telle sorte que les positions atomiques
dans la maille primitive soient données par :
\begin{equation}
    \mathbf{R}_m = m_1 \mathbf{a}_1 + m_2 \mathbf{a}_2 + m_3 \mathbf{a}_3
\end{equation}
où $m_1$, $m_2$ et $m_3$ sont des entiers. Le produit scalaire 
$\mathbf{q\cdot R}_m$ devient alors :

\begin{equation}
    \mathbf{q\cdot R}_m = m_1 \mathbf{q \cdot a}_1 + m_2 \mathbf{q \cdot a}_2 + m_3 \mathbf{q \cdot a}_3
\end{equation}

En outre, si l'on considère un cristal qui est un parallélépipède avec $N_1$
mailles selon la direction $\mathbf{a}_1$, $N_2$ selon la direction $\mathbf{a}_2$
et $N_3$ selon $\mathbf{a}_3$, alors la somme sur toutes les mailles du cristal
devient :

\begin{eqnarray}
    \sum_m^{N_c} e^{i \mathbf{q\cdot R_m}} & = &
        \sum_{m_1 = 0}^{N_1 - 1} \sum_{m_2 = 0}^{N_2 - 1} \sum_{m_3 = 0}^{N_3 - 1}
        \exp [i(m_1 \mathbf{q \cdot a}_1 + m_2 \mathbf{q \cdot a}_2 + m_3 \mathbf{q \cdot a}_3)]\\
        & = & \left( \sum_{m_1 = 0}^{N_1 - 1} e^{i m_1 \mathbf{q\cdot a}_1} \right) 
        \left( \sum_{m_2 = 0}^{N_2-1} e^{i m_2 \mathbf{q\cdot a}_2} \right)
        \left( \sum_{m_3 = 0}^{N_3-1} e^{i m_3 \mathbf{q\cdot a}_3} \right)\\
        & = & \prod_{j=1}^3 \left( \sum_{m_j = 0}^{N_j -1} e^{i m_j \mathbf{q\cdot a}_j} \right)
    \end{eqnarray}

Il s'agit d'une série géométrique de raison $e^{i \mathbf{q\cdot a}_j}$, d'où :
\begin{eqnarray}
    \sum_{m_j = 0}^{N_j - 1} e^{i m_j \mathbf{q\cdot a}_j} & = &
    \frac{1 - e^{i N_j \mathbf{q\cdot a}_j}}{1 - e^{i \mathbf{q \cdot a}_j}} \\
    & = & e^{i \phi_j} \frac{\sin \left( N_j \frac{\mathbf{q\cdot a}_j}{2} \right) }{\sin \left( \frac{\mathbf{q\cdot a}_j}{2} \right)}
\end{eqnarray}
où le terme de phase $\phi_j$ est donné par : $\frac{\mathbf{q\cdot a}_j}{2} (N_j
-1)$. Cela donne la somme sur toutes les mailles du cristal :

\begin{equation}
    \sum_m^{N_c} e^{i\mathbf{q\cdot R}_m} = \prod_{j=1}^3 \left\{ e^{i\phi_j} \left[ \frac{\sin \left(N_j \frac{\mathbf{q\cdot a}_j}{2}
    \right)}{\sin \left(\frac{\mathbf{q\cdot a}_j}{2} \right)} \right] \right\}
    \label{sommecristal}
\end{equation}

Ce résultat peut être injecté dans l'équation \ref{ampdiffract}, pour trouver
l'amplitude diffusée pur un cristal. Lorsque l'on effectue de la diffraction, on
mesure l'intensité, qui est en fait le carré du module de l'amplitude
$\epsilon \cdot \epsilon^* = |\epsilon|^2$, multipliée par la constante
$c\epsilon_0$. En prenant le carré complexe, le terme de phase disparaît et l'on
obtient alors :

\begin{equation}
    I = c\epsilon_0 \left( \frac{E_0 r_e}{R} \right)^2 |F(\mathbf{q})|^2
    \prod_{j=1}^3 \left[ \frac{\sin \left(N_j \frac{\mathbf{q\cdot a}_j}{2} \right)}{\sin \left( \frac{\mathbf{q\cdot a}_j}{2}\right)} \right]^2
\end{equation}

L'intensité diffractée contient le facteur de diffraction de Lorentz:

\begin{equation}
    \left[ \frac{\sin \left(N_j \frac{\mathbf{q\cdot a}_j}{2} \right)}{\sin \left( \frac{\mathbf{q\cdot a}_j}{2}\right)} \right]^2
\end{equation}

\TODO: cf ouvrage diffraction.

Ce facteur de Lorentz présente des pics lorsque :
\begin{equation}
    \frac{\mathbf{q \cdot a}_j}{2} = n \pi
\end{equation}

où n est un entier. Pour observer un pic d'intensité diffractée, il faut donc
que cela soit vrai dans chacune des trois directions, c'est à dire pour
$j = 1,2,3$. Le vecteur de diffusion $\mathbf{q}_B$ qui satisfait cette condition
est défini par :

\begin{eqnarray}
    \frac{\mathbf{q}_B \cdot \mathbf{a}_1}{2} & = & h \pi \\
    \frac{\mathbf{q}_B \cdot \mathbf{a}_2}{2} & = & k \pi \\
    \frac{\mathbf{q}_B \cdot \mathbf{a}_3}{2} & = & l \pi
\end{eqnarray}
où $h,k,l$ sont des entiers ; ce sont les indices de Miller définis précedemment.
Cette condition est appelée condition de Laue. Cela revient à écrire :

\begin{eqnarray}
    \mathbf{G}_{hkl} \cdot \mathbf{a}_1 = h \\
    \mathbf{G}_{hkl} \cdot \mathbf{a}_2 = k \\
    \mathbf{G}_{hkl} \cdot \mathbf{a}_3 = l
\end{eqnarray}
De telle sorte que les conditions de Laue soient satisfaites si :

\begin{equation}
    \mathbf{q}_B = 2\pi \mathbf{G}_{hkl}
\end{equation}

Cette équation représente le fait que pour un pic du spectre de diffraction d'un
cristal, le vecteur de diffusion est $2 \pi$ fois le vecteur du réseau réciproque
. Cela est la relation la plus importante à retenir dans ce chapitre. On peut
immédiatement voir que pour un pic, le vecteur de diffusion est perpendiculaire
aux plans de diffraction. En comparant les amplitudes des vecteurs, on peut aussi
remarquer que :

\begin{equation}
    q_B = \frac{4\pi}{\lambda} \sin \theta_B = 2\pi |\mathbf{G}_{hkl} | = \frac{2\pi}{d_{hkl}}
\end{equation}

Ce qui nous amène à :
\begin{equation}
    \lambda = 2 d_{hkl} \sin \theta_B
\end{equation}

et
\begin{equation}
    q_B = \frac{2\pi}{d_{hkl}}
\end{equation}
Ces deux dernières équations sont des formulations équivalentes de ce qui est
connu sous le nom de loi de Bragg.

\TODO: on dit von laue, pas laue.

\subsection{Loi de Bragg}

La diffraction est un concept très très important en physique du solide et en
science des matériaux, non seulement poru les techniques d'analyses qui en sont
rendues possibles, mais aussi pour la théorie des bandes d'énergie dans les
cristaux. Le phénomène de diffraction apparaît dans de nombreuses situations dans
les solides cristallins, et les chercheurs de différents domaines ont pu
construire différents moyens de représenter la diffraction. Nous nous intéressons
nous à certaines représentations graphiques de différents aspects de la
condition de la diffraction.

Nous avons vu précedemment qu'un moyen d'établir la condition de diffraction pour
trouver un maximum de difraction d'un ensemble de plans espacés d'une distance
$d$, illuminés par des rayons X de longueur d'onde $\lambda$ est :
\begin{equation}
    n \lambda = 2d \sin \theta_B
\end{equation}

Cette expression a été formulée pour la première fois par Bragg et est connue
sous le nom de loi de Bragg. Cette relation peut être déduite en considérent la
différence de phase entre les rayons X diffusés et les plans adjacents. Comme
présenté sur la figure \TODO, la différence de chemin entre les rayons X diffusés
à partir d'un plan et ceux qui sont diffusés par le plan prédédent est 
$2d sin \theta$, ce qui nous ramène à une différence de phase de
$\frac{2\pi}{\lambda} 2d \sin \theta$. Si cette différence de phase est égale à
un entier $n$ fois $2\pi$, alors les ondes diffusées à partir de plans succesifs
seront en phase et interféreront de façon constructive.

En appliquant cette condition, on voit :
\begin{equation}
    n 2 \pi = \frac{2 \pi}{\lambda} 2d\sin \theta_B \rightarrow
    n\lambda = 2d \sin \theta_B
\end{equation}

\TODO: construction de Bragg

Cette construction peut être utilisée pour visualiser l'expérience de diffraction
commune, où l'angle $\theta$ peut être varié en tournant le cristal sous un
faisceau de rayons X monochromatiques. Un détecteur est tourné à une fréquence
angulaire deux fois plus importante, suivant le meme axe, de sorte à maintenir la
symmétrie entre les rayons X incidents et diffusés, relativement aux plans
cristallins. En même temps que l'angle $\theta$ varie, dès que la condition de
Bragg est atteinte, un pic d'intensité apparaît. Cela est montré schématiquement
sur la figure \TODO.

La géométrie symmétrique entre $\mathbf{k}$ et $\mathbf{k'}$ relativement aux
plans de diffraction, maintient la condition que le vecteur de diffusion est
perpendiculaire aux plans, à la condition de Bragg. Dans de nombreux cas, la
géométrie symmétrique est également maintenue, relativement à la surface de
l'échantillon, de telle sorte à ce que les plans étudiés soient parallèles à la
surface de l'échantillon. Cependant, ce n'est pas toujours le cas ; nous verrons
plusieurs géométries de diffraction plus tard.

\subsection{Sphère d'Ewald}

La loi de Bragg est un postulat de la condition de la diffraction correct, mais
incomplet, du fait qu'il ne containet qu'une information scalaire, et ne
représente pas les aspects plus généraux, directionnels ou vectoriels, que nous
avons vu comme :
\begin{equation}
    (\mathbf{k'-k})_B = \mathbf{q}_B = 2\pi \mathbf{G}_{hkl}
\end{equation}

où $\mathbf{G}_{hk}$ est un vecteur de l'espace réciproque.

Un moyen facile de voir cette relation est sa représentation dans l'espace
réciproque, aussi connue sous le nom de sphère d'Ewald. On construit d'abord le
réseau réciproque pour le cristal qui nous intéresse. Ensuite, on place
l'extrémité du vecteur $\frac{\mathbf{k}}{2\pi}$ sur un site du réseau réciproque
. Ensuite, une sphère de rayon $\frac{k}{2\pi} = \frac{1}{\lambda}$ est tracé,
ave son centre à l'origine de $\frac{\mathbf{k}}{2\pi}$
\footnote{On peut noter que l'origine du vecteur $\frac{\mathbf{k}}{2\pi}$ n'est
pas nécessairement sur un nœud du réseau réciproque.}
Comme deux nœuds du réseau réciproque peuvent être connectés entre eux par un
vecteur du réseau réciproque $\mathbf{G}_{hkl}$, tout nœud du réseau réciproque
qui apparaît sur cette pshère (autre que celui tracé au début, qui termine à
$\frac{\mathbf{k}}{2\pi}$) sera à l'extrémité d'un vecteur
$\frac{\mathbf{k'}}{2\pi}$, qui satisfait la condition de la diffraction :

\begin{equation}
    \frac{1}{2\pi} (\mathbf{k'-k})_B = \mathbf{G}_{hkl} \rightarrow
    (\mathbf{k'-k})_B = \mathbf{q}_B = 2 \pi \mathbf{G}_{hkl}
\end{equation}

Ce qui peut être illustré sur la figure \TODO.

Sur les représentations dans l'espace réciproque, les expériences de diffraction 
sont des observations de l'intensité diffractée en fonction de l'orientation
ou de la longueur du vecteur de diffusion $\mathbf{q}$. Si l'expérience est faite
à une énergie constante (faisceau monochromatique), alors le diamètre de la
sphère d'Ewald est onstant, et son orientation est changée pendant l'expérience,
ce qui apporte un autre set de points du réseau réciporque en contact avec la
sphère d'Ewald.

La technique de la diffraction des électrons dans un microscope électronique à
transmission (TEM) utilise la nature ondulatoir edes électrons pour faire de la
diffraction à partir de cristaux. La longueur d'onde des électrons à hautes
énergies (\SI{100}{\kilo\volt} à \SI{1}{\mega\electronvolt}) utilisés est bien
plus courte que celle des rayons X typiques, ramenant la sphère d'Ewald à un
rayon très large, de telle sorte à ce que sa surface soit quasi planne.
En outre, l'échantillon cristallin est fin (nécessaire pour assurer la
transparence électronique), ce qui résulte en une élongation de la région de
diffraction dans la direction parallèle à $\mathbf{k}$.
Par conséquent, même s'il y a quelques courbures de la sphère d'Ewald, les nœuds
du réseau réciproque intersectent toujours la sphère d'Ewald. Cela signifie que
l'alignement d'une zone de l'axe avec le vecteur d'onde de l'électron
incident donnera des spots de diffraction pour quasiment tous les plans dans la
zone. Cela est présenté schématiquement sur la figure \TODO.



\section{Zones de Brillouin et condition de diffraction}

Les zones de Brillouin forment l'énoncé de la condition de diffraction la plus
utilisée en physique du solide : elle permet de retrouver la théorie des bandes
d'énergie électronique.

Nous avons vu précédement que la 1\iere zone de Brillouin comme la cellule de
Wigner-Seitz dans le réseau réciproque. Cela permet de former une interprétation
géométrique de la condition de la diffraction :
\begin{equation}
    2\mathbf{k} \cdot \mathbf{G} = G^2
\end{equation}
Que l'on peut aussi écrire :
\begin{equation}
    \mathbf{k} \cdot \left( \frac{1}{2} \mathbf{G} \right) = \left( \frac{1}{2} G \right)^2
\end{equation}

En travaillant dans l'espace réciproque, $\mathbf{k},\mathbf{G})$, on choisit un
vecteur $\mathbf{G}$ à l'origine d'un point du réseau réciproque. Le plan normal
à ce vecteur $\mathbf{G}$, au milieu du vecteur. Ce plan forme une partie de la
frontière de la zone de Brillouin.

Si une onde, de vecteur d'onde $\mathbf{k}$, est diffractée sur le réseau, le
faisceau diffracté aura la direction $\mathbf{k}-\mathbf{G}$, soit $\Delta\mathbf{k} = \Delta\mathbf{k} = -\mathbf{G}$.
La construction de Brillouin présente tous les vecteurs d'onde $\mathbf{k}$
réfléchis par le cristal en suivant la loi de Bragg.

Les plans qui sont les bissectrices des vecteurs du réseau réciproque sont d'une
grande importance en ce qui'il s'agit de la propagation d'ondes dans les
cristaux : une onde dont le vecteur d'onde tracé de l'origine termine sur un 
de ces plans respectera la condition de diffraction. Ces plans divisent l'espace
de Fourier du cristal en fragments. Le fragment central est la cellule de
Wigner-Seitz du réseau réciproque.

On peut former une définition plus précise des zones de Brillouin : \emph{
    La première zone de Brillouin est le plus petit volume entièrement entouré
    par les plans qui sont les bissectrices des vecteurs du réseau réciproque
tracés depuis l'origine.}




L'étude de réseau dans l'espace réel (K), généralement il est pratique et
intéressant de considérer un polyhèdre connu sous le nom de cellule de 
Wigner-Seitz, décrite précedement.

L'analogie entre la construction dans l'espace réciproque résulte en ce qui est
connu en tant que zore de Brillouin. Rappelons que la première zone de Brillouin 
est la cellule de Wigner-Seitz du réseau réciproque.

Généralement ,la construction de la zone de Brillouin est utiisée poru décrire
les électrons dans un solide périodique ; les plans bisecteurs et les
zones ont une signification pour la diffraction. Cela peut être remarqué en 
(ré)écrivant les conditions e la diffraction :
\begin{equation}
    \mathbf{q}_B = (\mathbf{k'-k})_B = 2\pi\mathbf{G} \rightarrow \mathbf{k} + 2\pi \mathbf{G} = \mathbf{k'}
\end{equation}

Lorsque $k=k'$, on peut écrire :
\begin{equation}
    (\mathbf{k} + 2\pi \mathbf{G})^2 = k^2
\end{equation}

mais :
\begin{equation}
    2 \mathbf{k\cdot G} = 2\pi G^2
\end{equation}

Soit le résultat :
\begin{equation}
    \mathbf{k}\cdot\frac{\mathbf{G}}{G} = 2\pi \left( \frac{G}{2} \right)
\end{equation}

où on a utiisé le fait que $-\mathbf{G}$ esta ussi un vecteur du réseau
réciproque.

Cette dernière équation pose le fait que l acondition de la diffraction est
satisfaite lorsque la composante de $\mathbf{k}$ selon $\mathbf{G}$ est égale à
$2\pi$ fois la demin longueur de $\mathbf{G}$. En fait, cette condition est
vérifiée pour tous les vecteurs $\frac{\mathbf{k}}{2\pi}$ qui ont leur origine
en un nœud du réseau réciproque, et terminent sur les plans bissecteurs
perpendiculaires au vecteur entre l'origine et un autre nœud du réseau réciproque. \TODO figure.


 
\subsection{Conclusions sur les conditions de la diffraction}

Les conditions de Laue, la realtion de Bragg et la construction d'Ewald sont des
représentations équivalentes du même phénomène : les directions de diffraction 
d'un réseau sont déterminées par son réseau réciproque.

La nature du motif influe uniquement sur l'intensité diffractée et pas sur les
directions de diffraction. La mesure des angles de diffraction des rayons X par
un cristal donne seulement des informations sur le réseau translatoire du cristal
. Pour obtenir la position des atomes dans la maille, il faut aussi utiliser les
intensités des figures de diffraction. Suivant la nature du problème étudié et
les techniques de diffraction employées, on utilisera poru déterminer les
directions de diffraction, l'une de ces trois méthodes.

 
\section{Méthodes expérimentales}
\subsection{Méthode de Laue}
monocristal fixe, rayons X polychromatiques
\subsection{Méthode du cristal mobile}
monocristal mobile, rayons X monochromatiques
\subsection{Méthode des poudres}
polycristaux, rayons X monochromatiques

\section{Exemples}

\subsection{Facteur de structure}

Aussi importante qu'elle puisse être, la somme de l'équation \ref{sommecristal}
ne donne pas une idée générale. Le facteur de structure $F(\mathbf{q})$ donne une
échelle de l'amplitude diffractée, et l'intensité diffractée est contrôlée
par le carré du facteur de structure. Comme le facteur de structure est une
fonction variant lentement par rapport au vecteur de diffusion, il est commun de
considérer sa valeur à la condition de Bragg exacte. On définit :

\begin{equation}
    F_{hkl} = F(\mathbf{q}_B) = \sum_n^{N_B} f_n e^{i2\pi \mathbf{G}_{hkl}\cdot\mathbf{r}_n}
\end{equation}

Ce facteur de structure reflète l'information de la distribution électronique
dans la maille. On verra q'il implique d'éliminer certains pics de diffraction
dus aux interférences entre les radiations diffusée des différents atomes dans
la maille.
Regardons plus en détail certaines structures caractéristiques.

\subsection{Structure cubique simple}

Dans un réseau cubique simple,

\begin{eqnarray}
    \mathbf{a}_1 & = & a \mathbf{\hat x} \\
    \mathbf{a}_2 & = & a \mathbf{\hat y} \\
    \mathbf{a}_3 & = & a \mathbf{\hat z}
\end{eqnarray}

Les vecteurs du réseau réciproque sont :

\begin{eqnarray}
    \mathbf{b}_1 & = & \frac{1}{a} \mathbf{\hat x} \\
    \mathbf{b}_2 & = & \frac{1}{a} \mathbf{\hat y} \\
    \mathbf{b}_3 & = & \frac{1}{a} \mathbf{\hat z}
\end{eqnarray}

Il y a un atome par maille primitive et il est situé sur la coordonnée (0,0,0).
Par conséquent, on trouve :

\begin{equation}
    F_{hkl} = \sum_n f(2\pi \mathbf{G}_{hkl}) e^{i 2\pi \mathbf{G}_{hkl}\cdot\mathbf{r}_n } = f(2\pi\mathbf{G}_{hkl})
\end{equation}
L'intensité diffusée présentera des pcis à chaque vecteur du réseau réciproque,
et ils seront pondérés par $|f(\mathbf{q})|^2$ évalué en $\mathbf{q = q}_B$. On a
vu préceddement que $f(\mathbf{q})$ est une fonction monotone décroissante,
de telle sorte à ce que les pics correspondant à de plus grands vecteur du
réseau réciproque, qui provionnent de plans espacés d'une distance $d$ plus
courte, auront une intensité décroissante.
\subsection{Structure cubique centré}

À des fins illustratives, calculons le facteur de structure en utilisant la
maille conventionnelle (et non pas la maille primitive). On a les mêmes vecteurs
du réseau direct et du réseau réciproque que dans le système cubique simple, mais
maintenant il y a un atome positionné à $(0,0,0)$ et un autre en
$\frac{a}{2}(1,1,1)$. On trouve par conséquent :

\begin{eqnarray}
    F_{hkl} & = & \sum_n f(2\pi\mathbf{G}_{hkl}) e^{i 2\pi \mathbf{G}_{hkl}\cdot\mathbf{r}_n} \\
    & = & f(2\pi \mathbf{G}_{hkl}) [1 + e^{i\pi(h+k+l)}] \\
    & = & f(2\pi \mathbf{G}_{hkl}) \times \begin{cases} 2 \text{ si } h+k+l \text{ est pair}\\ 0 \text{ si } h+k+l \text{ est impair}\end{cases}
\end{eqnarray}

Par conséquent, pour une structure cubique centrée dans l'espace réel, le réseau 
réciproque du cubique simple a une intensité nulle pour les points pour lesquels
$h + k + l$ est impair. Les pics avec un facteur de structure nul sont dits
interdits, alors que ceux qui ont un facteur de structure autorisé sont dits
autorisés (ou permis). Par exemple, les quatre premiers pis permis sont :
$(110)$,$(200)$,$(211)$ et $(220)$. Les indices $(100)$,$(111)$ et $(211)$ ont un
facteur de structure nul, qui correspond à des pics interdits. Si l'on consiidère
les pics autorisés pour un réseau cubique centré, et qu'on associe chacun d'entre
eux aux vecteurs du réseau réciproque corrrespondant, les points constitués par
les terminaisons de ces veucteurs produisent un réseau cubique faces-centrées,
avec des côtés de longueur $\frac{2}{a}$. Par conséquent, un réseau BCC dans
l'espace réel est considéré comme ayant une réseau réciproque FCC. 

\subsection{Cubique face centrée}

Ici, nous utilisons encore une fois la maille conventionnelle, mais cette fois-ci
nous avons un atome principal aux positions $(0,0,0)$,$\frac{a}{2}(1,1,0)$,
$\frac{a}{2}(1,0,1)$ et $\frac{a}{2}(0,1,1)$. On trouve, par conséquent un
facteur de structure:

\begin{eqnarray}
    F_{hkl} & = & f(2\pi \mathbf{G}_{hkl}) [ 1 + e^{i\pi(h+l)} + e^{i\pi(k+l)} + e^{i\pi(h+k)}]\\
    & = & f(2\pi\mathbf{G}_{hkl}) \times \begin{cases} 4 \text{ si } h,k,l \text{ sont de même parités}\\
    0\text{ sinon.} \end{cases}
\end{eqnarray}

Par conséquent, pour un réseau FCC dans l'espace réel, le réseau cubique simple a
des points manquants lorsque les indices $(hkl)$ ne sont pas de même parité. Les
quatre premiers pics autorisés sont $(111)$, $(200)$, $(220)$ et $(311)$. Les
pics correspondants aux plans $(100)$, $(110)$, $(210)$ et $(310)$ sont interdits
. Si encore une fois, on associe chaque pic avec un vecteur du réseau réciproque,
on produit un réseau FCC dans le réseau réciproque, avec un côté de longueur
$\frac{2}{a}$. Par conséquent, un réseau FCC dans l'espace réel donne un réseau
BCC dans l'espace réciproque.



\subsection{Structure diamant}
\subsection{NaCl}

\chapter{Méthodes expérimentales}

référence : DB cullity, annexe 1., chap 6, 7 , 8, 9

Dans la section précédente, on a vu que l'intensité diffractée agrémente les nœuds du réseau réciproque. Cette propriété simple est suffisante pour comprendre la plupart des phénomènes de diffraction.
Nous n'avons cependant pas encore développé suffisemment de formalisme pour prédire correctement les intensités observées, mais nous pouvons prédire les positions des pics. Explorons maintenant différentes méthodes de diffraction.

Un cristal simple dans un faisceau de rayons X monochromatique sera rarement orienté de telle sorte à ce que la sphère d'Ewald intersecte les nœuds du réseau réciproque. En conséquence, les maxima de diffraction seront difficilement observable. Différentes méthodes expérimentales sont décrites ici pour produire la condition de diffraction. Chacune a ses avantages et ses inconvénients.

\section{Méthode du cristal mobile}
monocristal mobile, rayons X monochromatiques

\begin{marginfigure}
    \includegraphics{./images/part1/cullity118}
    \caption{Schéma de principe de la méthode du cristal mobile}
    \label{fig:cristalmobile}
\end{marginfigure}

Dans un diffractomètre à rayons X, un faisceau de rayons X monochromatiques est dirigé sur un échantillon, monté sur un goniomètre qui permet des rotations suivant plusieurs axes (figure \ref{fig:cristalmobile}. Le détecteur est monté sur un bras rotatif du goniomètre. Cet arrangement permet de contrôler la norme et l'orientation du vecteur de diffusion $\v{q}$. L'expérience consiste en l'observation de l'intensité comme fonction de $\v{q}$ (norme, orientation, ou les deux). Certains des chemins possibles dans le réseau réciproque sont présentés sur la figure \ref{fig:cristalmobilerecip}. Les diffractomètres modernes, avec contrôle numérique, permettent de contrôler des chemins très généraux le long de l'espace réciproque.
Un réseau réciproque peut être sélectionné, et l'intensité diffractée peut être observée comme fonction de la position dans le réseau réciproque. C'est souvent très pratique de balayer en $\v{q}$ comme une fonction des variables $h,k,l$ :

\begin{equation}
    \v{q} = 2\pi (h\v{b}_1 + k\v{b}_2 + l\v{b}_3)
\end{equation}

\begin{figure}
    \includegraphics{./images/part1/cullity119-01}
    \caption{Schéma du réseau réciproque et des paramètres étudiés par la méthode du cristal mobile}
    \label{fig:cristalmobilerecip}
\end{figure}

Ainsi, par exemple, un balayage en $h$ peut diriger le vecteur de diffusion le long d'un chemin gardant constant $k$ et $l$. Ce type de balayage sera fait dans la direction $\v{b}_1$. En observant l'intensité, la forme et la position de ces pics, l'information sur la structure atomique, la densité en défauts et la contrainte peut être obtenue.


\TODO : la description correspond pas avec les figures.
\begin{figure}
    \includegraphics{./images/part1/cullity119-02}
    \caption{Tracé de contours logarithmiques de l'intersité dans le plan $(h,k,0)$ de l'espace réciproque pour des 10 mailles cubiques alignés selon la direction $x$, et 8 mailles selon la direction $y$, avec un facteur de diffusion atomique $f$ égal à 1. (a) : réseau cubique simple : intensités pour des valeurs entières de $h$ et $k$. (b) réseau cubique faces-centrées : les valeurs pour $h$ et $k$ impaires ne produisent pas de pic de diffraciton.}
    \label{fig:cristalmobilecubic}
\end{figure}

\section{Méthode de von Laue}
monocristal fixe, rayons X polychromatiques

La méthode de von Laue est effectuée sur un échantillon monocristallin, en utilisant un faisceau de rayons X collimaté, à large spectre. Celui-ci est générallement produit par un tube à rayons X classique. Un film photographique est placé soit avant, soit après l'échantillon, suivant que la géométrie soit en réflexion ou en transmission. Les points de diffraction s'arrengent en ensembles selon des courbes elliptiques ou hyperboliques (figure \ref{fig:laue}). Ces courbes résultent de l'intersection d'un cône avec le film plan. Les plans cristallographiques \emph{en zone} produiront des taches de diffraction qui formeront un cône. L'axe du cône correspond à l'\emph{axe de zone}.

\begin{figure}
    \includegraphics{./images/part1/cullity115}
    \caption{Schémas de cônes de diffractions formant des ellipses en transmission et des hyperboles en réflexion}
    \label{fig:laue}
\end{figure}

On peut résumer cela ainsi : les plans d'une même zone sont diffractés selon un cône. Cela est facilement visible dans l'espace réciproque. Pour cela, changons d'abord légèrement notre interprétation de l'espace réciproque, pour prendre en compte l'étalement en longueur d'ondes du réseau incident. Le réseau réciproque que nous étudions prend maintenant en compte l'étalement en longueur d'onde, qui est compris entr edeux longueur d'onde $\lambda_1$ et $\lambda_2$, où la plus petite des deux correspond à la longueur d'onde de coupure de la source à rayons X, et la plus grande est moins bien définie mais est généralement prise pour correspondre à la longueur d'absorption limite $K$ de l'argent( \SI{0.48}{\angstrom}).
La condition de diffraction :

\begin{equation}
    \v{q}_B = (\v{k'-k})_B = 2 \pi \v{G}_{hkl}
\end{equation}

peut s'écrire comme :

\begin{equation}
    \label{eq:lauerecip}
    \uv{s}' - \uv{s} = \lambda \v{G}_{hkl}
\end{equation}

avec
\begin{equation}
    \uv{s} = \frac{\v{k}}{k} = \frac{\lambda}{2\pi}\v{k}\quad\text{et}\quad
    \uv{s'} = \frac{\v{k'}}{k} = \frac{\lambda}{2\pi}\v{k'}
\end{equation}
ce sont les vecteurs unitaires correspondants aux directions de $\v{k}$ et $\v{k'}$.

Construisont maintenant l'espace réciproque basé sur l'équation \ref{eq:lauerecip}, o\ la distance entre chaque point de l'espace réciproque dépend de la longueur d'onde $\lambda$. Par conséquent, chaque point de l'espace réciproque est étalé sur une ligne connectant les points $\lambda_1\v{G}_{khl}$ et $\lambda_2\v{G}_{hkl}$. La longueur de cette ligne dans l'espace réciproque augmente avec la distance depuis l'origine. Cela est montré schématiquement sur la figure \ref{fig:lauerecip}. Si maintenant nous traçons une sphère de rayon unitaire, centrée sur le centre du vecteur $\uv{s}$, qui termine à l'origine du réseau réciproque, la condition de diffraction de l'équation \ref{eq:lauerecip} sera atetinte pour toute ligne du réseau réciproque qui s'intersecte avec cette sphère.

\begin{figure}
    \TODO
    \caption{Schéma de l'espace réciproque construit pour la méthode de von Laue. Les points de l'espace réciproque sont représentés come des lignes pour prendre en compte l'étalement en longueur d'onde du faisceau incidents. La condition de diffraction est satisfaite si la sphère de réflection intersecte ces lignes, comme montré dans plusieurs cas ici. Le vecteur $\uv{s'}$ et l'angle de diffusion $2\theta$ sont représentés pour les réflexions (310) et (220)}
    \label{fig:lauerecip}
\end{figure}

Tous les plans d'une même zone sont représentés par des lignes sur un plan perpendiculaire à l'axe de zone. Ce plan intersecte la sphère de réflexion en un cercle, et le vecteur $\uv{s'}$ pour tous les faisceaux incidents qui terminent sur ce cercle. Il lformera un cône correspondant à l'axe de zone.

Dans l'exemple de la figure \ref{fig:lauerecip}, le centre de la pshère de réflexion, qui est l'origine du vecteur $\uv{s}$, est placé dans le plan de zone, de telle sorte à ce que tous les vecteurs $\uv{s'}$ d'une même zone (la zone $[001]$) sont coplanaires avec le plan de zone. En outre, le cône est applati en un cercle, et sa trace sur le film sera une ligne. Cependant, si l'origine de la sphère de réflexion n'est pas dans le plan, alors les vecteurs $\uv{s}$ et $\uv{s'}$  forment un cône, comme montré sur la figure \ref{fig:lauecone}.

\begin{figure}
    \TODO
    \caption{Schéma des plans en zone formant un cône. Le plan de zone intersecte la sphère de réflexion en un cercle. Le vecteur direction de diffraciton $\uv{s'}$ est situé sur ce cercle et forme un cône dont l'axe correspond à l'axe de zone.}
    \label{fig:lauecone}
\end{figure}

La direction du vecteur $\uv{s}$ est dans le cône de diffraction, et l'axe de zone est l'axe du cône. Lorsque l'angle entre l'axe de zone et $\uv{s}$ est inférieur à \SI{45}{\degree}, alors le cône intersecte le film avec un motif elliptique. Cela n'est posisble qu'en géométrie en transmission. Lorsque cet angle est supérieur à \SI{45}{\degree}, alors le cône intersecte le film avec un motif hyperbolique. Cela peut se produire soit en transmission, soit en réflexion. La symétrie du motif produit par la méthode de von Laue comporte des similitudes avec la symétrie du cristal, et la détermination des zones responsables d'ue certains motifs de diffraction est possible. Cette technique est largement utilisée pour orienter les échantillons en méthode du cristal mobile pour l'étude de certaines surfaces caractéristiques.


\section{Méthode des poudres}
polycristaux, rayons X monochromatiques

La méthode des poudres est la technique d'analyse structurelle la plus utilisée. Un échantillon réduit sous forme d'une fine poudre est exposé à un faisceau collimaté de rayons X monochromatiques. Idéallement, la poudre est faite de cristaux qui ont une orientation aléatoire en fonction de $\v{k}$, la direction du faisceau incident. Cela est équivalent à la rotation d'un monocristal selon toutes les orientations possibles, en même temps. Cela signifie que haque vecteur du réseau réciproque peut prendre toutes les orientations possibles, en balayant une sphère de rayon $G_{hkl} = \frac{1}{d_{hkl}}$, centrée à l'origine du réseau réciproque. L'espace réciproque constiste alors en un ensemble de sphères concentriques de rayons $G_{hkl} = 1/ d_{hkl}$. L'intersection d'un de ces sphères du réseau réciproque avec la sphère d'Ewald forme un cercle. La condition de diffraction est alors satisfaite pour un faisceau diffracté $\v{k'}/2\pi$ dont l'origine est à $\v{k}/2\pi$ et termine à son intersection. Ce vecteur d'onde balaye un cône, comme présenté sur la figure \ref{fig:poudres}.

\begin{marginfigure}
    \TODO
    \caption{Schéma de la condition de diffraction pour un échantillon par la méthode des poudres, résultant en un cône de diffraction dans le réseau réciproque (a) et direct (b)}
    \label{fig:poudres}
\end{marginfigure}

Les diffractomètres pour la méthode des poudres utilisent férquemment la symétrie de réflexion, dans laquelle l'angle d'incidence et l'angle réfléchi sont égaux. Cela est accompagné par une rotation du détecteur de deux fois l'angle de l'échantillon, de telle sorte à ce que l'angle d'incidence et l'angle réfléchi soient toujours égaux. Le vecteur de diffusion $\v{q}$ est maintenu perpendiculaire à l'échantillon, et le balayage consiste à suivre l'intensité au fur et à mesure que $\v{q}$ varie. Dans l'espace réciproque, on peut voir que les pics apparaissent au fur et à mesure que $\v{q}$ balaye les sphères concentriques successives de l'espace réciproque.
Générallement, les diffractomètres à rayons X utilisent une focale pour augmenter l'intensité diffractée.

Un exemple de motif de diffraciton, où l'intensité diffractée est tracée en fonction de l'angle de diffusion $2\theta$, est représenté sur la figure \ref{fig:poudrepic}. Généralement, une série de pics sera observée, et on donne généralement la liste des intensités et des positions angulaires. Comme la distance inter-réticulaire est donnée par :
\begin{equation}
    d = \frac{\lambda}{2\sin\theta}
\end{equation}
 alors la position des pics peut être convertie en distance inter-réticulaire, ce qui donne une liste d-I. Cette liste peut être utilisée pour l'analise de phase quantitative ou qualitative.

\begin{marginfigure}
    \includegraphics{./images/part1/nacl}
    \caption{Motif de diffraction pour une poudre de NaCl.}
    \label{fig:poudrepic}
\end{marginfigure}


\subsection{Facteur d'intensité par la méthode des poudres}

POur mieux comprendre la diffraction expérimentale par la méthode des poudres, il faut regarder aux facteurs qui affectent l'intensité pour un pic de diffraction donné, pour une phase cristalline donnée. Nous avons déjà étudié certains de ces facteurs en étudiants des cristaux, et trouvé :

\begin{equation}
    I_{hkl} = I_0 \ 
        \left( \frac{r_e}{R} \right)^2 \ 
        |F|^2 \ 
        \prod_{j=1}^3 \left(
            \frac{ \sin \frac{N_j \v{q \cdot a}_j}{2} }{ \sin \frac{\v{q\cdot a}_j}{2} } \right)^2
    \label{eq:facteurintensite}
\end{equation}

où $\v{q}$ est le vecteur de diffusion, $\v{a}_j$ le vecteur du réseau direct, $N_j$ le nombre de maille dans la direction $j$, $r_e$ le rayon d'un électron classique, $R$ la distance échantillon-détecteur et $I_0 = c\epsilon_0 E_0^2/2$ est l'intensité incidente, où $E_0$ est l'intensité du champ électrique incident, $c$ la célérité de la lumière, $\epsilon_0$ la permittivité du vide. Cette expression néglige cependant des facteurs importants. Discutons de ces facteurs, parfois d'une façon qualitative, pour déduire une expression de l'intensité diffractée par la méthode des poudres.

\paragraph{Facteur de structure}

Le facteur de structure $F$ est donné par la somme sur les atomes de la maille primitive :
\begin{equation}
    F = \sum_n^{N_b} f_n e^{i\v{q\cdot r}_n}
\end{equation}

où $f_n$ et $\v{r}_n$ sont respectivement les facteurs de diffusion atomique et la position dans la maille du n\ieme atome. Si nous écrivons :
\begin{equation}
    \v{r}_n = x'_n \v{a}_1 + y'_n \v{a}_2 + z'_n \v{a}_3
\end{equation}

où $x'_n, y'_n, z'_n$ sont les coordonnées du n\ieme atome dans la maille (nombre rationnels), alors le facteur de structure à la condition de Bragg ($\v{q} = 2\pi \v{G}$) devient :

\begin{equation}
    F_{hkl} = \sum_n^{N_b} f_n e^{2\pi i (h x'_n + k y'_n + l z'_n)}
\end{equation}

\paragraph{Facteur de polarization}

En écrivant l'équation \ref{eq:facteurintensite}, on a considéré la diffusion d'une onde polrisée perpendiculairement à la fois au vecteur d'onde incidente $\v{k}$ et diffusé $\v{k}'$. Cependant, en règle générale, la polarisation peut avoir un angle arbitraire par rapport à ce plan. Ce paragraphe a pour objet de montrer que les composantes dans le plan, du champ incident $E_y$ produisent l'accélération des électrons dans le solide, qui se traduit en des composantes le long de la direction $x'$ du champ diffusé. Ces composantes ne peuvent pas contribuer à l'intensité diffusée, parce qu'elles produiraient un champ le long de sa direction de propagation.
Rappelons que la champ diffusé d'un élecrton simple s'écrit :

\begin{equation}
    \epsilon = \frac{E_0 r_e}{R} \cos \alpha
\end{equation}

où $\alpha$ est le complémentaire de l'angle entre la direction du champ incident et la direction de propagation du champ diffusé. Ainsi, l'expression de l'intensité implique que l'onde nicidente est polarisée selon la direction $z$, de telle sorte que $\alpha = 0$. Dans le cas général, il faut pourtant considérer la composante selon $y$ du champ incident, pour laquelle $\alpha = 2 \theta$. Dès lors, il faut remplacer $E_0^2$ dans l'équation \ref{eq:facteurintensite} par :

\begin{equation}
    E_{0z}^2 + E_{0y}^2 \cos^2 2\theta
    \label{eq:champincid}
\end{equation}

Dans le cas particulier où la polarisation est aléatoire, la moyenne temporelle des composantes du champ incident est la même pour chacune des composantes. Ainsi :

\begin{equation}
    <E>^2 = <E_y>^2 + <E_z>^2
\end{equation}
On peut voir que :

\begin{equation}
    <Ey>^2 = <E_z>^2 = \frac{<E>^2}{2}
\end{equation}

En plaçant cela dans l'expression \label{eq:champincid}, on se retrouve avec unecontribution de la polarisation sur l'intensité, comme un facteur de :

\begin{equation}
    \left( \frac{1+\cos^2 2\theta}{2} \right)
\end{equation}

Cette expression est le facteur de polarisation, et est valable pour une polarisation du faisceau incident aléatoire.

\paragraph{Facteur de Lorentz}

Le facteur de Lorentz est le produit de facteurs qui tient compte de différents effets. Décrivons cela en détail en dessous.

\paragraph{Intégration des pics}

L'intensité de pics est une quantité difficile à observer expérimentalement. De petites variatinos dans la résolution de l'instrument et de l'alignement peuvent avoir de grandes répercussions sur l'intensité mesurée. Cela fait la comparaison de rapports d'intensité difficiles. Une quantité plus robuste est l'aire d'un pic, également appelée intensité intégrée. En fait, certaines intégrations sont faites de base, car l'aire du détecteur est finie. L'expression de l'intensité sur le détecteur plan et suivant le profil du pic correspond à une intégration le long de l'espace réciproque. En convertissant la variation de volume des coordonnées angulaires de l'espace réciproque, on peut introduire un facteur de :

\begin{equation}
    \left(\frac{\lambda}{2\pi} \right)^3 \left( \frac{R^2}{V_c \sin 2\theta} \right)
    \label{eq:integ1}
\end{equation}

où $V_c$ est le volume de la maille. En intégrant le produit de l'équation \ref{eq:facteurintensite}, on se retrouve avec :

\begin{equation}
    (2\pi)^3 N_1N_2N_3 = (2\pi)^3 \frac{V}{V_c}
    \label{eq:integration}
\end{equation}

où $V$ est le volume de l'échantillon

\paragraph{Orientation}

\begin{marginfigure}
    \includegraphics{./images/part1/cullity155}
    \caption{Schéma montrant la sphère, uniformément garine des plans normaux de poudres orientés aléatoirements, faites de cristaux qui ont une distance inter-réticulaire $d$. La fraction de ces cristux orientés pour rdiffracter est simplement l'aire de la bande représentant ces cristaux, divisée par l'aire de la sphère entière.}
    \label{fig:}
\end{marginfigure}

Si un échantillon de méthode des poudres, composé de cristaux orientés aléatoirements, avec une distance inter-réticulaire $d$, est exposé à un faisceau de rayons X de longueur d'onde $\lambda$, alors une fraction de ces cristaux sera orienté de telle sorte à ce que les rayons X le frappent à un angle $\theta$ qui satisfait la loi de Bragg : $\lambda = 2d \sin \theta$. L'intensité des pics de Bragg correspondant sera proportionnelle à cette fraction. Dans l'objectif de trouver cette fraction de cristaux orientés, il faut d'abord remarquer que, pour une poudre orienté aléatoirement, la normale aux plans de distance inter-réticulaire $d$ sera uniformément placée à la surface d'une sphère qui sera centrée sur l'échantillon. Par conséquent, la fraction de cristaux orientés à un angle $\theta$, relatif au faisceau incident, sera simplement l'aire des bandes, divisée par l'aire de la sphère. Cette fraction peut s'écrire :

\begin{eqnarray}
    \text{fraction orientée à} \theta & = & \frac{\text{aire de la bande}}{\text{aire de la sphère}} \\
    & = & \frac{R\Delta \theta 2 \pi R \sin \left(\frac{pi}{2} - \theta \right)}{2\pi R^2}\\
    & = & \frac{\cos\theta \Delta \theta}{2}
    \label{eq:orientation}
\end{eqnarray}
où $R$ est la distance entre l'échantillon et le détecteur.

\paragraph{Segment du cône diffracté}

\begin{marginfigure}
    \TODO
    \caption{}
    \label{fig:diffconeseg}
\end{marginfigure}

Les cristaux qui sont orientés pour diffracter le font en suivant un cône qui a un demi-angle de $2\theta$, comme montré sur la figure \ref{fig:diffconeseg}. L'intensité diffractée est étalée sur un cercle à l'extrémité de ce cône. Cependant, le détecteur n'accepte que des petites longueurs de ce cercle, et pour chaque réflexion donnée, la fraction de l'intensité sur le cône total, acceptée par le détecteur, est inversement proportionnelle à la longueur du cercle à l'extrémité de ce cône. Cela introduit le facteur suivant dans l'expression de l'intensité :

\begin{equation}
    \frac{1}{\text{longueur du cercle}} = \frac{1}{2\pi R \sin 2\theta}
    \label{eq:diffconeseg}
\end{equation}

Si l'on combine les expressions \ref{eq:integ1}, \ref{eq:integration}, \ref{eq:orientation} et \ref{eq:diffconeseg}, on obtient :

\begin{flalign}
    \left( \frac{R^2\lambda^3}{V_c \sin 2 \theta} \right)
    \left( \frac{V}{V_c} \right)
    \left( \frac{\cos \theta}{2} \right)
    \left( \frac{1}{2 \pi R \sin 2 \theta} \right)
    & = \frac{R\lambda^3 V}{4\pi V_c^2} \left( \frac{\cos \theta}{\sin^2 2\theta} \right) &\\
    & = \frac{R\lambda^3 V}{8\pi V_c^2} \left[ \frac{1}{\sin 2\theta \sin \theta} \right] &
\end{flalign}

Le terme entre crochets est appelé facteur de Lorentz. Il est tracé sur la figure \ref{fig:lorentzfactor}.

\begin{marginfigure}
    \includegraphics{./images/part1/cullity157-01}
    \caption{tracé du facteur de Lorentz pour un angle de Bragg $\theta$ entre \SI{0}{} et \SI{90}{\degree}}
    \label{fig:lorentzfactor}
\end{marginfigure}


\paragraph{Multiplicité}

La multipliticé $m_{hkl}$ est le nombre de plans cristallographiques équivalents qui apparaissent dans un cristal. Par exemple, les plans $(00l)$ dans un cristal cubique sont équivalents aux plans $(0l0)$, $(l00)$, $(\bar{l}00)$, $(0\bar{l}0)$ et $(00\bar{l})$. La multiplicité de ce plan dans un cristal cubique est donc $m_{00l} = 6$. Pour les cristaux tétragonaux, $(00l)$ et $(00\bar{l})$ sont équivalents, donc dans ce cas, $m_{00l} = 2$. Le tableau \ref{tab:multiplicites} propose une liste des multiplicités. Le nombre de plans cristallins orientés pour diffracter une sertaine réflexion est proportionnel à $m_{hkl}$. Anisi, la multiplicité entre directement en compte dans l'expression de l'intensité.

\begin{table*}
    \begin{tabularx}{\textwidth}{lCCCCCCC}
        \toprule
        Cubique & $\frac{hkl}{48*}$ & $\frac{hhl}{24}$ & $\frac{0kl}{24*}$ & $\frac{0kk}{12}$ & $\frac{hhh}{8}$ & $\frac{00l}{6}$ & \\
         &  &  &  &  &  & & \\
        Tetragonal & $\frac{hkl}{16*}$ & $\frac{hhl}{8}$ & $\frac{0kl}{8}$ & $\frac{hk0}{8*}$ & $\frac{hh0}{4}$ & $\frac{0k0}{4}$& $\frac{00l}{2}$\\
         &  &  &  &  &  & & \\
        Orthorhombique & $\frac{hkl}{8}$ & $\frac{0kl}{4}$ & $\frac{h0l}{4}$ & $\frac{hk0}{4}$ & $\frac{h00}{2}$ & $\frac{0k0}{2}$& $\frac{00l}{2}$\\
         &  &  &  &  &  & & \\
        Hexagonal/trigonal & $\frac{hk\cdot l}{24*}$ & $\frac{hh \cdot l}{12*}$ & $\frac{0k\cdot l}{12*}$ & $\frac{hk\cdot 0}{12*}$ & $\frac{hh\cdot 0}{6}$ & $\frac{0k \cdot 0}{6}$& $\frac{00\cdot l}{2}$\\
         &  &  &  &  &  & & \\
        Monoclinique & $\frac{hkl}{4}$ & $\frac{h0l}{2}$ & $\frac{0k0}{2}$ &  &  & & \\
         &  &  &  &  &  & & \\
        Triclinique & $\frac{hkl}{2}$ &  &  &  &  & & \\
        \bottomrule
    \end{tabularx}
    \label{tab:multiplicites}
    \caption{Multiplicités pour la méthode des poudres.
    \footnotesize{* Les valeurs données ici sont les multiplicités usuelles. Dans certains cristaux, les plans qui ont ces indices peuvent avoir deux formes avec la même distance inter-réticulaire mais deux facteurs de structure différents, et la multiplicité pour chaque forme est la moitié de la valeur donnée ici. Dans le système cubique, par exemple, il y a certains cristaux dans lesquels des permutations des indices $(hkl)$ produit des plans qui ne sont pas structurellement équivalents. Dans de tels cristaux (AuBe par exemple), le plan $(123)$ par exemple, appartient à une forme et a un certain facteur de structure, alors que le plan $(321)$ appartient à une autre forme et a un facteur de structure différent. Il y a donc 48/2 = 24 plans dans la première forme, et 24 dans la seconde.}}
\end{table*}

\paragraph{Facteur d'absorption}

L'absorption des rayons X par un échantillon peut avoir des effets sur l'intensité observée. Considérons la variation d'intensité produite par un élément de volume d'épaisseur $dz$ à une profondeur $z$ d'un échantillon illuminé par un faisceau de rayons X sur une aire $A_0$. L'absorption des rayons X aura lieu avant et après que le faisceau soit diffracté par l'élément de volume.
Soit $\mathcal{L}$ la longeur totale du chemin parcourue par le faisceau. Alors la variation d'intensité produite par cet élément de volume infinitésimal peut s'écrire :

\begin{equation}
    dI = kI_0 A e^{-mu \mathcal{L}}\,dz
\end{equation}

où $I_0$ est l'intensité incidente, $k$ lest une constante proportionnelle représentant la force de difraction, $\mu$ est le coefficient d'absorption linéaire, et $A$ est l'aire de la section du volume donnée par :

\begin{equation}
    A = \frac{A_0}{\sin \theta}
\end{equation}

Pour le cas d'une réflexion symétrique, la longueur $\mathcal{L}$ est donnée par :

\begin{equation}
    \mathcal{L} = \frac{2z}{\sin \theta}
\end{equation}

Et alors on trouve :

\begin{equation}
    dI = k I_0 \frac{A_0}{\sin \theta} e^{\frac{-2\mu z}{\sin \theta}}\, dz
\end{equation}

\begin{marginfigure}
    \includegraphics{./images/part1/cullity159}
    \caption{Schéma représentant la diffraction d'un écanillon plat, avec un faisceau ayant une largeur de \SI{1}{\centi\metre} dans la direction normale au plan du dessin}
    \label{fig:facteurabs}
\end{marginfigure}

En intégrant sur l'épaisseur de l'échantillon que l'on considère grande devant l'inverse du coefficinet d'absorption linéaire, on trouve :

\begin{eqnarray}
    I & = & \int_0^\infty dI\\
    & = & \frac{k I_0A_0}{2 \mu}
\end{eqnarray}

Par conséquent, dans une symétrie géométrique, l'absorption n'introduit pas de dépendance angulaire, mais introduit un volume d'échantillon effectif de la forme $V = A_0 / 2\mu$.

\paragraph{Dépendance en température}

Les vibrations thermiques des atomes dans les solides résultent en une intensité diffractée réduite et une augmentation de l'intensité diffuse, qui se traduit par un fond large (broad background). La réduction de l'intensité des pics est représentée par un facteur de la forme :
\begin{equation}
    e^{-2M}
\end{equation}

où $M$ est donné par :

\begin{equation}
    M = 2 \pi^2 \frac{<u^2>}{d^2}
\end{equation}

où $<u^2>$ est le carré du déplacement thermique moyen des atomes dans une direction perpendiculaire aux plans diffractés, et $d$ la distance inter-réticulaire. Le facteur $M$ dépend à la fois ed la température et de l'angle de diffusion ; il est parfois écrit comme :

\begin{equation}
    M = B(T) \left( \frac{\sin \theta}{\lambda} \right)^2
\end{equation}

où $B(T)$ est la partie dépendante de la température, qui dépend également de la rugosité du matériau. Le facteur $B(T)$ est parfois tabulé pour certains matériaux.

\paragraph{Bilan}

En combinant les résultats des paragraphes précédens, on peut maintenant écrire l'intensité résultant de la méthode des poudres comme :

\begin{equation}
    I = 
        \frac{I_0 r_e^2 \lambda^3 m_{hkl}}{16\pi R V_c^2}\ 
        \left(\frac{A_0}{2\mu} \right) \
        |F_{hkl}|^2 \ 
        \left( \frac{1+\cos^2 2\theta}{\sin\theta \sin 2\theta} \right)\ 
        e^{-2M}
\end{equation}

\section{Production des rayons X}

% note ser ce qu'il va plutôt falloir faire :
% mettre d'abord cristaux moléculaires, ensuite
% cristaux ioniques puis liasons covalentes, etc.
% une fois qu'on a bien expliqué comment se lient les atomes,
% on parle de comment ils s'arrangent. donc bim réseau
% cristallin, réseau réciproque, classification
% et exemples de structures iono-covalentes simples.

\part[VIBRATIONS DU RÉSEAU]{Vibrations du réseau}

\chapter{Théorie classique des vibrations harmoniques dans un cristal}

\section{Gaz d'excitations}

On a pu étudier dans la première partie la structure d'un réseau
cristallin à l'état de repos. Cette image ne permet pas de décrire les
phénomènes de transport qui peuvent exister dans les matériaux. En particulier,
on peut décrire le phénomène de conduction de la chaleur en imaginant les atomes
vibrant autour de leur position d'équilibre, en poussant occasionnelelement
leurs voisins.
Si ces mécanismes sont corrects, on se heurte très rapidement à des calculs très
compliqués si l'on souhaite étudier le réseau dans son ensemble. Il devient
nécessaire de substituer, à l'étude de particules individuelles, un modèle
collectif.

Nous allons aborder dans cette partie la théorie de la chaleur spécifique à
travers le modèle d'Einstein, qui a mené au modèle de Debye, dans lequel
l'énergie thermique d'un solide est distribuée selon les modes de vibration du 
cristal dans son ensemble. Chaque mode normal, que l'on peut représenter par une
onde stationnaire, peut être analysé comme la somme de deux ondes propagatives
se propageant dans deux directions différentes. Ces ondes sont polarisées
longitudinalement : ce sont des ondes sonores.

Par analogie avec les photons des ondes électromagnétiques, les quanta du
champ vibrationnel du réseau sont appelés phonons.

Dans cette image, on considère le réseau comme un volume composé uniquement d'un
\emph{gaz} de phonons, dont le comportement peut être décrit par analogie avec
un corps noir.

Ces ondes de réseau peuvent transporter de grandes quantités d'énergie, avec
des vitesses de l'ordre de la vitesse du son. La difficulté maintenant est de
trouver un mécanisme pour interpréter le transport de la chaleur, qui correspond
à des mécanismes de diffusion plus qu'à des phénomènes radiatifs. L'idée d'un
gaz implique qu'il y a des \emph{collisions} entre des \emph{particules}, qui
contribuent à la résistance. Cela semble contradictoire avec l'hypothèse que les
états de phonons sont des modes normaux, qui, par définition, sont
indépendants et incapables d'interagir. Il est à noter que lorsque l'on
démontrera les équations du mouvement, on fera une approximation :
l'énergie potentielle est traitée comme si seuls les termes quadratiques
du déplacement par rapport aux positions d'équilibre, ont une influence.
Cette approximation (dite harmonique) n'est pas réelle et néglige, en grande
partie, les interactions phonons-phonons ainsi que la diffusion résultant
des imperfections du réseau.

En parallèle, on représentera la conduction électronique par un modèle collectif
basé sur la mécanique ondulatoire. Un électron ne sera plus simplement représenté
comme une particule simple tentant de pénétrer un réseau d'atomes.
On pourra construire dans les solides des ondes, qui peuvent se regrouper en
paquet d'ondes et être guidés dans le cristal, comme si les ions
n'existaient pas. Les vibrations thermiques et les interactions ondes-électrons
perturbent la régularité du réseau ; cela sera étudié dans une partie
ultérieure.

Ces modèles collectifs ont l'avantage d'être facile à modéliser : les
excitations se comportent comme des particules dans un gaz : on peut facilement 
transposer les concepts de distribution des viteses, de libre parcours moyens, 
etc. familiers dans la théorie cinétique classique.


\section{Approximation harmonique}

Supposons qu'une paire d'atomes d'un cristal, séparée d'une distance
$\mathbf{r}$, contribue d'une quantité $\phi(\mathbf{r})$ à l'énergie
potentielle du cristal. $\phi$ est par exemple le potentiel de Lennard-Jones.
Si le modèle du réseau statique est correct, chaque atome reste fixé sur
sa position du réseau de Bravais. Par conséquent, l'énergie potientelle du
cristal peut s'écrire comme la somme des contributions de toutes les paires
distinctes :

\begin{equation}
    U = \half \sum_{\mathbf{R,R'}} \phi(\mathbf{R-R'}) = \frac{N}{2} \sum_{R\neq0} \phi(\mathbf{R})
    \label{potentiela} 
\end{equation}

Si, dans le modèle, on permet aux atomes, dont la position d'équilibre est
$\mathbf{R}$, de se trouver à une position $\mathbf{r(R)} \neq \mathbf{R}$,
alors, on peut remplacer l'équation \ref{potentiela} par :

\begin{equation}
    U = \half \sum_{\mathbf{R,R'}} \phi(\mathbf{r(R) - r(R')})
    = \half \sum_{\mathbf{R,R'}} \phi(\mathbf{R - R' + u(R) - u(R')})
    \label{potentielu}
\end{equation}

Ici, l'énergie potentel dépend de la variable $\mathbf{u(R)}$, qui est la
distance qui sépare deux atomes.

Pour simplifier le problème, on est obligés d'utiliser l'approximation dite
\emph{harmonique}. Dans celle-ci, les atomes ne sont pas beaucoup déviés de leur
position d'équilibre. Si tous les $\mathbf{u(R)}$. sont petits, alors on peut
développer l'énergie potentiel $U$ autour de sa position d'équilibre, avec
un développement de Taylor à trois dimensions :

\begin{equation}
    f(\mathbf{r+a}) = f(\mathbf{r}) + \mathbf{a \cdot \nabla}f(\mathbf{r}) + \half (\mathbf{a\cdot\nabla})^2 f(\mathbf{r}) + \frac{1}{3!}
    (\mathbf{a \cdot \nabla})^3 f(\mathbf{r}) + \cdots
    \label{taylor}
\end{equation}

Ce qui donne, appliqué à $\mathbf{r=R-R'}$ et $\mathbf{a = u(R) - u(R')}$, on a :

\begin{equation}
    U = \frac{N}{2} \sum \phi(\mathbf{R}) + \half \sum_{\mathbf{R,R'}} (\mathbf{u(R) - u(R')}) \cdot \mathbf{\nabla} \phi(\mathbf{R-R'}) +
    \frac{1}{4} \sum_{\mathbf{R,R'}} [(\mathbf{u(R)-u(R'))\cdot\nabla}]^2 \phi(\mathbf{R-R'}) + O(u^3)
    \label{taylorpot}
\end{equation}

Le coefficient $\mathbf{u(R)}$ dans le terme d'ordre 1 est simplement :

\begin{equation}
    \sum_{\mathbf{R'}} \nabla \phi(\mathbf{R-R'})
    \label{coeflineaireapproxharmh}
 \end{equation}

C'est simplement le minimum de la force éxercé sur l'atome en position
$\mathbf{R}$  par tous les autres atomes, lorsque chacun d'entre eux est placé
à sa position d'équilibre. À l'équilibre, le bilan des forces sur les atomes
est nul. Par conséquent, ce terme d'ordre 1 est nul.

Ainsi, le premier terme non nul du potentiel est le terme quadratique.
L'approximation harmonique consiste à ne retenir que ce terme. On écrit ainsi
l'énergie potentielle comme :

\begin{equation}
    U = U^{eq} + U^{harm}
    \label{approxharm}
\end{equation}
 
Dans cette approximation, $U^{eq}$ est l'énergie potentielle à l'équilibre, et
en outre,

\begin{eqnarray}
    U^{harm} &  = &  \frac{1}{4} \sum_{\mathbf{R,R'}} [u_\mu(\mathbf{R}) - u_\mu(\mathbf{R'})] \phi_{\mu\nu}(\mathbf{R-R'})[u_\nu(\mathbf{R}) -
    u_\nu(\mathbf{R'})] \\
    \phi_{\mu\nu}(\mathbf{r}) & = & \frac{\partial^2 \phi(\mathbf{r})}{\partial r_\mu \partial r_\nu}
    \label{uharm}
\end{eqnarray}

Comme $U^{eq}$ est constante, on peut l'ignorer dans un grand nombre de
problèmes dynamiques, et se restreindre à $U^{harm}$

L'approximation harmonique est généralement le point de départ de toutes les
théories de dynamique du réseau. D'autres corrections sur $U$, en particulier
sur les termes du troisième et du quatrième ordre, permettent de comprendre
plus en profondeur certains phénomènes physiques.

Le potentiel harmonique est souvent écrit dans la forme donnée par \ref{uharm},
mais une forme plus générale est plutôt :

\begin{equation}
    U^{harm} = \half \sum_{\mathbf{R,R'}} u_\mu(\mathbf{R}) D_{\mu\nu} (\mathbf{R-R'})u_\nu(\mathbf{R'})
\end{equation}

Pour le cas décrit ici, $D_{\mu\nu}$ a la forme :

\begin{equation}
    D_{\mu\nu}(\mathbf{R-R'}) = \delta_{R,R'} \sum_{\mathbf{R''}} \phi_{\mu\nu}(\mathbf{R-R''}) - \phi_{\mu\nu} (\mathbf{R-R''})
\end{equation}

\section{Chaîne 1D à 1 atome par maille}

Considérons à présent les vibrations élastiques d'un cristal composé d'un atome
par maille primitive. On cherche la fréquene d'une onde élastique en fonction du
vecteur d'onde qui décrit l'onde avec les constantes élastiques.
Le modèle présenté ici consiste en fait à prendre un cristal tri-dimensionnel et
à réduire, à partir des symmétries, l'étude aux directions de propagation $[100]$, $[110]$ ou $[111]$ pour un cristal cubique par exemple.

Deux cas de figure peuvent se présenter : lorsqu'une onde se propage le long d'
une de ces directions, des plans d'atomes bougent en phase, soit parallèlement
au vecteur d'onde, soit perpendiculairement.

Posons $u_s$ le déplacement du plan d'atomes $s$ (on se restreint alors à un
problème à une dimension).
Pour chaque vecteur d'onde, il y a trois modes solutions : un de polarisation
longitudinale et deux de polarisations transverse.

Appliquons l'approximation harmonique au problème. On considère également que
seules les interacitons entre plus proches voisins ($p=\pm 1$) sont
significatives. La force totale induite par les plans $s\pm 1$ sur les plans $s$
s'écrit :

\begin{equation}
F_s = C(u_{s+1} - u_s) + C (u_{s-1} - u_s)
\end{equation}

Par conséquent, l'équation du mouvement peut s'écrire :
\begin{equation}
M \frac{du_s^2}{dt^2} = C (u_{s+1} -u_{s-1} - 2 u_s)
\end{equation}
où M est la masse d'un atome.

Considérons à présent les solutions sous formes d'ondes propagatives, c'est à
dire qui s'exprimenent en $exp(-i\omega t)$. On obtient alors :
\begin{equation}
-M\omega^2 u_s = C (u_{s+1} + u_{s-1} - 2u_s )
\label{eq:mvtreduite}
\end{equation}
Soit le résultat :
\begin{equation}
u_{s\pm 1} = u \exp(is\mathbf{k}\cdot\mathbf{a}) \exp \pm (i\mathbf{k}\cdot
\mathbf{a})
\label{eq:redmvt}
\end{equation}

où $\mathbf{a}$ correspond au vecteur séparant chacun des plans, et $\mathbf{k}$
est le vecteur d'onde de l'onde propagative.

En injectant \ref{eq:redmvt} dans \ref{eq:mvtreduite} et après calculs, on
retrouve la relation de dispersion :
\begin{equation}
    \omega^2 = \frac{2C}{M}\left(1-\cos \mathbf{k}\cdot\mathbf{a}\right)
    \label{eq:reldispphonon1D}
\end{equation}

La limite de la première zone de Brillouin se situe pour 
$\mathbf{k} = \pm \frac{\pi}{a}$. On peut montrer à partir de la relation de
dispersion \ref{eq:reldispphonon1D} que la pente de $\omega$ en fonction de
$\mathbf{k}$ est 0 en bord de zone :

\begin{equation}
\frac{d\omega^2}{d\mathbf{k}} = \frac{2Ca}{M} \sin \mathbf{k}\cdot\mathbf{a} = 0
\end{equation}

\begin{marginfigure}
\TODO
\caption{Fréquence $\omega$ tracée en fonction du vecteur d'onde $\mathbf{k}$.}
\label{fig:reldispphonon1D}
\end{marginfigure}

On peut alors tracer $\omega$ en fonction de $\mathbf{k}$, comme cela est reporté
sur la figure \ref{fig:reldispphonon1D} :
\begin{equation}
\omega = \sqrt{\frac{4C}{M}} \left| \sin \frac{1}{2} ka \right|
\end{equation}

\section{Chaîne 1D à deux atomes par maille primitive}

La relation de dispersion des phonons présente des caractéristiques différentes 
si l'on considère des cristaux comportant deux atomes ou plus par maille
primitive. Par exemple, la structure NaCl ou la structure diamant comportent
toutes deux deux atomes dans leur maille primitive. Pour chaque mode de
polarisation dans une direction de propagation donnée, la relation de
dispersion $\omega(\mathbf{k})$ présente deux branches : une branche acoustique
et une branche optique. Chacune de ces deux branches présentent des
modes longitudinaux et transverse.

Dans le cas général, si la maille primitive contient $p$ atomes, on se
retrouve avec $3p$ branches dans la relation de dispersion : 3 branches
acoustiques et $3p-3$ branches optiques. Par exemple, KBr ou le germanium
cristallisent avec 3 atomes par maille et disposent de 6 branches :
1 longitudinale acoustique, 2 transversale acoustique, 1 longitudinale optique
et 2 transverse optique.

Considérons, pour démontrer ceci, un cristal cubique composé d'atomes de masse
$M_1$ et d'atomes de masse $M_2$. En fait, il n'est pas nécessaire d'avoir des
masses différentes, mais il faut soit que la force séparant deux atomes dans des
sites non équivalents varie (dans ce cas, comme dans le cadre du cours de 
Pr Dimitri Roditchev, on définit deux constantes de raideur entre les atomes),
ou que le paramètre de maille $\mathbf{a}$ soit différents pour deux atomes dans
des sites non équivalents.

Appelons $u_s$ le déplacement de l'atome de masse $M_1$ dans la diade $s$ et
$v_s$ le déplacement de l'atome de masse $M_2$ dans la diade $s$. Considérons
une onde qui se propage dans la direction $\mathbf{a}$ (dans cette direction, un
plan ne contient qu'un type d'ion).

Écrivons à présent les équations du mouvement sous l'hypothèse que chaque plan
n'intergit qu'avec ses plus proches voisins :

\begin{eqnarray}
M_1 \frac{d^2u_s}{dt^2} & = & C (v_s + v_{s+1} - 2u_s)\\
M_2 \frac{d^2v_s}{dt^2} & = & C (u_{s+1} + u_s - 1v_s)
\label{eq:mvtdeuxatomes}
\end{eqnarray}

On cherche une solution sous la forme d'ondes propagatives, à savoir :
\begin{eqnarray}
u_s & = &  u \exp (iska) \exp (-i\omega t) \\
v_s & = & v \exp (iska) \exp (-i\omega t)
\label{eq:deuxatomessolexp}
\end{eqnarray}

En injectant \ref{eq:deuxatomessoleexp} dans \label{eq:mvtdeuxatomes}, on
retrouve :

\begin{eqnarray}
-\omega^2 M_1 u & = & C v [ 1 + \exp -ika ] - 2Cu\\
-\omega^2 M_2 v & = & C u [ 1 + \exp ika ] - 2Cv
\end{eqnarray}

On peut résoudre ce système d'équations seulement si le déterminant du système
est nul, ce qui revient à écrire :

\begin{equation}
M_1M_2\omega^4 - 2C(M_1 + M_2) \omega^2 + 2C^2 (1-\cos ka) = 0
\end{equation}

Si l'on prend $ka << 1$, alors 
$\omega^2 \approx 2C \left(\frac{1}{M_1} + \frac{1}{M_2} \right)$, 
ce qui correspond à la branche optique, ou alors
$\omega^2 \approx \frac{\frac{1}{2}C}{M_1+M_2} k^2 a^2$,
ce qui correspond à la branche accoustique.

Pour $k=\pm \frac{\pi}{a}$, on a donc $\omega^2 = \frac{2C}{M_1}$ ou
$\omega^2 = \frac{2C}{M_2}$.

Les déplacements de la particules dans les modes transverses acoustiques et
transverse optique sont :

\TODO figure

Pour la branche optique, à $k=0$, $\frac{u}{v} = - \frac{M_2}{M_1}$ : les atomes
vibrent en opposition de phase, avec un centre de masse fixe.

Si les deux atomes ont des charges différentes, on peut exciter un momouvement
de ce type avec le champ électrique d'une onde lumineuse. C'est pour cette
raison que l'on parle de branche optique.

Pour un $k$ quelconque, le rapport $\frac{u}{v}$ est complexe. Une autre 
solution pour le ratio d'amplitude aux petits $k$ est $u=v$, obtenu pour
$k=0$. Les atomes bougent en phase, comme pour des vibrations acoustiques de
grandes longueurs d'ondes.

Des solutions sous forme d'onde n'existent pas pour toutes les fréquences,
comme c'est par exemple le cas entre $\sqrt{\frac{2C}{M_1}}$ et
$\sqrt{\frac{2C}{M_2}}$. C'est une caractéristique des ondes élastiques dans les
réseaux polyatomiques : il y a une bande de fréquences enterdites au bord de la
première zone de Brillouin.

\section{Chaîne 3D}

\section{Loi de Dulong et Petit}

Le potentiel d'énergie dépend de la variable dynamique $\mathbf{u(R)}$ et peut
être abordée à partir du problème gouverné par l'hamiltonien :

\begin{equation}
    \mathcal{H} = \sum_{\mathbf{R}} \frac{\mathbf{P(R)}^2}{2M} + U
\end{equation}
où $\mathbf{P(R)}$ est la quantité de mouvement de l'atome dont la position
d'équilibre est $\mathbf{R}$.

La densité d'énergie thermique d'un cristal est donnée par :

\begin{equation}
    u = \frac{1}{V} \frac{\int d\Gamma e^{-\beta H} H}{\int d\Gamma e^{-\beta H}}
\end{equation}

Ici, $\beta = \frac{1}{k_B T}$ et $d\Gamma$ est l'élément de volume dans
l'espace des phases du cristal :

\begin{eqnarray}
    d\Gamma & = & \prod_{\mathbf{R}} d\mathbf{u(R)} d\mathbf{P(R)}\\
     & = & \prod_{\mathbf{R},\mu} du_\mu (\mathbf{R}) dp_\mu (\mathbf{R})
\end{eqnarray}

Cette formule est obtenue en moyennant les états selon toutes les configurations
ioniques possibles, en pondérant chaque état par le facteur $e^{-\frac{E}{k_BT}}$,
où $E$ est l'énergie de la configuration.

On peut écrire la densité d'énergie thermique dans une forme plus générale :

\begin{equation}
    u = - \frac{1}{V} \frac{\partial}{\partial \beta} \left( \ln \int d\Gamma e^{-\beta H} \right)
    \label{densiteenergiethermique}
\end{equation}

Dans l'approximation harmonique, la dépendance en température de l'intégrale
apparaissant dans \ref{densiteenergiethermique} peut être extraite à partir d'un
changement de variables :

\begin{align}
    \mathbf{\overline{u}(R)} & = & \sqrt{\beta} \mathbf{u(R)} & \qquad &
    d\mathbf{\overline{u}(R)} & = & \beta^{\frac{3}{2}} d\mathbf{u(R)} \\
    \mathbf{\overline{P}(R)} & = & \sqrt{\beta} \mathbf{u(R)} & \qquad &
    d\mathbf{\overline{P}(R)} & = & \beta^{\frac{3}{2}} d\mathbf{u(R)}
\end{align}

On obtient alors :

\begin{eqnarray}
    \int d\Gamma e^{-\beta H} & = & \int d\Gamma \exp \left[ -\beta \left( \sum
    \frac{\mathbf{P(R)}^2}{2M} + U^{eq} + U^{harm} \right) \right] \\
    & = & e^{-\beta U^{eq}} \beta^{-3N} \cdot I 
    \label{injected}
\end{eqnarray}

Où $I$ est une intégrale indépendante de la température. Par conséquent, elle ne
contribue pas dans la dérivée en fonction de $\beta$ quand on injecte
\ref{injected} dans \ref{densiteenergiethermique}. On peut alors réduire
la densité d'énergie thermique en :

\begin{eqnarray}
    u &= & -\frac{1}{V} \frac{\partial}{\partial \beta} \ln \left( e^{-\beta U^{eq}} \beta^{-3N} \cdot cst \right) \\
    & = & \frac{U^{eq}}{V} + \frac{3N}{V}k_BT\\
    & = & u^{eq} + 3nk_BT
\end{eqnarray}

\begin{itemize}
    \item À $T=0$, $u=u^{eq}$, ce qui correspond à la théorie du réseau statique;
    \item Pour $T\neq0$, on corrige $u^{eq}$ de la quantité $3nk_BT$
\end{itemize}

L'énergie $k_BT$ est de l'ordre du centième d'électron-volt ; par conséquent,
la correction est assez faible. Généralement, on considère plutôt la chaleur
spécifique :

\begin{eqnarray}
    c_v & =&  \frac{\partial u}{\partial T} \\
    & = & 3nk_BT
\end{eqnarray}

Ainsi, la chaleur spécifique due aux vibrations de résaeu n'est que de $3k_B$
par ion. Ce résultat constitue la loi de Dulong et Petit. Dans un solide
monoatomique, il y a \SI{6.022e23}{\per\mol} ions. Ainsi, la capacité molaire
est :

\begin{equation}
    c_v^{molaire} = \SI{5.96}{\calorie\per\mol\per\kelvin} = \SI{25}{\kilo\joule\per\mol\per\kelvin}
\end{equation}

En pratique, la variation est légèrement différente, comme présentée sur la
figure \ref{fig:capacitecalorifique}

\begin{marginfigure}
    \TODO
    \caption{Évolution de la capacité calorifique molaire de l'argon en fonction de la température, comparaison avec la loi de Dulong et Petit}
    \label{fig:capacitecalorifique}
\end{marginfigure}

On peut constater expérimentalement que :
\begin{enumerate}
    \item À faible température, la chaleur spécifique est bien plus faible que l valeur donnée par la loi de Dulong et Petit, elle est
        pratiquement nulle à température nulle ;
    \item À plus haute température, la valeur donnée par la loi de Dulong et Petit n'est pas atteinte expérimentallement.
\end{enumerate}

Le point 2 peut être expliqué par les défauts de l'approximation harmonique :
cette hypothèse est trop restrictive lorsque les atomes s'éloignent trop de leur
position d'équilibre.

Le point 1 ne peut pas être expliqué de façon classique. Pour cette raison, nous
allons introduire une quantification des vibrations du réseau.



\chapter{Quantification : les phonons}

L'énergie des vibrations de réseau peut être quantifiée. La quantité d'énergie
est appelée \emph{phonon}, par analogie avec le photon d'une onde
électromagnétique. L'énergie d'un mode élastique de fréquence angulaire $\omega$
est :
\begin{equation}
\mathcal{E} = \left( n + \frac{1}{2} \right) \hbar \omega
\end{equation}
Dans cette expression, le mode est excité au nombre quantique n. C'est le cas
lorsque le mode est occupé par $n$ photons. Le terme $\frac{1}{2}\hbar\omega$
correspond au niveau d'énergie fondamental. Il existe à la fois pour les photons
et les phonons en conséquence de leur équivalence à un oscillateur harmonique quantique de pulsation $\omega$, pour lequel les valeurs prpores de
l'énergie sont également $(n+\frac{1}{2})\hbar\omega$. On peut quantifier le
carré moyen de l'amplitude pour le mode stationnaire :
\begin{equation}
u = u_0 \cos kx \cos \omega t
\end{equation}

Où u est le déplacement d'un élément de volume de sa position d'équilibre à x
dans le cristal. L'énergie du mode, comme dans un oscillateur harmonique, est
composée pour moitié d'énergie cinétique, et pour moitié d'énergie potentielle
lorsque moyenné sur le temps. La densité d'énergie cinétique est :
\begin{equation}
\frac{1}{2}\rho \left( \frac{\partial u}{\partial t} \right)^2
\end{equation}

où $\rho$ est la densité de masse.

Dans un cristal de volume V, l'intégrale volumique de l'énergie cinétique
peut s'écrire :
\begin{equation}
\frac{1}{4}\rho V \omega^2 u_0^2 \sin^2 \omega t
\end{equation}
a moyenne au cours du temps s'écrit : 
\begin{equation}
    \frac{1}{8} \rho V \omega^2 u_0^2 = \frac{1}{2} \left( n+\frac{1}{2} \right) \hbar \omega
\end{equation}
Le carré de l'amplitude du mode s'écrit :
\begin{equation}
u_0^2 = \frac{4\left( n+\frac{1}{2} \right) \hbar}{\rho V \omega}
\end{equation}

Cela décrit le déplacement de l'occupation n d'un phonon dans un mode donné.

La question à se poser est sur le signe de $\omega$. L'énergie d'un phonon doit
être, en principe, positive. Par conséquent, $\omega > 0$. Si la structure
critalline est instable, en revanche, $\omega^2 < 0$ et alors $\omega \in i\mathcal{R}$.

Un phonon de vecteur d'onde $k$ peut interagir avec des particules comme des
photons, des neutrons et des électrons, comme s'il avait une quantité de
mouvement $\hbar k$. Cependant, un phonon ne transporte physiquement pas 
d'impulsion.

La raison pour laquelle un phonon dans unréseau n'a pas de quantité de mouvement
est que a coordonnée d'un phonon (sauf pour $k=0$) implique le problème soit
traité à partir des coordonnées relatives des atomes. Par conséquent, dans une
olécule comme \ch{H2}, la vibration internucléaire $\mathbf{r_1}-\mathbf{r_2}$
est en coordonnées relative et n'a pas d'impulsion linéaire . Le centre de masse
$\frac{1}{2}(\mathbf{r_1}-\mathbf{r_2}$ correspond au mode $k=0$ et peut avoir
une quantité de mouvement linéaire.

Dans les cristaux, il existe des règles de sélection des vecteurs d'ondes pour
les transitions autorisées (entre les états quantifiés).
Par exemple, pour la diffusion élastique en diffraction par les rayons X, la
règle de sélection est : 
\begin{equation}
\mathbf{k'} = \mathbf{k} + \mathbf{G}
\end{equation}




\section{Dynamique, modes normaux, propriétés}

\section{Capacité calorifique des phonons}

La capacité calorifique à volume constant peut s'écrire
$C_v = \left( \frac{\partial U}{\partial T}\right)_V$. La contribution des
phonons à la capacité calorifique est appelée chaleur latente du réseau et est
notée $C_{res}$.

L'énergie totale des phonons dans un cristal peut s'écrire comme la somme des
énergies sur tous les modes de phonons, ici indexés par le vecteur d'onde
$k$ et l'indice de polarisation $p$.

\begin{equation}
    U_{res} = \sum_k \sum_p U_{k,p} = \sum_k \sum_p <n_{k,p}> \hbar \omega_{k,p}
    \label{Ures}
\end{equation}
Dans cette formule, $<n_{k,p}>$ représente l'occuption des phonons de vecteur
d'onde $k$ et de polarisation $p$ à l'équilibre thermique. Il est donné par la
distribution de Planck :

\begin{equation}
    <n> = \frac{1}{\exp \left( \frac{\hbar \omega}{\tau} \right) - 1}
    \label{distrplanck}
\end{equation}

L'énergie d'un assemblage d'osscillateurs harmoniques de fréquence
$\omega_{k,p}$ à l'équilibre thermique est donné par l'équation \ref{Ures}.

Généralement, on préfère assimiler la somme sur $k$ à une intégrale. Pour cela,
supposons que le cristal a $D_p(\omega) d\omega$ modes d'une polarisation $p$
pour les fréquences comprises entre $\omega$ et $\omega + d\omega$.

On obtient alors :

\begin{equation}
    U = \sum_p \int d\omega D_p(\omega) \frac{\hbar \omega}{\exp \left( \frac{\hbar\omega}{\tau} \right) -1}
    \label{Uresint}
\end{equation}

La capacité calorifique du réseau est trouvée par différentiation de la fonction
de température $x = \frac{\hbar \omega}{\tau} = \frac{\hbar\omega}{k_B T}$ :

\begin{equation}
    C_{res} = k_B \sum_p \int d\omega D_p(\omega) \frac{x^2 \exp x}{(\exp x -1)^2}
    \label{Cres}
\end{equation}

À ce stade, tout le problème réside dans cette fonction $D(\omega)$ : le nombre
de modes pour une fréqence donnée. Cette fonction est appelée densité de modes,
ou plus généralement densité d'états.

\section{Densité d'états à une dimension}

Considérons le problème de vibrations unidimensionnelle : sur un segment de
longueur L, on transporte N+1 particules, chacune d'entre elles étant espacée
d'une distance a.

Supposons que les particules indexées par $s=0$ et $s=N$ sont fixées.
Chaque mode de vibration de polarisation p a la forme d'une onde stationnaire,
où $u_s$ est le déplacement de la particule $s$.

\begin{equation}
    u_s = u(0) \exp (-i \omega_{k,p} t) \sim (ska)
\end{equation}

Dans cette formule, $\omega_{k,p}$ est relié à $k$ par la relation de dispersion
appropriée.

Le vecteur d'onde $k$ est lui même restreint par les conditions limites :
$k = \frac{\pi}{L} ; \frac{2 \pi}{L} ; \frac{3\pi}{L} ; \cdots ; \frac{(N-1)\pi}{L} $.

La solution pour $k=\frac{\pi}{L}$ admet comme solution $u_s \propto \sin \left(\frac{s\pi a}{L} \right)$, et s'annule pour $s=0$ ou $s=N$.

La solution pour $k=\frac{N\pi}{L} = \frac{\pi}{a} = k_{frontière}$ est
$u_s \propto \sin (s\pi)$. Cela ne permet pas aux atomes de se déplacer car
$\sin s\pi$ est nul à chaque atome. Par conséquent, il y a $(N-1)$ valeurs de k
indépendantes autorisées. Ce nombre est égal au nobmre de particules qui peuvent
se mouvoir. Chaque valeur de $k$ est asociée à une onde stationnaire. Pour le
problème à une dimension, il y a un mode pour chaque intervalle
$\Delta k = \frac{\pi}{L}$, de sorte à ce que le nombre de modes par échelle
d'unité $k$ est $\frac{L}{\pi}$ pour $k \in ]-\frac{\pi}{a} ; \frac{\pi}{a} ]$.

Il y a trois polarisations $p$ possibles pour chaque $k$ :

À une dimension, deux d'entre elles sont transverses et une longitudinale.
À trois dimensions, la polarisation n'est simple que pour des vecteurs d'onde
dans les directions caractéristiques du crital. On peut utiliser les conditions
de Born von Karman : on considère le milieu infini, mais en supposant que les
solutions sont périodiques sur une distance $L$ grande, de sorte que
$u(sa) = u(sa + L)$.

\begin{marginfigure}
    \TODO
    \caption{BVK}
\end{marginfigure}

Considérons N particules contraintes sur un anneau circulaire. Les particules euvent oscillées si elles sont connectées par des ressorts.

Les modes normaux correspondent à des déplacements $u_s$ d'un atome $s$ de la
forme $\sin ska$ ou $\cos ska$ : ce sont des modes indépendants. La périodicité
impose les conditions aux limites : $u_{N+s} = u_s,\quad \forall s$. Par
conséquent, $Nka$ est un entier multiple de $2\pi$.

Pour une solution sous forme d'onde propagative, $u_s = u_0 \exp i (ska - \omega_k t)$, et donc les valeurs possibles pour k sont :

\begin{equation}
    k\in \{ 0,\pm\frac{2\pi}{L}, \pm\frac{4\pi}{L},\cdots, \frac{N\pi}{L} \} 
\end{equation}

Cette méthode de dénombrement donne le même nombre de modes (1 par atome mobile)
, mais on a maintenant des valeurs positives et négatives pour $k$, avec un
intervalle $\Delta k$ entre deux valeurs successives de $k$ égal à
$\frac{2\pi}{L}$. Pour des conditions limites périodiques (Born von Karman), le
nombre de modes par échelle de $k$ est $\frac{L}{2\pi}$ pour
$k \in ]\frac{-\pi}{a},\frac{\pi}{a} ]$ et 0 ailleurs.

Alors, on a besoin de connaître le nombre $D(\omega)$ de modes par unité de
fréquence pour une polarisation donnée. Le nombre de modes $D(\omega)d\omega$
entre $\omega$ et $\omega + d\omega$ est donné à une dimension par :

\begin{equation}
    D_1(\omega) d\omega = \frac{L}{\pi} \frac{dk}{d\omega} d\omega = \frac{L}{\pi} \cdot \frac{d\omega}{\frac{d\omega}{dk}}
\end{equation}

On peut obtenir la vitesse de groupe $\frac{d\omega}{dk}$ à partir de la
relation de dispersion $\omega = f(k)$.

Il y a une singularité dans $D_1(\omega)$ dès que $\omega(k)$ est horizontale,
c'est à dire lorsque la vitesse de groupe est nulle.


\section{Densité d'états à trois dimensions}

On applique les conditions périodiques sur $N^3$ mailles primitives. Dans un
cube de côté L, de sorte que $k$ soit déterminé par la condition :

\begin{equation}
    \exp \left[ i(k_x x + k_y y + k_z z) \right] = \exp i \left[ k_x (x+L) + k_y (y+L) + k_z (z+L) \right]
\end{equation}
Avec :

\begin{equation}
    k_x,k_y,k_z \in \left\{ 0, \pm\frac{2\pi}{L}, \pm \frac{4\pi}{L}, \cdots,\frac{N\pi}{L} \right\}
\end{equation}

Par conséquent, il n'y a qu'une valeur de $\mathbf{k}$ permise par volume de
$\left( \frac{2\pi}{L} \right)^3$ dans l'espace réciproque, c'est à dire
$\left( \frac{L}{2\pi} \right)^3 = \frac{V}{8\pi^3}$ valeurs de $\mathbf{k}$
permises par unité de volume dans l'espace réciproque, pour chaque branche ou
chaque polarisation. Le volume de l'échantillon est $V = L^3$.

Le nombre total des modes avec un vecteur d'onde plus petit que $k$ est le
produit du volume d'une sphère de rayon k par $\left( \frac{L}{2\pi} \right)^3$

Ainsi, pour chaque type de polarisation,

\begin{equation}
    N = \left( \frac{L}{2\pi} \right)^3 \left( \frac{4\pi k^3}{3} \right)
\end{equation}

La densité d'états de chaque polarisation est :

\begin{equation}
    D(\omega) = \frac{dN}{d\omega} = \left( \frac{Vk^2}{2\pi^2} \right) \frac{dk}{d\omega}
\end{equation}

\subsection{Modèle de Debye pour la densité d'états}

L'approximation de Debye consiste à dire que la vitess du son est constante pour
chaque type de polarisation, comme c'est le cas dans un milieu élastique continu
. La relation de dispersion s'écrit :

\begin{equation}
    \omega = v k
\end{equation}

Avec $v$ est la vitesse du son, qui est prise constante. Dans ce cas, la densité
d'états s'écrit :

\begin{equation}
    D(\omega) = \frac{V\omega^2}{2\pi^2v^3}
\end{equation}

Il y a N mailles primitives dans l'échantillon. Par conséquent, le nombre total 
de modes acoustiques de phonons est N.

La fréquence de coupure est $\omega_D$.
On peut écrire :

\begin{equation}
    \omega_D^3 = \frac{6\pi^2v^3 N}{V}
\end{equation}

À cette fréquence, on associe un vecteur d'onde de coupure dans l'espace
réciproque :

\begin{equation}
    k_D = \frac{\omega_D}{v} = \left( \frac{6\pi^2N}{V} \right)^{\frac{1}{3}}
\end{equation}

Le modèle de Debye n'autorise pas les modes de vecteurs d'ondes plus grands que
$k_D$. Le nombre de modes $k\leq k_D$ réduit le nombre de degrés de liberté
pour un réseau monoatomique.
L'énergie thermique est, pour chaque type de polarisation :

\begin{eqnarray}
U & = & \int d\omega D\omega <n(\omega)> \hbar \omega \\
 & = & \int_0^{\omega_D} d\omega \left( \frac{V\omega^2}{2\pi^2v^3} \right) \left( \frac{\hbar \omega}{e^{\frac{\hbar \omega}{\tau}} - 1} \right)
\end{eqnarray}

On admet que la vitesse des phonons est indépendante de la polarisation, donc on
obtient :

\begin{eqnarray}
    U &= & \frac{3V\hbar}{2\pi^2v^3} \int_0^{\omega_D} d\omega \frac{\omega^3}{\exp ( \frac{\hbar\omega}{\tau}) -1}\\
    \label{energieU}
    & = & \frac{3Vk_B^4 T^4}{2\pi^2v^3\hbar^3} \int_0^{x_D} dx \frac{x^3}{e^x -1}
\end{eqnarray}
Où on a effectué le changement de variables $x=\frac{\hbar\omega}{\tau}$. Posons
également la tepérature de Debye $\theta$, définie par la relation :
\begin{equation}
    x_D = \frac{\hbar \omega_D}{k_B T} = \frac{\theta}{T}
\end{equation}
Que l'on peut développer :

\begin{equation}
    \theta = \frac{\hbar v}{k_B} \left( \frac{6\pi^2 N}{V} \right)^3
\end{equation}

L'énergie totale de phonons est alors :

\begin{equation}
    U = 9 N k_B T \left( \frac{T}{\theta} \right)^3 \int_0^{x_D} dx \frac{x^3}{e^x -1}
\end{equation}

La capacité calorifique est trouvée facilement en différentiant l'expression
\ref{energieU}. Alors :

\begin{eqnarray}
    C_v &=& \frac{3V\hbar^2}{2\pi^2v^2k_B T^2} \int_0^{\omega_D} d\omega \frac{\omega^4 e^{\frac{\hbar \omega}{\tau}}}{\left(
        e^{\frac{\hbar\omega}{\tau}} -1 \right)^2}\\
        & = & 9 N k_B \left( \frac{T}{\theta} \right)^3 \int_0^{x_D} dx \frac{x^4 e^x}{(e^x -1)^2}
\end{eqnarray}

La capacité calorifique de Debye peut être alors tracée :

\begin{figure}
    \TODO
    \label{debye}
    \caption{capacité calorifique selon le modèle de Debye}
\end{figure}

Pour $T>>\theta$, la chaleur latente est approximativement celle obtenue par la
valeur classique de $3Nk_B$



\subsection{Einstein}

Prenons N oscillateurs de même fréquence $\omega_0$ et un à une dimension.
La densité d'état dans le modèle de Einstein est
$D(\omega) = N \delta(\omega - \omega_0)$. L'énergie thermique du système est:

\begin{equation}
    U = N <x> \hbar \omega = \frac{N \hbar \omega}{e^{\frac{\hbar \omega}{\tau}}-1}
\end{equation}

(Pour simplifier, j'écris $\omega$ à la place de $\omega_0$. On peut dériver
cette expression pour trouver la capacité calorifique :

\begin{equation}
    C_v = N k_B \left(\frac{\hbar \omega}{\tau} \right)^2 
    \frac{e^{\frac{\hbar\omega}{\tau}}}{(e^{\frac{\hbar \omega}{\tau}} - 1)^2}
\end{equation}

Le résultat pour la contribution de N oscillateurs identiques à la capacité
calorifique d'un solide. À 3 dimensions, N devient 3N.
La limite haute température de $C_v$ devient $3Nk_B$, ce qui correspond à la
valeur de Dulong et Petit.

À basse température, $C_v$ décroit de façon exponentielle :
$\exp \left( - \frac{\hbar\omega}{\tau} \right)$, alors qu'expérimenalement, 
c'est plus une décroissance en $T^3$ avec le modèle de Debye.

En revanche, le modèle d'Einstein est plus utilisé pour approximer la partie
phonon optique du spectre de phonons.

\subsection{haute température}

\subsection{$T^3$}

Pour $T<<\theta$, on a $x_D \rightarrow \infty$.

\begin{eqnarray}
    \int_0^\infty dx \frac{x^3}{e^x-1} & = & \int_0^\infty dx x^3 \sum_{s=1}^{\infty} \exp -sx\\
    & = & 6 \sum_1^\infty \frac{1}{s^4}\\
    & = & \frac{\pi^4}{15}
\end{eqnarray}

Par conséquent, pour $T<<\theta$,

\begin{equation}
    U ~= \frac{3 \pi^4 N k_B T^4}{5 \theta^3}
\end{equation}

Ainsi, on peut écrire la capacité calorifique :
\begin{eqnarray}
    C_v & = & \frac{12\pi^4}{5} N k_B \left( \frac{T}{\theta}\right)^3 \\
    & = & 234 N k_B \left( \frac{T}{\theta} \right) ^3
\end{eqnarray}

Pour de faibles températures, cette approximation est pas mauvaise. En fait,
ceci est valable lorsque seulement les modes acoustiques des grandes longueurs
d'onde sont excités therimquement. Ce sont juste les modes qui peuvent être
traités en tant que milieu élastique continu avec les constantes élastiques
macroscopiques.

Seuls les modes pour lesquels $\hbar \omega < k_B T$ seront excités de façon
suffisante à faible température. L'excitation de ces modes sera à peu près
classique, chacun d'entre eux avec une énergie $k_BT$.

Dans le volume autorisé dans l'espace réciproque, la fraction occupée par les
modes excités est de l'ordre de
$\left( \frac{\omega_T}{\omega_D} \right)^3$, soit
$\left( \frac{k_T}{k_D} \right)^3$ où $k_T$ est la longueur d'onde thermique,
définie par $\hbar v k_T = k_B T$ et $k_D$ est la longueur d'onde de coupure
du modèle de Debye.
La fraction occupée du volume total dans l'espace réciproque est
$\left( \frac{T}{\theta} \right)^3$. En outre, il y a de l'ordre de
$3N \left( \frac{T}{\theta} \right)^3$ modes excités, qui ont chacun une énergie
$k_B T$. L'énergie est approximativement
$3Nk_B T \left( \frac{T}{\theta} \right)^3$ et la capacité calorifique
$12Nk_B \left( \frac{T}{\theta} \right)^3$.

\begin{marginfigure}
    \TODO
    \caption{représentation des modes excités sur l'enseble des modes}
\end{marginfigure}

En pratique, pour que l'approximation soit correcte, il faut
$T=\frac{\theta}{50}$. Par exemple, pour le sicilium,
$\theta = \SI{645}{\kelvin}$ ; pour le germanium,
$\theta = \SI{374}{\kelvin}$, donc il faudrait travailler à des températures de 
l'ordre du kelvin pour que l'approximation soit correcte.

On peut noter que plus un atome est lourd, plus $\theta$ est petit. En fait,
la vitesse du son diminue lorsque la densité augmente, c'est ce qui provoque
cette variation en terme de vecteurs d'onde.

\section{Corps noir}
ppp

%\chapter{Interactions phonons-phonons, problèmes des modèles harmoniques}

abc

\section{Probabilité de transition}
abc
\section{Forces anharmoniques}
abc
\section{Conductivité thermique}
abc
\section{Expansion thermique}

%\chapter{Gaz de phonons}

sorte de conclusion de la partie 2. gaz classique ?


    \part[STRUCTURE ÉLECTRONIQUE]{Structure électronique}
%
%    %\include{chapter/01-liaisons} % électrons dans les solides
%    %\include{chapter/02-niveaux-energie}
%    %\include{chapter/03-repulsion}
%    \include{chapter/electrons-solide}
%    %\chapter{Transport d'électrons : le modèle de Drude}
phénomènes de conduction des électrons
hypothèses du modèle
vitesse entre deux collisions
libre parcours moyen
chaleur spécifique - capacité calorifique

réponse d'un gaz de drude en hautes fréquences
effet hall
statistique de maxwell boltzman

La théorie de Drude fournit des éléments de compréhensio des solides. C'est une
théorie classique, et c'est la première qui décrit les métaux. Elle comporte de
nombreux défauts, que nous expliquerons par les modèles qui suivent.

\section{phénomènes de conduction des électrons}

je sais plus ce que je voulais mettre ici.
\TODO

\section{Hypothèses du modèle}

Dans la théorie de Drude, on fait les hypothèses suivante :

- Les porteurs de charge sont les électrons qui constituent un gaz auquel on peut appliquer la théorie cinétique.

- Les porteurs de charge positive, beaucoup plus lourds, sont considérés comme immoobiles.

Seuls les électrons de valence des atomes participent à la conduction et les charges positives correspondent aux ions (noyau + électrons de cœur). Par conséquent, on obtient les informations suivantes :

- Entre les collisions, l'interaction d'un électron avec les autres électrons et avec les ions est nulle. Ainsi, en l'absence de champ extérieur, les électrons ont un mouvement rectiligne, uniforme et isotrope. En présence d'un champ extérieur, le mouvement des électrons entre les collisions est déterminé par les équations de Newton de la dynamique.

- les collisions sont des événements instantanés, qui modifient brusquement la vitesse des électrons. Cela revient à dire que l'on néglige totalement les forces à longue portée et on ne tient compte que des forces à courte portée.

- la probabilité qu'un électron subisse une collision durant l'intervalle de temps compris entre $t_0$ et $t_0 + dt$ est égale à $dt/\tau$. Elle est indépendante de $t_0$, de la position et de la vitesse de l'électron. On peut montrer que le temps moyen entre deux collisions successives d'un électron est égal à $\tau$. On note $\tau$ le temps de collision, temps de vol moyen et plus généralement \emph{temps de relaxation}.

- le gaz électronique atteint son équilibre themique par les collisions instantanées avec les ions. On admet que, arpès chaque collision, la vitesse d'un électron est l'orientation quelconque (distribution isotrope des vitesses juste après une collision) et que sa vitesse est reliée à la température locale (de la région de l'espace où a eu lieu la collision).

On fait donc deux hypothèses majeures :
\begin{description}
\item[approximation des électrons indépendants] on néglige l'interaction électron-électron, ce qui est valable dans de nombreux cas ;
\item[approximation des électrons libres] on néglige l'interaction entre électrons et ions. Ce modèle ne peut pas du tout définir les comportements différents des métaux et des semi-conducteurs.
\end{description}

\TODO

\section{vitesse entre deux collisions}

Si l'on impose un champ de \SI{1}{\volt\per\milli\metre} et un temps de relaxation $\tau$ de \SI{e-4}{\second}, alors $v = \SI{10}{\metre\per\second}$. Il s'agit de la vitesse de dérive des électrons.

Si on compare cette vitesse à $v_{th}$, alors

\begin{equation}
\sqrt{<v^2>} = \sqrt{\frac{3k_BT}{m}}
\end{equation}

Ce qui vaut, à 300K, $<v> \sim \SI{e5}{\metre\per\second}$.

Dans un métal, avec le modèle de Drude, l'ajout d'un champ est négligeable. On ne peut pas décrire les électrons comme dans un gaz :
deux atomes sont séparés d'environ \SI{2}{\angstrom}.

Ainsi, la densité d'électrons dans un métal est très grande devant la densité dans un gaz. Par exemple $n_e = 10^3 n_{air}$

\section{Statistique de Maxwell-Boltzman}

La distribution des charges dans un gaz est celle de Maxwell-Boltzmann, représentée sur la figure \ref{fig:maxwell-boltzmann}

\begin{marginfigure}
\TODO
\caption{Distribution de Maxwell-Boltzmann}
\label{fig:maxwell-boltzmann}
\end{marginfigure}

Cependant, Drude introduit les chocs entre particules : la formule $\sigma = \frac{ne^2\tau}{m}$ est valable et adaptable à la physique quantique.

\section{libre parcours moyen}

\begin{equation}
l = \sqrt{<v_{th}^2>} \tau
\end{equation}

Il s'agit du temps de parcours moyen d'une particule pendant un temps $\tau$, soit la distance typique entre deux chocs.
problème : il y a beaucoup plus de chocs par rapport au modèle de Drude. On est loin du gaz idéal (déviation représentée sur la figure \ref{fig:libreparcoursdrude})

\begin{marginfigure}
\TODO
\caption{Libre parcours moyen en fonction de la température ; déviation expérimentale par rapport à la prévision du modèle de Drude}
\label{fig:libreparcoursdrude}
\end{marginfigure}

$l$ dépend de la section efficace de l'ion.

Lorsque la température décroit, la section efficace diminue. de fait, la distance de choc diminue également.

\begin{equation}
S \sim T\quad \rightarrow l=\sqrt{T}
\end{equation}

Par conséquent,
\begin{equation}
\tau(T) \sim \frac{1/\sqrt{T}}{\sqrt{T}} \sim \frac{1}{T}
\end{equation}

Soit :
\begin{equation}
\rho = \frac{1}{\sigma} = \frac{m}{q^2nE} \sim \tau
\end{equation}

Finalement, le modèle de Drude permet de modéliser de façon plutôt fidèle l'évolution linéaire.
En revanche, on perd en performance dans le creux. Cela est dû aux impuretés.

\section{chaleur spécifique - capacité calorifique}

Soit $\mathcal{E}$ l'énergie totale du système. On peut écrire la chaleur spécifique $C$ comme :

\begin{equation}
C = n \frac{d\mathcal{E}}{dT}
\end{equation}

Cette capacité calorifique sert à repérer les transitions de phase. Si $C$ n'est pas lisse, alors il y a une transition de phase ou une transition chimique qui se produit. Cela est facile à mesurer lorsque l'énergie augmente. (à \SI{100}{\celsius} pour l'eau, $dI = 0$, ce qui divege ?????\TODO

Pour un gaz thermo, $\mathcal{E} = \frac{3}{2}k_BT$ donc

\begin{equation}
C = \frac{3}{2} n k_B
\end{equation}



\section{réponse d'un gaz de Drude à hautes fréquences}

\subsection{conduction de Drude, AC}

\TODO : d'où sort cette formule

\begin{equation}
\frac{d\mathbf{p}}{dt} = q\mathbf{E} - \frac{1}{\tau} \mathbf{p}
\label{eq:drudechampelec}
\end{equation}

On peut poser $\mathbf{E}(t) = \mathbf{E}(\omega) e^{-i\omega t}$, et chercher de cette façon $\mathbf{p}(t) = \mathbf{p}(\omega) e^{-i\omega t}$.

En injectant cela dans l'équation \ref{eq:drudechampelec}, on trouve :

\begin{equation}
-i\omega \mathbf{p}(\omega) = q\mathbf{E}(\omega) - \frac{1}{\tau} \mathbf{p}(\omega) 
\end{equation}

Soit :

\begin{equation}
\mathbf{p}(\omega) = \frac{\mathbf{p}(\omega)}{\frac{1}{\tau}-i\omega}
\end{equation}

On peut écrire la densité de courant en fonction de la fréquence. Comme $j(t) = j(\omega) e^{-i\omega t}$.

De plus, $j(t) = nqv = \frac{nq}{m}\mathbf{p}(t)$.

On obtient alors :
\begin{equation}
j(\omega) = \frac{nq}{m}\mathbf{p}(\omega) = \frac{nq^2/m}{\frac{1}{\tau}-i\omega} \mathbf{E}(\omega)
\end{equation}

\TODO


\section{Effet Hall}

    \chapter{Modèle des électrons libres}

Le modèle des électrons libres a été proposé par Arnold Sommerfeld en 1920, pour
décrire le comportement des électrons dans un solide comme un gaz.

On peut comprendre beaucoup des propritétés physiques des métaux, et pas
seulement des métaux simples, avec un modèle d'électrons libres. En suivant ce
modèle, les électrons de valence des atomes constituant le métal deviennent des
électrons de conduction est bougent librement dans le volume du métal. Même dans
les métaux pour lesquels le modèle des électrons libre marcherait le mieux, la
distribution de charge des électrons de conduction reflète le potentiel 
électrostatique important des cœurs ioniques. L'utilité d'un modèle d'électrons
libres est importante pour comprendre les propriétés qui dépendent
essentiellement des propriétés cinétiques des électrons de conduction.
L'interaction des électrons de conduction avec les ions du réseau est traitée
dans le chapitre suivant sur le modèle des électrons presque libres.

Les métaux les plus siples sont les métaux alcalins : le lithium, sodium, 
potassium, césium et rubidium. Dans un atome libre de sodium, l'électron de 
valence est dans un état 3s ; dans un métal, cet électron devient un électron de
conduction dans la bande de conduction 3s.

Un cristal monovalent qui contient N atomes aura N électrons de conduction et N
cœurs ioniques positifs. Le cœur ionique \ch{Na+} contient 10 électrons qui
occupent les couches 1s, 2s et 2p de l'ion libre, avec une distribution spatialle
qui est essentielleement la même qu'il soit métallique ou sous forme d'ion libre.
Les cœurs ioniques remplissent à peu près 15\% du volume dans un cristal de
sodium. Le rayon de l'ion libre \ch{Na+} est de \SI{0.98}{\angstrom}, alors que
la moitié de la distance aux plus proches voisins dans le métal est de
\SI{1.83}{\angstrom}.

L'interprétation des propriétés métalliques en terme de mouvement des électrons
libres a été développée bien avant l'invention de la mécanique quantique. La 
théorie classique a eu quelques succès, notamment en démontrant une forme de la
loi d'Ohm et la relation entre la conductivité électrique et thermique, que nous
aborderons plus tard avec le modèle de Drude pour la conduction électronique.
En revanche, la théorie classique a échoué à  expliquer la capacité calorifique
et la susceptibilité magnétique des élecrtons de conduction. En fait, ce ne sont
pas des problèmes sur le modèle de l'électron libre, mais des modèles sur la
distribution classique de Maxwell-Boltzman qui y est utilisée.

Il y a une difficulté supplémentaire avec le modèle classique. Plusieurs
expériences de plusieurs types donnent le résultat suivant : il est clair que les
électrons de conduction dans un métal peuvent se déplacer avec un chemin direct 
sur plusieurs distances atomiques, sans être défvés par des collisions avec
d'autres électrons de conduction ou par les collisions avec les noyaux atomiques.
Dans un échantillon très pur à basse température, le libre parcours moyen est
aussi long que \SI{e8}{} distances inter-atomiques (plus d'un centimètre !).

Pourquoi est la matière condensée si transparente à la conduction des électrons ?
La réponse à cette question peut être scindée en deux parties : premièrement, un
électron de conduction n'est pas dévié par les cœurs ioniques arrangés sur un
réseau \emph{périodique} parce que les ondes mécaniques peuvent se propager 
librement sur une structure périodique, comme conséquence des mathématiques 
traités dans ce chapitre. Deuxièmement, un électron de conduction est diffusé
seulement rarement par d'autres électrons de conduction. Cette proriété est une
conséquence du principe d'exclusion de Pauli. Quand on parle de gaz d'électrons
libres de Fermi, on doit plutôt vouloir dire un gaz d'électrons libres sujets au
principe de Pauli.


\section{Effets du principe de Pauli}

Considérons un électron libre dans un gaz 1D, en prenant en compte la théorie 
quantique et le principe de Pauli. Un électron qui a une masse m est confiné dans
un puit de largeur L. La fonction d'onde $\psi_n(x)$ de l'électron est une
solution de l'équation de Schrödinger $\mathcal{H}\psi = \epsilon \psi$, ce qui,
en négligeant l'énergie potentielle, donne $\mathcal{H} = p^2/2m$, où $p$ est la
quantité de mouvement. Dans la théorie quantique, $p$ peut représenter 
l'opérateur $-i\hbar d/dx$, de telle sorte que :
\begin{equation}
    \mathcal{H}\psi_n = -\frac{\hbar^2}{2m} \frac{d^2\psi_n}{dx^2}
    = \epsilon_n\psi_n
\end{equation}
où $\epsilon_n$ est l'énergie de l'électron sur son orbitale.

On utilise le terme orbitale pour dire une solution de l'équation d'onde pour un
système d'un seul électron. Le terme permet de distinguer entre un état quantique
exact de l'équation d'onde d'un système de N électrons interagissant et un état
quantique approximé que l'on construit en associant aux N électrons N
déffirentes orbitales, chacune d'entre elles étant une solution de l'équation
d'onde pour un électron. Le modèle orbitalaire est exact s'il n'y a aucune
interaction entre chacun des électrons.

Les conditions aux limites sont : $\psi_n (0) = 0$, $\psi_n(L)=0$, comme imposé 
par le puit de potentiel infini. Ces conditions sont satisfaites si la fonction
d'onde est sinusoidale, avec un nombre entier de demi-longeurs d'ondes entre 0
et L :
\begin{equation}
    \psi_n = A \sin\left( \frac{2\pi}{\lambda_n} \right) ; \hfill
    \frac{1}{2}n\lambda_n = L
\end{equation}
où A est constante.
On voit que c'est une solution de l'équation de Schrödinger, parce que :
\begin{equation}
    \frac{d\psi_n}{dx} = A \left( \frac{n\pi}{L} \right)
    \cos\left( \frac{n\pi}{L}x \right) ; \hfill \frac{d^2\psi_n}{dx^2} 
    = - A \left( \frac{n\pi}{L} \right)^2\sin\left( \frac{n\pi}{L}x \right)
\end{equation}

où l'énergie $\epsilon_n$ est donnée par :
\begin{equation}
    \epsilon_n = \frac{\hbar^2}{2m} \left( \frac{n\pi}{L}\right)^2
\label{en}
\end{equation}

On veut pouvoir loger N électrons sur la ligne. Comme prévu par le principe
d'exclusion de Pauli, deux électrons ne peuvent pas avoir le même nombre
quantique. Ansi, chaque orbitale ne peut être occupée que par deux électrons au
maximum. Ceci s'applique aux électrons dans les atomes, molécules, et solides.

Dans un solide linéaire, les nombres quantiques d'une orbitale d'un électron de
conduction sont $n$ et $m_s$, où $n$ est un nombre entier positif, et le nombre
quantique magnétique $m_s = \pm \frac{1}{2}$, suivant l'orientation de spin. Une
paire d'orbitales marquée par un nombre quantique $n$ peut recevoir deux
électrons, un avec un spin vers le haut et un avec un spin vers le bas. S'il y a
six électrons, alors l'état fondamental du système est rempli d'orbitales données
par le tableau suivant :

\begin{table}[ht]
    \begin{center}
        \begin{tabular}{rrr}
            \toprule
            n & $m_s$ & occupation par les électrons\\
            \midrule
            1 & $\uparrow$ & 1\\
            1 & $\downarrow$ & 1\\
            2 & $\uparrow$ & 1\\
            2 & $\downarrow$ & 1\\
            3 & $\uparrow$ & 1\\
            3 & $\downarrow$ & 1\\
            4 & $\uparrow$ & 0\\
            4 & $\downarrow$ & 0\\
            \bottomrule
        \end{tabular}
    \end{center}
    \label{}
    \caption{remplissage des orbitales pour un système à 6 électrons}
\end{table}

Plus d'une orbitale peut voir la même énergie. Le nombre d'orbitales avec la même
énergie est appelée \emph{dégénérescence}.

Soit $n_F$ le niveau d'énergie rempli le plus élevé, où on commence à remplir les
niveaux depuis le bas (n=1), et on continue à remplir les niveaux d'énergie 
supérieure, jusqu'à ce que tous les N électrons soient placés. Il est pratique de
supposer que N est un nombre pair. la condition $2n_F = N$ détermine $n_F$, la
valeur de n pour laquelle le niveau le plus élevé est rempli.

L'énergie de Fermi $\epsilon_F$ est définie comme l'énergie du plus haut niveau
rempli à l'état fondamental d'un système à N électrons. En utilisant la relation
\ref{en}, avec $n=n_F$, on a, à une dimension :
\begin{equation}
    \epsilon_F = \frac{\hbar^2}{2m}\left(\frac{n_F\pi}{L}\right)^2 = 
    \frac{\hbar^2}{2m}\left( \frac{N\pi}{2L} \right)^2
\end{equation}


\section{Statistique de Fermi-Dirac}

L'état fondamental est l'état d'un système de N électrons au zéro absolu. Ce qui
arrive quand on augmente la température ? C'est un problème standard dans la
mécanique statistique élémentaire, et la solution es donnée par la distribution
de Fermi-Dirac.

\begin{figure}
    \TODO
    \caption{La distribution de Fermi Dirac}
    \label{fermidir}
\end{figure}


L'énergie cinétique du gas d'électrons augmente quand la température augmente :
certains niveaux d'énergie sont occupés, d'autres restent vacants au zéro absolu,
et certains niveuax sont vides alors qu'ils sont occupés au zéro absolu. La
distribution de Fermi-Dirac donne la probabilité qu'une orbitale d'énergie
$\epsilon$ soit occupée par un gas d'électrons idéal à l'équilibre thermique.

\begin{equation}
    f(\epsilon) = \frac{1}{\exp\left( \frac{\epsilon - \mu}{k_BT} \right) + 1}
\end{equation}

La quantité $\mu$ est une fonction de la température. $\mu$ doit être choisie
pour un problème particulier d'une telle façon que le nombre total de
particules dans e système en dérive correctement, c'est à dire qu'on retrouve
bien N à la fin. Au zéro absolu, $\mu = \epsilon_F$, parce que la limite
$T\rightarrow 0$, la fonction $f(\epsilon)$ change de façon discontinue de la
valeur 1 (rempli) à la valeur zéro (vide), à $\epsilon = \epsilon_F = \mu$. Pour
toute température, $f(\epsilon)$ est égal à 1/2 lorsque $\epsilon = \mu$.

$\mu$ est le potentiel chimique, et on doit voir qu'au zéro absolu, le potentiel
chimique est égal à l'énergie de Fermi, définie comme l'énergie de la plus haute
orbitale remplie au zéro absolu.

La partie en queue de distribution, est cette parie pour laquelle $\epsilon - \mu
\ll k_BT$. Ici, le terme exponentiel est dominant dans le dénominateur, et ainsi
$f(\epsilon) \cong \exp(\frac{\mu - \epsilon}{k_BT})$. Cette limite est appelée
la distribution de Maxwell-Boltzmann.

\section{Généralisation à trois dimensions}

Les électrons sont des particules quantiques. Ils peuvent donc être décrits par
une fonction d'onde $\psi(\mathbf{r},t)$, qui est la solution de l'équation de
Schödinger suivante :


\section{conditions de Born von Karman}

\section{Sphère de Fermi et espace des états}

Dans l'état fondamental d'un système à N électrons libres, les orbitales occupées
peuvent représenter des points dans une sphère de l'espace réciproque. L'énergie
à la surface de la sphère est l'énergie de Fermi. Les vecteurs d'onde à la
surface de Fermi ont une intensité $k_F$ telle que :
\begin{equation}
    \epsilon_F = \frac{\hbar^2}{2m}k_F^2
    \label{eef}
\end{equation}

On peut alors voir qu'il n'y a qu'un seul vecteur d'onde permis, qui est distinct
du triplet de nombre quantiques $k_x, k_y, k_z$, pour l'élément de volume
$(2\pi/L)^3$ de l'espace réciproque. Ainsi, dans la sphère de volume $\frac{4\pi
k_F^3}{3}$, le nombre total d'orbitales est :

\begin{equation}
    2\cdot \frac{\frac{4\pi k_F^3}{3}}{\left( \frac{2\pi}{L}\right)^3} 
    = \frac{V}{3\pi^2} k_F^3 = N
\end{equation}

où le facteur 2 à gauche vient des deux valeurs permises par le nombre quantique
de spin pour chaque valeur de $\mathbf{k}$. Ainsi, on obtient :
\begin{equation}
    k_F = \left( \frac{3\pi^2N}{V}\right)^{1/3}
    \label{kkf}
\end{equation}

qui ne dépend que de la concentration en particules.

\begin{marginfigure}
    \TODO
    \caption{Sphère de Fermi}
    \label{spherefermi}
\end{marginfigure}

En combinant \ref{eef} et \ref{kkf}, on obtient :
\begin{equation}
    \epsilon_F = \frac{\hbar^2}{2m} \left(\frac{3\pi^2N}{V}\right)^{2/3}
    \label{ef}
\end{equation}

Ceci se rapporte à l'énergie de Fermi pour la concentation en électrons N/V. La
vitesse des électrons $v_F$ sur la surface de Fermi est :
\begin{equation}
    v_F = \left( \frac{\hbar k_F}{m}\right)
    = \left( \frac{\hbar}{m} \right)\left(\frac{3\pi^2 N}{V} \right)^{1/3}
\end{equation}

Les valeurs calculées de $k_F$, $v_F$ et $\epsilon_F$ sont données dans le
tableau suivant pour certains métaux. Le paramètre de rayon est un nombre sans
dimension $r_n = r_0/a_H$ où $a_H$ est le rayon de Bohr et $r_0$ le rayon de la
sphère qui contient un électron.

\begin{table*}[ht]
\footnotesize
\begin{center}
\begin{tabularx}{\textwidth}{rlRRRRRR}
\toprule
Valence & Métal & concentration en électrons ($10^{22} cm^{-3}$) & Paramètre de rayon $r_n$ & vecteur d'onde de Fermi ($10^8 cm^{-1}$) & vitesse de Fermi ($10^8 cm\cdot s^{-1}$) & énergie de Fermi (eV) & température de Fermi ($T_F =\epsilon_F / k_B$) ($10^4$ K)\\
\midrule
1 & Li & 4.70 & 3.25 & 1.11 & 1.29 & 4.72 & 5.48 \\
  & Na (78K) & 2.65 & 3.93 & 0.92 & 1.07 & 3.23 & 3.75\\
  & K (78K) & 1.40 & 4.86 & 0.75 & 0.86 & 2.12 & 2.46\\
  & Rb (78K) & 1.15 & 5.20 & 0.70 & 0.81 & 1.85 & 2.15\\
  & Cs (78K) & 0.91 & 5.63 & 0.64 & 0.75 & 1.58 & 1.83\\
  & Cu & 8.45 & 2.67 & 1.36 & 1.57 & 7.00 & 8.12\\
  & Ag & 5.85 & 3.02 & 1.20 & 1.39 & 5.48 & 6.36\\
  & Au & 5.90 & 3.01 & 1.20 & 1.39 & 5.51 & 6.39\\
2 & Be & 24.02 & 1.88 & 1.93 & 2.23 & 14.14 & 16.41\\
  & Mg & 8.60 & 2.65 & 1.37 & 1.58 & 7.13 & 8.27\\
  & Ca & 4.60 & 3.27 & 1.11 & 1.28 & 4.68 & 5.43\\
  & Sr & 3.56 & 3.56 & 1.02 & 1.18 & 3.95 & 4.58\\
  & Ba & 3.20 & 3.69 & 0.98 & 1.13 & 3.65 & 4.24\\
  & Zn & 13.10 & 2.31 & 1.57 & 1.82 & 9.39 & 10.90\\
  & Cd & 9.28 & 2.52 & 1.40 & 1.62 & 7.46 & 8.66\\
3 & Al & 18.06 & 2.07 & 1.75 & 2.02 & 11.63 & 13.49\\
  & Ga & 15.30 & 2.19 & 1.65 & 1.91 & 10.35 & 12.01\\
  & In & 22.49 & 2.41 & 1.50 & 1.74 & 8.60 & 9.98\\
4 & Pb & 13.20 & 2.30 & 1.57 & 1.82 & 9.37 & 10.87\\
  & Sn(w) & 14.48 & 2.23 & 1.62 & 1.88 & 10.03 & 11.64\\
\bottomrule
\end{tabularx}
\end{center}
\caption{Paramètres de la surface de Fermi pour le modèle de l'élecrton libre
pour quelques métaux à température ambiante}
\label{}
\end{table*}

Maintenant, on trouve une expression du nombre d'orbitales par gamme d'énergie
unitaire., $D(\epsilon)$, appelée densité d'états\footnote{si on est rigoureux,
$D(\epsilon)$ représente plutôt la densité d'états à une particules, ou la
densité des orbitales}. On utilise \ref{ef} pour obtenir le nombre total
d'orbitales d'énergie inférieure ou égale à $\epsilon$ :
\begin{equation}
    N = \frac{V}{3\pi^2} \left(\frac{2m\epsilon}{\hbar^2}\right)^{3/2}
\end{equation}

Ainsi, la densité d'états est :
\begin{equation}
    D(\epsilon) \equiv \frac{dN}{d\epsilon} = 
    \frac{V}{2\pi^2} \cdot 
    \left(\frac{2m}{\hbar^2}\right)^{3/2}\cdot\epsilon^{1/2}
\end{equation}
Ce résultat peut être exprimé plus simplement en comparant les deux formules,
pour obtenir, à $\epsilon$ :
\begin{equation}
    D(\epsilon) \equiv \frac{dN}{d\epsilon} = \frac{3N}{2\epsilon}
\end{equation}
Avec un facteur de l'ordre de l'unité, le nombre d'orbitales par unité d'énergie
à l'energie de Fermi est le nombre total d'électrons de conduction divisé par
l'énergie de Fermi, exactement ce qu'on espérait.

\section{Énergie et chaleur spécifique du gaz d'électrons libres}

\section{Densité d'états électroniques}

\section{conductivité thermique}

Constante de Wiedemann-Franz

\section{Effet Pelletier}

    \chapter{Modèle des électrons dans un potentiel périodique}

L'approximation de l'électron libre, prédécemment décrite, est un guide assez
utile pour comprendre les propriétés des métaux (particulièrement des métaux
    alcalins, dans lesquels les électrons de valence - uniques - sortent des
    couches compactes et fermées des ions métalliques, se comportent comme s'ils
étaient libres). La théorie de Sommerfeld (1928) est une théorie presque autonome
sur le transport électronqiue, en eutilisant cette approximation, et donne
presque tout le temps des résultast bon qualitativement. Mais avec notre
compréhension de plus en plus fine de la structure des métaux, il devient
indispensable de considérer les effets des termes plus importants, ceux dûs au
réseau ionique.

Comme on l'a remarqué, la fonction d'onde complète du système peput être traitée
comme s'il sagissait de deux facteurs séparables, chacun correspondant à des
coordonnées d'un électron simple, tant que l'on ne considère pas des termes comme
l'interaction coulombienne. Notre problème est donc de résoudre une équation de
Schrödinger de la forme :

\begin{equation}
    \left\{ - \frac{\hbar^2}{2m} \nabla^2 + \mathcal{U}(\mathbf{r})\right\}
    \psi(\mathbf{r}) = \mathcal{E}\psi(\mathbf{r})
\end{equation}

Sujette aux conditions limites habituelles.
\section{cas général, théorème de Bloch, sphère de Fermi, DOS}

\section{potentiel faible}
gap, plans de bragg


\section{liaisons fortes}
isolants

\subsection{généralités}
\subsection{LCAO}
\subsection{méthode cellulaire}
\subsection{méthode des ondes planes}

%    %\include{chapter/04-presque-libres}
%    %\include{chapter/05-liaisons-fortes}
    \chapter{Modèle semiclassique : théorie des bandes}

introduction, kronnig-penney, etc.

%    \include{chapter/dft}
%
%    %\part[INTERACTIONS ÉLECTRONS-RÉSEAU]{Interactions électrons-réseau}
%    %
%    %\include{chapter/basse-dimensionnalite}
%    %\include{chapter/polarons}
%    %\include{chapter/valence-mixte}
%    %\include{chapter/06-proprietes-meca}
%
    \part[DÉFAUTS]{Défauts}
%
    \chapter{Thermodynamique}

La description des structures cristallines et l'étude du modèle ionique ont été
réalisées dans le cadre du cristal parfait. Dans la réalité, le solide présente
des défauts dont les principaux sont : les phonons (vibration thermiques), les
défauts atomiques (lacunes, interstitiels, impuretés), les défauts électroniques
(électrons, trous, excitons), les imperfections dans l'arrangement atomique
(dislocations, fautes d'empilement) et la surface où sont localisésdes atomes
particuliers du point de vue énorgétique et structurale.

Dans le cristal réel, de nombreuses propriétés physiques et chimiques proviennent
directement de l'existence de ces défauts. Nous nous limitons ici à l'étude des
défauts atomiques qui sont à l'origine de la conduction ionique. Dans quelques
cas particuliers, la conductivité dans les solides ioniques est voisine de celle
d'un électrolyte liquide. Le matériau est alors potentiellement utilisable comme
électrolyte solide dans un système électrochimique.

On peut montrer aisément que la présence de défauts, jusqu'à une certaine
concentration conduit à une réduction d'enthalpie libre, donc à une stabilisation
du réseau. L'introduction d'un défaut ponctuel (impureté, lacune, interstitiel)
dans un cristal supposé parfait nécessite une augmentation d'enthalpie
assimilable à une quantité d'énergie
\footnote{$H_f = E_f + p\delta v$ où $\delta v$ est approximativement le volume
    atomique (\SI{20}{\cubic\angstrom}). Pour p = 1atm, on a
    $p \delta v = \SI{e-5}{\electronvolt}$ négligeable devant
$E_f \approx \SI{1}{\electronvolt}$ }
$E_f$, énergie de formation du défaut. Mais elle produit aussi une augmentation
importante d'entropie de configuration $\Delta S_c$, car ce défaut peut occuper
un grand nombre de positions. Dans le cas le plus simple où le défaut occupe un
site anionique et possède la symétrie de l'atome qu'il remplace,
l'entropie\footnote{Le terme entropique $\Delta S_v$ dû aux variations des modes
de vibrations automiques est généralement négligé.} calculée pour n défauts
disposés sur N sites atomiques est\footnote{On écrit $S_c = k \log(P)$ où le
nombre de complexions P dans le cristal est le nombre d'arrangements possibles de
n défauts edes N-n atomes dans les N positions du réseau : $P =
\frac{N!}{(N-n)!n!}$. On utilise l'approximation de stirling : $\log N! = N \log
N - N$} :

\begin{equation}
    \Delta S_c = - N k (x \log x + (1-x) \log (1-x))
\end{equation}

où l'on a posé $x = n/N$ la concentration en défauts. Ce terme est toujours
positif et inférieur à 1. Il varie très brutalement pour x petit : $dS/dx
\rightarrow \infty$ pour $x\rightarrow 0$. L'énergie, elle, ne varie que comme
$NxE_f$. En conséquence, l'introduction de défauts dans le solide parfait
provoque une diminution de l'enthalppie libre.

\begin{marginfigure}
    \TODO
    \caption{variation d'énergie par introduction de défauts dans un cristal
    parfait}
    \label{varenergcristparf}
\end{marginfigure}

L'enthalpie libre, $G = NxE_f + NkT (x \log x + (1-x)\log(1-x))$, est minimale
lorsque x vérifie la relation :
\begin{equation}
    \frac{x}{1-x} = \exp - \left( \frac{E_f}{kT} \right)
\end{equation}

Soit encore, pour x petit, c'est à dire $E_f$ assez grand devant kT :
\begin{equation}
    x \approx \exp - (E_f / kT)
\end{equation}

À une température donnée, il existe donc une certaine concentration de défauts
qui minimise G. Le défaut prédominant est évidemment celui associé à la plus
petite valeur de $E_f$ et il est très largement fonction de la structure
cristalline.

\begin{table*}[ht]
    \begin{tabularx}{\textwidth}{lXX}
        \toprule
        Cristal & Structure & Défaut prédominant \\
        \midrule
        Halogénures alcalins & NaCl & Schottky \\
        Oxydes alcalino-terreux & NaCl & Schottky \\
        AgCl, AgBr & NaCl & Frenkel cationique \\
        Halogénures de césium, TlCl & CsCl & Schottky \\
        BeO & Wurtzite, ZnS & Schottky \\
        Fluorures d'alcalino-terreux, $CeO_2$,$ThO_2$ & Fluorine, $CaF_2$ &
        Frenkel anionique\\
        \bottomrule
    \end{tabularx}
    \label{}
    \caption{défaut ponctuel prédominant dans différents cristaux}
\end{table*}

\begin{figure}
    \TODO
    \caption{Défauts de Schottky et de Frenkel}
    \label{schottkyfrenkel}
\end{figure}

Dans des structures compactes, le défaut prédominant est le défauts de Schottky
avec même nombre de lacunes cationiques et anioniques pour assurer
l'électroneutralité. L'énergie de formation $E_s$ de la paire de Schottky
correspond à l'extraction d'un cation (énergie $E_{fc}$) et d'un anion (énergie
$E_{fa}$) qui se localisent à la surface du cristal.

Calculons les concentrations en volume des lacunes cationiques et anioniques pour
une température donnée, en fixant la contrainte $x_c = x_a$.

Le passage d'un ion à la surface du cristal revient à faire passer sa constate de
Madelung de M à M/2. Les énergies de formation des lacunes cationiques et
anioniques devraient donc être égales à la moitié de la contribution coulombienne
à l'énergie réticulaire. Mais, à cette énergie, il faut soustraire l'énergie de
polarisation érsultant du processus de relaxation ionique (une lacune anionique,
par exemple, porte une charge positive qui attire les anions). Les contributions
dues à la relaxation ionique n'ont aucune raison d'être identiques pour les deux
types d'ions. En conséquence, $E_{fc} \neq E_{fa}$ et $x_c \neq x_a$.

La condition de neutralité électrique n'est plus respectée dans le cristal.
Celui-ci réagit en disposant l'exès de lacunes chargées sous la surface de façon
à créer une couche dipolaire (couche de Debye) qui restitue la neutralité
électrique en volume et diminue considérablement la portée du champ électrique dû
à la surface.

Avec la cotnraine $x_c = x_a$, la minimisation de l'enthalpie libre, avec $dx_c =
dx_a$, conduit à la loi d'action de masse pour l'équilibre cristal-lacunes :
\begin{equation}
    x_c \cdot x_a = \exp(-E_s/kT)
\end{equation}
avec $E_s = E_{fc} + E_{fa}$, soit, avec nos hypothèses : 
$x_c = x_a = exp(-E_s/2kT)$.
(Dans le cas de NaCl : $E_s \approx 2.3 eV$, $x_c = x_a = 3\cdot 10^{-17}$ à
300K\footnote{cette valeur est sous-estimée d'au moins un à deux ordres de
grandeur. Les termes correctifs proviennent d'une part de la modification des
virbations ioniques ($\Delta S_v$) et d'autre part de la variation de $E_S$ avec
la température que l'on corrèle à la dilatation du cristal}, ce qui correspond à
$5\cdot10^5$ défauts par \si{\cubic\centi\metre}).

Le défaut de Frenkel est prédominant dans des structures ouvertes (faible nombre
de coordination) et concerne principalement les cations (taille inférieure à
celle des anions). Il existe deux exceptions importantes à cette règle :

\begin{itemize}
    \item le cas de la structure fluorine dans laquelle l'anion a un faible
        nombre de coordination (4 au lieu de 8 pour le cation), ce qui lui permet
        d'aller elativement facilement en position interstitielle (cas des ions
        $F^-$ dans $CaF_2$ et $O^{2-}$ dans $ZrO_2$). On parle dans ce cas de
        défauts anti-Frenkel.
    \item le cas des halogénures d'argent qui possèdent une structure type NaCl
        (donc relativement compacte) et dans laquelle des proportions importantes
        d'ions $Ag^+$ peuvent occuper une position interstitielle. Dans cette
        position, un ion $Ag^+$ est entouré tétraédriquement par 4 ions $Cl^-$ et
        également à la même distance par 4 ions $Ag^+$. La stabilisation du
        défaut est due à une interaction covalente maruée entre les atomes
        d'argent et de chlore.
\end{itemize}

La concentration en défauts de Frenkel à l'équilibre est donnée par :

\begin{equation}
    x_i \cdot x_v = \exp (-E_F / kT)
\end{equation}
où $x_v$ et $x_i$ sont respectivement les concentrations en lacunes et en
insterstitiels. $E_F$ l'énergie de formation du défaut de Frenkel ($E_F$ =
\SI{1.35}{\electronvolt} pour AgCl).
 
Dans le cristal pur, on considère en général que $x_i = x_v = \exp -E_F/2kT$.
 
Pour les deux types de défauts (Frenkel et Schottky), on peut observer des
associations de défauts atomiques par interaction électrostatique, par exemple
entre une lacune anionique de charge nette +e et une lacune anionique de charge
nette -e. Ces associations se comportent comme des dipôles.
 
Les défauts atomiques ont également la possibilité de piéger des défauts
électroniques. Ainsi, la charge positive de la lacune anionique lui permet de
piéger un électron. Le défaut constitue un obget hydrogénoïde donnant lieu,
comme un atome d'hydrogène, à des niveaux d'énergie et des absorptions optiques
caractéristiques. L'absorption sélective a souvent lieu dans le visible, d'où le
nom de centre F\footnote{de Farbzentrum, centre colloré en allemand} donné à
l'ensemble lacune-électron.
 
Le centre F peut être considéré, pour simplifier, comme une cage cubique où se
trouve localisé l'électron, l'arête de la cage est peu différente de l'arête a
de la maille cristalline. Si le potentiel est pris nul dans la cage et infini à
l'extérieur, les avelurs propres de l'énergie de l'électron sont :
 
\begin{equation}
    E = \frac{\hbar^2}{2m}\frac{\pi^2}{a^2} (n_x^2 + n_y^2 + n_z^2)
\end{equation}

où $n_i$ sont des entiers non nuls.
 
L'énergie correspondant au passage de l'état fondamental au premier état excité
est :
 
\begin{equation}
    \Delta E = 3 \frac{\hbar^2}{2m} \frac{\pi^2}{a^2}
\end{equation}
 
La variation de l'énergie en $a^{-2}$ est observée pour les halogénures alcalins.
La taille des lacunes est un peu supérieure à la taille de la maille du fait des
interactions attractives de l'électron du défaut par les actions voisins.
 
\begin{marginfigure}
    \TODO
    \caption{absorption lumineuse pour les halogénures alcalins contenant des
    centres F. l'énergie deltaE de la première transition est reportée en
coordonnées logarithmiques en fonction de l'arête a de la maille cristalline}
    \label{abslumhalogalcal}
\end{marginfigure}
 
Un cristal de NaCl contenant des centres F est obtenu par chauffage du cristal
en présence de vapeur de sodium (ou de potassium). Il se créé un excès de
cations par rapport à la stoechiométrie et des lacunes anioniques que les
électrons provenant de l'ionisation du sodium transforment en centre F. La
transition vers le premier état excité est responsable d'une couleur jaune pour
NaCl.
 
D'autres exemples d'associations de défauts atomiques et électroniques sont
monrtés dans la figure suivante. Les centres colorés sont aussi formés par
irradiation (rayonnement X ou $\gamma$) et sont à l'origine de la couleur de
nombreuses pierres précieuses : topaze bleu, améthyste, etc.
 
\begin{marginfigure}
    \TODO
    \caption{représentation chématique de centres colorés dans des cristaux
    ioniques}
    \label{centrescolor}
\end{marginfigure}

    \chapter{Création de défauts par la présence d'ions étrangers}

\section{Solution solide : phase cristalline à composition variable}


\section{Substitution d'un ion par un autre de charge différente. Solutions solides complexes}

cf aschcroft p 623 : centres colorés

\section{Centres colorés}

On a indiqué que la neutralité de charge impose des lacunes d'un des constituants d'un cristal ionique diatomique, pour être contrebalancée, soit par un nombre égal d'interstices du même onstituent (Frenkel), soit par un nombre égal de lacunes de l'autre composé (Schottky). Il est également possible, malgré cela, de contrebalancer la charge manquante d'une lacune ionique négative avec un électron localisé au voisinage d'un défaut ponctuel, dont la charge est replacée.

Un tel électron doit être vu comme lié à un un centre chargé positivement, et aura, en général, un spectre de niveaux énergétiques. Les excitations entre ces niveaux produiront une série de bandes d'absorption plutôt analogues à celles d'un atome isolé. Ces énergies d'excaitation apparaissent dans la bande optique interdite entre $\hbar\omega_T$ et $\hbar\omega_L$, pour un cristal parfait (définir, cf aschcroft chap 27), et par conséquent se démarqueront avec des pics intenses dans le spectre d'absportion optique. Ces défauts, avec d'autres structures électroniques électron-induits, sont connus comme des centres colorés, parce que leur présence induit une couleur intense sur les cristaux parfaits qui seraient sinon transparents.

Les centres colorés ont été étudiés de façon extensive dans les halogénures alcalins, qui peuvent être colorés par expostion à une radiation de rayons X ou $\gamma$ (avec la production induite de défauts par les photons à très haute énergie), ou, plus instructivement, en chauffant les cristaux d'halogénures alcalins dans une vapeur de métal alculin. Dans ce cas, les atomes d'alcalins en excès (ceux dont le nombre est compris entre un pour \SI{e7}{} et 1 pour \SI{e3}{} ), sont incorporés dans le cristal, comme l'analyse chimique peut le démontrer. Cependant, la densité massique des cristaux colorés diminue en proportion avec la concentration en atome alcalins en excès, ce qui montre que les atomes ne sont pas absorbés aux interstices.  À la place, les atomes de métaux alcalins sont ionisés et tiennent leur emplacement dans les sites d'un sous-réseau parfait chargé positivement, et les électrons en excès sont liés à un nombre égal de lacunes ioniques négatives.

Une évidence frappente de la validité de cette image est donnée par le fait que le spectre d'absorption produit ainsi n'est pas vraiment changé si, par exemple ,no chauffe du chlorure de potassium sous une vapeur de sodium, plutôt que du potassium métal. Cela confirme de fait que le rôle primaire de la vapeur de métaux alcalins est là pour introduire des lacunes ioniques négatives et pour fournir l'électron supplémentaire, dont leniveau d'énergie produit le spectre d'absorption.

Un électron lié à une lacune ionique négative (connue sous le nom de centre $F$\footnote{de l'allemand \emph{farbzentrum}}), est capable de reproduire plusieurs fonctions qualitatives du spectre atomique ordinaire, avec la complication ajoutée qu'il bouge dans un champ de symétrie cubique, plutôt que sphérique\footnote{cela va te permettre de réviser la théorie des groupes}. En fait, en contraignant le cristal, on peut réduire la symétrie cubique, ce qui produira des perturbations induites, qui seront utiles pour démêler une série de structures aditionnelles dans le spectre d'absorptio. La structure additionnelele est présente parce que le simple centre $F$ n'est pas le seul moyen qu eles électrons et les lacunes peuvent utiliser pour colorer le cristal. Deux autres possibilités sont :
\begin{enumerate}
    \item un centre $M$, dans lequel deux lacunes ioniques négatives voisines dans un plan $(100)$ lient deux électrons ;
    \item un centre $R$, dans lequel trois lacunes ioniques négatives voisines dans un plan $(111)$ lient trois électrons.
\end{enumerate}

Cela demande becauop d'ingéniosité de démentrer que ces catégories variées de défauts sont en fait responsables du spectre observé. L'identification est rendue possible en ramarquant que chaque a une réponse caractéristiuque aux effets de la contrainte ou de hamp électrique à sa structure de niveau.

La résonnance dans le spectre d'absorption optique, produite par les centres colorés, n'est pas aussi nette que celle produite par l'excitation d'atomes isolés. Cela est en fait à cause du fait que l'épaisseur de ligne est inversement proportionnelle au temps de vie de l'état excité. Des atomes isolés peuvent uniquement décroiter radiativement, ce qui est un processus relativement lent, mais les atomes représentés par un centre $F$ sont couplés de façon importante, comme le reste des solides, et peuvent donc perdre leur énergie en émettant des photons.

On peut penser qu'en chauffant les cristaux d'halogénures alcalins dans un gaz halogène, on peut aussi introduire des lacunes métaliques alcalines auxquelles des trous peuvent être liées, mais cela sont des \emph{antimorphes} des centres $F$ et ne sont pas observés. Les trous peuvent être liées à des imperfections ponctuelles, mais les imperfections n'ont pas été observées être des lacunes positives ioniques. En fait, le centre de trous le plus étudié, le centre $V_K$, n'est pas basé sur une lacune du tout, mais sur la possibilité pour un trou de se lier à deux ions négatifs voisins (par exemple \ch{Cl-}). dans quelque chose qui a un spectre plus comme \ch{Cl2-}.

En ayant commencé avec le diagnostique de la construction d'un centre coloré, on peut continuer assez loin. Par exemple, on peut regarder, ou fabriquer, un centre $F$ simple, dans luqeul in des six plus proches voisins ions positifs a été remplacé par une impureté. On se retrouve alors avec un centre $F_A$, qui a une symétrie réduite.

Finalement, continuer dans la recherche d'opposés que nous pouvons inculre, avec soit l'antimorphe du cnetre $V_K$ a été obsérvée : un électron localisé, servant à lier deux ions voisins chargés positivement, ensemble. Comme (par exemple) les molécules de \ch{Cl2} existent (liées de façon covalente) et les molécules de \ch{Na2} en général non, la réponse est non. En fait l'asymétrie entre les électrons et les trous est précisemment due à la différence dans les électrons de valence du sodium (niveau $s$) et du chlore (niveau $p$), qui forment des liaisons covalentes seulement dans le second cas. Cependant, quelque chose de plus localisé que l'antimorphe $V_K$ existe, et c'est connu en tant que \emph{polaron}.

\section{Polarons et excitons}

Lorsqu'un électron est introduit dans la bande de conduction d'un cristal ionique parfait, il peut être énegrétiquement favorable parce qu'il peut bouger sur un niveau spatialement localisé, accompagnié d'une déformation locale dans l'arrangement ionique initialement parfait (\ie une polarisation du réseau), qui sert à écranter son champ et à réduire son énergie électrostatique. Une telle entité (un électron et une polarisation induite du réseau), se révèle être plus mobile que les défauts décrits précédemment, et est généralement pas vue commme un défaut du tout, mais plutôt comme une complication dans la théoie de la mobilité électronique dans les cristaux ioniques ou partiellement ioniques. Les théories des polarons sont un peu compliquées, parce qu'il faut considérer la dynamique d'un électron qui est très reliée au degrée de liberté ionique.

Les formes de défauts ponctuels la plus évidente consiste en un ion vancant (lacune), un ion en excès (interstices), ou alors un mauvais type d'ion (impuretés de substitution). Une façon plus subtile est le cas d'un ion dans un cristal parfait, qui diffère des ses collègues en étant dans un état excité. Un tel \emph{défaut} est appelé un \emph{exciton de Frenkel}.

Comme n'importe quel ion est capable d'être excité, et qu'à partir du moment ou le couplage entre les ions du niveau électronique est important, l'énergie d'excitation peut en fait être transférée d'ion à ion. Par conséquent, l'exciton de Frenkel peut se mouvoir dans le cristal entier, sans que les ions eux-mêmes aient à changer de place, ce qui résulte du fait que (comme le polaron), bien plus mobile que les lacunes, interstices ou défauts de substitution. De fait, pour la plupart des études, il est mieux de ne pas penser qu'un exciton puisse être localisé. Il est plus précis de décrire la structure électronique d'un cristal contenant un exciton, omme une superposition quantique d'états, dans lequel il est équiprobable que l'excitation soit associée avec n'importe quel ion du cristal. Cette vison porte la même relation que les ions spécifiquement excités, comme les niveaux des liaison forte de la théorie de Bloch portent les niveaux atomiques individuels, dans la théorie des bandes.

Par conséquent, l'exciton est probablement mieux vu comme une manifestation plus complexe de la structure de bande électronique, que comme un défaut cristallin. De fait, une fois que l'on reconnait la description correcte qu'un exciton est réellement un problème de structure de bande, on peut avoir un point de vue différent sur ce même phénomène :

Supposons que nous avons calculé le niveau électronique fondamental d'un isolant dans l'approximation des électrons indépendants. Le niveau excité le plus bas de l'isolant sera évidemment donné en enlevant un électron du niveau le plus haut, dans la bande haute occupée. (la bande de valence) et en le placant sur un niveau plus bas de la bande de conduction. Un tel réarrangement de la distribution des électrons n'altère par le potentiel périodique auto-cohérent, dans lequel ils bougent. Cela est du au fait que les électrons de Bloch ne sont pas localisés (comme $\abs{\psi_{nk}(r)}^2$ est périodique), et par conséquent le changement de ladensité de charge localeproduit en changennt la charge d'un électron sera de l'ordre de $1/N$ (comme seul $1/N$ de la charge électronique sera celle d'une maille donnée), \ie négligeable. Par conséquent, le niveau d'énergie électronique n'a pas à être recalculé pour la configuration excitée et pour le premier état excité, qui sera à l'énergie $\epsilon_c - \epsilon_v$ au dessus de l'énergie de l'état fondamental, où $\epsilon_c$ est l'énergie du minimum de la bande de conduction et $\epsilon_v$ est l'énergie du maximum de la bande de valence.

Cependant, il y a un autre moyen de produire un état excité. Supposons que l'on forme un niveau à un électron en superposant sffisemment de niveaux près du minimum de la bande de conduction, pour former un paquet d'ondes bien localisés. Parce que nous avons besoins de niveaux au voisinage du minimum pour produire le paquet d'ondes, l'énergie $\bar{\epsilon}_c$ du paquet d'onde sera légèrement plus grande que $\epsilon_c$. Supposons, en plus, que le niveau de la bande de valence que nous dépeuplons est aussi un paquet d'onde, formé de niveaux au voisinage du maximum de la bande de valence (de sorte que son énergie $\bar{\epsilon}_v$ est très légèrement plus petit que $\epsilon_v$), et choisi de sorte que le centre du paquet d'onde est spatiallement très proche du centre du paquet d'onde de la bande de conduction. Si on ignore les interactions électrons-électrons, alors l'énergie nécessaire pour pouger un électron de la bande de valence à la bande de conduction (paquets d'onde), sera $\bar{\epsilon}_c - \bar{\epsilon}_v > \epsilon_c - \epsilon_v$, mais parce que les niveaux sont localisés, ils seront, en plus, dans des quantités non négligeables d'énergie coulombienne, due à l'attraction électrostatique de l'électron (localisé) de la bande de conduction et du trou (localisé aussi) de la bande de valence.

Cette énergie négative électrostatique additionnelle peut être réduite en une énergie totale d'excitation à un montant plus petit que $\epsilon_c - \epsilon_v$, de sorte à ce que le type compliqué d'état excités, dans lequel l'électron de la bande de conduction est spatiallement corrélé avec le trou de la bnade de valence qu'il a laissé, il est le vrai plus petit état excité du cristal. La preuve de cela est que :
\begin{enumerate}
    \item le début de la bande d'absorption optique à des énergie s sous le continuum interbandes et
    \item l'argument théorique qui va suivre, qui indique que l'on fait toujours mieux en utilisant l'attraction électron-trou.
\end{enumerate}

Considérons le cas dans lequel les niveaux des électrons et trous localisés s'étendent sur de nombreuses constantes de réseau. On peut alors faire le même type d'argument semiclassique que celui que l'on utilise pour former les niveaux d'impuretés dans les semi-conducteurs (chap 28 aschcroft). On voit l'électron et le trou comme des particules de masses $m_c$ et $m_v$ (les masses effectives des bnades de conduction et de valence), que  nous prenons, pour simplifier, isotropiques. Elles interagissent à travers une interaction coulombienne attractive, écrantée par la constante diélectrique $\epsilon$ du cristal. Évidemment, c'est juste un problème d'atome d'hydrogène, avec la masse réduite de l'atome d'hydrogène $\mu$ remplacée par la masse réduite effective $m*$ (où $\frac{1}{m*} = \frac{1}{m_c} + \frac{1}{m_v}$), et la charge électronique remplacée par $\frac{e^2}{\epsilon}$. Par conséquent, il y a deux états liés, le plus bas en énergie s'étent au delà du rayon de Bohr donné par :

\begin{equation}
    a_{ex} = \frac{\hbar^2}{m*(e^2/\epsilon)} = \epsilon \frac{m}{m*} a_0
\end{equation}

Cette énergie de l'état lié sera plus faible que l'énergie $(\epsilon_c - \epsilon_v)$ du couple électron-troun qui n'intéragit pas, par

\begin{eqnarray}
    E_ex & = & \frac{e^2/\epsilon}{1a*_0} = \frac{m*}{m} \frac{1}{\epsilon^2} \frac{e^2}{2a0}\\
    & = & \frac{m*}{m} \frac{1}{\epsilon^2} (\SI{13.6}{\electronvolt})
\end{eqnarray}

La validité de ce modèle requiert que $a_{ex}$ soit grand à l'échelle du réseau cristallin (soit $a_{ex} >> a_0$), mais comme les isolants avec des gaps d'énergie tendent à avoir des masses effectives petites et de grandes constantes diélectriques, alors il n'estp as difficile de résoudre, particulièrement dans les semi-conducteurs. De tels spectres de l'hydrogène ont en fait été observés dans l'absorption otpqiue qui se produit plus bas que le seuil interbande.

L'exctiton décrit dans ce modèle est connu comme l'\emph{exciton de Mott-Wannier}. Évidemment, comme les niveaux atomiques parmi lesquels les niveaux de bande sont formés deviennent faiblement liés, $\epsilon$ diminuera, $m*$ augmentera, $a*_0$ diminuera, l'exciton deviendra plus localisé, et l'image de Mott-Wannier s'effondrera. L'exciton de Mott-Wannier et l'exciton de Frenkel sont les extrèmes opposés du même phénomène. Dans le cas de Frenkel, basé sur un niveau ionique excité simple, l'électron est le trou sont finement localisés à l'échelle atomique. Le spectre de l'exciton des solides de gas rares tombe dans ce cas.

    \chapter{Transport}

\TODO : ashcroft : p 621

\section{Conductivité électrique des cristaux ioniques}

Les cristaux ioniques, qui sont d'excellents isolants, ont une conductivité électrique qui ne s'annule pas. Les résistivitées typiques dépendent sensiblement de la température et de la pureté de l'échantillon, et peuvent se ranger, dans les cristaux d'halogénures alcalins, de \SI{e2}{} à \SI{e8}{\ohm\centi\metre} (qui doit être comparé avec la résistivité typique d'un métal, qui est de l'ordre du microhm-centimètre). La conduction ne peut pas être due à l'excitation thermique des électrons de la bande de valence à la bande de conduction, comme dans les semi-conducteurs, parce que la bande interdite est tellement grande que seuls très peu des \SI{e23}{} électrons peuvent s'exciter ainsi.
Il y a une preuve directe que la charge est portée non pas par les électrons, mais par les ions eux-même : après le passage d'un courrant, les atomes correspondant aux ions appropriés sont trouvés déposés sur les électrodes dans des nombres proportionnels à la charge totale transportée par le courrant.

La possibilité pour ces ions de conduire est énormément augmentée par la présence de lacunes. Cela requiert bien moins de travail de bouger une lacune à travers un cristal que de forcer un ion à travers un réseau ionique dense.

Il y a de très nombreuses preuves que la conduction ionique dépend du mouvement des lacunes. Il est observé que la conductivité aumente avec la température comme $\exp \frac{1}{T}$, ce qui reflète la dépendance en température de la concentration en lacunes à l'équilibre thermique. De plus, à de faibles températures, la conductivité d'un crystal ionique monovalent dopé avec des impuretés divalentes (par exemple \ch{Ca} dans \ch{NaCl}), est proportionnelle à la concentration en impureté divalentes, en dépit du fait que le matériau déposé sur la cathode continue à être l'atome monovalent.

La fonction importante de l'impureté est sa force, via une neutralité de charge, la création d'une lacune \ch{Na+} pour chaque ion \ch{Ca^2+} incorporé en substitution dans le réseau sur un site \ch{Na+}. Par conséquent, le plus de \ch{Ca} sont introduits, le plus grand nombre de lacunes \ch{Na+} il y a, et la conduction augmente alors.

\section{autres}

Dans les solides cristallins, les atomes en vibration échangent de l'énergie et occasionnelelement on peut avoir pour un atome une énergie bien plus élevée que la moyenne, permettant à cet atome de se déplacer vers un site adjacent inoccupé. Dans ce nouveau site il est piégé à nouveau jusqu'à un prochain saut. Les sauts atomiques sont donc des processus activés et ont pu être analysés à l'aide de la théorie absolue des vitesses de réaction. La probabilité pour qu'un saut atomique se produise par unité de temps dépend exponentiellement d'une enthalpie libre d'activation $G_m$ :

\begin{equation}
\mu = \mu_0 \exp (-G_m / kT)
\end{equation}

$\mu_0$ est généralement assimilé à la fréquence de vibration dans la direction du saut ($\approx 10^{13}$Hz). Les énergies d'activation sont typiquement de l'ordre de 2-3 eV mais nous verrons que pour certains matériaux à structure très ouverte elles tombent à 0.1 eV.

Plusieurs mécanismes ont été proposés pour rendre compte de la succession de sauts élémentaires. Le plus simple est probablement le mécanisme lacunaire ou diffusion de Schottky. Un certain nombre de sites sont vacants dans le réseau. Ils échangent leur position avec des sites voisins de même nature. Les atomes interstitiels peuvent se dépalcer directement par un saut vers un site interstitiel voisin. Ils peuvent aussi provoquer le déplacemnent direct ou indirect d'un autre atome.

\begin{figure}
\TODO
\caption{mécanismes de diffusion lacunaire, interstitiel direct, collinéaire, non-collinéaire}
\label{mecanismestransport}
\end{figure}

La relation la plus caractéristique des cristaux inoniques est la relation de Nernst-Einstein\footnote{c'est aussi une relation fondamentale dans les semi-conducteurs où ell es'applique au mouvement des électrons} qui relie le coefficient de diffusion à la conductivité ionique.

\section{Expression du coefficient de diffusion}

Le calcul du coefficient de diffusion se fait à l'aide de la théorie du mouvement Brownien : mouvement aléatoir ed particules sous l'effet de chocs. On définit le parcours quadratique moyen : $<R^2(t)>$.

On appelle coefficient de diffusion la quantité :
\begin{equation}
D = \frac{<R^2>}{6t}
\end{equation}

Dans le cas d'une maille cubique où la diffusion est isotrope pour N sauts, $R_N$ est un vecteur somme de N déplacements $r_i$ d'un atome :
\begin{equation}
R_N = \sum_{i=1}^N r_i
\end{equation}

%cf photo 05/10/2017, page 134-142

%    \include{chapter/conducteurs-ioniques}
%    \include{chapter/non-stoechiometrie}
%
    \part[TRANSPORT ÉLECTRONIQUE]{Transport électronique}
%
%    %\include{chapter/bande-interdite}
    \chapter{Transport d'électrons : le modèle de Drude}
phénomènes de conduction des électrons
hypothèses du modèle
vitesse entre deux collisions
libre parcours moyen
chaleur spécifique - capacité calorifique

réponse d'un gaz de drude en hautes fréquences
effet hall
statistique de maxwell boltzman

La théorie de Drude fournit des éléments de compréhensio des solides. C'est une
théorie classique, et c'est la première qui décrit les métaux. Elle comporte de
nombreux défauts, que nous expliquerons par les modèles qui suivent.

\section{phénomènes de conduction des électrons}

je sais plus ce que je voulais mettre ici.
\TODO

\section{Hypothèses du modèle}

Dans la théorie de Drude, on fait les hypothèses suivante :

- Les porteurs de charge sont les électrons qui constituent un gaz auquel on peut appliquer la théorie cinétique.

- Les porteurs de charge positive, beaucoup plus lourds, sont considérés comme immoobiles.

Seuls les électrons de valence des atomes participent à la conduction et les charges positives correspondent aux ions (noyau + électrons de cœur). Par conséquent, on obtient les informations suivantes :

- Entre les collisions, l'interaction d'un électron avec les autres électrons et avec les ions est nulle. Ainsi, en l'absence de champ extérieur, les électrons ont un mouvement rectiligne, uniforme et isotrope. En présence d'un champ extérieur, le mouvement des électrons entre les collisions est déterminé par les équations de Newton de la dynamique.

- les collisions sont des événements instantanés, qui modifient brusquement la vitesse des électrons. Cela revient à dire que l'on néglige totalement les forces à longue portée et on ne tient compte que des forces à courte portée.

- la probabilité qu'un électron subisse une collision durant l'intervalle de temps compris entre $t_0$ et $t_0 + dt$ est égale à $dt/\tau$. Elle est indépendante de $t_0$, de la position et de la vitesse de l'électron. On peut montrer que le temps moyen entre deux collisions successives d'un électron est égal à $\tau$. On note $\tau$ le temps de collision, temps de vol moyen et plus généralement \emph{temps de relaxation}.

- le gaz électronique atteint son équilibre themique par les collisions instantanées avec les ions. On admet que, arpès chaque collision, la vitesse d'un électron est l'orientation quelconque (distribution isotrope des vitesses juste après une collision) et que sa vitesse est reliée à la température locale (de la région de l'espace où a eu lieu la collision).

On fait donc deux hypothèses majeures :
\begin{description}
\item[approximation des électrons indépendants] on néglige l'interaction électron-électron, ce qui est valable dans de nombreux cas ;
\item[approximation des électrons libres] on néglige l'interaction entre électrons et ions. Ce modèle ne peut pas du tout définir les comportements différents des métaux et des semi-conducteurs.
\end{description}

\TODO

\section{vitesse entre deux collisions}

Si l'on impose un champ de \SI{1}{\volt\per\milli\metre} et un temps de relaxation $\tau$ de \SI{e-4}{\second}, alors $v = \SI{10}{\metre\per\second}$. Il s'agit de la vitesse de dérive des électrons.

Si on compare cette vitesse à $v_{th}$, alors

\begin{equation}
\sqrt{<v^2>} = \sqrt{\frac{3k_BT}{m}}
\end{equation}

Ce qui vaut, à 300K, $<v> \sim \SI{e5}{\metre\per\second}$.

Dans un métal, avec le modèle de Drude, l'ajout d'un champ est négligeable. On ne peut pas décrire les électrons comme dans un gaz :
deux atomes sont séparés d'environ \SI{2}{\angstrom}.

Ainsi, la densité d'électrons dans un métal est très grande devant la densité dans un gaz. Par exemple $n_e = 10^3 n_{air}$

\section{Statistique de Maxwell-Boltzman}

La distribution des charges dans un gaz est celle de Maxwell-Boltzmann, représentée sur la figure \ref{fig:maxwell-boltzmann}

\begin{marginfigure}
\TODO
\caption{Distribution de Maxwell-Boltzmann}
\label{fig:maxwell-boltzmann}
\end{marginfigure}

Cependant, Drude introduit les chocs entre particules : la formule $\sigma = \frac{ne^2\tau}{m}$ est valable et adaptable à la physique quantique.

\section{libre parcours moyen}

\begin{equation}
l = \sqrt{<v_{th}^2>} \tau
\end{equation}

Il s'agit du temps de parcours moyen d'une particule pendant un temps $\tau$, soit la distance typique entre deux chocs.
problème : il y a beaucoup plus de chocs par rapport au modèle de Drude. On est loin du gaz idéal (déviation représentée sur la figure \ref{fig:libreparcoursdrude})

\begin{marginfigure}
\TODO
\caption{Libre parcours moyen en fonction de la température ; déviation expérimentale par rapport à la prévision du modèle de Drude}
\label{fig:libreparcoursdrude}
\end{marginfigure}

$l$ dépend de la section efficace de l'ion.

Lorsque la température décroit, la section efficace diminue. de fait, la distance de choc diminue également.

\begin{equation}
S \sim T\quad \rightarrow l=\sqrt{T}
\end{equation}

Par conséquent,
\begin{equation}
\tau(T) \sim \frac{1/\sqrt{T}}{\sqrt{T}} \sim \frac{1}{T}
\end{equation}

Soit :
\begin{equation}
\rho = \frac{1}{\sigma} = \frac{m}{q^2nE} \sim \tau
\end{equation}

Finalement, le modèle de Drude permet de modéliser de façon plutôt fidèle l'évolution linéaire.
En revanche, on perd en performance dans le creux. Cela est dû aux impuretés.

\section{chaleur spécifique - capacité calorifique}

Soit $\mathcal{E}$ l'énergie totale du système. On peut écrire la chaleur spécifique $C$ comme :

\begin{equation}
C = n \frac{d\mathcal{E}}{dT}
\end{equation}

Cette capacité calorifique sert à repérer les transitions de phase. Si $C$ n'est pas lisse, alors il y a une transition de phase ou une transition chimique qui se produit. Cela est facile à mesurer lorsque l'énergie augmente. (à \SI{100}{\celsius} pour l'eau, $dI = 0$, ce qui divege ?????\TODO

Pour un gaz thermo, $\mathcal{E} = \frac{3}{2}k_BT$ donc

\begin{equation}
C = \frac{3}{2} n k_B
\end{equation}



\section{réponse d'un gaz de Drude à hautes fréquences}

\subsection{conduction de Drude, AC}

\TODO : d'où sort cette formule

\begin{equation}
\frac{d\mathbf{p}}{dt} = q\mathbf{E} - \frac{1}{\tau} \mathbf{p}
\label{eq:drudechampelec}
\end{equation}

On peut poser $\mathbf{E}(t) = \mathbf{E}(\omega) e^{-i\omega t}$, et chercher de cette façon $\mathbf{p}(t) = \mathbf{p}(\omega) e^{-i\omega t}$.

En injectant cela dans l'équation \ref{eq:drudechampelec}, on trouve :

\begin{equation}
-i\omega \mathbf{p}(\omega) = q\mathbf{E}(\omega) - \frac{1}{\tau} \mathbf{p}(\omega) 
\end{equation}

Soit :

\begin{equation}
\mathbf{p}(\omega) = \frac{\mathbf{p}(\omega)}{\frac{1}{\tau}-i\omega}
\end{equation}

On peut écrire la densité de courant en fonction de la fréquence. Comme $j(t) = j(\omega) e^{-i\omega t}$.

De plus, $j(t) = nqv = \frac{nq}{m}\mathbf{p}(t)$.

On obtient alors :
\begin{equation}
j(\omega) = \frac{nq}{m}\mathbf{p}(\omega) = \frac{nq^2/m}{\frac{1}{\tau}-i\omega} \mathbf{E}(\omega)
\end{equation}

\TODO


\section{Effet Hall}

    \chapter{Équations du mouvement}

Nous démontrons les équations du mouvement d'un électron dans une bande
d'énergie. Nous regardons le mouvement d'un paquet d'ondes dans un champ
électrique donné. Supposons que le paquet d'ondes est fait de fonctions d'ondes
assemblées au voisinage d'un vecteur d'onde $k$. La vitesse de groupe, est par
définition $v_g = \frac{d\omega}{dk}$. La fréquence associée à cette fonction
d'onde d'énergie $\epsilon$, dans la théorie quantique est $\omega =
\frac{\epsilon}{\hbar}$, et ainsi :

\begin{equation}
    v_g = \frac{1}{\hbar} \frac{d\epsilon}{dk}
\end{equation}

Les effets du cristal dues au mouvement d'électrons sont contenus dans la
relation de dispersion $\epsilon(k)$.

Le travail $\delta \epsilon$ fourin par le champ électrique E sur l'électron en
un temps $\delta t$ est :
\begin{equation}
    \delta \epsilon = - e E v_g \delta t
\end{equation}
On peut observer que :
\begin{equation}
    \delta \epsilon = (\frac{d\epsilon}{dk})\delta k = \hbar v_g \delta k
\end{equation}

En combinant ces trois  équations, on obtient :
\begin{equation}
    \delta k = -(eE/\hbar)\delta t
\end{equation}
d'où $\hbar \frac{dk}{dt} = -eE$
On peut réécrire cette équation en faisant intervenir la force $\mathbf{F}$ : 
\begin{equation}
    \hbar \frac{d\mathbf{k}}{dt} = \mathbf{F}
\end{equation}

C'est une relation importante : dans un cristal, $\hbar d\mathbf{k}/dt$ est égal
à la force subie par l'électron. Dans l'espace libre, $d(m\mathbf{v})/dt$ est
égal à la force. Mais attention, on n'est pas en contradiction avec la seconde
loi de Newton : l'électron dans le cristal est sujet à des forces du réseau
cristallin autant que de sources externes.

Le terme de force dans cette équation inclut aussi un champ électrique et une
force de Lorentz sur un élecrton dans un champ magnétique, sous certaines
conditions, où le champ magnétique est assez faible pour ne pas casser la
structure de bandes. Ainsi, l'équiation du mouvement d'un élecrton, d'une vitesse
de groupe $\mathbf{v}$ dans un champ magnétique constant est :
\begin{equation}
    \hbar \frac{d\mathbf{k}}{dt} = -e\mathbf{v}\times{B}
\end{equation}

où le terme de droite est la force de Lorentw sur l'électron. Avec la vitesse de
groupe $\mathbf{v} = \frac{1}{\hbar} \grad_\mathbf{k}\mathbf{\epsilon}$, le taux
de variation du vecteur d'onde est
\begin{equation}
    \frac{d\mathbf{k}}{dt} = -\frac{e}{\hbar} \Delta_\mathbf{k}\mathbf{\epsilon}
    \times \mathbf{B}
\end{equation}
où chaque côté de l'équation se réfère aux coordonnées de l'espace réciproque.

Nous voyons que le produit vectoriel dans cette dernière équation, que dans un
champ magnétique, un électron se déplace dans l'espace réciproque dans la
direction normale à la direction du gradient d'énergie $\epsilon$, de telle sorte
à ce qu'\emph{un électron se déplace sur les surfaces isoénergétiques}. La valeur
de la profection $k_\mathbf{B}$ de $\mathbf{k}$ sur $\mathbf{B}$ est constante
durant le mouvement. Le mouvement dans l'espace réciproque est sur un plan normal
à la direction de $\mathbf{B}$, et l'orbite est définie par l'intersection de ce
plan avec la surface isoénergétique.

\section{démonstration physique de $\hbar \mathbf{k} = \mathbf{F}$}
Considérons la fonction propre de Bloch $\psi_k$, associée à l'énergie propre
$\mathbf{\epsilon_k}$ et au vecteur d'onde $\mathbf{k}$ :
\begin{equation}
    \psi_\mathbf{k} = \sum_\mathbf{G} C(\mathbf{k+G})\exp [i(\mathbf{k+G})\cdot
    \mathbf{r}]
\end{equation}
La valeur attendue du moment d'un électron dans l'état $\mathbf{k}$ de Bloch est
:
\begin{equation}
    \mathbf{P_el} = <\mathbf{k}|-i\hbar\Delta|\mathbf{k}> = \sum_\mathbf{G} \hbar
    (\mathbf{k+G}) |C(\mathbf{k+G})|^2 = \hbar (\mathbf{k} + \sum_\mathbf{G}
    \mathbf{G}|C(\mathbf{k+G})|^2)
\end{equation}
en utilisant l'égalité $\sum |C(\mathbf{k+G})|^2 = 1$

Nousévaluons le transfert de moments entre l'électron et le réseau, lorsque
l'électron pass de $\mathbf{k}$ à $\mathbf{k}+\Delta\mathbf{k}$, à cause d'une
force extérieure. Nous imaginons un cristal isolant, neutre électrostatiquement,
excepté pour un seul électron dans l'espace réciproque d'une autre bnade vide.
Supposons qu'une force externe faible soit appliquée pour un intervalle de temps
tel que l'implusion totale donnée au système cristallin entier soit $\mathbf{J} =
\int \mathbf{F} dt$. Si l'électron de conduction est libre ($m*=m$), alors le
moment total transim au système cristallin par l'impulsion apparaîtrait comme la
différentiell edu moment de l'électron de conduction :
\begin{equation}
    \mathbf{J} = \Delta\mathbf{p_{tot}} = \Delta\mathbf{p_{el}} = \hbar \Delta
    \mathbf{k}
\end{equation}
Le cristal neutre ne subut aucune interaction avec le champ électrique, que ce
soit directement ou indirectement à travers les électrons libres.
Si l'électron de conduction interagit avec le potentiel périodique du réseau
cristallin, nous devons alors avoir :
\begin{equation}
    \mathbf{J} = \Delta{p_{tot}} = \Delta \mathbf{p_{rés}} +
    \Delta\mathbf{p_{el}}
\end{equation}
Ce qui donne :
\begin{equation}
    \Delta \mathbf{p_{el}} = \hbar \Delta \mathbf{k} + \sum_\mathbf{G}
    \hbar\mathbf{G}[(\nabla_k|C(\mathbf{k+G})|^2)\cdot\Delta\mathbf{k}]
\end{equation}
La variation $\Delta \mathbf{p_{rés}}$ dans le moment du réseau résultant de la
variation de l'état de l'électron peut être démontrée par des considérations
physiques élémentaires. Un élecrton, réfléchi par le réseau, transfeère un moment
au réseau. Si un électron indicdent avec une composante du moment sous forme
d'onde plane $\hbar \mathbf{k}$ est réfléchi avec un moment
$\hbar(\mathbf{k+G})$, le réseau acquierre un moment $-\hbar \mathbf{G}$, c'est
la conservation du moment. Le transfert de moments au réseau lorsque l'état varie
de $\psi_\mathbf{k}$ à $\psi_{\mathbf{k+\Delta k}}$ est :
\begin{equation}
    \Delta \mathbf{p_{rés}} = -\hbar \sum_\mathbf{G}
    \mathbf{G}[(\nabla_k|C(\mathbf{k+G})|^2)\cdot\Delta\mathbf{k}]
\end{equation}
car la partie $(\nabla_k|C(\mathbf{k+G})|^2)\cdot\Delta\mathbf{k}$ de chaque
composante individuelle de l'état initial est réfléchie pendat le chamgement
d'état $\Delta \mathbf{k}$. La différentielle du moment devient alors :
\begin{equation}
    \Delta \mathbf{p_{el}} + \Delta \mathbf{p_{rés}} = \mathbf{J} = \hbar \Delta
    \mathbf{k}
\end{equation}
ce qui est exactement la forme pour des électrons libres. Par conséquent, à
partir de la définition de $\mathbf{J}$, on obtient :
\begin{equation}
    \hbar \frac{d\mathbf{k}}{dt} = \mathbf{F}
\end{equation}
démontré plus tôt par une méthode différente. Une démonstration riguoureuse est
page 655 de Kittel. à regarder. peut-être à ajouter. à voir.

\section{Trous}

Les propriétés des orbitales vacantes dans une bande remplie sont importantes
dans la physique des semiconductors et dans l'électronique de l'état solide. Les
orbitales vacantes dans une bande sont communément appelés trous, et sans trous,
il n'y aurait pas de transistors. Un trou agit dans un champ électrique et
magnétique, comme s'il avait une charge positive $+e$. Cela peut être démontré de
la façon suivante :

page 194 à 196 kittel.

\section{Masse effective}
Lorsque l'on regarde à la relation d'onde en énergie $\epsilon =
(\frac{\hbar^2}{2m})k^2$ pour des électrons libres, on voit que le coefficient de
$k^2$ détermine la courbure de $\epsilon$ en fonction de $k$. Ainsi, on peut dire
que $1/m$, la masse réciproque, détermine la courbure. Pour des électrons dans
une bande, il y a des réguions de courbure anormalement hautes, près de la bande
interdite, et aux frontières, comme on a pu le voir précédemment, à partir des
solutions de l'équation d'onde aux limites de zone de Brillouin. Si l'énergie du
gap est petite devant l'anérgie $\lambda$ de l'électron libre à la limite de la
zone, la courbure est augmentée d'un facteur $\lambda/E_g$.

Dans les semiconducteurs, la largeur de bande, qui est telle que l'énergie de
l'électron libre est de l'ordre de 20eV, tandis que le bandgap est de l'ordre de
0.2 à 2eV. Ainsi, la masse réciproque est augmentée d'un facteur 10 ou 100, et la
masse effective est réduite au dixième ou au centième de la masse de l'électron
libre. Ces valeurs s'appliquent près du bandgap, lorque l'on s'en éloigne, la
courbure et les masses se rapprochent de celles de l'électron libre.

Pour résumer les solutions que nous avons vu précedemment pour U positif, un
électron proche de la partie la plus basse de la seconde bande a une énergie qui
peut être écrite comme :
\begin{equation}
    \epsilon(K) = \epsilon_c + (\hbar^2/2m_e)K^2 ; \hfill m_e/m =
    1/[(2\lambda/U)-1]
\end{equation}
Ici, K est le vecteur d'onde mesuré de la bordure de zone, et $m_e$ représente la
masse effective d e l'électon proche de la frontière de la seconde bande. Un
électron proche du haut de la première bande a l'énergie :
\begin{equation}
    \epsilon(K) = \epsilon_v - (\hbar^2/2m_h)K^2 ; \hfill m_h/m =
    1/[(2\lambda/U)+1]
\end{equation}
cf tut pour ajout d'informations.

La courbure et par conséquent la masse seront négatives proches du haut de la
première bande, mais nous avons introduit un signe moins dans la dernière
équation, de sorte à ce que le symbole $m_h$ pour la masse du trou ait une valeur
positive.

Le cristal ne pèse pas moins si la masse effective d'un porteur est mois que la
masse d'un électron libre, et de même, la seconde loi de Newton est respecté pour
le cristal \emph{pris dans son ensemble}, avec les ions plus les porteurs. Le
point important est qu'un électron dans un potentiel périodique est accéléré
relativement au résaeu dans un champ électrique ou magéntique, comme si la masse
de l'électron était égale à la masse éffective que nous venons de définir.

On dérive le résultat pour la vitess ede groupe et on obtient alors :

\begin{equation}
    \frac{dv_g}{dt} = \frac{1}{hbar} \frac{d^2\epsilon}{dk dt} = \frac{1}{\hbar}
    \left( \frac{d^2e}{dk^2} \frac{dk}{dt}\right)
\end{equation}
On sait depuis la section précédete que $dk/dt = F/\hbar$, d'où on tire :

\begin{equation}
    \frac{dv_g}{dt} = \left( \frac{1}{\hbar^2} \frac{d^2\epsilon}{dk^2} \right) F
    ; \text{ soit } F = \frac{\hbar^2}{d^2 \epsilon/dk^2} \frac{dv_g}{dt}
\end{equation}
En identifiant $\hbar^2/(d^2\epsilon/dk^2)$ comme une masse, alors la précédente
équation peut se voir comme une formulation de la seconde loi de Newton. On
définit la masse effective $m*$ par :

\begin{equation}
    \frac{1}{m*} = \frac{1}{\hbar^2} \frac{d^2\epsilon}{dk^2}
\end{equation}
Il devient facile de gééraliser ceci pour prendre en compte la surface d'énergie
d'un électron anisotropique, comme pour les électrons du silicium ou du
germanium. On introduit alors les composantes du tenseur de masse éffective
réciproque :

\begin{equation}
    \left( \frac{1}{m*} \right)_{\mu \nu} = \frac{1}{\hbar}
    \frac{d^2\epsilon_k}{dk_\mu dk_\nu} ; \hfill \frac{dv_\mu}{dt} =
    \left(\frac{1}{m*}\right)_{\mu \nu} F_\nu
\end{equation}
où $\mu, \nu$ sont les coordonnées cartésiennes

\section{interprétation physique de la masse effective}
Comment un électron de masse $m$ mis dans un crital répond-il à un champ appliqué
comme si sa masse était $m*$ ? Il est utile de penser au processus de réflection
de Bragg d'ondes délectrons dans un réseau. Considérons l'approximation de
l'interaction faible, traitée précédemment. Près du bas de la bande faible,
l'orbitale est représentée de façon plutôt juste par une onde plane $\exp(ikx)$
avec un moment $\hbar k$. La composant ede l'onde $\exp(i(k-G)x)$ avec le moment
$\hbar(k-G)$ est petite est augmente faiblement avec k, et dans cette région
$m*\simeq$. Une augemntation de la composante réfléchie $\exp(i(k-G)x)$ avec k
représente un transfert du moment du réseau sur l'électron.

Près des limites de zone, la composante réfléchie est importante ; à la limite,
elle devient égale en amplitude avec la composante directe, et en ce point les
fonctions propres sont des ondes stationnaires, plutôt que propagatives. Ici, la
composante du moment $\hbar(-1/2 G)$ annule la composante du moment $\hbar(1/2
G)$.

Un électron seul dans une bande d'énergie peut avoir une masse effective positive
ou négative : les états de masse effective positive apparaissent près du bas de
la bande, à cause du fait que la masse effective positive implique que la bande a
une courbure vers le haut ($d^2\epsilon/dk^2$ positif). Les états de masse
effective négative apparaissent près du haut de la bnade. Une masse effective
négative signifique qu'en passant d'un état $k$ à un état $k+\Delta k$, le
transfert de moment de l'électron au réseau est plus grand que le transfer de
moment de la force appliquée à l'électron. Même si $k$ est augmenté de $\Delta k
$ par le champ électrique appliqué, l'approche de la réflexion de Bragg peut
donner une décroissance globale du moment direct des électrons, quand cela
arrive, la masse effective est négative.

Lorsque nous travaillons sur la seconde bande, loin des bords de zone,
l'amplitude de $\exp[i(k-G)x]$ décroit rapidement et $m*$ a une valeur positive
faible. Ici, l'augmentation de la vitesse des électrons, résultant d'un stimulus
extérieur donné est plus grande que celle qu'aurait donné l'expérience d'un
électro libre. Le réseau construit une différence à travers le léger recul qu'il
subit lorsque l'amplitude de $\exp[i(k-G)x]$ diminue.

Si l'énergie dans une bande dépend seulement faiblement de $k$, alors la masse
effective sera très importante. Cela étant, $m*/m \gg 1$ lorsque $d^2\epsilon /
dk^2 \ll 1$. Le modèle des liaisons fortes, discuté précedemment, donne un aperçu
de la formation de bandes étroites. Si les fonctions d'onde centrées aux
voisinages d'atomes se superposent un peu, alors l'intégrale de recouvrement sera
petite ; la largeur de bande étroite, et la masse effective importantee. La
superposition des fonctions d'ode centrée sur un atome voisin est petite pour les
électrons internes ou de cœur. Les électrons 4f des métaux rares, par exemple, se
superposent faiblement.

\section{Masses effectives dans les semiconducteurs}

Dans beaucoup de semiconducteurs, il a été possible de déterminer, par résonnance
cyclotron, la masse effective des porteurs des bandes de valence et de conduction
près des bords de bandes. La détermination de la surface d'énergie est
équivalente à la détermination du tenser deu masse effective. La résonnance
cyclotron dans un semiconducteur est effectuée avec des ondes d'une longueur
d'onde du milli ou du centimètre, à de faibles concentrations en porteurs.

Les porteurs de courrant sont accélérés sur des horbites hélicoïdales sur l'axe
d'un champ magnétique statique. La fréquence de rotation angulaire $\omega_c$ est
donnée par :
\begin{equation}
\omega_c = \frac{eB}{m*}
\end{equation}
où $m*$ est la masse effective du cyclotron approprié. L'absorption de résonnance
de l'énergie pour un champ électrique rf perpendiculaire au champ magnétiqu
statique apparaît lorsque la fréquence rf est égale à la fréquence cyclotron. Les
trous et les électrons tournent dans des sens opposés dans un champ magnétique.

On considère l'expérience pour $m*/m=0.1$. À $f_c=24GHz$ ou $\omega_c = 1.5
\times 10^{11} s^{-1}$, nous avons $ B = 860 G$ à la résonnance. La largeur de
ligne est déterminée par le temps de relaxation de collision $\tau$, et pour
obtenir une résonnance distinctive, il est nécessaire que $\omega_c \tau \geq 1$.
Le libre parcours moyen doit être suffisemmet long pour permettre aux porteurs
moyens de parcourir un radian autour d'un cercle avant d'entrer en collision. Il
est nécessaire d'utiliser une fréquence importante et un champ magnétique
importante, avec des cristaux purs dans l'hélium liquide.

Dans des semiconducteurs à gap direct, avec des bords de bandes aux centre de la
zone de Brillouin, les bandes ont la structure donnée par la figure suivante
\TODO figure 13kittel. Le bord de la bande de coduction est sphérique avec la
masse effective $m_0$ :
\begin{equation}
    \epsilon_c = E_g + \hbar^2 k^2 / 2m_e
\end{equation}

La bande de valence est généralement repliée près du bord, avec les bandes des
trous lourds hh et des trous légers lh dégénérées au centre, et une bande soh
splittée par le couplage spin-orbite $\Delta$ :
\begin{eqnarray}
    \epsilon_v(hh) \simeq -\hbar^2 k^2/2m_{hh}\\
    \epsilon_v(lh) \simeq -\hbar^2 k^2/2m_{lh}\\
    \epsilon_v(soh) \simeq - \Delta - \hbar^2 k^2/2m_{soh}
\end{eqnarray}

Les valeurs des paramètres de masse sont données dans le tableau suivant
\TODO tableau (page 201 kittel)

Les formes données ici ne sont que des approximations, parce que même proche de
$k = 0$, les bandes des trous lourds et légers ne sont pas sphériques, comme nous
allons le montrer dans le cas du Silicium et du Gemanium.

La théorie des perturbations aux bords des bandes usggère que la masse effective
de l'électron doit être proportionelle au bandgap, approximativement, pour un
cristal à gap direct. Les deux tables précédentent montrent que ces valeurs sont
cnastantes pour la série InSb, InAs et InP, conformément à cette hypothèse

\section{Silicium et Germanium}

Les bandes de conduction et de valence du germanium sont montrées sur la figure
suivante, basés sur une combinaisons de résultats théoriques et expérimentaux. Le
bord de la bande de valence, à la fois pour le silicium et le germanium, est à
$\mathbf{k}=0$ et cela peut être démontré à partir des états $p_{3/2}$ et
$p_{1/2}$ des atomes libres, clairement dans le modèle des liaisons fortes pour
les fonctions d'onde.

Le niveau $p_{3/2}$ est quadruplement dégénéré comme pour l'atome ; les quatre
états correspondent à des valeurs de $m_J$ de $\pm \frac{3}{2}$ et $\pm
\frac{1}{2}$. Le niveau $p_{1/2}$ est doublement dégénéré, avec $m_J = \pm
\frac{1}{2}$. Les niveau $p_{3/2}$ sont plus haut en énergie que les niveaux
$p_{1/2}$ ; la différence d'énergie $\Delta$ est une mesure de l'interaction
spin-orbite.

Les bords de la bande de valence ne sont pas simples. Les trous près du bord de
la bande sont caractérisés par deux masses effectives, légère et lourde.
Celles-ci proviennent de deux bandes formées des niveaux $p_{3/2}$ des atomes. Il
y a également une bande formée depuis le niveau $p_{1/2}$, séparé du niveau
$p_{3/2}$ par le couplage spin-orbite. Les surfaces d'énergie ne sont pas
sphériques, mais déformées :
\begin{equation}
    \epsilon(\mathbf{k}) = Ak^2 \pm [B^2 k^4 + C^2(k_x^2k_y^2 + k_y^2k_z^2 +
    k_z^2k_x^2)]^{1/2}
\end{equation}
Le choix du signe distingue les deux masses. La bande séparée correspond à
l'équation $\epsilon(k) = -\Delta + Ak^2$. L'expérience donne, en unités
$\hbar^2/2m$ :

\begin{tabular}{lllll}
Si : & A = -4.29 ; & |B| = 0.68 ; & |C|=4.87 ; & $\Delta$=0.044eV \\
Ge : & A = -13.38; & |B| = 8.48; & |C| = 13.15 ; & $\Delta$=0.29eV
\end{tabular}

Grosso modo, les trous légiers et lourds du germanium ont une masse de 0.043m et
0.34m. Dans le silicium, 0.16m et 0.52m. Dans le diamant, 0.7m et 2.12m.

Les bords de la bande de conduction dans Ge sont à des points équivalents L de la
zone de Brillouin. Chaque bord de bande a une surface d'énergie d'orientation
sphéroïdale, suivant l'axe <111>, avec une masse longitudinale $m_l = 1.59m$ et
une masse transverse $m_t = 0.082m$. Pour un champ magnétique statique à un angle
$\theta$ avec l'axe longitudinal d'une sphéroide, la masse effective
cyclotronique $m_c$ est :
\begin{equation}
    \frac{1}{m_c} = \frac{\cos^2\theta}{m_t^2} + \frac{\sin^2 \theta}{m_t m_l}
\end{equation}

Les résultas pour le germanium sont motrées sur la figure ... (masse effective
cyclotronique)
Dans le silicium, las bords de la bnade de conduction sont des sphéroides
orietées selon les directions équivalentes <100> dans la zone de Brillouin, avec
des paramètres de masse $m_l = 0.92m$ et $m_t = 0.19m$, motnré sur la figure ...
Les bords de bande se placent selon les lignes marqées $\Delta$ dans la zone, un
peu déplacées des bords X.

Dans GaAs, on a A = -6.98, B = -4.5, |C| = 6.2, $\Delta$= 0.341eV. la structure
de bande est montrée ... il y a un gap direct avec une masse des électron de
conduction isotropique de 0.067m.

\TODO : points classiques pour réseaux bcc, fcc, etc. gamma, delta, sigma, G, N,
P, etc. et rajouter les autres en annexes



\begin{marginfigure}
\TODO
\caption{bande du germanium}
\label{bandeGe}
\end{marginfigure}

    \chapter{Concentration en porteurs intrinsèques}

Nous souhaitons la concentration en porteurs intrinsèques comme une fonction de
la température, en fonction du bandgap. Nous calculons cela pour un bord de bande
simple parabolique. D'abord, on calcule en terme du potentiel chimique $\mu$, le
nombre d'électrons excités dans la bande de conduction à la température T. Dans
la physique des semiconducteurs. $\mu$ est appeeelé le niveau de Fermi. À la
température d'intérêt, on peut supposer que pour la bande de conduction d'un
semiconducteur, $\epsilon - \mu \gg k_B T$, et par conséquent la distribution de
Fermi-Dirac se réduit à :
\begin{equation}
    f_e \simeq \exp[(\mu - \epsilon)/k_B T]
\end{equation}

C'est  la probabilité que l'orbitale d'un électron de conduction est occupée,
dans une approximation valide lorsque $f_e \ll 1$.

L'énergie d'un électro dans la bande de conduction est :
\begin{equation}
    \epsilon_k * E_c + \hbar^2k^2/2m_e
\end{equation}

où $E_c$ est l'énergie au bord de la bande de conduction. Ici, $m_e$ est la masse
effective d'un électron. Par conéquent, la densité d'état à $\epsilon$ est :
\begin{equation}
    D_e(\epsilon) = \frac{1}{2\pi^2}
    \left( \frac{2m_e}{\hbar^2}\right)^{3/2}(\epsilon - E_c)^{1/2}
\end{equation}
La concentration des électrons dans la bande de conduction est :
\begin{equation}
    n = \int_{E_c}^\infty D_e(\epsilon)f_e(\epsilon) d\epsilon = \frac{1}{2\pi^2}
    \left( \frac{2m_e}{\hbar^2}\right)^{3/2} \exp(\frac{\mu}{k_BT})\times
    \int_{E_c}^\infty (\epsilon - E_c)^{1/2} \exp(-\frac{\epsilon}{k_B
    T})d\epsilon
    \label{integn}
\end{equation}

Soit en intégrant, on obtient :
\begin{equation}
    n = 2 \left(\frac{m_e k_B T}{2 \pi \hbar^2}\right)^{3/2} \exp\left[\frac{(\mu
    - E_c)}{k_BT}\right]
    \label{n}
\end{equation}

Le problème est résolu pour n lorsque $\mu$ est connu. C'est pratique de calculer
la concentration des trous p à l'équilibre. La distribution $f_h$ pour les trous
est reliée à la distribution des électrons $f_e$ par $f_h + f_e = 1$, parce qu'un
trou est l'absence d'un électron. On a alors, dans le cas où $(\mu - \epsilon)
\gg k_BT$ :
\begin{align}
    f_h & = & 1 - \frac{1}{\exp\left[\frac{\epsilon - \mu}{k_BT}\right] + 1} =
    \frac{1}{\exp\left[\frac{\mu-\epsilon}{k_BT}\right] + 1}\\
    & \cong & \exp\left[\frac{\epsilon - \mu}{k_BT}\right]
    \label{fh}
\end{align}

Si les trous près du haut de la bande de valence se comportent comme des
particules avec une masse effective $m_h$, la densité des trous est donnée par :
\begin{equation}
    D_h(\epsilon) = \frac{1}{2\pi^2}\left( \frac{2m_h}{\hbar^2} \right)^{3/2}
    (E_v - \epsilon)^{1/2}
\end{equation}

Où $E_v$ est l'énergie au bord de la bande de valence. Si on fait de la même
façon que pour \ref{integn}, on obtient alors :

\begin{equation}
    p =r \int_{-\infty}^{E_c}D_h(\epsilon) f_h(\epsilon) d\epsilon = 2 \left(
    \frac{m_hk_BT}{2\pi\hbar^2} \right)^{3/2} \exp\left[ \frac{E_c - \mu}{k_BT}
    \right]
\end{equation}

pour la concentration $p$ en trous dans la bande de valence.

On peut multiplier ces expressions ensemble pour $n$ et $p$ et ainsi obtenir la
relation d'équilibre, avec l'énergie du gap $E_g = E_c - E_v$ :
\begin{equation}
    np = 4 \left( \frac{k_BT}{2\pi\hbar^2} \right)^3 (m_c m_h)^{3/2}
    \exp(-E_g/k_BT)
    \label{np}
\end{equation}

Ce résultat utile ne fait pas intervenir le niveau de Fermi $\mu$. À 300K, la
valeur de $np$ est \SI{2.10e19}{\centi\metre\tothe{-6}} pour le silicium,
\SI{2.89e26}{\centi\metre\tothe{-6}} pour le germanium et
\SI{6.55e12}{\centi\metre\tothe{-6}} pour l'arsenure de gallium.

Nous n'avons jamais, dans cette démonstration, supposé que la matériau est
intrisèque : le résultat tient pour l'ionisation d'impuretés aussi bien. La seul
hypothèse faite, est que la distance du niveau de Fermi depuis les bords des
bandes est grande devant $k_BT$.

Un simple argument cinétique montre en quoi le produit $np$ est constant à une
température donnée. Supposons que la population d'équilibre d'électrons et de
trous est maintenue par les photons de radiation du corps noir à une température
\begin{equation}
    \frac{dn}{dt} = A(T) - B(T)np = \frac{dp}{dt}
\end{equation}
À l'équilibre, $\frac{dn}{dt} = 0$,$\frac{dp}{dt} = 0$, d'où $np =
\frac{A(T)}{B(T)}$.

Comme le produit des concentrations en électrons et en trous est une constante
indépendante de la concentration en impuretés à une température donnée,
l'introduction d'une petite proportion d'une impureté convenable, pour augmenter
$n$(par exemple), devrait réduire $p$.
Ce résultat est important dans la pratique : on peut réduire la concentration
totale en porteurs $n+p$ dans un cristal impur, parfois énormément, en
controllant l'introduction d'impuretés. Une telle réduction est appelée
\emph{compensation}.

Dans un semiconducteur intrinsèque, le nombre d'électrons est égal au nombre de
trous parce que l'excitation thermique d'un électron laisse derrière un trou dans
la bande de valence. Ainsi, d'après \ref{np}, on peut obtenir la relation
suivante, où $i$ représente $n$ ou $p$ et $E_g = E_c - E_v$ :
\begin{equation}
    n_i = p_i = 2 \left( \frac{k_BT}{2\pi\hbar^2} \right)^{3/2} (m_em_h)^{3/4}
    \exp(-E_g / 2k_BT)
\end{equation}

La concentration en porteurs intrinsèques dépend de façon exponentielle de
$E_g/2k_BT$, où $E_g$ est l'énergie du gap. On peut réunir les équations \ref{n}
et \ref{fh} pour obtenir, pour le niveau de Fermi mesuré du bord supérieur de la
bande de valence :
\begin{equation}
    \exp(2\mu/k_BT) = (m_h/m_e)^{3/2}\exp(E_g/k_BT)
\end{equation}
Ce qui donne le niveau de Fermi :
\begin{equation}
    \mu = \frac{1}{2}E_g + \frac{3}{4} k_BT \ln(m_h/m_e)
\end{equation}

Si $m_h = m_e$, alors $\mu = \frac{1}{2}E_g$, et le niveau de Fermi est au milieu
de la bande interdite.

\section{Mobilité intrisèque}

La mobilité est l'intensité de la vitesse de dérive d'un porteur de charge par
unité de champ électrique :
\begin{equation}
    \mu = \frac{|v|}{E}
\end{equation}
La mobilité est définie comme étant positive à la fois pour les électrons et les
trous, même si leurs vitesses de dérives sont opposées pour un champ électrique
donné. En écrivant $\mu_e$ ou $\mu_h$, les mobilités des électrons ou des trous,
on peut éviter toute confusion avec $\mu$ le potentiel chimique.

La conductivité électrique est la somme des contributions des électrons et des
trous :
\begin{equation}
    \sigma = (ne\mu_e + pe\mu_h)
\end{equation}
où $n$ et $p$ sont les concentrations en électrons et en trous. Dans le chapitre
..., la vitesse de dérive d'une charge q a été établie comme $v = q\tau E/m$ avec
:
\begin{equation}
    \mu_e = e \tau_e / m_e ; \hfill \mu_h = e \tau_h / m_h
\end{equation}
avec $\tau$ le temps de collision.

Les mobilités dépendent de la température, avec une loi de puissance faible. La
dépendance en température de la conductivité dans la région intrinsèque sera
dominée par la dépendance exponentielle $\exp(-E_g/2k_BT)$ de la concentration en
pourteurs.

Le tableau suivant donne les valeurs expérimentales des mobilitésà température
ambiante pour certains semiconducteurs classiques. Pour la plupart des matériaux,
les valeurs données sont limitées par la diffusion de charge par les phonons
thermiques. Les mobilités des trous sont générallement plus petites que les
mobilités électroniques, parce que l'apparition d'une dégénérescence au bord de
la bande de valence au centre de la zone de Brillouin rend possible une diffusion
interbande qui réduit considérablement la mobilité.

%\begin{table}[ht]
%\begin{tabular}{lrrlrr}
%\toprule
%Cristal & électrons & trous & cristal & électrons & trous \\
%\midrule
%Diamant & 0.18 & 0.12 & GaAs & 0.8 & 0.03 \\
%Si & 0.135 & 0.048 & GaSb & 0.5 & 0.1 \\
%Ge & 0.36 & 0.18 & PbS & 0.055 & 0.06 \\
%InSb & 0.08 & 0.045 & PbSe & 0.102 & 0.093 \\
%InAs & 3 & 0.045 & PbTe & 0.25 & 0.1 \\
%InP & 0.45 & 0.01 & AgCl & 0.005 & -- \\
%AlAs & 0.028 & -- & KBr(100K) & 0.01 & -- \\
%AlSb & 0.09 & 0.04 & SiC & 0.01 & 0.001-0.002\\
%\bottomrule
%\end{tabular}
%\label{}
%\caption{Mobilités des porteurs de charge à température ambiante ($m^2/V-s$)}
%\end{table}

Dans certains cristaux, particulièrement les cristaux ioniques, les trous sont
essentielleent immobiles et ne se déplacent que par \emph{hopping} (saut) d'un
ion à un autre, activé thermiquement. La principale cause de cet
\emph{auto-piégeage} est la distortion du réseau, associée à un effet Jahn-Teller
des états dégénérés. La dégénérescence des orbitales nécessaire pour un
\emph{auto-piégeage} est plus fréquente pour les trous que pour les électrons.

Il y a une tendance remarquable pour les critsaux avec de petits gaps et des
bords directs : ils ont de grandes valeurs de mobilité électronqiue. Les petits
gaps donnent une masse effective petite, qui favorise les hautes mobilités. La
plus grande mobilité observée dans le bulk d'un semiconducteur est
\SI{5e2}{\square\metre\per\volt\per\second} dans le PbTe à 4K, pour lequel le gap
est de \SI{0.19}{\electronvolt}.

    \chapter{Conductivité due aux impuretés}

Certaines impuretés et imperfections affectent dramatiqument les propriétés des
semiconducteurs. L'addition de bore dans le silicium dans des proportions de 1
atomes de bore pour $10^5$ de silicium augmente la conductivité du silicium pur à
température ambiante d'un facteur $10^3$. Dans un composé semiconducteur, une
déficience stoechiométrique en un constituent peut agir comme une impureté ; de
tels semiconducteurs sont appelés \emph{semiconducteurs déficitaires}. L'addition
délibérée d'impuretés à un semiconducteur est appelé \emph{dopage}.

On considère l'effet des impuretés dans le silicium et le germanium. Ces éléments
cristallisent dans la structure diamant. Chaque atome forme quatre liaisons
covalentes, une avec chacun de ses plus proches voisins, correspondant à la
valence de l'élément qui est de 4. Si une impureté, par exemple un atome de
valence 5 comm ele phosphore, l'arsenic au l'antimoine, est substituée dans le
réseau à la place d'un atome normal, il y aura un électron de valence de l'atome
de l'impureté laissé vide après que les 4 laisons covalentes se soient faites
avec les plus proches voisins.. C'est, après que l'atome d'impureté a été
accomodé dans la structure avec une légèe perturbation, possible. Les atomes
d'impureté qui peuvent donner des électrons sont appelés \emph{donneurs}.

\section{État donneur}
La structure présentée sur la figure suivante a une charge positive portée par
l'impureté (qui a perdu un électron). L'étude du réseau a vérifié que les
impuretés pentavalentes entrent dans le réseau par substitution d'atomes normaux,
et non pas aux positions interstitielles. Le cristal dans son ensemble demeure
neutre car les électrons restent dans le cristal.

\begin{marginfigure}
    \TODO
    \caption{charges associées avec une impureté d'arsenic dans le silicium. avec
    niveau d'énergie de la perturbation}
\end{marginfigure}

L'électron supplémentaire se déplace dans el potentiel de Coulomb $e/\epsilon r$
de l'impureté ionique, où $\epsilon$ est la constante diélectrique statique du
milieu dans un cristal covalent. Le facteur $1/\epsilon$ prend en compte la
réduction des forces de Coulomb enntre les charges causées par la polarisation
électronique du milieu. Ce traitement est valide pour les orbites larges en
comparaison avec la distance entr eles atomes, et pour des mouvements lents des
électrons, de telle sorte à ce que la fréquence d'orbite soit faible en
comparaison avec la fréquence $\omega_g$ correspondant au gap d'énergie. Ces
conditions sont vérifiées, et plutôt bien, dans le germanium et le silicium, pour
des électrons donneurs de P, As ou Sb.

On peut estimer l'énergie d'ionisation de l'impureté donneuse. La théorie de Bohr
de l'atome d'hydrogène peut être modifiée pour prendre en compte la constante
diélectrique du milieu et la masse effective d'un électron dans le potentiel
périodique du cristal. L'énergie d'ionisation d'hydrogène atomique est
$-e^4m/2(4\pi\epsilon_0\hbar)^2$.

Dans les semiconducteurs avec une constante diélectrique $\epsilon$, on remplace
$e^2$ par $e^2/\epsilon$ et $m$ par la masse effective $m_e$, et on obtient
l'énergie d'ionisation du doneur dans le semiconducteur :
\begin{equation}
    E_d = \frac{e^4 m_e}{2(4\pi\epsilon\epsilon_0 \hbar)^2}
\end{equation}

Le rayon de Bohr du bas de la bande de l'hydrogène est $4 \pi \epsilon_0\hbar^2 /
me^2$. Ainsi, le rayon de Bohr du donneur est :
\begin{equation}
    a_d = \frac{4 \pi \epsilon \epsilon_0 \hbar^2}{m_ee^2}
\end{equation}

L'ajout d'un état d'impureté au germanium et au silicium est compliquée par la
masse effective anisotropique des électrons de conduction. En revanche, la
constante diélectrique a un effet plus important sur l'énergie du donneur parce
qu'elle est au carré, alors que la masse effective apparaît directement sans
puissance.

Pour obtenir une impression générale des niveaux d'impureté, on doit utiliser
$m_e \approx 0.1m$ pour les électrons du germanium et $m_e \approx 0.2m$ dans le
silicium. Les constantes diélectriques statiques sont données dans le tableau
suivant L'énergie d'ionisation d'un atome d'hydrogène libre est 13.6eV. Pour l
germanium, l'énergie d'ionisation du donneur $E_d$ dans ce modèle est de $5meV$,
réduite avec l'hydrogène d'un facteur $m_e/m \epsilon^2 = 4\cdot 10^{-4}$. Le
résultat correspondant pour le silicium est de $20meV$. Les calculs utilisant le
tenseur de masse anisotropique approprié prédisent $9.05meV$ pour le germanium et
$29.8meV$ pour le silicum. Les valeurs observées des énergies d'ionisations des
doneurs dans le silicium et le germanium sot données dans la table suivante. Pour
l'arsenure de gallium, on a $E_d\approx \SI{6}{\milli\electronvolt}$.

\begin{table}[ht]
    \begin{tabularx}{\textwidth}{lRlRlR}
        \toprule
        cristal & $\epsilon$ & cristal & $\epsilon$ & cristal & $\epsilon$ \\
        \midrule
        Diamant & 5.5 & InAs & 14.55 & AlAs & 10.1 \\
        Si & 11.7 & InP & 12.37 & AlSb & 10.3 \\
        Ge & 15.8 & GaSb & 15.69 & SiC & 10.2 \\
        InSb & 17.88 & GaAs & 13.13 & $Cu_2O$ & 7.1\\
        \bottomrule
    \end{tabularx}
    \label{}
    \caption{constantes diélectriques statiques relatives des semiconducteurs}
\end{table}

\begin{table}[ht]
    \begin{tabularx}{\textwidth}{XRRR}
        \toprule
        & P & As & Sb \\
        \midrule
        Si & 45.0 & 49.0 & 39.0 \\
        Ge & 12.0 & 12.7 & 9.6\\
        \bottomrule
    \end{tabularx}
    \label{}
    \caption{Énergies d'ionisation de donneurs $E_d$ pour des impretés
    pentavalentes (\si{\milli\electronvolt})}
\end{table}
 Le rayon de la première orbite de Bohr est augmentée de $\epsilon m/m_e$ de
 \SI{0.53}{\angstrom} pour l'atome d'hydrogène libre. La rayon correspondant est
 $(160)(0.53) \simeq \SI{80}{\angstrom}$ pour le germanium et $(60)(0.53) \simeq
 \SI{30}{\angstrom}$ pour le silicium. Ce ssont de grands rayons, et par conséquent
 l'orbite du donneur se recouvre à des concentrations en donneurs relativements
 basses, comparés au nombre d'atome hôtes. Avec un recouvrement orbitaluire
 suffisnant, une \emph{bande d'impuretés} est formée à partir d'état donneur.
 
 Le semiconducteur peut conduire dans la bande d'impuretés grâce au
 \emph{hopping} des électrons de donneurs en donneurs. Le processus de conduction
 de bande d'impureté établit à une concentration en donneur faible un niveau s'il
 y a assez d'atomes acepteurs présents, de telle sorte à ce que les donneurs sont
 toujours ionisés. Il est plus facile pour un électrodonneur de sauter sur un
 doneur ionisé (inoccupé) que  sur un atome d'oneur occupé, pour que deux
 électrons n'occupent pas le même site pendant le transport de charge.
 
 \section{États accepteurs}
 
 Un trou peut être lié à une impureté trivalente dans le germunium ou le
 silicium, de la même façon qu'un électron est lié à une impureté pentavalente.
 Les impuretés trivalentes, comme B, Al, Ga ou In sont appelées des
 \emph{accepteurs} car ils acceptent des électros de la bande de valence pour
 compléter des liaisons covalentes avec les atomes voisins, laissant des trous
 dans la bande.
 
 
\begin{table}[ht]
    \begin{tabularx}{\textwidth}{XRRRR}
        \toprule
        & B & Al & Ga & In \\
        \midrule
        Si & 45.0 & 57.0 & 65.0 & 157.0 \\
        Ge & 10.4 & 10.2 & 10.8 & 11.2 \\
        \bottomrule
    \end{tabularx}
    \label{}
    \caption{Énergies d'ionisation d'accepteurs $E_a$ pour des impretés
    trivalentes (\si{\milli\electronvolt})}
\end{table}

Quand un accepteur est ionisé, un trou est libéré, ce qui requière une source
d'énergie. Dans le diagramme de bande usuel, un électron monte lorsqu'il gagne de
l'énergie, alors qu'un trou descend s'il en gagne.

L'ionisation expérimentale des énergies d'accepteurs dans le germanium et le
sicilium sont données sur la table ci-dessus. Le modèle de Bohr s'applique
qualitativement pour les trous de la même façon que poru les électrons, mais la
dégénerescence en haut de la bande de valence complique le problème de la masse
effective.

Les tables montrent que les énergies d'ionisation des doneurs et accepteurs dans
le silicium sont comparables avec $k_BT$ à température ambiante (26meV), de telle
sorte à ce que l'ionisation thermique des doneurs et accepteurs est importante
dans la conductivité électrique du silicium à température ambiante. Si les atomes
doneurs sont présents en nombre considérablement supérieur que les accepteurs,
l'ionisation thermique des doneurs libérera des électrons dans la bande de
conduction. La conductivité des espèces sera contrôlée par les électrons (charges
négatives) , et le matériau sera dit de type $n$.

Si les accepteurs sont dominants, les trous seront relachés dans la bande de
valence en même temps que la conductivité sera contrôlée par les trous (charges
positives) : le matériau est de typee $p$. Une autre technique pratique est le
signe du potentiel thermoélectrique, dont nous ne traiteront pas ici. peut être
trouvé chez Kittel, chap 8.

Le nombre de trous e t d'électrons est égal dans le régime intrinsèque. La
concentration intrinsèque en électrons $n_i$ à 300K est
\SI{1.7e13}{\per\cubic\centi\metre} dans le germanium et
\SI{4.6e9}{\per\cubic\centi\metre} dans le silicium. La résistivité électrique de
matériaux intrinsèques est \SI{43}{\ohm\centi\metre} pour le germanium et
\SI{2.6e5}{\ohm\centi\metre} pour le silicium.

Le germanium a \SI{4.42e22}{} atomes par centimètre cube. La purification du
germanium a été réalisée plus loin que pour d'autres éléments. La concentration
en impuretés communément électriquement actives - les impuretés donneuses ou
accepteuses superficielles - ont été réduites en dessous de une impureté pour
$10^11$ atomes de germanium. Par exemple, la concentration en potassium dans le
germanium peut être réduite sous \SI{4e10}{\per\cubic\centi\metre}. Il y a des
impuretés (H,O,Si,C) dont les concentrations dans le germanium ne peuvent
généralement pas être réduite sous \SI{e12}{} - \SI{e14}{\per\cubic\centi\metre},
mais cela n'affecte pas les mesures électriques et par conséquent est difficile à
détecter.

\section{Ionisation thermique des donneurs et accepteurs}

Le calcul des concentration d'équilibre des électrons de conduction depuis des
donneurs ionisés est identique avec le calcul standard de la mécanique
statistique de l'ionisation thermique des atomes d'hydrogène. S'il n'y a pas
d'accepteurs présents, le résultat à la limite basse température $k_BT \ll E_d$
donne :
\begin{equation}
    n \cong (n_0N_d)^{1/2} \exp (-E_d / 2k_BT)
\end{equation}
avec $n_0 \equiv 2(m_ek_BT/2\pi\hbar^2)^{3/2}$. Ici, $N_d$ est la concentration
en donneurs. Pour obtenir l'équation ci-dessus, on applique les lois de
l'équilibre chimique au ratio $[e][N_d^+]/[N_d]$ et on établit $[N_d^+]=[e]=n$.
Des résultats identiques sont valables pour les accepeteurs, avec les
considération des atomes non donneurs.

Si les conctentrations en donneurs et accepteurs sont comparables, les choses se
compliquent et les équations peuvent être résolues avec des méthodes numériques.
Ainsi, la loi d'action de masse requière que le produit $np$ doit être constant à
une température donnée. Un excès de donneurs augmentera la concentration en
électrons et une diminution de la concentration en trous. La somme $n+p$
augmentera. La conductivité augmente en $n+p$ si les mobilités sont égales.
%\ref{trucmuche}

\begin{marginfigure}
    \TODO
    \caption{à propos du dernier paragraphe}
    \label{trucmuche}
\end{marginfigure}

\section{Effets thermoélectriques}

Considérons un semiconducteur maintenu à une température constante, parcouru par
un champ électrique de densité de courrant $j_q$. Si le courrant est porté
uniquement par les électrons, le flux de charge est :
\begin{equation}
    j_q = n(-e)(-\mu_e)E = ne\mu_e E
\end{equation}
où $\mu_e$ est la mobilité électronique.. L'énergie moyenne transportée par un
électron dépend du niveau de Fermi $\mu$ :
\begin{equation*}
    (E_c - \mu) + \frac{3}{2}k_BT
\end{equation*}
où $E_c$ est l'énergie du bord de la bande de conduction. On se rapporte à
l'énergie du niveau de Fermi parce que différents conducteurs en contact ont un
même niveau de Fermi, comme nous allons le voir dans le chapitre suivant. Le flux
d'énergie qui accompagne le flux de charge est :
\begin{equation}
    j_U = n(E_c - \mu +\frac{3}{2}k_BT)(-\mu_e)E
\end{equation}
Le \emph{coefficient de Peltier} $\Pi$ est défini par $j_U = \Pi j_q$ ou
l'énergie transportée par charge unitaire. Pour des électrons,
\begin{equation}
    \Pi_e = -(E_c - \mu + \frac{3}{2}k_BT)/e
\end{equation}
qui est négatif car le flux d'énergie est opposé au flux de charge. Pour des
trous,
\begin{equation}
    j_q = pe\mu_hE; \hfill j_U = p(\mu - E_v + \frac{3}{2}k_BT)\mu_h E
\end{equation}
où $E_v$ est l'énergie au bord de la bande de valence. Par conséquent,
\begin{equation}
    \Pi_h = (\mu - E_v + \frac{3}{2}k_BT)/e
\end{equation}
et est positif. Ces deux équations sont le résultat de notre simple théorie de la
vitesse de dérive. Un traitement par l'équation du transport de Bortzmann donne
des valeurs numériques différentes.

Le \emph{pouvoir thermoélectrique absolu Q} est défini comme le champ électrique
en circuit ouvert, créé par un gradient de température :
\begin{equation}
    E = Q \grad T
\end{equation}
Le coefficient de Peltier $\Pi$ est relié au pouvoir thermoélectrique Q par la
relation :
\begin{equation}
    \Pi = QT
\end{equation}

C'est la fameuse relation de Kelvin de l'irréversibilité thermodynamique. Une
mesure du signe du voltage le long d'un échantillon semiconducteur, à une
extrémité chauffée, est un moyen efficace et brut de dire si l'échantillon est
dopé n ou p.

\begin{marginfigure}
    \TODO
    \caption{coefficient s de peltier pour du silicium n et p en fonction de la
    température}
    \label{peltier}
\end{marginfigure}

    \chapter{La jonction p-n}

une jonction p-n est faite d'un monocrystal modifié dans deux régions séparées.
Des impuretés accepteuses sont ajoutées dans une partie, pour produire une région
p dans laquelle la majorité des porteurs sont des trous. Des impuretés donneuses
sont ajoutées dans l'autre partie, pour produire une région n, dans laquelle les
porteurs majoritaires sont des électrons. La région à l'interface est moins
épaisse que $10^-4 cm$. Loin de l'interface, du côté p, il y a des impuretés
accepteuses ionisées (-) et une concentration égale en trous libres. Du côté n,
il y a des atomes donneurs ionisés (+), et une concentration égale en électrons
libres. Par conséquent, les porteurs majoritaires sont les trous du côté p et les
électrons du côté n.

Les trous sont concentrés du côté p voudront diffuser pour remplir le cristal de
façon uniforme. Les électrons voudront diffuser depuis le côté n. Mais la
diffusion va contrarier la neutralité électrique locale du système.

Un léger transfert de charge par diffusion laisse derrière, du côté p, un excès
d'accepteurs ionisés (-), et du côté n, un excès de donneurs ionisés (+). Cette
double couche de charges créé un champ électrique dirigé de la région n à p, qui
empêche la diffusion et ainsi maintient la séparation des deux types de porteurs.
À cause de cette double couche, le potentiel électrostatique dans le cristal fait
un saut en passant d'un côté à l'autre de la jonction.

À l'équilibre thermique, le potentiel chimique de chaque porteur est partout
constant dans le cristal, même à la jonction. Pour les trous :
\begin{equation}
    k_BT \ln p(\mathbf{r}) + e\phi(\mathbf{r}) = cste
\end{equation}
à travers le cristal, où $p$ est la concentration en trous et $\phi$ le potentiel
électrostatique. Par conséquent, $p$ est faible et $\phi$ est important. Pour les
électrons :
\begin{equation}
    k_BT \ln n(\mathbf{r}) - e\phi(\mathbf{r}) = cste
\end{equation}
et $n$ sera faible quand $\phi$ petit.

Le potentiel chimique total est constant partout dans le cristal. L'effet du
gradient de concentration annule parfaitement le potentiel électrostatique, et le
lux net de particules de chaque porteur est nul. Cependant, même à l'équilibre
thermique, il y a un léger flux d'électrons de $n$ vers $p$ où les électrons
finissent leur vie en se combinant avec des trous. Le courrant de recombination
$J_{nr}$ est contrebalancé par un courant $J_{ng}$ d'électrons qui sont générés
thermiquement dans la région $p$ et qui sont poussés par le champ auto-alimenté
dans la région $n$. Par conséquent, en l'absence de champ élecrtique appliqué :
\begin{equation}
    J_{nr}(0) + J_{ng}(0) = 0
\end{equation}

pour que sinon les électrons s'accumulent indéfiniment d'un côté de la barrière.

\section{Rectification}

Une jonction p-n peut agir comme un rectifieur. Un courant important apparaîtra
si on applique une tension le long de la jonction dans une direction, mais si la
tension est dans la direction opposée, juste un courrant faible découlera. Si une
tension alternative est appliquée le long de la jonction, le courrant s'écoulera
principalement dans une direction -- la jonction a rectifié le courrant.

\begin{marginfigure}
    \TODO
    \caption{caractéristique de la rectification d'une jonction p-n dans du
    germanium}
    \label{caracpn}
\end{marginfigure}

Pour des biais de tension inverse, une tension négative est appliquée dans la
région p et une tension positive dans la région n. Par conséquent, augmenter la
différence de potentiel entre les deux régions. Maintenant, pratiquement aucun
électron ne peut grimper la barrière d'énergie potentielle du côté bas en énergie
jusqu'en haut. Le courrant de recombinaison est réduit d'un facteur de Boltzmann
:
\begin{equation}
    J_{nr}(V back) = J_{nr}(0) \exp (-e|V|/k_BT)
\end{equation}

Le facteur de Boltzmann contrôle le nombre d'électrons avec suffisemment
d'énergie pour traverser la barrière.

La génération thermique d'un courant d'électrons n'est pas particulièrement
affectée par le courant inverse parce que la génération d'électron descend la
barrière (de p vers n). Par conséquent :
\begin{equation}
    J_{ng}(V back)=J_{ng}(0)
\end{equation}
On a vu précédemment que $J_{nr}(0) = -J_{ng}(0)$. Par conséquent, la génération
de courrant domine les courrants de recombinaison pour les biais inverses.

Lorsqu'une tension inverse est appliquée ,le courrant de recombinaison augmente
car la barrière d'énergie potentielle est abaissée, et par conséquent permet à
plus d'électrons de passer du côté n au côté p :
\begin{equation}
    J_{nr}(V forward) = J_{nr}(0) \exp (e|V|/k_BT)
\end{equation}
Encore une fois, le courrang de génération est inchangé :
\begin{equation}
    J_{ng}(V forward) = J_{ng}(0)
\end{equation}
Le courrant de trous traversant la jonction se comporte de façon similaire au
courant d'électrons. La tension appliquée, qui diminue la hauteur de la barrière
pour les électrons le fait également pour les trous, de sorte à ce qu'un grand
nombre d'électrons traverse de la région n àsous les mêmes conditions de tension
qui produisent un grand courrant de trous dans la direction opposée.

Les courrants électriques de trous et d'électrons sont additifs, et ainsi le
courrant total est :
\begin{equation}
    I = I_s [\exp(eV/k_BT)-1]
\end{equation}
où $I_s$ est la somm edes deux courrants générés. Cette équation est bien
satisfaite pour des jonctions p-n dans le germanium, mais pas trop pour d'autres
semiconducteurs.

\section{Cellules solaires et détecteurs photovoltaiques}

Si l'on illumine une jonction p-n, sans tension extérieure qui biaise le truc,
chaque photon absorbé créé un électron et un trou. Lorsque ces porteurs diffusent
dans la jonction, le champ électrique auto-alimenté de la jonction sépare les
charges à la barrière d'énergie. La séparation des poruteurs produit une tension
directe le long de la barrière. Directe, parce que le champ électrique des
porteurs photoexcités est opposé au champ intégré de la jonction.

L'apparition d'une tension inverse le long de la jonction illimunée est appelée
l'\emph{effet photovoltaique}. Une jonction illimunée peut délivrer de la
puissance à un circuit externe. Les jonctions p-n à grande surface en silicium
sont majoritairement utilisés dans les panneaux solaires, pour convertir des
photons d'origine solaire en énergie électrique.

\section{Barrière de Schottky}

Lorsqu'un semiconducteur est mis en contact avec un métal, on observe la
formation dans le semiconducteur d'une couche barrière à partir de laquelle les
porteurs de charge diminuent drastiquement. La couche barrière est aussi appelée
couche de déplétion (depletive layer ou exhaustion layer).

\begin{marginfigure}
    \TODO
    \caption{semiconducteur en contact avec un métal}
    \label{semicondmetal}
\end{marginfigure}

Sur la figure \ref{semicondnmetal}, un semiconducteur de type n est mis en
contact avec un métal. Les niveaux de Fermi sont coincident après le transfert
d'électrons à la bande de conduction du métal. Les donneurs chargés positivement
sont écartés de cette région qui est partiquement uniquement composée
d'électrons. Ici, l'équation de Poisson  donne :
\begin{equation}
    \div \mathbf{D} = ne/\epsilon_0
\end{equation}
où n est la concentration en donneur. Le potentiel électrostatique est déterminé
par :
\begin{equation}
    \frac{d^2\phi}{dx^2} = \frac{ne}{\epsilon \epsilon_0}
\end{equation}
Qui admet une solution de la forme :
\begin{equation}
    \phi = -(\frac{-ne}{2\epsilon \epsilon_0})x^2
\end{equation}

L'originedes x est prise, pour des raisons pratiques, au bord droit de la
barrirère. Le contact est à $-x_b$, et ici l'énergie potentielle relative au côté
droit est $-e\phi_0$, d'où une épaisseur de la barrière qui peut s'écrire sous la
forme :
\begin{equation}
    x_b = \left(\frac{2\epsilon \epsilon_0 |\phi_0|}{ne}\right)^{1/2}
\end{equation}

Avec $\epsilon = 16$, $e\phi_0 =$ \SI{0.5}{\electronvolt},
n=\SI{e16}{\per\cubic\centi\metre}, on trouve $x_b = $\SI{0.3}{\micro\metre}.
C'est une vision plutôt simplifiée.

%
%    \part[SYNTHÈSE DE MATÉRIAUX CRISTALLINS]{Synthèse de matériaux cristallins}
%
%    \include{chapter/haute-temperature}
%    \include{chapter/sol-gel}
%    \include{chapter/ei-i-g} % échange ionique, intercalations, greffage
%
    \appendix

    \part[ANNEXES]{Annexes}
%
%    \include{appendices/magnetisme} % magnétisme et structure spinelle
%    \include{appendices/optique} % propriétés optiques des composés d'éléments de transition
    \chapter{Structure du graphene}

Le graphite est constitué d'un empilement de feuillets faiblement connectés
(liaisons de van der Waals). Les propriétés d'un seul feuillet (graphène) sont
très différentes de celles du graphite.

Il existe plusieurs techniques pour la synthèse du graphène : le dépot sur
membrane, la suspension, etc.
Du point de vue thermique, il semble très difficile d'imaginer un matériau
purement 2D. Lorsqu'il est suspendu, c'est la modulation vibrationnelle qui le
stabilise.

Du point de vue électronique, on pourra montrer que c'est un conducteur et
calculer sa structure de bande. On montrera que les électrons ont des propriétés
particulières dans ce type de matériau 2D. Les électrons relativistes -> vitesse
de fermi.
relation de dispersion linéaire. énergie proportionnelle à k et non plus à $k^2$
(fermion de dirac)

effet Hall quantique à température ambiante dans le graphène.
protection des états quantiques.

il s'agit d'un matériau topologique, on peut faire des isolants topologiques,
c'est à dire isolants dans le bulk, conducteurs en surface. exemple : fermion de
majorana : particule prédite qui est sa propre antiparticule. il est observé dans
des isolants topologiques, couplé à des semi-conducteurs.
Approche des électrons fortement liés. On part d'orbitales en considérant le
recouvrement, structure de bande.

Orbitales $sp^2$, forme triangulaire à 120\textdegree.

\begin{figure}
    \TODO
    \caption{figure des $sp^2$ et $2p_z$}
    \label{fig:orbitales}
\end{figure}

La maille élémentaire du graphène contient deux sites différents. Le carbone
possède 4 électrons de valence. Trois de ces électrons sont engagés dans des
liaisons $sp^2$ à 120\textdegree les unes des autres, et qui pointent vers les
plus proches voisins. Ces liaisons étant très fortes (bandes profondes et
étroites) ; cela nous permet de négliger ces électrons. Il reste ainsi un
électron qui occupe une orbitale $2p_z$ qui se trouve dans les parages, sans
vraiment savoir quoi faire.

La maille élémentaire est définie par les vecteurs $\mathbf{a}_i$ et
$\mathbf{a}_2$, et on définit $a$ tel que $a=a_1=a_2 = \sqrt{3}a_0$ est la
distance entre deux atomes dans le plan. On définit le vecteur $\mathbf{b}$ comme
celui qui relie deux atomes de carbone les plus proces entre eux. Les angles
$\alpha$ et $\gamma$ valent respectivement 30\textdegree et 120\textdegree.


\begin{figure}
    \TODO
    \caption{Mailles du graphène dans l'espace direct et l'espace réciproque}
    \label{fig:maillegraphene}
\end{figure}

Les vecteurs du réseau direct sont :

\begin{equation}
    \mathbf{a}_1 =
    (\frac{1}{2}\mathbf{\hat{x}}+\frac{\sqrt{3}}{2}\hat{\mathbf{y}}),\quad
    \mathbf{a}_2 = a \mathbf{\hat{x}},\quad
    \mathbf{b} = a_0
    (\frac{\sqrt{3}}{2}\hat{\mathbf{x}}+\frac{1}{2}\hat{\mathbf{y}}) =
    \frac{1}{3}(\mathbf{a}_1 + \mathbf{a}_2)
\end{equation}

Ceux du réseau réciproque peuvent s'écrire :

\begin{equation}
    \mathbf{a}_1^* = \frac{2\pi}{a_0} \cdot \frac{2}{3} \mathbf{\hat{y}} =
    \frac{4\pi}{3a_0}\mathbf{\hat{y}},\quad
    \mathbf{a}_2^* = \frac{2\pi}{3a_0}(\sqrt{3}\mathbf{\hat{x}}+\mathbf{\hat{y}})
\end{equation}

Posons $V(\mathbf{r}-\mathbf{R})$ le potentiel d'un site atomique arbitraire.
Nous pouvons écrire l'hamiltonien d'un élecron dans le feuillet de graphène comme
:

\begin{equation}
    \mathcal{H} = \frac{p^2}{2m} + V(\mathbf{r}-\mathbf{R}) + (V(\mathbf{r-R-b}))
\end{equation}

Nous allons déterminer les énergies $\mathcal{E}(\mathbf{k})$ à partir d'une
approche variationnelle.

Prenons la fonction d'onde :

\begin{equation}
    |\psi_k> = \frac{1}{\sqrt{N}} \sum_{\mathbf{R}} \exp (i
    \mathbf{k}\cdot\mathbf{R}) \left( \mathbf{a}_k |\phi_{R_0}> + \mathbf{b}_k
    |\phi_{R_b}> \right)
\end{equation}

Ici, $<r|\psi_{R_0}> = \phi(\mathbf{r}-\mathbf{R})$ et représente l'orbitale
$2p_z$ sur le site 0.

$<r|\psi_{R_b}> = \phi(\mathbf{r}-\mathbf{R}-\mathbf{b})$ erprésente l'orbitale
$2p_z$ au site 1.

L'énergie peut s'écrire comme :
\begin{equation}
    \mathcal{E}(\mathbf{k}) = \frac{<\psi_k|\mathcal{H}|\psi_k>}{<\psi_k|\psi_k>}
\label{eq:grapheneenergie}
\end{equation}

On veut minimiser $\mathcal{E}(\mathbf{k})$. Par conséquent, il faut trouver
$\mathbf{a}_k$ et $\mathbf{b}_k$.

On peut alors réécrire l'équation \ref{eq:grapheneenergie}
\begin{equation}
    <\psi_k|\mathcal{H}|\psi_k> = \mathcal{E}(\mathbf{k}) <\psi_k|\psi_k>
\end{equation}

En dérivant, on obtient :
\begin{eqnarray}
    \frac{\partial}{\partial \mathbf{a}_k} <\psi_k|\mathcal{H}|\psi_k> & = &
    \frac{\partial \mathcal{E}_k}{\partial \mathbf{a}_k} <\psi_k|\psi_k> +
    \mathcal{E}_k \frac{\partial}{\partial\mathbf{a}_k} <\psi_k | \psi_k>\\
    \frac{\partial}{\partial \mathbf{b}_k} <\psi_k|\mathcal{H}|\psi_k> & = &
    \frac{\partial \mathcal{E}_k}{\partial \mathbf{b}_k} <\psi_k|\psi_k> +
    \mathcal{E}_k \frac{\partial}{\partial\mathbf{b}_k} <\psi_k | \psi_k>
\end{eqnarray}

Dans cette formulation, les énergies $\mathcal{E}_k$ des électrons sont
indépendantes de leurs positions, donc des $\mathbf{a}_k$ et des $\mathbf{b}_k$.
Par conséquent, les dérivées sont nulles, et on peut réécrire :

\begin{eqnarray}
    \frac{\partial}{\partial \mathbf{a}_k} <\psi_k|\mathcal{H}|\psi_k> & = &
    \mathcal{E}_k \frac{\partial}{\partial\mathbf{a}_k} <\psi_k | \psi_k>\\
    \frac{\partial}{\partial \mathbf{b}_k} <\psi_k|\mathcal{H}|\psi_k> & = &
    \mathcal{E}_k \frac{\partial}{\partial\mathbf{b}_k} <\psi_k | \psi_k>
\end{eqnarray}

\begin{equation}
    <\psi_k|\psi_k> = \int dk (...)
\end{equation}

Soit finalement,

\begin{eqnarray*}
    <\psi_k | \psi_k> & = & \frac{1}{N} \sum_{\mathbf{R}'}
    e^{i\mathbf{k}\cdot\mathbf{R}'} (\mathbf{a}_k^* <\psi_{\mathbf{R}_0'}| +
    \mathbf{b}_k^* <\psi_{\mathbf{R}_b'}|) \cdot \sum_\mathbf{R}
    e^{i\mathbf{k}\cdot \mathbf{R}} (\mathbf{a}_k |\psi_{R_0}> + \mathbf{b}_k |
    \psi_{\mathbf{R}_b}>) \\
    . & = & \frac{1}{N} \sum_{\mathbf{R,R}'}
    e^{i(\mathbf{k}\cdot\mathbf{R}-\mathbf{k}'\cdot\mathbf{R}')}
    (\mathbf{a}_k^*\cdot\mathbf{a}_k <\psi_{\mathbf{R}_0'}|\psi_{\mathbf{R}_0}> +
    \mathbf{b}_k^* \cdot\mathbf{b}_k <\psi_{\mathbf{R}_b'}|\psi_{\mathbf{R}_b}>)
\end{eqnarray*}

D'où on tire :

\begin{align}
    <\psi_k|\psi_k> & = &\frac{1}{N} \sum_R (a_k^2 <\phi_{\mathbf{R}_0'} |
    \phi_{\mathbf{R}_0}> + b_k^2 <\phi_{R_b'}|\phi_{R_b}>)\\
    & = & a_k^2 + b_k^2
    & = & \mathbf{a}_k\cdot\mathbf{a}_k^* + \mathbf{b}_k\cdot\mathbf{b}_k^*
\end{align}

On peut donc retrouver :

\begin{eqnarray}
    \frac{\partial}{\partial \mathbf{a}_k} <\psi_k|\psi_k> & = & \mathbf{a}_k^* \\
    \frac{\partial}{\partial \mathbf{b}_k} <\psi_k|\psi_k> & = & \mathbf{b}_k^*
\end{eqnarray}

\TODO

    \chapter{Historique de la matière condensée}

"on ne saurait connaître une science sans en connaître l'histoire" auguste comte


rien ne vaut d'aller aux sources, de se mettre en contact aussi fréquent et complet que possible avec ceux qui ont fait la science et qui en ont le mieux représenté l'aspect vivant. paul langevin
\begin{description}
\item[1611] Johannes Kepler (1571-1630) fait l'hypothèse que la symétrie hexagonale des cristaux de neige est due à un arrangement hexagonal de particules d'eau sphériques.
\item[1669] Steno donne la loi de la constance des angles entre les faces d'un cristaux.
\item[1784] Haüy (1743-1822) établi que les faces des cristaux peuvent être repérées par un ensemble de trois nombres entiers appelés indices de Miller, ce qui l'amène à associier à chaque cristal un réseau de petits volumes de matière qu'il nomme « molécules intégrantes » et qui correspondent aujourd'hui à la notion de maille élémentaire
\item[ ] Bravais (1811-1863) et Schoenflied (1853-1928) montrent à l'aide de la théorie des groupes que les cristaux peuvent être catégorisés en 32 groupes ponctuels de symétrie et en 230 groupes d'espace
\item[19e siècle] apparition de nouveaux domaines scientifiques:
    \begin{itemize}
    \item mécanique des milieux continus
    \item électromagnétisme
    \item thermodynamique, les duex qui attribuent des propriétés macroscopiques mesurables à la matière
    \item module d'Young
    \item susceptibilité optique
    \item conductivité électrique et thermique
    \item grandeurs macroscopiques qui permettent une description phénoménologique satisfaisante des solides
    \end{itemize}
\item[1808]  modèle de Dalton (atome)
\item[1826] Naumann énonce l'idée des 7 systèmes cristallins
\item[1830] Hessel énonce le fait qu'il y a 32 classes cristallines car il y a 32 groupes ponctuels sont consistens avec l'idée qu'il y a des symmétries translationnelles.
\item[1849] Auguste Bravais fait l'hypothèse d'une structure réticulaire des cristaux
\item[1869] rayons cathodiques découverts par Hittorf.
\item[1878] William Crookes (1832--1919) étudie la conduction de l'électricité dans des gaz à faible pression (tubes à vide appelés tubes de Crookes). 
\item[1894] différents gorupes sont énumérés parmi les 230 groupes d'espaces, cohérent avec le fait qu'il n'y aque 230 possiblitisé de symmétries cristallines.
\item[1895] découverte des rayons X par Wilhelm Röntgen (1843 -- 1923) dans un tube de Crookes.
\item[1897] modèle de Thomson (atome)
\item[1900] Paul Karl Ludwig Drude (1863-1906) développe un modèle quasi-classique de la conductiondes métaux en supposant ceux-ci remplis d'un gaz d'électrons libres, auxquels il applique la physique statistique de Ludwig Boltzmann
\item[1901] Röntgen reçoit le premier prix Nobel de physique pour la découverte des rayons X. Il fait le premier cliché de la main d'Anna Bertha Röntgen (pose de 25min).
\item[1902] Modèle de Lewis (atome)
\item[1904] modèle de nagoaka (atome)
\item[1904] Lord Rayleigh (john william strutt) nobel de physique pour ses recherches sur les densités des gaz les plus importants et pour sa découvert ed l'argon
\item[1905] philipp eduard et anton von lenard nobel travaux rayons cathodiques
\item[1906] joseph john thomson nobel théorie cinétique et conduction de l'électricité dans les gaz
\item[1910] johannes diderik van der Waals nobel équation d'état des gaz et liquides
\item[1911] modèle de Rutherford (atome)
\item[1912] Max von Laue (1879 - 1960) réalise la première expérience de diffraction des rayons X sur un cristal
diffraction d'électrons et de neutrons
\item[1913] Heike Kamerlingh onnes nobel recherches propriétés de la matière aux basses températures, conduisant à la production d'hélium liquide
\item[1913] modèle de Bohr (atome)
\item[1914] Max von Laue  reçoit le prix Nobel de physique pour sa découverte de la diffraction des rayons X par les cristaux
\item[1915] sir William Henry Brag et son fils sir William Lawrence Bragg recoivent le prix Nobel de physique pour leurs contributions à l'analyse de la structure cristalline au moyen des rayons X
\item[1917] charles glover barkla nobel physique pour sa découverte des rayonnements röntgen caractéristiques des divers éléments
\item[1918] nobel max karl ernst ludwig planck découverte des quanta d'énergie
\item[1919] Albert W. Hull, Peter Debye et Paul Scherrer] méthode de Debye et Scherrer pour la diffraction sur poudre
\item[1920] charles édouard guillaume nobel physique anomalies alliages acier au nickel
\item[1921] nobel einstein effet photoélectrique
\item[1922] nobel physique niels henrik david bohr contribution à la recherche sur la structure des atomes et sur le rayonnement qu'ils émettent
\item[1923] nobel physique robert andrews millikan travaux charge élémentaire
\item[1924] De Broglie prédit que les particules doivent se comporter comme des ondes.
\item[1924] nobel physique karl manne georg siegbahn découverte dans la spectroscopie des rayons X
\item[1925] modèle de Schrödinger (atome)
\item[1925] principe d'exclusion de Pauli, à la base de la statistique de fermi-dirac
\item[1926] Emnrico Fermi et Paul Dirac font la statistique de Fermi-Dirac
\item[1927?] George Paget Thomson fait passer un faisceu d'électrons au travers d'un film mince de métal et observe les figures de diffraction prédites par De Broglie
\item[1927?] Clinton Joseph Davisson et Lester Halbert Germer (bell) font passer leur faisceu par une grille cristalline (cristal de nickel).
\item[1927] Arthur Holly Compton nobel physique effet compton
\item[1929] Louis de Broglie prix nobel de physique pour sa découverte de la nature ondulatoire des électrons
\item[1933] Arnold Sommerfeld (1868--1951) et Hans Bethe(1906--2005) reprennent le modèle de Drude
\item[1933] Erwin Schrödinger et Paul Adrien Maurice Dirac nobel physique pour la découverte de nouvelles formes productives de la théorie atomique
\item[1935] James Chadwick nobel physique pour sa découverte du neutron
\item[1936] Seitz commence à démontrer les représentations irréductibles des groupes d'espace
\item[1937] Thomson et Davisson obtiennent le prix nobel de physique. pour leur découverte expérimentale de la diffraction des électrons par les cristuax
\item[1939] enrest orlando lawrence nobel physique pour l'invention et le développement du cyclotron??
\item[1945] Ernest O. Wollan fait la première expérience de diffraction de neutrons. il est rejoint en 1946 par Clifford Shull.
\item[1945] Wolfgang Pauli nobel physique pour le principe d'exclusion
\item[1952] Felix Bloch (1905-1983) et Edward Mills Purcell obtiennent le prix nobel de physique pour leur développement de nouvelles méthodes de mesure magnétiques nucléaires fines (RMN)
\item[1956] William Bradford Shockley, John Bardeen et Walter Houser Brattain] nobel physique pour leurs recherches sur les semiconducteurs et leur découverte de l'effet transistor
\item[1962] Lev Davidovitch Landau nobel physique pour ses théories d'avant-garde sur la matière condensée, en particulier l'hélium liquide.
\item[1963] nobel physique Eugene Paul Wigner pour ses contributions à la théorie du noyau atomique et des particules élémentaires, en particulier par la découverte et l'application de principes fondamentaux de symétrie.
\item[1967] Bethe prix nobel de physique pour sa contrubution à la compréhension de la nucléosynthèse stellaire
\item[1970] Louis Eugène Félix Néel nobel physique pour ses travaux fondamentaux et ses découvertes sur l'antiferromagnétisme et le ferrimagnétisme qui ont conduit à des applications importantes en physique du solides
\item[1972] John Bardeen, Leon Neil Cooper, John Robert Schieffer nobel physique pour leur théorie développée conjointement sur les supraconducteurs, BCS.
\item[1973] Leo Esaki, Ivar Giaever nobel physique pour leurs découvertes expérimentales sur les phénomènes d'effet tunnel dans les semi-conducteurs et les supraconducteurs respectivement
\item[1973] Brian David Josephson nobel physique poru ses prédictions théoriques sur les propriétés d'un super-courant à travers une barrière tunnel, en particulier l'effet Josephson
\item[1984] Shectmann, Steinhardt et al découvrent les quasi-cristaux, des substances qui ne sont ni cristallines, ni vitreuses, mais néanmoins ordonnées d'une façon quasi-périodique
\item[1977] philip warren anderson, nevill francis mott, john hasbrouck van vleck nobel physique pour leurs recherches théoriques fondamentales sur la structure électronique des systèmesma gnétiques désordonnés.
\item[1985] nobel physique klaus von klitzing pour la découverte de l'effet hall quantique entier.
\item[1986] ernst ruska nobel physique pour ses travaux fondamentaux en optique électronique ; gerd binning, heinrich rohrer pour leur conception du microscope à effet tunnel à balayage
\item[1987] johannes georg bednorz, karl alexander müller pour leur percée importante dans la découverte de la supraconductivité de matériaux céramiques
\item[1991] Pierre-Gilles de Gennes pour sa découverte du fait uqeles méthodes développées pour l'étude des phénomènes d'ordre dans des systèmes simples peuvent être généralisées à des formes plus complexes de la matière, en particulier aux cristaux liquides et aux polymères.
\item[1994] Shull obtient le prix Nobel de physique pour la mise au point de la technique de diffraction neutronique
\item[1994] Bertram Brockhouse obtient le prix nobel de physique pour le développement de la diffusion inélastique de neutrons et la mise au point du spectromètre trois-axes
\item[20e siècle] découvertes, nouveaux outils
\item[-] étude des propriétés des solides aux basses températures
\item[-] introduction de la mécanique quantique
\item[-] apparition du microscope électronique
\item[2003] alexeï alexeïevitch abrikosov, vitaly lazarevich ginzburg, anthony J. Leggett pour des travaux novateurs dans la théorie des supraconducteurs et des superfluides.
\item[2007] albert fert, peter grünberg nobel physique pour la découverte de la magnétorésistance géante
\item[2010] andre geim, konstantin novoselov nobel physique pour des expériences fondamentales concernant le graphène, matériau bi-dimensionnel
\item[2012] structure perovskite au plomb ?
\end{description}

rayleigh (diffusion élastique)
compton (diffusion inélastique)
effet photoélectrique, ionisation
auger (émission auger)
rutherford (spectroscopie de rétrodiffusion)
TEM, basé sur la diffusion d'électrons
SEM idem
EBSD] diffraction d'électrons rétrodiffusés.
paul peter ewald (1888--1985)

méthode poudre] géométrie bragg-brentano
monocristaux] méthode de laue, diffractomètre quatre cercles, etc.

modèle de bohr de l'atome -> structure fine des raies spectrales de l'hydrogène (modèle de Bohr-Sommerfeld).
constante de structure fine alpha introduite par Sommerfeld
théorie des rayons X de sommerfeld
étudiants célèbres de Sommerfeld] Werner Heisenberg, Wolfgang Pauli, Peter Debye, Hans Bethe, Paul Sophus Epstein, Walter Heitler, Paul Peter Ewald, Karl Bechert, Alfred Landé, Wilhelmm Lenz, Albrecht Unsöld, Gregor Wentzel, Helmut Hönl, Fritz Bopp. post doctorants] Isidor Isaac Rabi, Linus Pauling.

DFT] density functional theory. origines dans le modèle de Llewellyn Thomas et Enrico Fermi fin 1920. ; 1960] Pierre Hohenberge, Walter Kohn et Lu Sham] formalisme héorique sur lequel repose la méthode actuelle.
théorie hartree-fock

étudiants de Langevin] Louis de Brogile, Léon Brillouin, Irène Joliot-Curie
théorie du magnétisme par Langevin, nature microscopique. interpréter l'observation de Pierre Curie] la susceptibilité des matériaux paramagnétiques varie avec la température.

hall, effet

- effet Hall, edwin hall, USA
-Sir Joseph Thomson. 1897 découverte de l'électron. trouve que] atome divisible, il existe des électrons, q/m de l'électron et de h+, modèle atome plum puddig.
- paul karl ludwig drud, introduit le symbole c, célérité. 1900 modèle de drude pour la conduction des électrons. professeur de arnold sommerfeld.
- wolfgang pauli] 1925] principe d'exclusion  pour les fermions
- arnold sommerfeld] 1928, 1933] 1er modèle quantique du solide. professeur de Debye et Heisenberg.

% graphène
% mosfet & gaz 2D
% unités CGS/SI
% historique
% annexe sur la projection stéréographique
% références des figures et des tableaux
% glossaire
% constantes physiques
% tableau périodique
% page de fin


\backmatter

%\bibliography{biblio.bib}
%\bibliographystyle{plainnat}


\printindex

\end{document}
