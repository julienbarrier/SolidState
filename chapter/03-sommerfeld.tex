\chapter{Modèle des électrons libres}

Le modèle des électrons libres a été proposé par Arnold Sommerfeld en 1920, pour
décrire le comportement des électrons dans un solide comme un gaz.

On peut comprendre beaucoup des propritétés physiques des métaux, et pas
seulement des métaux simples, avec un modèle d'électrons libres. En suivant ce
modèle, les électrons de valence des atomes constituant le métal deviennent des
électrons de conduction est bougent librement dans le volume du métal. Même dans
les métaux pour lesquels le modèle des électrons libre marcherait le mieux, la
distribution de charge des électrons de conduction reflète le potentiel 
électrostatique important des cœurs ioniques. L'utilité d'un modèle d'électrons
libres est importante pour comprendre les propriétés qui dépendent
essentiellement des propriétés cinétiques des électrons de conduction.
L'interaction des électrons de conduction avec les ions du réseau est traitée
dans le chapitre suivant sur le modèle des électrons presque libres.

Les métaux les plus siples sont les métaux alcalins : le lithium, sodium, 
potassium, césium et rubidium. Dans un atome libre de sodium, l'électron de 
valence est dans un état 3s ; dans un métal, cet électron devient un électron de
conduction dans la bande de conduction 3s.

Un cristal monovalent qui contient N atomes aura N électrons de conduction et N
cœurs ioniques positifs. Le cœur ionique \ch{Na+} contient 10 électrons qui
occupent les couches 1s, 2s et 2p de l'ion libre, avec une distribution spatialle
qui est essentielleement la même qu'il soit métallique ou sous forme d'ion libre.
Les cœurs ioniques remplissent à peu près 15\% du volume dans un cristal de
sodium. Le rayon de l'ion libre \ch{Na+} est de \SI{0.98}{\angstrom}, alors que
la moitié de la distance aux plus proches voisins dans le métal est de
\SI{1.83}{\angstrom}.

L'interprétation des propriétés métalliques en terme de mouvement des électrons
libres a été développée bien avant l'invention de la mécanique quantique. La 
théorie classique a eu quelques succès, notamment en démontrant une forme de la
loi d'Ohm et la relation entre la conductivité électrique et thermique, que nous
aborderons plus tard avec le modèle de Drude pour la conduction électronique.
En revanche, la théorie classique a échoué à  expliquer la capacité calorifique
et la susceptibilité magnétique des élecrtons de conduction. En fait, ce ne sont
pas des problèmes sur le modèle de l'électron libre, mais des modèles sur la
distribution classique de Maxwell-Boltzman qui y est utilisée.

Il y a une difficulté supplémentaire avec le modèle classique. Plusieurs
expériences de plusieurs types donnent le résultat suivant : il est clair que les
électrons de conduction dans un métal peuvent se déplacer avec un chemin direct 
sur plusieurs distances atomiques, sans être défvés par des collisions avec
d'autres électrons de conduction ou par les collisions avec les noyaux atomiques.
Dans un échantillon très pur à basse température, le libre parcours moyen est
aussi long que \SI{e8}{} distances inter-atomiques (plus d'un centimètre !).

Pourquoi est la matière condensée si transparente à la conduction des électrons ?
La réponse à cette question peut être scindée en deux parties : premièrement, un
électron de conduction n'est pas dévié par les cœurs ioniques arrangés sur un
réseau \emph{périodique} parce que les ondes mécaniques peuvent se propager 
librement sur une structure périodique, comme conséquence des mathématiques 
traités dans ce chapitre. Deuxièmement, un électron de conduction est diffusé
seulement rarement par d'autres électrons de conduction. Cette proriété est une
conséquence du principe d'exclusion de Pauli. Quand on parle de gaz d'électrons
libres de Fermi, on doit plutôt vouloir dire un gaz d'électrons libres sujets au
principe de Pauli.


\section{Effets du principe de Pauli}

Considérons un électron libre dans un gaz 1D, en prenant en compte la théorie 
quantique et le principe de Pauli. Un électron qui a une masse m est confiné dans
un puit de largeur L. La fonction d'onde $\psi_n(x)$ de l'électron est une
solution de l'équation de Schrödinger $\mathcal{H}\psi = \epsilon \psi$, ce qui,
en négligeant l'énergie potentielle, donne $\mathcal{H} = p^2/2m$, où $p$ est la
quantité de mouvement. Dans la théorie quantique, $p$ peut représenter 
l'opérateur $-i\hbar d/dx$, de telle sorte que :
\begin{equation}
    \mathcal{H}\psi_n = -\frac{\hbar^2}{2m} \frac{d^2\psi_n}{dx^2}
    = \epsilon_n\psi_n
\end{equation}
où $\epsilon_n$ est l'énergie de l'électron sur son orbitale.

On utilise le terme orbitale pour dire une solution de l'équation d'onde pour un
système d'un seul électron. Le terme permet de distinguer entre un état quantique
exact de l'équation d'onde d'un système de N électrons interagissant et un état
quantique approximé que l'on construit en associant aux N électrons N
déffirentes orbitales, chacune d'entre elles étant une solution de l'équation
d'onde pour un électron. Le modèle orbitalaire est exact s'il n'y a aucune
interaction entre chacun des électrons.

Les conditions aux limites sont : $\psi_n (0) = 0$, $\psi_n(L)=0$, comme imposé 
par le puit de potentiel infini. Ces conditions sont satisfaites si la fonction
d'onde est sinusoidale, avec un nombre entier de demi-longeurs d'ondes entre 0
et L :
\begin{equation}
    \psi_n = A \sin\left( \frac{2\pi}{\lambda_n} \right) ; \hfill
    \frac{1}{2}n\lambda_n = L
\end{equation}
où A est constante.
On voit que c'est une solution de l'équation de Schrödinger, parce que :
\begin{equation}
    \frac{d\psi_n}{dx} = A \left( \frac{n\pi}{L} \right)
    \cos\left( \frac{n\pi}{L}x \right) ; \hfill \frac{d^2\psi_n}{dx^2} 
    = - A \left( \frac{n\pi}{L} \right)^2\sin\left( \frac{n\pi}{L}x \right)
\end{equation}

où l'énergie $\epsilon_n$ est donnée par :
\begin{equation}
    \epsilon_n = \frac{\hbar^2}{2m} \left( \frac{n\pi}{L}\right)^2
\label{en}
\end{equation}

On veut pouvoir loger N électrons sur la ligne. Comme prévu par le principe
d'exclusion de Pauli, deux électrons ne peuvent pas avoir le même nombre
quantique. Ansi, chaque orbitale ne peut être occupée que par deux électrons au
maximum. Ceci s'applique aux électrons dans les atomes, molécules, et solides.

Dans un solide linéaire, les nombres quantiques d'une orbitale d'un électron de
conduction sont $n$ et $m_s$, où $n$ est un nombre entier positif, et le nombre
quantique magnétique $m_s = \pm \frac{1}{2}$, suivant l'orientation de spin. Une
paire d'orbitales marquée par un nombre quantique $n$ peut recevoir deux
électrons, un avec un spin vers le haut et un avec un spin vers le bas. S'il y a
six électrons, alors l'état fondamental du système est rempli d'orbitales données
par le tableau suivant :

\begin{table}[ht]
    \begin{center}
        \begin{tabular}{rrr}
            \toprule
            n & $m_s$ & occupation par les électrons\\
            \midrule
            1 & $\uparrow$ & 1\\
            1 & $\downarrow$ & 1\\
            2 & $\uparrow$ & 1\\
            2 & $\downarrow$ & 1\\
            3 & $\uparrow$ & 1\\
            3 & $\downarrow$ & 1\\
            4 & $\uparrow$ & 0\\
            4 & $\downarrow$ & 0\\
            \bottomrule
        \end{tabular}
    \end{center}
    \label{}
    \caption{remplissage des orbitales pour un système à 6 électrons}
\end{table}

Plus d'une orbitale peut voir la même énergie. Le nombre d'orbitales avec la même
énergie est appelée \emph{dégénérescence}.

Soit $n_F$ le niveau d'énergie rempli le plus élevé, où on commence à remplir les
niveaux depuis le bas (n=1), et on continue à remplir les niveaux d'énergie 
supérieure, jusqu'à ce que tous les N électrons soient placés. Il est pratique de
supposer que N est un nombre pair. la condition $2n_F = N$ détermine $n_F$, la
valeur de n pour laquelle le niveau le plus élevé est rempli.

L'énergie de Fermi $\epsilon_F$ est définie comme l'énergie du plus haut niveau
rempli à l'état fondamental d'un système à N électrons. En utilisant la relation
\ref{en}, avec $n=n_F$, on a, à une dimension :
\begin{equation}
    \epsilon_F = \frac{\hbar^2}{2m}\left(\frac{n_F\pi}{L}\right)^2 = 
    \frac{\hbar^2}{2m}\left( \frac{N\pi}{2L} \right)^2
\end{equation}


\section{Statistique de Fermi-Dirac}

L'état fondamental est l'état d'un système de N électrons au zéro absolu. Ce qui
arrive quand on augmente la température ? C'est un problème standard dans la
mécanique statistique élémentaire, et la solution es donnée par la distribution
de Fermi-Dirac.

\begin{figure}
    \TODO
    \caption{La distribution de Fermi Dirac}
    \label{fermidir}
\end{figure}


L'énergie cinétique du gas d'électrons augmente quand la température augmente :
certains niveaux d'énergie sont occupés, d'autres restent vacants au zéro absolu,
et certains niveuax sont vides alors qu'ils sont occupés au zéro absolu. La
distribution de Fermi-Dirac donne la probabilité qu'une orbitale d'énergie
$\epsilon$ soit occupée par un gas d'électrons idéal à l'équilibre thermique.

\begin{equation}
    f(\epsilon) = \frac{1}{\exp\left( \frac{\epsilon - \mu}{k_BT} \right) + 1}
\end{equation}

La quantité $\mu$ est une fonction de la température. $\mu$ doit être choisie
pour un problème particulier d'une telle façon que le nombre total de
particules dans e système en dérive correctement, c'est à dire qu'on retrouve
bien N à la fin. Au zéro absolu, $\mu = \epsilon_F$, parce que la limite
$T\rightarrow 0$, la fonction $f(\epsilon)$ change de façon discontinue de la
valeur 1 (rempli) à la valeur zéro (vide), à $\epsilon = \epsilon_F = \mu$. Pour
toute température, $f(\epsilon)$ est égal à 1/2 lorsque $\epsilon = \mu$.

$\mu$ est le potentiel chimique, et on doit voir qu'au zéro absolu, le potentiel
chimique est égal à l'énergie de Fermi, définie comme l'énergie de la plus haute
orbitale remplie au zéro absolu.

La partie en queue de distribution, est cette parie pour laquelle $\epsilon - \mu
\ll k_BT$. Ici, le terme exponentiel est dominant dans le dénominateur, et ainsi
$f(\epsilon) \cong \exp(\frac{\mu - \epsilon}{k_BT})$. Cette limite est appelée
la distribution de Maxwell-Boltzmann.

\section{Généralisation à trois dimensions}

Les électrons sont des particules quantiques. Ils peuvent donc être décrits par
une fonction d'onde $\psi(\mathbf{r},t)$, qui est la solution de l'équation de
Schödinger suivante :


\section{conditions de Born von Karman}

\section{Sphère de Fermi et espace des états}

Dans l'état fondamental d'un système à N électrons libres, les orbitales occupées
peuvent représenter des points dans une sphère de l'espace réciproque. L'énergie
à la surface de la sphère est l'énergie de Fermi. Les vecteurs d'onde à la
surface de Fermi ont une intensité $k_F$ telle que :
\begin{equation}
    \epsilon_F = \frac{\hbar^2}{2m}k_F^2
    \label{eef}
\end{equation}

On peut alors voir qu'il n'y a qu'un seul vecteur d'onde permis, qui est distinct
du triplet de nombre quantiques $k_x, k_y, k_z$, pour l'élément de volume
$(2\pi/L)^3$ de l'espace réciproque. Ainsi, dans la sphère de volume $\frac{4\pi
k_F^3}{3}$, le nombre total d'orbitales est :

\begin{equation}
    2\cdot \frac{\frac{4\pi k_F^3}{3}}{\left( \frac{2\pi}{L}\right)^3} 
    = \frac{V}{3\pi^2} k_F^3 = N
\end{equation}

où le facteur 2 à gauche vient des deux valeurs permises par le nombre quantique
de spin pour chaque valeur de $\mathbf{k}$. Ainsi, on obtient :
\begin{equation}
    k_F = \left( \frac{3\pi^2N}{V}\right)^{1/3}
    \label{kkf}
\end{equation}

qui ne dépend que de la concentration en particules.

\begin{marginfigure}
    \TODO
    \caption{Sphère de Fermi}
    \label{spherefermi}
\end{marginfigure}

En combinant \ref{eef} et \ref{kkf}, on obtient :
\begin{equation}
    \epsilon_F = \frac{\hbar^2}{2m} \left(\frac{3\pi^2N}{V}\right)^{2/3}
    \label{ef}
\end{equation}

Ceci se rapporte à l'énergie de Fermi pour la concentation en électrons N/V. La
vitesse des électrons $v_F$ sur la surface de Fermi est :
\begin{equation}
    v_F = \left( \frac{\hbar k_F}{m}\right)
    = \left( \frac{\hbar}{m} \right)\left(\frac{3\pi^2 N}{V} \right)^{1/3}
\end{equation}

Les valeurs calculées de $k_F$, $v_F$ et $\epsilon_F$ sont données dans le
tableau suivant pour certains métaux. Le paramètre de rayon est un nombre sans
dimension $r_n = r_0/a_H$ où $a_H$ est le rayon de Bohr et $r_0$ le rayon de la
sphère qui contient un électron.

\begin{table*}[ht]
\footnotesize
\begin{center}
\begin{tabularx}{\textwidth}{rlRRRRRR}
\toprule
Valence & Métal & concentration en électrons ($10^{22} cm^{-3}$) & Paramètre de rayon $r_n$ & vecteur d'onde de Fermi ($10^8 cm^{-1}$) & vitesse de Fermi ($10^8 cm\cdot s^{-1}$) & énergie de Fermi (eV) & température de Fermi ($T_F =\epsilon_F / k_B$) ($10^4$ K)\\
\midrule
1 & Li & 4.70 & 3.25 & 1.11 & 1.29 & 4.72 & 5.48 \\
  & Na (78K) & 2.65 & 3.93 & 0.92 & 1.07 & 3.23 & 3.75\\
  & K (78K) & 1.40 & 4.86 & 0.75 & 0.86 & 2.12 & 2.46\\
  & Rb (78K) & 1.15 & 5.20 & 0.70 & 0.81 & 1.85 & 2.15\\
  & Cs (78K) & 0.91 & 5.63 & 0.64 & 0.75 & 1.58 & 1.83\\
  & Cu & 8.45 & 2.67 & 1.36 & 1.57 & 7.00 & 8.12\\
  & Ag & 5.85 & 3.02 & 1.20 & 1.39 & 5.48 & 6.36\\
  & Au & 5.90 & 3.01 & 1.20 & 1.39 & 5.51 & 6.39\\
2 & Be & 24.02 & 1.88 & 1.93 & 2.23 & 14.14 & 16.41\\
  & Mg & 8.60 & 2.65 & 1.37 & 1.58 & 7.13 & 8.27\\
  & Ca & 4.60 & 3.27 & 1.11 & 1.28 & 4.68 & 5.43\\
  & Sr & 3.56 & 3.56 & 1.02 & 1.18 & 3.95 & 4.58\\
  & Ba & 3.20 & 3.69 & 0.98 & 1.13 & 3.65 & 4.24\\
  & Zn & 13.10 & 2.31 & 1.57 & 1.82 & 9.39 & 10.90\\
  & Cd & 9.28 & 2.52 & 1.40 & 1.62 & 7.46 & 8.66\\
3 & Al & 18.06 & 2.07 & 1.75 & 2.02 & 11.63 & 13.49\\
  & Ga & 15.30 & 2.19 & 1.65 & 1.91 & 10.35 & 12.01\\
  & In & 22.49 & 2.41 & 1.50 & 1.74 & 8.60 & 9.98\\
4 & Pb & 13.20 & 2.30 & 1.57 & 1.82 & 9.37 & 10.87\\
  & Sn(w) & 14.48 & 2.23 & 1.62 & 1.88 & 10.03 & 11.64\\
\bottomrule
\end{tabularx}
\end{center}
\caption{Paramètres de la surface de Fermi pour le modèle de l'élecrton libre
pour quelques métaux à température ambiante}
\label{}
\end{table*}

Maintenant, on trouve une expression du nombre d'orbitales par gamme d'énergie
unitaire., $D(\epsilon)$, appelée densité d'états\footnote{si on est rigoureux,
$D(\epsilon)$ représente plutôt la densité d'états à une particules, ou la
densité des orbitales}. On utilise \ref{ef} pour obtenir le nombre total
d'orbitales d'énergie inférieure ou égale à $\epsilon$ :
\begin{equation}
    N = \frac{V}{3\pi^2} \left(\frac{2m\epsilon}{\hbar^2}\right)^{3/2}
\end{equation}

Ainsi, la densité d'états est :
\begin{equation}
    D(\epsilon) \equiv \frac{dN}{d\epsilon} = 
    \frac{V}{2\pi^2} \cdot 
    \left(\frac{2m}{\hbar^2}\right)^{3/2}\cdot\epsilon^{1/2}
\end{equation}
Ce résultat peut être exprimé plus simplement en comparant les deux formules,
pour obtenir, à $\epsilon$ :
\begin{equation}
    D(\epsilon) \equiv \frac{dN}{d\epsilon} = \frac{3N}{2\epsilon}
\end{equation}
Avec un facteur de l'ordre de l'unité, le nombre d'orbitales par unité d'énergie
à l'energie de Fermi est le nombre total d'électrons de conduction divisé par
l'énergie de Fermi, exactement ce qu'on espérait.

\section{Énergie et chaleur spécifique du gaz d'électrons libres}

\section{Densité d'états électroniques}

\section{conductivité thermique}

Constante de Wiedemann-Franz

\section{Effet Pelletier}
