\chapter{Historique de la matière condensée}

"on ne saurait connaître une science sans en connaître l'histoire" auguste comte


rien ne vaut d'aller aux sources, de se mettre en contact aussi fréquent et complet que possible avec ceux qui ont fait la science et qui en ont le mieux représenté l'aspect vivant. paul langevin
\begin{description}
\item[1611] Johannes Kepler (1571-1630) fait l'hypothèse que la symétrie hexagonale des cristaux de neige est due à un arrangement hexagonal de particules d'eau sphériques.
\item[1669] Steno donne la loi de la constance des angles entre les faces d'un cristaux.
\item[1784] Haüy (1743-1822) établi que les faces des cristaux peuvent être repérées par un ensemble de trois nombres entiers appelés indices de Miller, ce qui l'amène à associier à chaque cristal un réseau de petits volumes de matière qu'il nomme « molécules intégrantes » et qui correspondent aujourd'hui à la notion de maille élémentaire
\item[ ] Bravais (1811-1863) et Schoenflied (1853-1928) montrent à l'aide de la théorie des groupes que les cristaux peuvent être catégorisés en 32 groupes ponctuels de symétrie et en 230 groupes d'espace
\item[19e siècle] apparition de nouveaux domaines scientifiques:
    \begin{itemize}
    \item mécanique des milieux continus
    \item électromagnétisme
    \item thermodynamique, les duex qui attribuent des propriétés macroscopiques mesurables à la matière
    \item module d'Young
    \item susceptibilité optique
    \item conductivité électrique et thermique
    \item grandeurs macroscopiques qui permettent une description phénoménologique satisfaisante des solides
    \end{itemize}
\item[1808]  modèle de Dalton (atome)
\item[1826] Naumann énonce l'idée des 7 systèmes cristallins
\item[1830] Hessel énonce le fait qu'il y a 32 classes cristallines car il y a 32 groupes ponctuels sont consistens avec l'idée qu'il y a des symmétries translationnelles.
\item[1849] Auguste Bravais fait l'hypothèse d'une structure réticulaire des cristaux
\item[1869] rayons cathodiques découverts par Hittorf.
\item[1878] William Crookes (1832--1919) étudie la conduction de l'électricité dans des gaz à faible pression (tubes à vide appelés tubes de Crookes). 
\item[1894] différents gorupes sont énumérés parmi les 230 groupes d'espaces, cohérent avec le fait qu'il n'y aque 230 possiblitisé de symmétries cristallines.
\item[1895] découverte des rayons X par Wilhelm Röntgen (1843 -- 1923) dans un tube de Crookes.
\item[1897] modèle de Thomson (atome)
\item[1900] Paul Karl Ludwig Drude (1863-1906) développe un modèle quasi-classique de la conductiondes métaux en supposant ceux-ci remplis d'un gaz d'électrons libres, auxquels il applique la physique statistique de Ludwig Boltzmann
\item[1901] Röntgen reçoit le premier prix Nobel de physique pour la découverte des rayons X. Il fait le premier cliché de la main d'Anna Bertha Röntgen (pose de 25min).
\item[1902] Modèle de Lewis (atome)
\item[1904] modèle de nagoaka (atome)
\item[1904] Lord Rayleigh (john william strutt) nobel de physique pour ses recherches sur les densités des gaz les plus importants et pour sa découvert ed l'argon
\item[1905] philipp eduard et anton von lenard nobel travaux rayons cathodiques
\item[1906] joseph john thomson nobel théorie cinétique et conduction de l'électricité dans les gaz
\item[1910] johannes diderik van der Waals nobel équation d'état des gaz et liquides
\item[1911] modèle de Rutherford (atome)
\item[1912] Max von Laue (1879 - 1960) réalise la première expérience de diffraction des rayons X sur un cristal
diffraction d'électrons et de neutrons
\item[1913] Heike Kamerlingh onnes nobel recherches propriétés de la matière aux basses températures, conduisant à la production d'hélium liquide
\item[1913] modèle de Bohr (atome)
\item[1914] Max von Laue  reçoit le prix Nobel de physique pour sa découverte de la diffraction des rayons X par les cristaux
\item[1915] sir William Henry Brag et son fils sir William Lawrence Bragg recoivent le prix Nobel de physique pour leurs contributions à l'analyse de la structure cristalline au moyen des rayons X
\item[1917] charles glover barkla nobel physique pour sa découverte des rayonnements röntgen caractéristiques des divers éléments
\item[1918] nobel max karl ernst ludwig planck découverte des quanta d'énergie
\item[1919] Albert W. Hull, Peter Debye et Paul Scherrer] méthode de Debye et Scherrer pour la diffraction sur poudre
\item[1920] charles édouard guillaume nobel physique anomalies alliages acier au nickel
\item[1921] nobel einstein effet photoélectrique
\item[1922] nobel physique niels henrik david bohr contribution à la recherche sur la structure des atomes et sur le rayonnement qu'ils émettent
\item[1923] nobel physique robert andrews millikan travaux charge élémentaire
\item[1924] De Broglie prédit que les particules doivent se comporter comme des ondes.
\item[1924] nobel physique karl manne georg siegbahn découverte dans la spectroscopie des rayons X
\item[1925] modèle de Schrödinger (atome)
\item[1925] principe d'exclusion de Pauli, à la base de la statistique de fermi-dirac
\item[1926] Emnrico Fermi et Paul Dirac font la statistique de Fermi-Dirac
\item[1927?] George Paget Thomson fait passer un faisceu d'électrons au travers d'un film mince de métal et observe les figures de diffraction prédites par De Broglie
\item[1927?] Clinton Joseph Davisson et Lester Halbert Germer (bell) font passer leur faisceu par une grille cristalline (cristal de nickel).
\item[1927] Arthur Holly Compton nobel physique effet compton
\item[1929] Louis de Broglie prix nobel de physique pour sa découverte de la nature ondulatoire des électrons
\item[1933] Arnold Sommerfeld (1868--1951) et Hans Bethe(1906--2005) reprennent le modèle de Drude
\item[1933] Erwin Schrödinger et Paul Adrien Maurice Dirac nobel physique pour la découverte de nouvelles formes productives de la théorie atomique
\item[1935] James Chadwick nobel physique pour sa découverte du neutron
\item[1936] Seitz commence à démontrer les représentations irréductibles des groupes d'espace
\item[1937] Thomson et Davisson obtiennent le prix nobel de physique. pour leur découverte expérimentale de la diffraction des électrons par les cristuax
\item[1939] enrest orlando lawrence nobel physique pour l'invention et le développement du cyclotron??
\item[1945] Ernest O. Wollan fait la première expérience de diffraction de neutrons. il est rejoint en 1946 par Clifford Shull.
\item[1945] Wolfgang Pauli nobel physique pour le principe d'exclusion
\item[1952] Felix Bloch (1905-1983) et Edward Mills Purcell obtiennent le prix nobel de physique pour leur développement de nouvelles méthodes de mesure magnétiques nucléaires fines (RMN)
\item[1956] William Bradford Shockley, John Bardeen et Walter Houser Brattain] nobel physique pour leurs recherches sur les semiconducteurs et leur découverte de l'effet transistor
\item[1962] Lev Davidovitch Landau nobel physique pour ses théories d'avant-garde sur la matière condensée, en particulier l'hélium liquide.
\item[1963] nobel physique Eugene Paul Wigner pour ses contributions à la théorie du noyau atomique et des particules élémentaires, en particulier par la découverte et l'application de principes fondamentaux de symétrie.
\item[1967] Bethe prix nobel de physique pour sa contrubution à la compréhension de la nucléosynthèse stellaire
\item[1970] Louis Eugène Félix Néel nobel physique pour ses travaux fondamentaux et ses découvertes sur l'antiferromagnétisme et le ferrimagnétisme qui ont conduit à des applications importantes en physique du solides
\item[1972] John Bardeen, Leon Neil Cooper, John Robert Schieffer nobel physique pour leur théorie développée conjointement sur les supraconducteurs, BCS.
\item[1973] Leo Esaki, Ivar Giaever nobel physique pour leurs découvertes expérimentales sur les phénomènes d'effet tunnel dans les semi-conducteurs et les supraconducteurs respectivement
\item[1973] Brian David Josephson nobel physique poru ses prédictions théoriques sur les propriétés d'un super-courant à travers une barrière tunnel, en particulier l'effet Josephson
\item[1984] Shectmann, Steinhardt et al découvrent les quasi-cristaux, des substances qui ne sont ni cristallines, ni vitreuses, mais néanmoins ordonnées d'une façon quasi-périodique
\item[1977] philip warren anderson, nevill francis mott, john hasbrouck van vleck nobel physique pour leurs recherches théoriques fondamentales sur la structure électronique des systèmesma gnétiques désordonnés.
\item[1985] nobel physique klaus von klitzing pour la découverte de l'effet hall quantique entier.
\item[1986] ernst ruska nobel physique pour ses travaux fondamentaux en optique électronique ; gerd binning, heinrich rohrer pour leur conception du microscope à effet tunnel à balayage
\item[1987] johannes georg bednorz, karl alexander müller pour leur percée importante dans la découverte de la supraconductivité de matériaux céramiques
\item[1991] Pierre-Gilles de Gennes pour sa découverte du fait uqeles méthodes développées pour l'étude des phénomènes d'ordre dans des systèmes simples peuvent être généralisées à des formes plus complexes de la matière, en particulier aux cristaux liquides et aux polymères.
\item[1994] Shull obtient le prix Nobel de physique pour la mise au point de la technique de diffraction neutronique
\item[1994] Bertram Brockhouse obtient le prix nobel de physique pour le développement de la diffusion inélastique de neutrons et la mise au point du spectromètre trois-axes
\item[20e siècle] découvertes, nouveaux outils
\item[-] étude des propriétés des solides aux basses températures
\item[-] introduction de la mécanique quantique
\item[-] apparition du microscope électronique
\item[2003] alexeï alexeïevitch abrikosov, vitaly lazarevich ginzburg, anthony J. Leggett pour des travaux novateurs dans la théorie des supraconducteurs et des superfluides.
\item[2007] albert fert, peter grünberg nobel physique pour la découverte de la magnétorésistance géante
\item[2010] andre geim, konstantin novoselov nobel physique pour des expériences fondamentales concernant le graphène, matériau bi-dimensionnel
\item[2012] structure perovskite au plomb ?
\end{description}

rayleigh (diffusion élastique)
compton (diffusion inélastique)
effet photoélectrique, ionisation
auger (émission auger)
rutherford (spectroscopie de rétrodiffusion)
TEM, basé sur la diffusion d'électrons
SEM idem
EBSD] diffraction d'électrons rétrodiffusés.
paul peter ewald (1888--1985)

méthode poudre] géométrie bragg-brentano
monocristaux] méthode de laue, diffractomètre quatre cercles, etc.

modèle de bohr de l'atome -> structure fine des raies spectrales de l'hydrogène (modèle de Bohr-Sommerfeld).
constante de structure fine alpha introduite par Sommerfeld
théorie des rayons X de sommerfeld
étudiants célèbres de Sommerfeld] Werner Heisenberg, Wolfgang Pauli, Peter Debye, Hans Bethe, Paul Sophus Epstein, Walter Heitler, Paul Peter Ewald, Karl Bechert, Alfred Landé, Wilhelmm Lenz, Albrecht Unsöld, Gregor Wentzel, Helmut Hönl, Fritz Bopp. post doctorants] Isidor Isaac Rabi, Linus Pauling.

DFT] density functional theory. origines dans le modèle de Llewellyn Thomas et Enrico Fermi fin 1920. ; 1960] Pierre Hohenberge, Walter Kohn et Lu Sham] formalisme héorique sur lequel repose la méthode actuelle.
théorie hartree-fock

étudiants de Langevin] Louis de Brogile, Léon Brillouin, Irène Joliot-Curie
théorie du magnétisme par Langevin, nature microscopique. interpréter l'observation de Pierre Curie] la susceptibilité des matériaux paramagnétiques varie avec la température.

hall, effet

- effet Hall, edwin hall, USA
-Sir Joseph Thomson. 1897 découverte de l'électron. trouve que] atome divisible, il existe des électrons, q/m de l'électron et de h+, modèle atome plum puddig.
- paul karl ludwig drud, introduit le symbole c, célérité. 1900 modèle de drude pour la conduction des électrons. professeur de arnold sommerfeld.
- wolfgang pauli] 1925] principe d'exclusion  pour les fermions
- arnold sommerfeld] 1928, 1933] 1er modèle quantique du solide. professeur de Debye et Heisenberg.
