\chapter{Théorie classique des vibrations harmoniques dans un cristal}

\section{Gaz d'excitations}

On a pu étudier dans la première partie la structure d'un réseau
cristallin à l'état de repos. Cette image ne permet pas de décrire les
phénomènes de transport qui peuvent exister dans les matériaux. En particulier,
on peut décrire le phénomène de conduction de la chaleur en imaginant les atomes
vibrant autour de leur position d'équilibre, en poussant occasionnelelement
leurs voisins.
Si ces mécanismes sont corrects, on se heurte très rapidement à des calculs très
compliqués si l'on souhaite étudier le réseau dans son ensemble. Il devient
nécessaire de substituer, à l'étude de particules individuelles, un modèle
collectif.

Nous allons aborder dans cette partie la théorie de la chaleur spécifique à
travers le modèle d'Einstein, qui a mené au modèle de Debye, dans lequel
l'énergie thermique d'un solide est distribuée selon les modes de vibration du 
cristal dans son ensemble. Chaque mode normal, que l'on peut représenter par une
onde stationnaire, peut être analysé comme la somme de deux ondes propagatives
se propageant dans deux directions différentes. Ces ondes sont polarisées
longitudinalement : ce sont des ondes sonores.

Par analogie avec les photons des ondes électromagnétiques, les quanta du
champ vibrationnel du réseau sont appelés phonons.

Dans cette image, on considère le réseau comme un volume composé uniquement d'un
\emph{gaz} de phonons, dont le comportement peut être décrit par analogie avec
un corps noir.

Ces ondes de réseau peuvent transporter de grandes quantités d'énergie, avec
des vitesses de l'ordre de la vitesse du son. La difficulté maintenant est de
trouver un mécanisme pour interpréter le transport de la chaleur, qui correspond
à des mécanismes de diffusion plus qu'à des phénomènes radiatifs. L'idée d'un
gaz implique qu'il y a des \emph{collisions} entre des \emph{particules}, qui
contribuent à la résistance. Cela semble contradictoire avec l'hypothèse que les
états de phonons sont des modes normaux, qui, par définition, sont
indépendants et incapables d'interagir. Il est à noter que lorsque l'on
démontrera les équations du mouvement, on fera une approximation :
l'énergie potentielle est traitée comme si seuls les termes quadratiques
du déplacement par rapport aux positions d'équilibre, ont une influence.
Cette approximation (dite harmonique) n'est pas réelle et néglige, en grande
partie, les interactions phonons-phonons ainsi que la diffusion résultant
des imperfections du réseau.

En parallèle, on représentera la conduction électronique par un modèle collectif
basé sur la mécanique ondulatoire. Un électron ne sera plus simplement représenté
comme une particule simple tentant de pénétrer un réseau d'atomes.
On pourra construire dans les solides des ondes, qui peuvent se regrouper en
paquet d'ondes et être guidés dans le cristal, comme si les ions
n'existaient pas. Les vibrations thermiques et les interactions ondes-électrons
perturbent la régularité du réseau ; cela sera étudié dans une partie
ultérieure.

Ces modèles collectifs ont l'avantage d'être facile à modéliser : les
excitations se comportent comme des particules dans un gaz : on peut facilement 
transposer les concepts de distribution des viteses, de libre parcours moyens, 
etc. familiers dans la théorie cinétique classique.


\section{Approximation harmonique}

Supposons qu'une paire d'atomes d'un cristal, séparée d'une distance
$\mathbf{r}$, contribue d'une quantité $\phi(\mathbf{r})$ à l'énergie
potentielle du cristal. $\phi$ est par exemple le potentiel de Lennard-Jones.
Si le modèle du réseau statique est correct, chaque atome reste fixé sur
sa position du réseau de Bravais. Par conséquent, l'énergie potientelle du
cristal peut s'écrire comme la somme des contributions de toutes les paires
distinctes :

\begin{equation}
    U = \half \sum_{\mathbf{R,R'}} \phi(\mathbf{R-R'}) = \frac{N}{2} \sum_{R\neq0} \phi(\mathbf{R})
    \label{potentiela} 
\end{equation}

Si, dans le modèle, on permet aux atomes, dont la position d'équilibre est
$\mathbf{R}$, de se trouver à une position $\mathbf{r(R)} \neq \mathbf{R}$,
alors, on peut remplacer l'équation \ref{potentiela} par :

\begin{equation}
    U = \half \sum_{\mathbf{R,R'}} \phi(\mathbf{r(R) - r(R')})
    = \half \sum_{\mathbf{R,R'}} \phi(\mathbf{R - R' + u(R) - u(R')})
    \label{potentielu}
\end{equation}

Ici, l'énergie potentel dépend de la variable $\mathbf{u(R)}$, qui est la
distance qui sépare deux atomes.

Pour simplifier le problème, on est obligés d'utiliser l'approximation dite
\emph{harmonique}. Dans celle-ci, les atomes ne sont pas beaucoup déviés de leur
position d'équilibre. Si tous les $\mathbf{u(R)}$. sont petits, alors on peut
développer l'énergie potentiel $U$ autour de sa position d'équilibre, avec
un développement de Taylor à trois dimensions :

\begin{equation}
    f(\mathbf{r+a}) = f(\mathbf{r}) + \mathbf{a \cdot \nabla}f(\mathbf{r}) + \half (\mathbf{a\cdot\nabla})^2 f(\mathbf{r}) + \frac{1}{3!}
    (\mathbf{a \cdot \nabla})^3 f(\mathbf{r}) + \cdots
    \label{taylor}
\end{equation}

Ce qui donne, appliqué à $\mathbf{r=R-R'}$ et $\mathbf{a = u(R) - u(R')}$, on a :

\begin{equation}
    U = \frac{N}{2} \sum \phi(\mathbf{R}) + \half \sum_{\mathbf{R,R'}} (\mathbf{u(R) - u(R')}) \cdot \mathbf{\nabla} \phi(\mathbf{R-R'}) +
    \frac{1}{4} \sum_{\mathbf{R,R'}} [(\mathbf{u(R)-u(R'))\cdot\nabla}]^2 \phi(\mathbf{R-R'}) + O(u^3)
    \label{taylorpot}
\end{equation}

Le coefficient $\mathbf{u(R)}$ dans le terme d'ordre 1 est simplement :

\begin{equation}
    \sum_{\mathbf{R'}} \nabla \phi(\mathbf{R-R'})
    \label{coeflineaireapproxharmh}
 \end{equation}

C'est simplement le minimum de la force éxercé sur l'atome en position
$\mathbf{R}$  par tous les autres atomes, lorsque chacun d'entre eux est placé
à sa position d'équilibre. À l'équilibre, le bilan des forces sur les atomes
est nul. Par conséquent, ce terme d'ordre 1 est nul.

Ainsi, le premier terme non nul du potentiel est le terme quadratique.
L'approximation harmonique consiste à ne retenir que ce terme. On écrit ainsi
l'énergie potentielle comme :

\begin{equation}
    U = U^{eq} + U^{harm}
    \label{approxharm}
\end{equation}
 
Dans cette approximation, $U^{eq}$ est l'énergie potentielle à l'équilibre, et
en outre,

\begin{eqnarray}
    U^{harm} &  = &  \frac{1}{4} \sum_{\mathbf{R,R'}} [u_\mu(\mathbf{R}) - u_\mu(\mathbf{R'})] \phi_{\mu\nu}(\mathbf{R-R'})[u_\nu(\mathbf{R}) -
    u_\nu(\mathbf{R'})] \\
    \phi_{\mu\nu}(\mathbf{r}) & = & \frac{\partial^2 \phi(\mathbf{r})}{\partial r_\mu \partial r_\nu}
    \label{uharm}
\end{eqnarray}

Comme $U^{eq}$ est constante, on peut l'ignorer dans un grand nombre de
problèmes dynamiques, et se restreindre à $U^{harm}$

L'approximation harmonique est généralement le point de départ de toutes les
théories de dynamique du réseau. D'autres corrections sur $U$, en particulier
sur les termes du troisième et du quatrième ordre, permettent de comprendre
plus en profondeur certains phénomènes physiques.

Le potentiel harmonique est souvent écrit dans la forme donnée par \ref{uharm},
mais une forme plus générale est plutôt :

\begin{equation}
    U^{harm} = \half \sum_{\mathbf{R,R'}} u_\mu(\mathbf{R}) D_{\mu\nu} (\mathbf{R-R'})u_\nu(\mathbf{R'})
\end{equation}

Pour le cas décrit ici, $D_{\mu\nu}$ a la forme :

\begin{equation}
    D_{\mu\nu}(\mathbf{R-R'}) = \delta_{R,R'} \sum_{\mathbf{R''}} \phi_{\mu\nu}(\mathbf{R-R''}) - \phi_{\mu\nu} (\mathbf{R-R''})
\end{equation}

\section{Chaîne 1D à 1 atome par maille}

Considérons à présent les vibrations élastiques d'un cristal composé d'un atome
par maille primitive. On cherche la fréquene d'une onde élastique en fonction du
vecteur d'onde qui décrit l'onde avec les constantes élastiques.
Le modèle présenté ici consiste en fait à prendre un cristal tri-dimensionnel et
à réduire, à partir des symmétries, l'étude aux directions de propagation $[100]$, $[110]$ ou $[111]$ pour un cristal cubique par exemple.

Deux cas de figure peuvent se présenter : lorsqu'une onde se propage le long d'
une de ces directions, des plans d'atomes bougent en phase, soit parallèlement
au vecteur d'onde, soit perpendiculairement.

Posons $u_s$ le déplacement du plan d'atomes $s$ (on se restreint alors à un
problème à une dimension).
Pour chaque vecteur d'onde, il y a trois modes solutions : un de polarisation
longitudinale et deux de polarisations transverse.

Appliquons l'approximation harmonique au problème. On considère également que
seules les interacitons entre plus proches voisins ($p=\pm 1$) sont
significatives. La force totale induite par les plans $s\pm 1$ sur les plans $s$
s'écrit :

\begin{equation}
F_s = C(u_{s+1} - u_s) + C (u_{s-1} - u_s)
\end{equation}

Par conséquent, l'équation du mouvement peut s'écrire :
\begin{equation}
M \frac{du_s^2}{dt^2} = C (u_{s+1} -u_{s-1} - 2 u_s)
\end{equation}
où M est la masse d'un atome.

Considérons à présent les solutions sous formes d'ondes propagatives, c'est à
dire qui s'exprimenent en $exp(-i\omega t)$. On obtient alors :
\begin{equation}
-M\omega^2 u_s = C (u_{s+1} + u_{s-1} - 2u_s )
\label{eq:mvtreduite}
\end{equation}
Soit le résultat :
\begin{equation}
u_{s\pm 1} = u \exp(is\mathbf{k}\cdot\mathbf{a}) \exp \pm (i\mathbf{k}\cdot
\mathbf{a})
\label{eq:redmvt}
\end{equation}

où $\mathbf{a}$ correspond au vecteur séparant chacun des plans, et $\mathbf{k}$
est le vecteur d'onde de l'onde propagative.

En injectant \ref{eq:redmvt} dans \ref{eq:mvtreduite} et après calculs, on
retrouve la relation de dispersion :
\begin{equation}
    \omega^2 = \frac{2C}{M}\left(1-\cos \mathbf{k}\cdot\mathbf{a}\right)
    \label{eq:reldispphonon1D}
\end{equation}

La limite de la première zone de Brillouin se situe pour 
$\mathbf{k} = \pm \frac{\pi}{a}$. On peut montrer à partir de la relation de
dispersion \ref{eq:reldispphonon1D} que la pente de $\omega$ en fonction de
$\mathbf{k}$ est 0 en bord de zone :

\begin{equation}
\frac{d\omega^2}{d\mathbf{k}} = \frac{2Ca}{M} \sin \mathbf{k}\cdot\mathbf{a} = 0
\end{equation}

\begin{marginfigure}
\TODO
\caption{Fréquence $\omega$ tracée en fonction du vecteur d'onde $\mathbf{k}$.}
\label{fig:reldispphonon1D}
\end{marginfigure}

On peut alors tracer $\omega$ en fonction de $\mathbf{k}$, comme cela est reporté
sur la figure \ref{fig:reldispphonon1D} :
\begin{equation}
\omega = \sqrt{\frac{4C}{M}} \left| \sin \frac{1}{2} ka \right|
\end{equation}

\section{Chaîne 1D à deux atomes par maille primitive}

La relation de dispersion des phonons présente des caractéristiques différentes 
si l'on considère des cristaux comportant deux atomes ou plus par maille
primitive. Par exemple, la structure NaCl ou la structure diamant comportent
toutes deux deux atomes dans leur maille primitive. Pour chaque mode de
polarisation dans une direction de propagation donnée, la relation de
dispersion $\omega(\mathbf{k})$ présente deux branches : une branche acoustique
et une branche optique. Chacune de ces deux branches présentent des
modes longitudinaux et transverse.

Dans le cas général, si la maille primitive contient $p$ atomes, on se
retrouve avec $3p$ branches dans la relation de dispersion : 3 branches
acoustiques et $3p-3$ branches optiques. Par exemple, KBr ou le germanium
cristallisent avec 3 atomes par maille et disposent de 6 branches :
1 longitudinale acoustique, 2 transversale acoustique, 1 longitudinale optique
et 2 transverse optique.

Considérons, pour démontrer ceci, un cristal cubique composé d'atomes de masse
$M_1$ et d'atomes de masse $M_2$. En fait, il n'est pas nécessaire d'avoir des
masses différentes, mais il faut soit que la force séparant deux atomes dans des
sites non équivalents varie (dans ce cas, comme dans le cadre du cours de 
Pr Dimitri Roditchev, on définit deux constantes de raideur entre les atomes),
ou que le paramètre de maille $\mathbf{a}$ soit différents pour deux atomes dans
des sites non équivalents.

Appelons $u_s$ le déplacement de l'atome de masse $M_1$ dans la diade $s$ et
$v_s$ le déplacement de l'atome de masse $M_2$ dans la diade $s$. Considérons
une onde qui se propage dans la direction $\mathbf{a}$ (dans cette direction, un
plan ne contient qu'un type d'ion).

Écrivons à présent les équations du mouvement sous l'hypothèse que chaque plan
n'intergit qu'avec ses plus proches voisins :

\begin{eqnarray}
M_1 \frac{d^2u_s}{dt^2} & = & C (v_s + v_{s+1} - 2u_s)\\
M_2 \frac{d^2v_s}{dt^2} & = & C (u_{s+1} + u_s - 1v_s)
\label{eq:mvtdeuxatomes}
\end{eqnarray}

On cherche une solution sous la forme d'ondes propagatives, à savoir :
\begin{eqnarray}
u_s & = &  u \exp (iska) \exp (-i\omega t) \\
v_s & = & v \exp (iska) \exp (-i\omega t)
\label{eq:deuxatomessolexp}
\end{eqnarray}

En injectant \ref{eq:deuxatomessoleexp} dans \label{eq:mvtdeuxatomes}, on
retrouve :

\begin{eqnarray}
-\omega^2 M_1 u & = & C v [ 1 + \exp -ika ] - 2Cu\\
-\omega^2 M_2 v & = & C u [ 1 + \exp ika ] - 2Cv
\end{eqnarray}

On peut résoudre ce système d'équations seulement si le déterminant du système
est nul, ce qui revient à écrire :

\begin{equation}
M_1M_2\omega^4 - 2C(M_1 + M_2) \omega^2 + 2C^2 (1-\cos ka) = 0
\end{equation}

Si l'on prend $ka << 1$, alors 
$\omega^2 \approx 2C \left(\frac{1}{M_1} + \frac{1}{M_2} \right)$, 
ce qui correspond à la branche optique, ou alors
$\omega^2 \approx \frac{\frac{1}{2}C}{M_1+M_2} k^2 a^2$,
ce qui correspond à la branche accoustique.

Pour $k=\pm \frac{\pi}{a}$, on a donc $\omega^2 = \frac{2C}{M_1}$ ou
$\omega^2 = \frac{2C}{M_2}$.

Les déplacements de la particules dans les modes transverses acoustiques et
transverse optique sont :

\TODO figure

Pour la branche optique, à $k=0$, $\frac{u}{v} = - \frac{M_2}{M_1}$ : les atomes
vibrent en opposition de phase, avec un centre de masse fixe.

Si les deux atomes ont des charges différentes, on peut exciter un momouvement
de ce type avec le champ électrique d'une onde lumineuse. C'est pour cette
raison que l'on parle de branche optique.

Pour un $k$ quelconque, le rapport $\frac{u}{v}$ est complexe. Une autre 
solution pour le ratio d'amplitude aux petits $k$ est $u=v$, obtenu pour
$k=0$. Les atomes bougent en phase, comme pour des vibrations acoustiques de
grandes longueurs d'ondes.

Des solutions sous forme d'onde n'existent pas pour toutes les fréquences,
comme c'est par exemple le cas entre $\sqrt{\frac{2C}{M_1}}$ et
$\sqrt{\frac{2C}{M_2}}$. C'est une caractéristique des ondes élastiques dans les
réseaux polyatomiques : il y a une bande de fréquences enterdites au bord de la
première zone de Brillouin.

\section{Chaîne 3D}

\section{Loi de Dulong et Petit}

Le potentiel d'énergie dépend de la variable dynamique $\mathbf{u(R)}$ et peut
être abordée à partir du problème gouverné par l'hamiltonien :

\begin{equation}
    \mathcal{H} = \sum_{\mathbf{R}} \frac{\mathbf{P(R)}^2}{2M} + U
\end{equation}
où $\mathbf{P(R)}$ est la quantité de mouvement de l'atome dont la position
d'équilibre est $\mathbf{R}$.

La densité d'énergie thermique d'un cristal est donnée par :

\begin{equation}
    u = \frac{1}{V} \frac{\int d\Gamma e^{-\beta H} H}{\int d\Gamma e^{-\beta H}}
\end{equation}

Ici, $\beta = \frac{1}{k_B T}$ et $d\Gamma$ est l'élément de volume dans
l'espace des phases du cristal :

\begin{eqnarray}
    d\Gamma & = & \prod_{\mathbf{R}} d\mathbf{u(R)} d\mathbf{P(R)}\\
     & = & \prod_{\mathbf{R},\mu} du_\mu (\mathbf{R}) dp_\mu (\mathbf{R})
\end{eqnarray}

Cette formule est obtenue en moyennant les états selon toutes les configurations
ioniques possibles, en pondérant chaque état par le facteur $e^{-\frac{E}{k_BT}}$,
où $E$ est l'énergie de la configuration.

On peut écrire la densité d'énergie thermique dans une forme plus générale :

\begin{equation}
    u = - \frac{1}{V} \frac{\partial}{\partial \beta} \left( \ln \int d\Gamma e^{-\beta H} \right)
    \label{densiteenergiethermique}
\end{equation}

Dans l'approximation harmonique, la dépendance en température de l'intégrale
apparaissant dans \ref{densiteenergiethermique} peut être extraite à partir d'un
changement de variables :

\begin{align}
    \mathbf{\overline{u}(R)} & = & \sqrt{\beta} \mathbf{u(R)} & \qquad &
    d\mathbf{\overline{u}(R)} & = & \beta^{\frac{3}{2}} d\mathbf{u(R)} \\
    \mathbf{\overline{P}(R)} & = & \sqrt{\beta} \mathbf{u(R)} & \qquad &
    d\mathbf{\overline{P}(R)} & = & \beta^{\frac{3}{2}} d\mathbf{u(R)}
\end{align}

On obtient alors :

\begin{eqnarray}
    \int d\Gamma e^{-\beta H} & = & \int d\Gamma \exp \left[ -\beta \left( \sum
    \frac{\mathbf{P(R)}^2}{2M} + U^{eq} + U^{harm} \right) \right] \\
    & = & e^{-\beta U^{eq}} \beta^{-3N} \cdot I 
    \label{injected}
\end{eqnarray}

Où $I$ est une intégrale indépendante de la température. Par conséquent, elle ne
contribue pas dans la dérivée en fonction de $\beta$ quand on injecte
\ref{injected} dans \ref{densiteenergiethermique}. On peut alors réduire
la densité d'énergie thermique en :

\begin{eqnarray}
    u &= & -\frac{1}{V} \frac{\partial}{\partial \beta} \ln \left( e^{-\beta U^{eq}} \beta^{-3N} \cdot cst \right) \\
    & = & \frac{U^{eq}}{V} + \frac{3N}{V}k_BT\\
    & = & u^{eq} + 3nk_BT
\end{eqnarray}

\begin{itemize}
    \item À $T=0$, $u=u^{eq}$, ce qui correspond à la théorie du réseau statique;
    \item Pour $T\neq0$, on corrige $u^{eq}$ de la quantité $3nk_BT$
\end{itemize}

L'énergie $k_BT$ est de l'ordre du centième d'électron-volt ; par conséquent,
la correction est assez faible. Généralement, on considère plutôt la chaleur
spécifique :

\begin{eqnarray}
    c_v & =&  \frac{\partial u}{\partial T} \\
    & = & 3nk_BT
\end{eqnarray}

Ainsi, la chaleur spécifique due aux vibrations de résaeu n'est que de $3k_B$
par ion. Ce résultat constitue la loi de Dulong et Petit. Dans un solide
monoatomique, il y a \SI{6.022e23}{\per\mol} ions. Ainsi, la capacité molaire
est :

\begin{equation}
    c_v^{molaire} = \SI{5.96}{\calorie\per\mol\per\kelvin} = \SI{25}{\kilo\joule\per\mol\per\kelvin}
\end{equation}

En pratique, la variation est légèrement différente, comme présentée sur la
figure \ref{fig:capacitecalorifique}

\begin{marginfigure}
    \TODO
    \caption{Évolution de la capacité calorifique molaire de l'argon en fonction de la température, comparaison avec la loi de Dulong et Petit}
    \label{fig:capacitecalorifique}
\end{marginfigure}

On peut constater expérimentalement que :
\begin{enumerate}
    \item À faible température, la chaleur spécifique est bien plus faible que l valeur donnée par la loi de Dulong et Petit, elle est
        pratiquement nulle à température nulle ;
    \item À plus haute température, la valeur donnée par la loi de Dulong et Petit n'est pas atteinte expérimentallement.
\end{enumerate}

Le point 2 peut être expliqué par les défauts de l'approximation harmonique :
cette hypothèse est trop restrictive lorsque les atomes s'éloignent trop de leur
position d'équilibre.

Le point 1 ne peut pas être expliqué de façon classique. Pour cette raison, nous
allons introduire une quantification des vibrations du réseau.


