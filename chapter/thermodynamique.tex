\chapter{Thermodynamique}

\TODO revoir ça, page 616 aschcroft :
Par défaut cristallin, on entend généralement une région où l'arrangement atomique des ions diffère drastiquement de celui du cristal parfait. Les défauts sont appelés défauts surfaciques, linéaires ou ponctuels, suivant que la région imparfaite se propage à l'échelle atomique sur une, deux ou trois dimensions.

Dans cette partie, nous décrirons quelques unes des imperfections possibles et en quoi leur présence a de grandes implications sur les propriétés des solides.
Deux grandes classes de défauts :

\begin{enumerate}
    \item les lacunes ou interstitiels : ce sont des défauts ponctuels, consistant en l'absence d'ions (ou la présence d'ions étrangers). De tels défauts sont entièrement responsables de la conductivité électrique observée dans les cristaux ioniques, et peuvent profondément altérer leurs propriétés optiques (en particulier, leur couleur). De plus, leur présence est un phénomène normal à l'équilibre thermique, donc ils sont une fonction intrinsèque des cristaux réels.
    \item les dislocations. Ce sont des défauts linéaires qui, même si absents du cristal idéal à l'équilibre thermique, sont pratiquement toujours présents dans un échantillon réel. Les dislocations sont essentielles pour expliquer la force observée de cristaux réels, et les taux de croissance cristalline observés.
\end{enumerate}

\section{Défauts ponctuels, considération thermodynamiques}

Les défauts ponctuels sont présents même dans les cristaux à l'équilibre thermodynamique. Nous pouvons l'illustrer en considérant le type le plus simple : une \emph{lacune}, ou \emph{défaut Schottky}, dans un réseau de bravais monoatomique. Une lacune apparaît à chaque fois que'un nœud du réseau de Bravais serait normalement occupé par un ion dans le cristal parfait, mais n'a pas d'ion associé. Si le nombre $n$ de telles lacunes à la température $T$ est une variable thermodynamique extensive (\ie s'il est proportionnel au nombre total d'ions $N$ lorsque $N$ est très grand), alors on peut estimer sa taille en minimisant le potentiel thermodynamique appropiré. Si le cristal est à une pression constante $P$, alors l'énergie libre de Gibbs s'écrira :

\begin{equation}
    G = U - TS + PV
\end{equation}

Pour comprendre comment $G$ dépend de $n$, il est facile de penser qu'un cristal de $N$ ions contenant $n$ sites de lacunes peut être vu comme un cristal parfait de $(N+n)$ ions, dans lequel $n$ ions ont été enlevés. Par conséquent, le volume $V(n)$ sera, en première approximation $(N+n)v_0$ où $v_0$ est le volume libre par ion dans le cristal parfait.

Pour un choix particulier de $n$ sites privés de leurs ions, on peut, en principe, calculer $F_0(n) = U - TS$ pour un cristal imparfait particulier. Si $n$ est très petit devant $N$, alors on peut s'attendre à ce qu'il ne dépende que du nombre de lacunes, et non de l'arrangement spatial. On doit ajouter l'entropie $S$ pour une configuration fixée de lacunes, une contribution ultérieure $S^{config}$ rendant compte du désordre induit par les $\frac{(N+n)!}{N!n!}$ façons de choisir les $n$ sites de lacunes parmi $N+n$ :

\begin{equation}
    S^{config} = k_B \log \frac{(N+n)!}{N!n!}
\end{equation}

Par conséquent, l'énergie libre de Gibbs s'écrit

\begin{equation}
    G(n) = F_0 (n) - TS^{config}(n) + P(N+n)v_0
\end{equation}
En utilisant la formule de Stirling, valable ici pour $N/n >> 1$, on peut approximer :

\begin{equation}
    \pd{S^{config}}{n} = k_B \log \left( \frac{N+n}{n} \right) \approx k_B \log \left( \frac{N}{n} \right)
\end{equation}

Par conséquent,

\begin{equation}
    \pd{G}{n} = \pd{F_0}{n} + Pv_0 - k_B T \log \left( \frac{N}{n} \right)
    \label{eq:gibbsthermo}
\end{equation}

Lorsque $n<<N$, on peut écrire :

\begin{equation}
    \pd{F_0}{n} \approx \pdc{F_0}{n}{n=0} = \epsilon
\end{equation}

où $\epsilon$ est indépendant de $n$. Par conséquent, l'équation \ref{eq:gibbsthermo} nous dit que $G$ est minimal pour

\begin{equation}
    n = N \exp - \frac{\epsilon + P v_0}{k_B T}
    \label{eq:minimizgibbs}
\end{equation}

Pour calculer $\epsilon$, il faudrait écrire l'énergie potentielle totale d'un réseau de $N+n$ ions avec $n$ lacunes sous la forme de l'approximation harmonique :

\begin{equation}
    U = U^{eq} + U^{harm}
\end{equation}

De là, on peut calculer $F_0$ depuis la fonction de partition :

\begin{eqnarray}
    e^{-\beta F_0} & = & \sum_E e^{-\beta E}\\
    & = & e^{-\beta U^{eq}} \sum_{E_{harm}} e^{-\beta E_{harm}}
\end{eqnarray}
ici, $\beta = \frac{1}{k_B T}$, et la somme sur $E_{harm}$ est une somme sur toutes les avleurs propres de la contribution harmonique de l'hamiltonien. Évidemment, cela donnera un $F_0$ qui sera l'énergie potentielle d'équilibre pour un réseau avec des lacunes plus l'énergie libre des phonons :

\begin{equation}
    F_0 = U^{eq} + F^{ph}
    \label{eq:energieeq}
\end{equation}

Le second terme est générallement petit comparé au premier, de telle sorte à ce qu'en première approximation, $\epsilon$ soit :

\begin{equation}
    \epsilon_0 = \pdc{U^{eq}}{n}{n=0}
\end{equation}

Ceci est l'énergie potentielle indépendante de la température, nécessaire pour enlever un ion. À des pressions normales (atmosphériques), $P v_0$ est négligeable en comparaison, et donc :

\begin{equation}
    n = N e^{-\beta \epsilon_0}
    \label{eq:phononcorrec}
\end{equation}

Comme $\epsilon_0$ peut être estimé de l'ordre de quelques électron volts\footnote{on peut l'estimer, grosso modo, à la taille de l'énergie de cohésion par particule}, $n/N$ sera en fait petit, mais pas nul.

La correction de phonons de l'équation \ref{eq:phononcorrec} du second terme de \ref{eq:energieeq} augmentera en $n$. C'est à cause de l'introduction de lacunes, qui tendent à diminuer certaines fréquences de mode normaux (et donc les énergies de phonon associées). Par conséquent, on aura un terme $\pd{F^{ph}}{n}$ négatif.

L'analyse ci-dessous n'est valable que pour un type de défauts ponctuels : les lacunes sur un nœud du réseau de Bravais. En général, évidemment, il peut y avoir plus d'un type de lacune (dans les réseaux polyatomiques). Il y a également la possibilité que des ions supplémentaires occupent des positions non occupées dans le cristal parfait, un type de défaut ponctuel nommé \emph{interstitiel}. Par conséquent, on peut généraliser cette analyse, pour permettre à $n_j$ défauts ponctuels du $j$\ieme type : si tous les $n_j$ sont petits devant $N$, alors chaque type de défaut ne se produira que dans des nombres donnés par la généralisation de l'équation \ref{eq:minimizgibbs} (en ignorant la correction $P v_0$) :

\begin{equation}
    n_j = N_j e^{-\beta \epsilon_j},\quad \epsilon_j = \pdc{F_0}{n_j}{n_j=0}
\end{equation}

où $N_j$ est le nombre de sites où un défaut du type $j$ peut se produire.

Les $\epsilon_j$ sont généralement très grand devant $k_B T$, et si, en plus, la plus petite des deux valeurs de $\epsilon_j$ (disons $\epsilon_1$ et $\epsilon_2$) sont toutes deux loin devant $k_B T$, alors $n_j$ sera bien plus grand que tous les autres $n_j$, \ie le défaut avec le plus petit $\epsilon_j$ sera, en très grande proportion, le plus abondant.

Cependant, cette dernière relation n'est valable que si le nombre de chaque type de défaut est indépendant, ce qui suit la minimisation de l'énergie libre, indépendamment de la relation avec les $n_j$. S'il y a des contraintes sur les $n_j$, alors il faut refaire le probèlme.
La plus importante de ces contraintes est la neutralité de charge. On ne peut pas avoir un ensemble de défauts consistant entièrement en des lacunes d'ions positifs dans un cristal ionique, par exemple, sans contrebalancer avec une charge positive, qui aurait une énergie Coulombienne très importante. Cet excès de charge doit être balancé soit pas des interstices d'ions positifs, soit pas des lacunes d'ions négatifs, ou alors une combinaison de cela.

Par conséquent, l'énergie libre doit être minimisée relativement à la contrainte :

\begin{equation}
    0 = \sum_j q_j n_j
    \label{eq:minimisfreeenerg}
\end{equation}

où $q_j$ est lacharge du $j$\ieme type de défauts ($q_j = +e $pour une lacune d'ion négatif ou pour un site interstitiel positif, ou $q_j = -e$ pour une lacune positive ou un interstitiel négatif). Introduisons le multiplcateur de Lagrange $\lambda$, pour ensuite prendre la contrainte en compte pour minimiser non plus $G$, mais $G + (\lambda \sum_j q_j n_j)$. Cela nous ramène à :

\begin{equation}
    n_j = N_j e^{-\beta(\epsilon_j + \lambda q_j)}
\end{equation}
où l'inconnue $\lambda$ est déterminée par la conditition \ref{eq:minimisfreeenerg}.

Généralement, les petits $\epsilon_j$ pour chaque type de harge sont séparés en énergie d'un nombre grand devant $k_B T$.\footnote{et si ce n'est pas le cas, le formalisme des défauts de Schottky et de Frenkel ne peut pas être construit.} Par conséquent, il y a un type de défauts dominants pour chaque charge, dont les nombres sont donnés par :

\begin{eqnarray}
    n_+ & = & N_+ e^{-\beta(\epsilon_+ + \lambda e)}\\
    n_- & = & N_- e^{-\beta(\epsilon_- - \lambda e)}, \quad \epsilon_\pm = \min_{q_j = \pm e} (\epsilon_j)
    \label{eq:condition}
\end{eqnarray}

Comme la densité de tous les autres défauts satisfait :

\begin{eqnarray}
    n_j & << & n_+, \quad q_j = +e\\
    n_j & << & n_-, \quad q_j = -e
\end{eqnarray}

La neutralité de charge requiert à très haute précision que :
\begin{equation}
    n_+ = n_-
\end{equation}

Comme l'équation \ref{eq:condition} requiert :
\begin{equation}
    n_+ n_- = N_+ N_- e^{-\beta(\epsilon_+ + \epsilon_-)}
\end{equation}

On trouve alors :
\begin{equation}
    n_+ = n_- = \sqrt{N_+ N_-} e^{-\beta(\epsilon_+ + \epsilon_-)/2}
\end{equation}

Par conséquent, la contrainte de la neutralité de charge réduit la concentration des défauts du type le plus abondant et augmente la concentration du type de défaut le plus abondant de la charge opposée.

Même dans les cristaux ioniques les plus simples (diatomiques), il y a plusieurs moyens parmi lesquels la neutralité de charge peut être mise en œuvre. Il peut y avoir essentiellement des nombres égaux de lacunes ioniques positives ou négatives, connues, dans ce contexte, comme des défauts de \emph{Schottky}. D'un autre côté, il peut y avoir essentiellement des nombres égaux de lacunes et d'interstices du même ion, ce qui est appelé \emph{défaut de Frenkel}. Les halogénures d'alcalins ont des défauts de type Schottky, les halogénures d'argent, de type Frenkel. (La troisième possibilité, des nombres égaux d'interstices ioniques positifs et négatifs, ne semble pas arriver, les interstices sont généralement plus coûteux en énergie que les lacunes du même ion).


\section{Équilibre thermodynamique}

Il est moins probable qu'un défaut linéaire ou surfacique peut, comme un défaut ponctuel, avoir des concentrtions qui ne s'annulent pas à l'équilibre thermodynamique. L'énergie de formation d'un de ces défauts étendus sera proportionnelle aux dimensions linéaires ($N^\frac{1}{3}$), ou à l'aire de la section ($N^\frac{2}{3}$) du cristal. Cependant, le nombre de chemins possible pour en introduire (suivant que les lignes ne soient pas trop zig-zag, ou les surfaces en vagues), le nombre n'apparaît pas plus logarithmique en $N$ que pour les défauts ponctuels. Par conséquent, même si le coût en énergie d'un défaut ponctuel (indépendant de $N$) est plus que compensé par le gain en entropie (de l'ordre de $\log N$), cela n'est probablement pas le cas pour un défaut linéaire ou surfacique.

Les défauts linéaires ou surfacique sont, en toute vraissemblance, des configurations métastables de cristaux. Cependant, l'équilibre thermique peut les faire approcher lentement, de sorte que les buts pratiques des défauts soinet considérés comme gelés. Il est aussi facile d'arranger des concentrations en défauts ponctuels hors équilibre, qui peuvent avoir une persistance considérable (par exemple, en refroidissant très rapidement un cristal qui a été à l'équilibre). La concentration à l'équilibre des défauts ponctuels peut nous ramener à l aforme de Maxwell-Boltzmann, et ladensité des défauts linéaires et ponctuels réduit de fawon correspondante vers zéro, par le lent chauffage ou refroidissement. La restauration de la concentartion en défauts à l'équilibre est connu sous le nom de \emph{recuit}.

\section{Croissance des cristaux}
dislocations, etc.

\section{autres :}

La description des structures cristallines et l'étude du modèle ionique ont été
réalisées dans le cadre du cristal parfait. Dans la réalité, le solide présente
des défauts dont les principaux sont : les phonons (vibration thermiques), les
défauts atomiques (lacunes, interstitiels, impuretés), les défauts électroniques
(électrons, trous, excitons), les imperfections dans l'arrangement atomique
(dislocations, fautes d'empilement) et la surface où sont localisésdes atomes
particuliers du point de vue énorgétique et structurale.

Dans le cristal réel, de nombreuses propriétés physiques et chimiques proviennent
directement de l'existence de ces défauts. Nous nous limitons ici à l'étude des
défauts atomiques qui sont à l'origine de la conduction ionique. Dans quelques
cas particuliers, la conductivité dans les solides ioniques est voisine de celle
d'un électrolyte liquide. Le matériau est alors potentiellement utilisable comme
électrolyte solide dans un système électrochimique.

On peut montrer aisément que la présence de défauts, jusqu'à une certaine
concentration conduit à une réduction d'enthalpie libre, donc à une stabilisation
du réseau. L'introduction d'un défaut ponctuel (impureté, lacune, interstitiel)
dans un cristal supposé parfait nécessite une augmentation d'enthalpie
assimilable à une quantité d'énergie
\footnote{$H_f = E_f + p\delta v$ où $\delta v$ est approximativement le volume
    atomique (\SI{20}{\cubic\angstrom}). Pour p = 1atm, on a
    $p \delta v = \SI{e-5}{\electronvolt}$ négligeable devant
$E_f \approx \SI{1}{\electronvolt}$ }
$E_f$, énergie de formation du défaut. Mais elle produit aussi une augmentation
importante d'entropie de configuration $\Delta S_c$, car ce défaut peut occuper
un grand nombre de positions. Dans le cas le plus simple où le défaut occupe un
site anionique et possède la symétrie de l'atome qu'il remplace,
l'entropie\footnote{Le terme entropique $\Delta S_v$ dû aux variations des modes
de vibrations automiques est généralement négligé.} calculée pour n défauts
disposés sur N sites atomiques est\footnote{On écrit $S_c = k \log(P)$ où le
nombre de complexions P dans le cristal est le nombre d'arrangements possibles de
n défauts edes N-n atomes dans les N positions du réseau : $P =
\frac{N!}{(N-n)!n!}$. On utilise l'approximation de stirling : $\log N! = N \log
N - N$} :

\begin{equation}
    \Delta S_c = - N k (x \log x + (1-x) \log (1-x))
\end{equation}

où l'on a posé $x = n/N$ la concentration en défauts. Ce terme est toujours
positif et inférieur à 1. Il varie très brutalement pour x petit : $dS/dx
\rightarrow \infty$ pour $x\rightarrow 0$. L'énergie, elle, ne varie que comme
$NxE_f$. En conséquence, l'introduction de défauts dans le solide parfait
provoque une diminution de l'enthalppie libre.

\begin{marginfigure}
    \TODO
    \caption{variation d'énergie par introduction de défauts dans un cristal
    parfait}
    \label{varenergcristparf}
\end{marginfigure}

L'enthalpie libre, $G = NxE_f + NkT (x \log x + (1-x)\log(1-x))$, est minimale
lorsque x vérifie la relation :
\begin{equation}
    \frac{x}{1-x} = \exp - \left( \frac{E_f}{kT} \right)
\end{equation}

Soit encore, pour x petit, c'est à dire $E_f$ assez grand devant kT :
\begin{equation}
    x \approx \exp - (E_f / kT)
\end{equation}

À une température donnée, il existe donc une certaine concentration de défauts
qui minimise G. Le défaut prédominant est évidemment celui associé à la plus
petite valeur de $E_f$ et il est très largement fonction de la structure
cristalline.

\begin{table*}[ht]
    \begin{tabularx}{\textwidth}{lXX}
        \toprule
        Cristal & Structure & Défaut prédominant \\
        \midrule
        Halogénures alcalins & NaCl & Schottky \\
        Oxydes alcalino-terreux & NaCl & Schottky \\
        AgCl, AgBr & NaCl & Frenkel cationique \\
        Halogénures de césium, TlCl & CsCl & Schottky \\
        BeO & Wurtzite, ZnS & Schottky \\
        Fluorures d'alcalino-terreux, $CeO_2$,$ThO_2$ & Fluorine, $CaF_2$ &
        Frenkel anionique\\
        \bottomrule
    \end{tabularx}
    \label{}
    \caption{défaut ponctuel prédominant dans différents cristaux}
\end{table*}

\begin{figure}
    \TODO
    \caption{Défauts de Schottky et de Frenkel}
    \label{schottkyfrenkel}
\end{figure}

Dans des structures compactes, le défaut prédominant est le défauts de Schottky
avec même nombre de lacunes cationiques et anioniques pour assurer
l'électroneutralité. L'énergie de formation $E_s$ de la paire de Schottky
correspond à l'extraction d'un cation (énergie $E_{fc}$) et d'un anion (énergie
$E_{fa}$) qui se localisent à la surface du cristal.

Calculons les concentrations en volume des lacunes cationiques et anioniques pour
une température donnée, en fixant la contrainte $x_c = x_a$.

Le passage d'un ion à la surface du cristal revient à faire passer sa constate de
Madelung de M à M/2. Les énergies de formation des lacunes cationiques et
anioniques devraient donc être égales à la moitié de la contribution coulombienne
à l'énergie réticulaire. Mais, à cette énergie, il faut soustraire l'énergie de
polarisation érsultant du processus de relaxation ionique (une lacune anionique,
par exemple, porte une charge positive qui attire les anions). Les contributions
dues à la relaxation ionique n'ont aucune raison d'être identiques pour les deux
types d'ions. En conséquence, $E_{fc} \neq E_{fa}$ et $x_c \neq x_a$.

La condition de neutralité électrique n'est plus respectée dans le cristal.
Celui-ci réagit en disposant l'exès de lacunes chargées sous la surface de façon
à créer une couche dipolaire (couche de Debye) qui restitue la neutralité
électrique en volume et diminue considérablement la portée du champ électrique dû
à la surface.

Avec la cotnraine $x_c = x_a$, la minimisation de l'enthalpie libre, avec $dx_c =
dx_a$, conduit à la loi d'action de masse pour l'équilibre cristal-lacunes :
\begin{equation}
    x_c \cdot x_a = \exp(-E_s/kT)
\end{equation}
avec $E_s = E_{fc} + E_{fa}$, soit, avec nos hypothèses : 
$x_c = x_a = exp(-E_s/2kT)$.
(Dans le cas de NaCl : $E_s \approx 2.3 eV$, $x_c = x_a = 3\cdot 10^{-17}$ à
300K\footnote{cette valeur est sous-estimée d'au moins un à deux ordres de
grandeur. Les termes correctifs proviennent d'une part de la modification des
virbations ioniques ($\Delta S_v$) et d'autre part de la variation de $E_S$ avec
la température que l'on corrèle à la dilatation du cristal}, ce qui correspond à
$5\cdot10^5$ défauts par \si{\cubic\centi\metre}).

Le défaut de Frenkel est prédominant dans des structures ouvertes (faible nombre
de coordination) et concerne principalement les cations (taille inférieure à
celle des anions). Il existe deux exceptions importantes à cette règle :

\begin{itemize}
    \item le cas de la structure fluorine dans laquelle l'anion a un faible
        nombre de coordination (4 au lieu de 8 pour le cation), ce qui lui permet
        d'aller elativement facilement en position interstitielle (cas des ions
        $F^-$ dans $CaF_2$ et $O^{2-}$ dans $ZrO_2$). On parle dans ce cas de
        défauts anti-Frenkel.
    \item le cas des halogénures d'argent qui possèdent une structure type NaCl
        (donc relativement compacte) et dans laquelle des proportions importantes
        d'ions $Ag^+$ peuvent occuper une position interstitielle. Dans cette
        position, un ion $Ag^+$ est entouré tétraédriquement par 4 ions $Cl^-$ et
        également à la même distance par 4 ions $Ag^+$. La stabilisation du
        défaut est due à une interaction covalente maruée entre les atomes
        d'argent et de chlore.
\end{itemize}

La concentration en défauts de Frenkel à l'équilibre est donnée par :

\begin{equation}
    x_i \cdot x_v = \exp (-E_F / kT)
\end{equation}
où $x_v$ et $x_i$ sont respectivement les concentrations en lacunes et en
insterstitiels. $E_F$ l'énergie de formation du défaut de Frenkel ($E_F$ =
\SI{1.35}{\electronvolt} pour AgCl).
 
Dans le cristal pur, on considère en général que $x_i = x_v = \exp -E_F/2kT$.
 
Pour les deux types de défauts (Frenkel et Schottky), on peut observer des
associations de défauts atomiques par interaction électrostatique, par exemple
entre une lacune anionique de charge nette +e et une lacune anionique de charge
nette -e. Ces associations se comportent comme des dipôles.
 
Les défauts atomiques ont également la possibilité de piéger des défauts
électroniques. Ainsi, la charge positive de la lacune anionique lui permet de
piéger un électron. Le défaut constitue un obget hydrogénoïde donnant lieu,
comme un atome d'hydrogène, à des niveaux d'énergie et des absorptions optiques
caractéristiques. L'absorption sélective a souvent lieu dans le visible, d'où le
nom de centre F\footnote{de Farbzentrum, centre colloré en allemand} donné à
l'ensemble lacune-électron.
 
Le centre F peut être considéré, pour simplifier, comme une cage cubique où se
trouve localisé l'électron, l'arête de la cage est peu différente de l'arête a
de la maille cristalline. Si le potentiel est pris nul dans la cage et infini à
l'extérieur, les avelurs propres de l'énergie de l'électron sont :
 
\begin{equation}
    E = \frac{\hbar^2}{2m}\frac{\pi^2}{a^2} (n_x^2 + n_y^2 + n_z^2)
\end{equation}

où $n_i$ sont des entiers non nuls.
 
L'énergie correspondant au passage de l'état fondamental au premier état excité
est :
 
\begin{equation}
    \Delta E = 3 \frac{\hbar^2}{2m} \frac{\pi^2}{a^2}
\end{equation}
 
La variation de l'énergie en $a^{-2}$ est observée pour les halogénures alcalins.
La taille des lacunes est un peu supérieure à la taille de la maille du fait des
interactions attractives de l'électron du défaut par les actions voisins.
 
\begin{marginfigure}
    \TODO
    \caption{absorption lumineuse pour les halogénures alcalins contenant des
    centres F. l'énergie deltaE de la première transition est reportée en
coordonnées logarithmiques en fonction de l'arête a de la maille cristalline}
    \label{abslumhalogalcal}
\end{marginfigure}
 
Un cristal de NaCl contenant des centres F est obtenu par chauffage du cristal
en présence de vapeur de sodium (ou de potassium). Il se créé un excès de
cations par rapport à la stoechiométrie et des lacunes anioniques que les
électrons provenant de l'ionisation du sodium transforment en centre F. La
transition vers le premier état excité est responsable d'une couleur jaune pour
NaCl.
 
D'autres exemples d'associations de défauts atomiques et électroniques sont
monrtés dans la figure suivante. Les centres colorés sont aussi formés par
irradiation (rayonnement X ou $\gamma$) et sont à l'origine de la couleur de
nombreuses pierres précieuses : topaze bleu, améthyste, etc.
 
\begin{marginfigure}
    \TODO
    \caption{représentation chématique de centres colorés dans des cristaux
    ioniques}
    \label{centrescolor}
\end{marginfigure}
