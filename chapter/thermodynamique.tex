\chapter{Thermodynamique}

La description des structures cristallines et l'étude du modèle ionique ont été
réalisées dans le cadre du cristal parfait. Dans la réalité, le solide présente
des défauts dont les principaux sont : les phonons (vibration thermiques), les
défauts atomiques (lacunes, interstitiels, impuretés), les défauts électroniques
(électrons, trous, excitons), les imperfections dans l'arrangement atomique
(dislocations, fautes d'empilement) et la surface où sont localisésdes atomes
particuliers du point de vue énorgétique et structurale.

Dans le cristal réel, de nombreuses propriétés physiques et chimiques proviennent
directement de l'existence de ces défauts. Nous nous limitons ici à l'étude des
défauts atomiques qui sont à l'origine de la conduction ionique. Dans quelques
cas particuliers, la conductivité dans les solides ioniques est voisine de celle
d'un électrolyte liquide. Le matériau est alors potentiellement utilisable comme
électrolyte solide dans un système électrochimique.

On peut montrer aisément que la présence de défauts, jusqu'à une certaine
concentration conduit à une réduction d'enthalpie libre, donc à une stabilisation
du réseau. L'introduction d'un défaut ponctuel (impureté, lacune, interstitiel)
dans un cristal supposé parfait nécessite une augmentation d'enthalpie
assimilable à une quantité d'énergie
\footnote{$H_f = E_f + p\delta v$ où $\delta v$ est approximativement le volume
    atomique (\SI{20}{\cubic\angstrom}). Pour p = 1atm, on a
    $p \delta v = \SI{e-5}{\electronvolt}$ négligeable devant
$E_f \approx \SI{1}{\electronvolt}$ }
$E_f$, énergie de formation du défaut. Mais elle produit aussi une augmentation
importante d'entropie de configuration $\Delta S_c$, car ce défaut peut occuper
un grand nombre de positions. Dans le cas le plus simple où le défaut occupe un
site anionique et possède la symétrie de l'atome qu'il remplace,
l'entropie\footnote{Le terme entropique $\Delta S_v$ dû aux variations des modes
de vibrations automiques est généralement négligé.} calculée pour n défauts
disposés sur N sites atomiques est\footnote{On écrit $S_c = k \log(P)$ où le
nombre de complexions P dans le cristal est le nombre d'arrangements possibles de
n défauts edes N-n atomes dans les N positions du réseau : $P =
\frac{N!}{(N-n)!n!}$. On utilise l'approximation de stirling : $\log N! = N \log
N - N$} :

\begin{equation}
    \Delta S_c = - N k (x \log x + (1-x) \log (1-x))
\end{equation}

où l'on a posé $x = n/N$ la concentration en défauts. Ce terme est toujours
positif et inférieur à 1. Il varie très brutalement pour x petit : $dS/dx
\rightarrow \infty$ pour $x\rightarrow 0$. L'énergie, elle, ne varie que comme
$NxE_f$. En conséquence, l'introduction de défauts dans le solide parfait
provoque une diminution de l'enthalppie libre.

\begin{marginfigure}
    \TODO
    \caption{variation d'énergie par introduction de défauts dans un cristal
    parfait}
    \label{varenergcristparf}
\end{marginfigure}

L'enthalpie libre, $G = NxE_f + NkT (x \log x + (1-x)\log(1-x))$, est minimale
lorsque x vérifie la relation :
\begin{equation}
    \frac{x}{1-x} = \exp - \left( \frac{E_f}{kT} \right)
\end{equation}

Soit encore, pour x petit, c'est à dire $E_f$ assez grand devant kT :
\begin{equation}
    x \approx \exp - (E_f / kT)
\end{equation}

À une température donnée, il existe donc une certaine concentration de défauts
qui minimise G. Le défaut prédominant est évidemment celui associé à la plus
petite valeur de $E_f$ et il est très largement fonction de la structure
cristalline.

\begin{table*}[ht]
    \begin{tabularx}{\textwidth}{lXX}
        \toprule
        Cristal & Structure & Défaut prédominant \\
        \midrule
        Halogénures alcalins & NaCl & Schottky \\
        Oxydes alcalino-terreux & NaCl & Schottky \\
        AgCl, AgBr & NaCl & Frenkel cationique \\
        Halogénures de césium, TlCl & CsCl & Schottky \\
        BeO & Wurtzite, ZnS & Schottky \\
        Fluorures d'alcalino-terreux, $CeO_2$,$ThO_2$ & Fluorine, $CaF_2$ &
        Frenkel anionique\\
        \bottomrule
    \end{tabularx}
    \label{}
    \caption{défaut ponctuel prédominant dans différents cristaux}
\end{table*}

\begin{figure}
    \TODO
    \caption{Défauts de Schottky et de Frenkel}
    \label{schottkyfrenkel}
\end{figure}

Dans des structures compactes, le défaut prédominant est le défauts de Schottky
avec même nombre de lacunes cationiques et anioniques pour assurer
l'électroneutralité. L'énergie de formation $E_s$ de la paire de Schottky
correspond à l'extraction d'un cation (énergie $E_{fc}$) et d'un anion (énergie
$E_{fa}$) qui se localisent à la surface du cristal.

Calculons les concentrations en volume des lacunes cationiques et anioniques pour
une température donnée, en fixant la contrainte $x_c = x_a$.

Le passage d'un ion à la surface du cristal revient à faire passer sa constate de
Madelung de M à M/2. Les énergies de formation des lacunes cationiques et
anioniques devraient donc être égales à la moitié de la contribution coulombienne
à l'énergie réticulaire. Mais, à cette énergie, il faut soustraire l'énergie de
polarisation érsultant du processus de relaxation ionique (une lacune anionique,
par exemple, porte une charge positive qui attire les anions). Les contributions
dues à la relaxation ionique n'ont aucune raison d'être identiques pour les deux
types d'ions. En conséquence, $E_{fc} \neq E_{fa}$ et $x_c \neq x_a$.

La condition de neutralité électrique n'est plus respectée dans le cristal.
Celui-ci réagit en disposant l'exès de lacunes chargées sous la surface de façon
à créer une couche dipolaire (couche de Debye) qui restitue la neutralité
électrique en volume et diminue considérablement la portée du champ électrique dû
à la surface.

Avec la cotnraine $x_c = x_a$, la minimisation de l'enthalpie libre, avec $dx_c =
dx_a$, conduit à la loi d'action de masse pour l'équilibre cristal-lacunes :
\begin{equation}
    x_c \cdot x_a = \exp(-E_s/kT)
\end{equation}
avec $E_s = E_{fc} + E_{fa}$, soit, avec nos hypothèses : 
$x_c = x_a = exp(-E_s/2kT)$.
(Dans le cas de NaCl : $E_s \approx 2.3 eV$, $x_c = x_a = 3\cdot 10^{-17}$ à
300K\footnote{cette valeur est sous-estimée d'au moins un à deux ordres de
grandeur. Les termes correctifs proviennent d'une part de la modification des
virbations ioniques ($\Delta S_v$) et d'autre part de la variation de $E_S$ avec
la température que l'on corrèle à la dilatation du cristal}, ce qui correspond à
$5\cdot10^5$ défauts par \si{\cubic\centi\metre}).

Le défaut de Frenkel est prédominant dans des structures ouvertes (faible nombre
de coordination) et concerne principalement les cations (taille inférieure à
celle des anions). Il existe deux exceptions importantes à cette règle :

\begin{itemize}
    \item le cas de la structure fluorine dans laquelle l'anion a un faible
        nombre de coordination (4 au lieu de 8 pour le cation), ce qui lui permet
        d'aller elativement facilement en position interstitielle (cas des ions
        $F^-$ dans $CaF_2$ et $O^{2-}$ dans $ZrO_2$). On parle dans ce cas de
        défauts anti-Frenkel.
    \item le cas des halogénures d'argent qui possèdent une structure type NaCl
        (donc relativement compacte) et dans laquelle des proportions importantes
        d'ions $Ag^+$ peuvent occuper une position interstitielle. Dans cette
        position, un ion $Ag^+$ est entouré tétraédriquement par 4 ions $Cl^-$ et
        également à la même distance par 4 ions $Ag^+$. La stabilisation du
        défaut est due à une interaction covalente maruée entre les atomes
        d'argent et de chlore.
\end{itemize}

La concentration en défauts de Frenkel à l'équilibre est donnée par :

\begin{equation}
    x_i \cdot x_v = \exp (-E_F / kT)
\end{equation}
où $x_v$ et $x_i$ sont respectivement les concentrations en lacunes et en
insterstitiels. $E_F$ l'énergie de formation du défaut de Frenkel ($E_F$ =
\SI{1.35}{\electronvolt} pour AgCl).
 
Dans le cristal pur, on considère en général que $x_i = x_v = \exp -E_F/2kT$.
 
Pour les deux types de défauts (Frenkel et Schottky), on peut observer des
associations de défauts atomiques par interaction électrostatique, par exemple
entre une lacune anionique de charge nette +e et une lacune anionique de charge
nette -e. Ces associations se comportent comme des dipôles.
 
Les défauts atomiques ont également la possibilité de piéger des défauts
électroniques. Ainsi, la charge positive de la lacune anionique lui permet de
piéger un électron. Le défaut constitue un obget hydrogénoïde donnant lieu,
comme un atome d'hydrogène, à des niveaux d'énergie et des absorptions optiques
caractéristiques. L'absorption sélective a souvent lieu dans le visible, d'où le
nom de centre F\footnote{de Farbzentrum, centre colloré en allemand} donné à
l'ensemble lacune-électron.
 
Le centre F peut être considéré, pour simplifier, comme une cage cubique où se
trouve localisé l'électron, l'arête de la cage est peu différente de l'arête a
de la maille cristalline. Si le potentiel est pris nul dans la cage et infini à
l'extérieur, les avelurs propres de l'énergie de l'électron sont :
 
\begin{equation}
    E = \frac{\hbar^2}{2m}\frac{\pi^2}{a^2} (n_x^2 + n_y^2 + n_z^2)
\end{equation}

où $n_i$ sont des entiers non nuls.
 
L'énergie correspondant au passage de l'état fondamental au premier état excité
est :
 
\begin{equation}
    \Delta E = 3 \frac{\hbar^2}{2m} \frac{\pi^2}{a^2}
\end{equation}
 
La variation de l'énergie en $a^{-2}$ est observée pour les halogénures alcalins.
La taille des lacunes est un peu supérieure à la taille de la maille du fait des
interactions attractives de l'électron du défaut par les actions voisins.
 
\begin{marginfigure}
    \TODO
    \caption{absorption lumineuse pour les halogénures alcalins contenant des
    centres F. l'énergie deltaE de la première transition est reportée en
coordonnées logarithmiques en fonction de l'arête a de la maille cristalline}
    \label{abslumhalogalcal}
\end{marginfigure}
 
Un cristal de NaCl contenant des centres F est obtenu par chauffage du cristal
en présence de vapeur de sodium (ou de potassium). Il se créé un excès de
cations par rapport à la stoechiométrie et des lacunes anioniques que les
électrons provenant de l'ionisation du sodium transforment en centre F. La
transition vers le premier état excité est responsable d'une couleur jaune pour
NaCl.
 
D'autres exemples d'associations de défauts atomiques et électroniques sont
monrtés dans la figure suivante. Les centres colorés sont aussi formés par
irradiation (rayonnement X ou $\gamma$) et sont à l'origine de la couleur de
nombreuses pierres précieuses : topaze bleu, améthyste, etc.
 
\begin{marginfigure}
    \TODO
    \caption{représentation chématique de centres colorés dans des cristaux
    ioniques}
    \label{centrescolor}
\end{marginfigure}
