\chapter{Modèle des électrons dans un potentiel périodique}

L'approximation de l'électron libre, prédécemment décrite, est un guide assez
utile pour comprendre les propriétés des métaux (particulièrement des métaux
    alcalins, dans lesquels les électrons de valence - uniques - sortent des
    couches compactes et fermées des ions métalliques, se comportent comme s'ils
étaient libres). La théorie de Sommerfeld (1928) est une théorie presque autonome
sur le transport électronqiue, en eutilisant cette approximation, et donne
presque tout le temps des résultast bon qualitativement. Mais avec notre
compréhension de plus en plus fine de la structure des métaux, il devient
indispensable de considérer les effets des termes plus importants, ceux dûs au
réseau ionique.

Comme on l'a remarqué, la fonction d'onde complète du système peput être traitée
comme s'il sagissait de deux facteurs séparables, chacun correspondant à des
coordonnées d'un électron simple, tant que l'on ne considère pas des termes comme
l'interaction coulombienne. Notre problème est donc de résoudre une équation de
Schrödinger de la forme :

\begin{equation}
    \left\{ - \frac{\hbar^2}{2m} \nabla^2 + \mathcal{U}(\mathbf{r})\right\}
    \psi(\mathbf{r}) = \mathcal{E}\psi(\mathbf{r})
\end{equation}

Sujette aux conditions limites habituelles.
\section{cas général, théorème de Bloch, sphère de Fermi, DOS}

\section{potentiel faible}
gap, plans de bragg


\section{liaisons fortes}
isolants

\subsection{généralités}
\subsection{LCAO}
\subsection{méthode cellulaire}
\subsection{méthode des ondes planes}
