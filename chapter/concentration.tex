\chapter{Concentration en porteurs intrinsèques}

Nous souhaitons la concentration en porteurs intrinsèques comme une fonction de
la température, en fonction du bandgap. Nous calculons cela pour un bord de bande
simple parabolique. D'abord, on calcule en terme du potentiel chimique $\mu$, le
nombre d'électrons excités dans la bande de conduction à la température T. Dans
la physique des semiconducteurs. $\mu$ est appeeelé le niveau de Fermi. À la
température d'intérêt, on peut supposer que pour la bande de conduction d'un
semiconducteur, $\epsilon - \mu \gg k_B T$, et par conséquent la distribution de
Fermi-Dirac se réduit à :
\begin{equation}
    f_e \simeq \exp[(\mu - \epsilon)/k_B T]
\end{equation}

C'est  la probabilité que l'orbitale d'un électron de conduction est occupée,
dans une approximation valide lorsque $f_e \ll 1$.

L'énergie d'un électro dans la bande de conduction est :
\begin{equation}
    \epsilon_k * E_c + \hbar^2k^2/2m_e
\end{equation}

où $E_c$ est l'énergie au bord de la bande de conduction. Ici, $m_e$ est la masse
effective d'un électron. Par conéquent, la densité d'état à $\epsilon$ est :
\begin{equation}
    D_e(\epsilon) = \frac{1}{2\pi^2}
    \left( \frac{2m_e}{\hbar^2}\right)^{3/2}(\epsilon - E_c)^{1/2}
\end{equation}
La concentration des électrons dans la bande de conduction est :
\begin{equation}
    n = \int_{E_c}^\infty D_e(\epsilon)f_e(\epsilon) d\epsilon = \frac{1}{2\pi^2}
    \left( \frac{2m_e}{\hbar^2}\right)^{3/2} \exp(\frac{\mu}{k_BT})\times
    \int_{E_c}^\infty (\epsilon - E_c)^{1/2} \exp(-\frac{\epsilon}{k_B
    T})d\epsilon
    \label{integn}
\end{equation}

Soit en intégrant, on obtient :
\begin{equation}
    n = 2 \left(\frac{m_e k_B T}{2 \pi \hbar^2}\right)^{3/2} \exp\left[\frac{(\mu
    - E_c)}{k_BT}\right]
    \label{n}
\end{equation}

Le problème est résolu pour n lorsque $\mu$ est connu. C'est pratique de calculer
la concentration des trous p à l'équilibre. La distribution $f_h$ pour les trous
est reliée à la distribution des électrons $f_e$ par $f_h + f_e = 1$, parce qu'un
trou est l'absence d'un électron. On a alors, dans le cas où $(\mu - \epsilon)
\gg k_BT$ :
\begin{align}
    f_h & = & 1 - \frac{1}{\exp\left[\frac{\epsilon - \mu}{k_BT}\right] + 1} =
    \frac{1}{\exp\left[\frac{\mu-\epsilon}{k_BT}\right] + 1}\\
    & \cong & \exp\left[\frac{\epsilon - \mu}{k_BT}\right]
    \label{fh}
\end{align}

Si les trous près du haut de la bande de valence se comportent comme des
particules avec une masse effective $m_h$, la densité des trous est donnée par :
\begin{equation}
    D_h(\epsilon) = \frac{1}{2\pi^2}\left( \frac{2m_h}{\hbar^2} \right)^{3/2}
    (E_v - \epsilon)^{1/2}
\end{equation}

Où $E_v$ est l'énergie au bord de la bande de valence. Si on fait de la même
façon que pour \ref{integn}, on obtient alors :

\begin{equation}
    p =r \int_{-\infty}^{E_c}D_h(\epsilon) f_h(\epsilon) d\epsilon = 2 \left(
    \frac{m_hk_BT}{2\pi\hbar^2} \right)^{3/2} \exp\left[ \frac{E_c - \mu}{k_BT}
    \right]
\end{equation}

pour la concentration $p$ en trous dans la bande de valence.

On peut multiplier ces expressions ensemble pour $n$ et $p$ et ainsi obtenir la
relation d'équilibre, avec l'énergie du gap $E_g = E_c - E_v$ :
\begin{equation}
    np = 4 \left( \frac{k_BT}{2\pi\hbar^2} \right)^3 (m_c m_h)^{3/2}
    \exp(-E_g/k_BT)
    \label{np}
\end{equation}

Ce résultat utile ne fait pas intervenir le niveau de Fermi $\mu$. À 300K, la
valeur de $np$ est \SI{2.10e19}{\centi\metre\tothe{-6}} pour le silicium,
\SI{2.89e26}{\centi\metre\tothe{-6}} pour le germanium et
\SI{6.55e12}{\centi\metre\tothe{-6}} pour l'arsenure de gallium.

Nous n'avons jamais, dans cette démonstration, supposé que la matériau est
intrisèque : le résultat tient pour l'ionisation d'impuretés aussi bien. La seul
hypothèse faite, est que la distance du niveau de Fermi depuis les bords des
bandes est grande devant $k_BT$.

Un simple argument cinétique montre en quoi le produit $np$ est constant à une
température donnée. Supposons que la population d'équilibre d'électrons et de
trous est maintenue par les photons de radiation du corps noir à une température
\begin{equation}
    \frac{dn}{dt} = A(T) - B(T)np = \frac{dp}{dt}
\end{equation}
À l'équilibre, $\frac{dn}{dt} = 0$,$\frac{dp}{dt} = 0$, d'où $np =
\frac{A(T)}{B(T)}$.

Comme le produit des concentrations en électrons et en trous est une constante
indépendante de la concentration en impuretés à une température donnée,
l'introduction d'une petite proportion d'une impureté convenable, pour augmenter
$n$(par exemple), devrait réduire $p$.
Ce résultat est important dans la pratique : on peut réduire la concentration
totale en porteurs $n+p$ dans un cristal impur, parfois énormément, en
controllant l'introduction d'impuretés. Une telle réduction est appelée
\emph{compensation}.

Dans un semiconducteur intrinsèque, le nombre d'électrons est égal au nombre de
trous parce que l'excitation thermique d'un électron laisse derrière un trou dans
la bande de valence. Ainsi, d'après \ref{np}, on peut obtenir la relation
suivante, où $i$ représente $n$ ou $p$ et $E_g = E_c - E_v$ :
\begin{equation}
    n_i = p_i = 2 \left( \frac{k_BT}{2\pi\hbar^2} \right)^{3/2} (m_em_h)^{3/4}
    \exp(-E_g / 2k_BT)
\end{equation}

La concentration en porteurs intrinsèques dépend de façon exponentielle de
$E_g/2k_BT$, où $E_g$ est l'énergie du gap. On peut réunir les équations \ref{n}
et \ref{fh} pour obtenir, pour le niveau de Fermi mesuré du bord supérieur de la
bande de valence :
\begin{equation}
    \exp(2\mu/k_BT) = (m_h/m_e)^{3/2}\exp(E_g/k_BT)
\end{equation}
Ce qui donne le niveau de Fermi :
\begin{equation}
    \mu = \frac{1}{2}E_g + \frac{3}{4} k_BT \ln(m_h/m_e)
\end{equation}

Si $m_h = m_e$, alors $\mu = \frac{1}{2}E_g$, et le niveau de Fermi est au milieu
de la bande interdite.

\section{Mobilité intrisèque}

La mobilité est l'intensité de la vitesse de dérive d'un porteur de charge par
unité de champ électrique :
\begin{equation}
    \mu = \frac{|v|}{E}
\end{equation}
La mobilité est définie comme étant positive à la fois pour les électrons et les
trous, même si leurs vitesses de dérives sont opposées pour un champ électrique
donné. En écrivant $\mu_e$ ou $\mu_h$, les mobilités des électrons ou des trous,
on peut éviter toute confusion avec $\mu$ le potentiel chimique.

La conductivité électrique est la somme des contributions des électrons et des
trous :
\begin{equation}
    \sigma = (ne\mu_e + pe\mu_h)
\end{equation}
où $n$ et $p$ sont les concentrations en électrons et en trous. Dans le chapitre
..., la vitesse de dérive d'une charge q a été établie comme $v = q\tau E/m$ avec
:
\begin{equation}
    \mu_e = e \tau_e / m_e ; \hfill \mu_h = e \tau_h / m_h
\end{equation}
avec $\tau$ le temps de collision.

Les mobilités dépendent de la température, avec une loi de puissance faible. La
dépendance en température de la conductivité dans la région intrinsèque sera
dominée par la dépendance exponentielle $\exp(-E_g/2k_BT)$ de la concentration en
pourteurs.

Le tableau suivant donne les valeurs expérimentales des mobilitésà température
ambiante pour certains semiconducteurs classiques. Pour la plupart des matériaux,
les valeurs données sont limitées par la diffusion de charge par les phonons
thermiques. Les mobilités des trous sont générallement plus petites que les
mobilités électroniques, parce que l'apparition d'une dégénérescence au bord de
la bande de valence au centre de la zone de Brillouin rend possible une diffusion
interbande qui réduit considérablement la mobilité.

%\begin{table}[ht]
%\begin{tabular}{lrrlrr}
%\toprule
%Cristal & électrons & trous & cristal & électrons & trous \\
%\midrule
%Diamant & 0.18 & 0.12 & GaAs & 0.8 & 0.03 \\
%Si & 0.135 & 0.048 & GaSb & 0.5 & 0.1 \\
%Ge & 0.36 & 0.18 & PbS & 0.055 & 0.06 \\
%InSb & 0.08 & 0.045 & PbSe & 0.102 & 0.093 \\
%InAs & 3 & 0.045 & PbTe & 0.25 & 0.1 \\
%InP & 0.45 & 0.01 & AgCl & 0.005 & -- \\
%AlAs & 0.028 & -- & KBr(100K) & 0.01 & -- \\
%AlSb & 0.09 & 0.04 & SiC & 0.01 & 0.001-0.002\\
%\bottomrule
%\end{tabular}
%\label{}
%\caption{Mobilités des porteurs de charge à température ambiante ($m^2/V-s$)}
%\end{table}

Dans certains cristaux, particulièrement les cristaux ioniques, les trous sont
essentielleent immobiles et ne se déplacent que par \emph{hopping} (saut) d'un
ion à un autre, activé thermiquement. La principale cause de cet
\emph{auto-piégeage} est la distortion du réseau, associée à un effet Jahn-Teller
des états dégénérés. La dégénérescence des orbitales nécessaire pour un
\emph{auto-piégeage} est plus fréquente pour les trous que pour les électrons.

Il y a une tendance remarquable pour les critsaux avec de petits gaps et des
bords directs : ils ont de grandes valeurs de mobilité électronqiue. Les petits
gaps donnent une masse effective petite, qui favorise les hautes mobilités. La
plus grande mobilité observée dans le bulk d'un semiconducteur est
\SI{5e2}{\square\metre\per\volt\per\second} dans le PbTe à 4K, pour lequel le gap
est de \SI{0.19}{\electronvolt}.
