\chapter{Création de défauts par la présence d'ions étrangers}

\section{Solution solide : phase cristalline à composition variable}


\section{Substitution d'un ion par un autre de charge différente. Solutions solides complexes}

cf aschcroft p 623 : centres colorés

\section{Centres colorés}

On a indiqué que la neutralité de charge impose des lacunes d'un des constituants d'un cristal ionique diatomique, pour être contrebalancée, soit par un nombre égal d'interstices du même onstituent (Frenkel), soit par un nombre égal de lacunes de l'autre composé (Schottky). Il est également possible, malgré cela, de contrebalancer la charge manquante d'une lacune ionique négative avec un électron localisé au voisinage d'un défaut ponctuel, dont la charge est replacée.

Un tel électron doit être vu comme lié à un un centre chargé positivement, et aura, en général, un spectre de niveaux énergétiques. Les excitations entre ces niveaux produiront une série de bandes d'absorption plutôt analogues à celles d'un atome isolé. Ces énergies d'excaitation apparaissent dans la bande optique interdite entre $\hbar\omega_T$ et $\hbar\omega_L$, pour un cristal parfait (définir, cf aschcroft chap 27), et par conséquent se démarqueront avec des pics intenses dans le spectre d'absportion optique. Ces défauts, avec d'autres structures électroniques électron-induits, sont connus comme des centres colorés, parce que leur présence induit une couleur intense sur les cristaux parfaits qui seraient sinon transparents.

Les centres colorés ont été étudiés de façon extensive dans les halogénures alcalins, qui peuvent être colorés par expostion à une radiation de rayons X ou $\gamma$ (avec la production induite de défauts par les photons à très haute énergie), ou, plus instructivement, en chauffant les cristaux d'halogénures alcalins dans une vapeur de métal alculin. Dans ce cas, les atomes d'alcalins en excès (ceux dont le nombre est compris entre un pour \SI{e7}{} et 1 pour \SI{e3}{} ), sont incorporés dans le cristal, comme l'analyse chimique peut le démontrer. Cependant, la densité massique des cristaux colorés diminue en proportion avec la concentration en atome alcalins en excès, ce qui montre que les atomes ne sont pas absorbés aux interstices.  À la place, les atomes de métaux alcalins sont ionisés et tiennent leur emplacement dans les sites d'un sous-réseau parfait chargé positivement, et les électrons en excès sont liés à un nombre égal de lacunes ioniques négatives.

Une évidence frappente de la validité de cette image est donnée par le fait que le spectre d'absorption produit ainsi n'est pas vraiment changé si, par exemple ,no chauffe du chlorure de potassium sous une vapeur de sodium, plutôt que du potassium métal. Cela confirme de fait que le rôle primaire de la vapeur de métaux alcalins est là pour introduire des lacunes ioniques négatives et pour fournir l'électron supplémentaire, dont leniveau d'énergie produit le spectre d'absorption.

Un électron lié à une lacune ionique négative (connue sous le nom de centre $F$\footnote{de l'allemand \emph{farbzentrum}}), est capable de reproduire plusieurs fonctions qualitatives du spectre atomique ordinaire, avec la complication ajoutée qu'il bouge dans un champ de symétrie cubique, plutôt que sphérique\footnote{cela va te permettre de réviser la théorie des groupes}. En fait, en contraignant le cristal, on peut réduire la symétrie cubique, ce qui produira des perturbations induites, qui seront utiles pour démêler une série de structures aditionnelles dans le spectre d'absorptio. La structure additionnelele est présente parce que le simple centre $F$ n'est pas le seul moyen qu eles électrons et les lacunes peuvent utiliser pour colorer le cristal. Deux autres possibilités sont :
\begin{enumerate}
    \item un centre $M$, dans lequel deux lacunes ioniques négatives voisines dans un plan $(100)$ lient deux électrons ;
    \item un centre $R$, dans lequel trois lacunes ioniques négatives voisines dans un plan $(111)$ lient trois électrons.
\end{enumerate}

Cela demande becauop d'ingéniosité de démentrer que ces catégories variées de défauts sont en fait responsables du spectre observé. L'identification est rendue possible en ramarquant que chaque a une réponse caractéristiuque aux effets de la contrainte ou de hamp électrique à sa structure de niveau.

La résonnance dans le spectre d'absorption optique, produite par les centres colorés, n'est pas aussi nette que celle produite par l'excitation d'atomes isolés. Cela est en fait à cause du fait que l'épaisseur de ligne est inversement proportionnelle au temps de vie de l'état excité. Des atomes isolés peuvent uniquement décroiter radiativement, ce qui est un processus relativement lent, mais les atomes représentés par un centre $F$ sont couplés de façon importante, comme le reste des solides, et peuvent donc perdre leur énergie en émettant des photons.

On peut penser qu'en chauffant les cristaux d'halogénures alcalins dans un gaz halogène, on peut aussi introduire des lacunes métaliques alcalines auxquelles des trous peuvent être liées, mais cela sont des \emph{antimorphes} des centres $F$ et ne sont pas observés. Les trous peuvent être liées à des imperfections ponctuelles, mais les imperfections n'ont pas été observées être des lacunes positives ioniques. En fait, le centre de trous le plus étudié, le centre $V_K$, n'est pas basé sur une lacune du tout, mais sur la possibilité pour un trou de se lier à deux ions négatifs voisins (par exemple \ch{Cl-}). dans quelque chose qui a un spectre plus comme \ch{Cl2-}.

En ayant commencé avec le diagnostique de la construction d'un centre coloré, on peut continuer assez loin. Par exemple, on peut regarder, ou fabriquer, un centre $F$ simple, dans luqeul in des six plus proches voisins ions positifs a été remplacé par une impureté. On se retrouve alors avec un centre $F_A$, qui a une symétrie réduite.

Finalement, continuer dans la recherche d'opposés que nous pouvons inculre, avec soit l'antimorphe du cnetre $V_K$ a été obsérvée : un électron localisé, servant à lier deux ions voisins chargés positivement, ensemble. Comme (par exemple) les molécules de \ch{Cl2} existent (liées de façon covalente) et les molécules de \ch{Na2} en général non, la réponse est non. En fait l'asymétrie entre les électrons et les trous est précisemment due à la différence dans les électrons de valence du sodium (niveau $s$) et du chlore (niveau $p$), qui forment des liaisons covalentes seulement dans le second cas. Cependant, quelque chose de plus localisé que l'antimorphe $V_K$ existe, et c'est connu en tant que \emph{polaron}.

\section{Polarons et excitons}

Lorsqu'un électron est introduit dans la bande de conduction d'un cristal ionique parfait, il peut être énegrétiquement favorable parce qu'il peut bouger sur un niveau spatialement localisé, accompagnié d'une déformation locale dans l'arrangement ionique initialement parfait (\ie une polarisation du réseau), qui sert à écranter son champ et à réduire son énergie électrostatique. Une telle entité (un électron et une polarisation induite du réseau), se révèle être plus mobile que les défauts décrits précédemment, et est généralement pas vue commme un défaut du tout, mais plutôt comme une complication dans la théoie de la mobilité électronique dans les cristaux ioniques ou partiellement ioniques. Les théories des polarons sont un peu compliquées, parce qu'il faut considérer la dynamique d'un électron qui est très reliée au degrée de liberté ionique.

Les formes de défauts ponctuels la plus évidente consiste en un ion vancant (lacune), un ion en excès (interstices), ou alors un mauvais type d'ion (impuretés de substitution). Une façon plus subtile est le cas d'un ion dans un cristal parfait, qui diffère des ses collègues en étant dans un état excité. Un tel \emph{défaut} est appelé un \emph{exciton de Frenkel}.

Comme n'importe quel ion est capable d'être excité, et qu'à partir du moment ou le couplage entre les ions du niveau électronique est important, l'énergie d'excitation peut en fait être transférée d'ion à ion. Par conséquent, l'exciton de Frenkel peut se mouvoir dans le cristal entier, sans que les ions eux-mêmes aient à changer de place, ce qui résulte du fait que (comme le polaron), bien plus mobile que les lacunes, interstices ou défauts de substitution. De fait, pour la plupart des études, il est mieux de ne pas penser qu'un exciton puisse être localisé. Il est plus précis de décrire la structure électronique d'un cristal contenant un exciton, omme une superposition quantique d'états, dans lequel il est équiprobable que l'excitation soit associée avec n'importe quel ion du cristal. Cette vison porte la même relation que les ions spécifiquement excités, comme les niveaux des liaison forte de la théorie de Bloch portent les niveaux atomiques individuels, dans la théorie des bandes.

Par conséquent, l'exciton est probablement mieux vu comme une manifestation plus complexe de la structure de bande électronique, que comme un défaut cristallin. De fait, une fois que l'on reconnait la description correcte qu'un exciton est réellement un problème de structure de bande, on peut avoir un point de vue différent sur ce même phénomène :

Supposons que nous avons calculé le niveau électronique fondamental d'un isolant dans l'approximation des électrons indépendants. Le niveau excité le plus bas de l'isolant sera évidemment donné en enlevant un électron du niveau le plus haut, dans la bande haute occupée. (la bande de valence) et en le placant sur un niveau plus bas de la bande de conduction. Un tel réarrangement de la distribution des électrons n'altère par le potentiel périodique auto-cohérent, dans lequel ils bougent. Cela est du au fait que les électrons de Bloch ne sont pas localisés (comme $\abs{\psi_{nk}(r)}^2$ est périodique), et par conséquent le changement de ladensité de charge localeproduit en changennt la charge d'un électron sera de l'ordre de $1/N$ (comme seul $1/N$ de la charge électronique sera celle d'une maille donnée), \ie négligeable. Par conséquent, le niveau d'énergie électronique n'a pas à être recalculé pour la configuration excitée et pour le premier état excité, qui sera à l'énergie $\epsilon_c - \epsilon_v$ au dessus de l'énergie de l'état fondamental, où $\epsilon_c$ est l'énergie du minimum de la bande de conduction et $\epsilon_v$ est l'énergie du maximum de la bande de valence.

Cependant, il y a un autre moyen de produire un état excité. Supposons que l'on forme un niveau à un électron en superposant sffisemment de niveaux près du minimum de la bande de conduction, pour former un paquet d'ondes bien localisés. Parce que nous avons besoins de niveaux au voisinage du minimum pour produire le paquet d'ondes, l'énergie $\bar{\epsilon}_c$ du paquet d'onde sera légèrement plus grande que $\epsilon_c$. Supposons, en plus, que le niveau de la bande de valence que nous dépeuplons est aussi un paquet d'onde, formé de niveaux au voisinage du maximum de la bande de valence (de sorte que son énergie $\bar{\epsilon}_v$ est très légèrement plus petit que $\epsilon_v$), et choisi de sorte que le centre du paquet d'onde est spatiallement très proche du centre du paquet d'onde de la bande de conduction. Si on ignore les interactions électrons-électrons, alors l'énergie nécessaire pour pouger un électron de la bande de valence à la bande de conduction (paquets d'onde), sera $\bar{\epsilon}_c - \bar{\epsilon}_v > \epsilon_c - \epsilon_v$, mais parce que les niveaux sont localisés, ils seront, en plus, dans des quantités non négligeables d'énergie coulombienne, due à l'attraction électrostatique de l'électron (localisé) de la bande de conduction et du trou (localisé aussi) de la bande de valence.

Cette énergie négative électrostatique additionnelle peut être réduite en une énergie totale d'excitation à un montant plus petit que $\epsilon_c - \epsilon_v$, de sorte à ce que le type compliqué d'état excités, dans lequel l'électron de la bande de conduction est spatiallement corrélé avec le trou de la bnade de valence qu'il a laissé, il est le vrai plus petit état excité du cristal. La preuve de cela est que :
\begin{enumerate}
    \item le début de la bande d'absorption optique à des énergie s sous le continuum interbandes et
    \item l'argument théorique qui va suivre, qui indique que l'on fait toujours mieux en utilisant l'attraction électron-trou.
\end{enumerate}

Considérons le cas dans lequel les niveaux des électrons et trous localisés s'étendent sur de nombreuses constantes de réseau. On peut alors faire le même type d'argument semiclassique que celui que l'on utilise pour former les niveaux d'impuretés dans les semi-conducteurs (chap 28 aschcroft). On voit l'électron et le trou comme des particules de masses $m_c$ et $m_v$ (les masses effectives des bnades de conduction et de valence), que  nous prenons, pour simplifier, isotropiques. Elles interagissent à travers une interaction coulombienne attractive, écrantée par la constante diélectrique $\epsilon$ du cristal. Évidemment, c'est juste un problème d'atome d'hydrogène, avec la masse réduite de l'atome d'hydrogène $\mu$ remplacée par la masse réduite effective $m*$ (où $\frac{1}{m*} = \frac{1}{m_c} + \frac{1}{m_v}$), et la charge électronique remplacée par $\frac{e^2}{\epsilon}$. Par conséquent, il y a deux états liés, le plus bas en énergie s'étent au delà du rayon de Bohr donné par :

\begin{equation}
    a_{ex} = \frac{\hbar^2}{m*(e^2/\epsilon)} = \epsilon \frac{m}{m*} a_0
\end{equation}

Cette énergie de l'état lié sera plus faible que l'énergie $(\epsilon_c - \epsilon_v)$ du couple électron-troun qui n'intéragit pas, par

\begin{eqnarray}
    E_ex & = & \frac{e^2/\epsilon}{1a*_0} = \frac{m*}{m} \frac{1}{\epsilon^2} \frac{e^2}{2a0}\\
    & = & \frac{m*}{m} \frac{1}{\epsilon^2} (\SI{13.6}{\electronvolt})
\end{eqnarray}

La validité de ce modèle requiert que $a_{ex}$ soit grand à l'échelle du réseau cristallin (soit $a_{ex} >> a_0$), mais comme les isolants avec des gaps d'énergie tendent à avoir des masses effectives petites et de grandes constantes diélectriques, alors il n'estp as difficile de résoudre, particulièrement dans les semi-conducteurs. De tels spectres de l'hydrogène ont en fait été observés dans l'absorption otpqiue qui se produit plus bas que le seuil interbande.

L'exctiton décrit dans ce modèle est connu comme l'\emph{exciton de Mott-Wannier}. Évidemment, comme les niveaux atomiques parmi lesquels les niveaux de bande sont formés deviennent faiblement liés, $\epsilon$ diminuera, $m*$ augmentera, $a*_0$ diminuera, l'exciton deviendra plus localisé, et l'image de Mott-Wannier s'effondrera. L'exciton de Mott-Wannier et l'exciton de Frenkel sont les extrèmes opposés du même phénomène. Dans le cas de Frenkel, basé sur un niveau ionique excité simple, l'électron est le trou sont finement localisés à l'échelle atomique. Le spectre de l'exciton des solides de gas rares tombe dans ce cas.
