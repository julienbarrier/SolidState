\chapter{Classification des réseaux cristallins}
\label{ch:classification}

Mon but ici n'est pas de rentrer en détails dans les détails de la classification
mais d'en donner un apperçu pratique et utile. On trouvera plus de détails à ce
sujet dans les ouvrages suivants :
\begin{description}
    \item[Diffraction from Materials] L.H. Schwartz et J.B: Cohen,
        Springer-Verlag.
    \item[Essentials of Crystallography] D. et C. Mc Kie, Blackwell Scientific
        Publications, 1986.
    \item[The Basics of Crystallography and Diffraction] C. Hammond,
        International Union of Crystallography Text on Crystallography, Oxford
        University Press, 1997.
\end{description}

Le polycopié de N. Lequeux de \pc est largement plus complet que ce document de ce
point de vue. En particulier, j'ai choisi de ne pas détailler les aspects
mathématiques liées aux symétrie et de me concentrer sur la géométrie. Je ne fais ni
mention des 32 groupes ponctuels, ni de la notation de Schonflies ou des 230 groupes
d'espace.

\section{Opérations de symétrie}

La périodicité d'un réseau est due aux translations de réseau. Celles-ci sont
définies par le postulat de Bravais, complété par celui de Schönflies-Fedorov.

L'ensemble des points d'un réseau, appelés \emph{nœuds du réseau}, constitue un
réseau spatial périodique. On construit celui-ci en appliquant à chaque nœud
l'ensemble des translations :
\begin{equation}
\mathbf{t}_n = u \mathbf{t}_1 + v \mathbf{t}_2 + w \mathbf{t}_3
\end{equation}
où $u$, $v$ et $w$ sont des entiers et $\mathbf{t}_1$, $\mathbf{t}_2$, $\mathbf{t}_3$
forment une base primitive.

La périodicité du réseau est une contrainte forte qui limite le nombre et la nature des opérations de symétrie assurant l'invariance du réseau.

Avant de comprendre comment les opérateurs de symétrie sont inclues dans le
réseau, il est nécessaire de compredre comment les éléments de symétrie
agissent sur leur environnement. Cette partie considèrera des objets quelconques (que j'appelle corps) avant d'étudier les réseaux plus en détail.

Notons en premier lieu que pour un objet à une
position donnée d'un élément de symétrie, le type d'élément de symétrie
impose l'emplacement et l'orientation d'un objet identique.
De la même façon, un corps est dit symétrique lorsque ses composants sont
arrangés de sorte à ce que certaines opérations de symétries peuvent être
effectuées en son sein, pour le reconstruire en superposant ses éléments.
Par exemple, si un corps est symétrique par rapport à un plan qui le traverse,
alors la réflexion de chaque moitié de ce corps par le miroir plan
produira un corps qui coincidera avec l'autre moitié. On en déduit qu'un cube a
plusieurs plans de symétrie, dont l'un d'entre eux est représenté sur la figure
\ref{cullity8} (a). Les points $A_1$ et $A_2$ sur cette figure doivent être
identiques à cause de ce miroir plan coupant $A_1A_2$ perpendiculairement.
$A_1$ et $A_2$ sont reliés par une réflexion.

\begin{figure}
    \includegraphics{./images/part1/cullity46.eps}
    \caption{Certains éléments de symétrie sur un cube. (a) réflexion plane :
    $A_1$ devient $A_2$. (b): rotation d'ordre 4: $A_1$ devient $A_2$ ; rotation
    d'ordre 3 : $A_1$ devient $A_3$ ; rotation d'ordre 2 : $A_1$ devient $A_4$.
    (c) centre d'inversion. (d) rotation d'ordre 4 suvie d'une inversion : $A_1$
    devient $A'_1$ par la rotation d'ordre 4 puis $A_2$ par
l'inversion.}
    \label{cullity8}
\end{figure}

Il y a quatre opérations de symétrie macroscopiques\sidenote{on peut également définir des opérations microscopiques, dont font partie par exemple les miroirs de glissement. Ces opérations n'ont pas d'influence dans les cristaux à l'échelle macroscopique} : réflexion (miroir), rotation,
inversion et rotation-inversion. Un corps a une symmétrie de rotation d'ordre $n$ selon
un axe si une rotation de $\SI{360}{\degree}/n$ coincide avec le cristal. Par conséquenct, un
cube a un axe de rotation d'ordre 4 (quaternaire) normal à chaque face, un axe de rotation
d'ordre 3 (ternaire) le long de chaque grande diagonale, et un axe d'ordre 2 (binaire) liant le centre
de chaque côté opposé. Certains de ceux-ci sont représentés sur la figure
\ref{cullity8}. En général, les axes de rotation peuvent être d'ordre 1, 2, 3, 4
donc pas représenté. En revanche, les axes de rotation d'orde 5 ou d'ordre
supérieur à 6 sont impossibles, parce qu'une maille primitive qui possèderait de une
telles symétries ne pourrait pas paver tout l'espace sans laisser de lacunes.

Un corps possède un centre d'inversion s'il possède des points qui sont symétriques
par rapport à un centre unique. C'est à dire qu'une ligne passant par ce centre d'inversion
peut relier les points à distance égale du centre.
Un corps possédant un centre d'inversion se superposera parfaitement avec lui
même en chaque point du corps s'il est inversé, ou réfléchi par le centre
d'inversion. Par exemple, un cube possède un centre d'inversion à l'intersection de ses grandes
diagonales.
Finalement, un corps peut avoir un axe de rotation-ineversion, d'ordre 1, 2, 3, 4
ou 6. S'il a un axe de rotation-inversion d'ordre $n$, alors il peut être ramené à
lui-même par une rotation de $\SI{360}{\degree}/n$ par l'axe, suivie d'une inversion par le
centre, qui est lui-même situé sur l'axe.

Considérons a présent toutes les positions et orientations qu'un objet ou un
motif peut prendre suite aux opérations de symétrie de différent types (figure
\ref{cullity9}). Le motif doit apparaître encore plus fréquemment si, par
exemple, deux opérations de symétrie passent par le même point. L'opération
combinée d'un axe d'ordre 2 situé sur un miroir plan produit un second miroir
plan, perpendiculaire au premier, et contenant également un axe d'ordre 2.
Lorsqu'un axe d'ordre 4 est situé sur un miroir plan, la symétrie requiert qu'un
total de 8 motifs identiques (dans des orientations diverses) et 4 miroirs plans
soient présents.

\begin{figure}
    \includegraphics{./images/part1/cullity48.eps}
    \caption{Opérations de symétrie et symboles associées pour des rotations
        d'ordre 1 (a), d'ordre 2 (b), d'ordre 3 (c), d'ordre 4 (d), d'ordre 6
        (e). (f) représente un miroir plan, (g) un miroir plan et un axe d'ordre
    2 et (h) un miroir plan et un axe d'ordre 4.}
    \label{cullity9}
\end{figure}

Les différentes opérations de symétrie agissant sur un point forment ce que l'on appelle un
\emph{groupe ponctuel}. À deux dimensions, il y a dix groupe ponctuels qui
peuvent être inclus dans des réseaux. À trois dimensions, le nombre de groupes
ponctuels est de trente-deux : contrairement aux réseaux bi-dimensionnels, les
centres d'inversions ne sont plus équivalents à un axe de rotation d'ordre 2, et
les combinaisons comme celles des miroirs perpendiculaires à des axes de rotation
sont possibles.
Il est important d'insister sur le fait que les éléments de symétries agissent
sur l'ensemble de l'espace. La discusison jusqu'à présent s'est concentrée sur
l'espace réel, mais tous les principes ici présents s'appliquent également dans
l'espace réciproque.

\section{Systèmes cristallins}

Un réseau (3D) peut être défini par trois vecteurs non coplanaires. On peut alors
définir des mailles primitives de diverses formes, selon la longueur et l'orientation
des vecteurs générateurs du réseau.
Par exemple, si les vecteurs $\mathbf{a}$,$\mathbf{b}$,$\mathbf{c}$ sont de
longueur égale et à des angles droits les uns des autres, \ie $\mathbf{a = b = c}$
et $\alpha = \beta = \gamma = \SI{90}{\degree}$, alors la maille primitive est
cubique. En imposant des valeurs différentes pour les longueurs axiales et les angles,
on pourra définir des mailles de différentes formes, et par conséquent de différent
groupe ponctuel, car les points du réseau sont situés aux sommets de la
maille primitive. Il vient alors qu'il n'y a que sept sortes de mailles, et elles
sont nécessairement inclues dans tous les réseaux ponctuels possibles.
Celles-ci correspondent aux sept \emph{systèmes cristallins} parmi lesquels
les cristaux peuvent être classifiés.
Ces systèmes sont listés sur le tableau \ref{tab:syscrist}. \footnote{Le système
trigonal est parfois appelé rhomboédrique.}

\begin{table*}[t]
    %\resizebox{\textwidth}{!}{
    \begin{tabularx}{\textwidth}{lClr}
    \toprule
    Système & axes et angles & réseaux de Bravais & Symbole\\
    \midrule
    \multirow{3}{*}{Cubique} & \multirow{3}{*}{ $a=b=c$, $\alpha = \beta = \gamma
    = \SI{90}{\degree}$} & Simple & P \\
    & & Centré & I \\
    & & Faces-centrées & F \\
    & & & \\
    \multirow{2}{*}{Tétragonal} & \multirow{2}{*}{$a=b\neq c$, $\alpha = \beta =
    \gamma = \SI{90}{\degree}$} & Simple & P \\
    & & Centré & I \\
    & & & \\
    \multirow{4}{*}{Orthorhombique} & \multirow{4}{*}{$a \neq b \neq c$, $\alpha
    = \beta = \gamma = \SI{90}{\degree}$} & Simple & P \\
    & & Centré & I \\
    & & Base-centré & C \\
    & & Faces-centrées & F \\
    & & & \\
    Trigonal & $a = b = c$,
    $\alpha = \beta = \gamma \neq \SI{90}{\degree}$ & Simple & R \\
    & & & \\
    Hexagonal & $a = b \neq c$, $\alpha = \beta = \SI{90}{\degree}$, $\gamma =
    \SI{120}{\degree}$ & Simple & P \\
    & & & \\
    \multirow{2}{*}{Monoclinique} & \multirow{2}{*}{$a \neq b \neq c$, $\alpha =
    \gamma = \SI{90}{\degree} \neq \beta$} & Simple & P\\
    & & Base-centré & C\\
    & & & \\
    Triclinique & $a \neq b \neq c$, $\alpha \neq \beta \neq \gamma \neq
    \SI{90}{\degree}$ & Simple & P\\
    \bottomrule
\end{tabularx}%}
\caption{Systèmes cristallins et résaux de Bravais}
\label{tab:syscrist}
\end{table*}

Sept réseaux ponctuels différents peuvent être obtenus en plaçant les
points aux côtés des mailles primitives des sept systèmes cristallins. Cependant,
il y a d'autres arrangements de points qui peuvent respecter les conditions d'un
réseau ponctuel, à savoir que chaque point du réseau a un environnement
identique. Le cristallographe français Bravais a travaillé sur ce problème et a
démontré en 1848 qu'il y a 14 réseaux ponctuels possibles, et pas plus. Ce
résultat est très important ; en hommage, le terme de \emph{réseau de Bravais}
est devenu synonyme de \emph{réseau ponctuel}. Par exemple, si un point est placé
au centre de chaque maille d'un réseau ponctuel cubique, le nouvel arrangement de
points forme également un réseau de Bravais. De façon similaier, un autre réseau
ponctuel peut être basé sur une maille cubique n'ayant des nœuds du réseau qu'à
chaque sommet, et au centre de chaque face.

Les 14 réseaux de Bravais sont décrits dans la table \ref{tab:syscrist}.
Certaines mailles sont simples (ou primitives) (symbole P ou R), et certaines
sont non-primitives (les autres symboles). Les mailles primitives n'ont qu'un
nœud du réseau par maille, alors que les non primitives en ont plus qu'une. Un
nœud du réseau à l'intérieur d'une maille appartient à cette maille, alors qu'un
point sur une face ou sur un sommet ne sera pas à compter plusieurs fois.

\begin{figure}
    \includegraphics{./images/part1/cullity50.eps}
    \caption{les 14 réseaux de Bravais}
    \label{fig:bravaisschema}
\end{figure}

Chaque maille contenant des points du réseau sur ses sommets est primitive, alors
qu'une maille contenant des points en son centre ou sur ses faces est
non-primitive. Les symboles $F$ et $I$ se réfèrent respectivement aux mailles à
faces centrées et centrées, alors que $A$, $B$ et $C$ se réfèrent aux mailles
base-centrée, avec un atome au centre de deux faces $A$, $B$ ou $C$
opposées\footnote{la face $A$ est définie par les axes $b$ et $c$, la face $B$
par les axes $a$ et $c$ et la face $C$ par les axes $a$ et $b$}. Le symbole $R$
est utilisé principalement pour le système trigonal (ou rhomboédrique). Sur la
figure \ref{fig:bravaisschema}, les axes de longueur égale dans un système
particulier ont le même symbole, par exemple les axes du système cubique sont
tous marqués $a$, dans le système tétragonal (dans lequel $a = b \neq c$), deux
axes sont marqués $a$ et un $c$.

À première vue, la liste des réseaux de Bravais dans le tableau
\ref{tab:syscrist} est incomplète : pourquoi, par exemple, on n'a pas de réseau
tétragonal à base centrée ? En fait, si l'on trace un réseau tétragonal C de
paramètre de maille $a$, on se rend compte que celui-ci peut se réduire à un
réseau tétragonal P de paramètre de maille $\frac{a}{\sqrt{2}}$.

Les ponits du réseau d'une maille non primitive peuvent être étendus dans tous
l'espace par des translations des vecteurs unitaires $\mathbf{a}$, $\mathbf{b}$
et $\mathbf{c}$. Les points du réseau associés à ces mailles unitaires peuvent
être translatés un a un comme un groupe. Dans chacun des cas, les points
équivalents du réseau dans les mailles unitaires sont séparés par un des vecteurs
primitifs, peu importe où ces points sont localisés dans la maille.

À présent, les systèmes cristallins sont définis par la possession d'un certain
nombre d'éléments de symétrie. Chaque système se différencie d'un autre à partir
du nombre d'opérations de symétrie dont il dispose et par les valeurs des
longueurs axiales et des angles. En fait, ceux-ci sont interdépendants. Par
exemple, l'existence d'un axe de rotation d'ordre 4, normal aux faces d'une
maille cubique requiert que les bords de la cellule soient de même longueur et à
\SI{90}{\degree} les uns des autres. D'un autre côté, une maille tétragonale n'a
qu'un axe de rotation d'ordre 4, et cette symmétrie requiert qu'il n'y a que deux
bords de maille qui doivent être égau, \ie les deux qui sont normaux à l'axe.

Le nombre minimal d'opération de symétrie que possède chaque système cristallin
est listé dans le tableau \ref{tab:minsym}. Certains cristaux peuvent posséder
plus de symmétrie que ce nombre minimal requis par le système cristallin auquel
ils appartiennent, mais aucun n'en a moins. L'existence d'une certaine opération
de symétrie implique généralement l'existence d'autres. Par exemple, un cristal
qui possèdent trois axes de rotation d'ordre 4 a nécessairement, quatre axes de
rotation d'ordre 3 et appartient aux système cubique. La réciproque n'est pas
forcément vraie : il y a des systèmes cubques qui n'ont pas forcément trois axes
de rotation d'ordre 4.

\begin{table}
    \begin{tabularx}{\textwidth}{lX}
        \toprule
        Système & Nombre minimal d'éléments de symétrie \\
        \midrule
        Cubique & 4 axes de rotation d'ordre 4 \\
        Tétragonal & Un axe de rotation (ou rotation-inversion) d'ordre 4\\
        Orthorhombique & Trois axes de rotation (ou rotation-inversion) d'ordre
        2, orthogonaux \\
        Trigonal & Un axe de rotation (ou rotation-inversion) d'ordre 3 \\
        Hexagonal & Un axe de rotation (ou rotation-inversion) d'ordre 6 \\
        Monoclinique & Un axe de rotation (ou rotation-inversion) d'ordre 2 \\
        Triclinique & Aucun\\
        \bottomrule
    \end{tabularx}
    \label{tab:minsym}
    \caption{Éléments de symétrie minimums retrouvés dans chacun des systèmes
    cristallins}
\end{table}

\section{Mailles primitives et non-primitives}

Dans chacun des réseau ponctuels, une maille unitaire peut être choisie d'une
infinité de façons différentes et peut contenir un ou plusieurs nœuds du réseau.
Il est important de remarquer qu'une maille primitive n'\emph{existe} pas
forcément dans un réseau : il s'agit d'une construction mentale et est choisie
pour son utilité. Les mailles conventionnelles présentées en figure
\ref{fig:bravaisschema} sont pratiques et conformes avec les éléments de symétrie
du réseau. Dans certains cas, on pourra en choisir d'autres.

Chacun des 14 réseaux de Bravais peut être réduit à une maille primitive. Par
exemple, le réceau cubique faces-centrées, présenté en figure \ref{fig:fcc} peut
être considéré dans le système trigonal (figure \ref{fig:fcctrigonal}). Chaque
maille cubique a 4 nœuds qui y sont associés; une maille trigonale n'en a qu'un :
le réseau cubique faces-centrées (avec une maille cubique) a donc un
volume quatre fois supérieur à la maille primitive (dans le système trigonal).
Cependant, il est souvent plus pratique de considérer une maille cubique plutôt
que trigonale parce que sa forme suggère immédiatement la symétrie cubique que le
réseau possède. De façon similaire, les autres mailles non-primitives listées
dans le tableau \ref{tab:syscrist} sont souvent préférées aux mailles primitives.

\begin{marginfigure}
    \includegraphics{./images/part1/cullity53.eps}
    \caption{Le réseau cubique faces-centrées appartient au système trigonal : la
    maille en pointillés est la maille primitive}
    \label{fig:fcctrigonal}
\end{marginfigure}

Dès lors, pourquoi les réseaux centrais apparraissent dans la liste des 14
réseaux de Bravais ? Si deux mailles peuvent décrire le même ensemble de nœuds du
réseau, alors, pourquoi ne pas éliminer la maille cubique et n'utiliser que la
maille trigonale ? La réponse est que cette maille est une maille particulière du
réseau trigonale, avec un angle $\alpha = \SI{60}{\degree}$. Dans le réseau
trigonal classique, aucune restriction n'est faite sur l'angle $\alpha$ ; le
résultat est un réseau de points avec un axe de symétrie d'orde 3. Lorsque
$\alpha = \SI{60}{\degree}$, alors le réseau a 4 axes de rotation d'ordre 3, et
cette symétrie le place dans le système cubique.

Si des mailles non primitives sont utilisées, le vecteur de l'origine de
n'importe quel nœud du réseau aura des composentes qui seront des multiples non
entiers des  vecteurs du réseau $\mathbf{a}$, $\mathbf{b}$ et $\mathbf{c}$. La
position de n'importe quel point du réseau dans la maille sera donnée en terme de
ses coordonnées ; si le vecteur de l'origine de la maille unitaire à un nœud
donné a des composantes $x\mathbf{a}, y\mathbf{b}, z\mathbf{c}$ où $x,y,z$ sont
des nombres rationels, alors les coordonnées des points sont $x\,y\,z$. Par
conséquent, le point $A$ sur la figure \ref{fig:fcctrigonal}, pris comme
l'origine, a comme coordonnées $0\,0\,0$, alors que les points $B, C$ et $D$,
dans le système cubique, ont des coordonnées respectives $0\,\half\,\half$,
$\half\,0\,\half$ et $\half\,\half\,0$. Le point $E$ a pour coordonnées
$\half\,\half\,1$ et est équivalent au point $D$, sépéré du vecteur $\mathbf{c}$.
Les coordonnées des points équivalents dans différentes mailles peuvent être
rendues identiques par l'addition ou la soustraction par un ensemble de
coordonnées entières : dans ce cas, la soustraction de $\half\,\half\,1$ par
$0\,0\,1$ (la coordonnée de $E$) donne $\half\,\half\,0$ (la coordonnée de D).

Notons que la coordonnée d'un nœud d'un réseau centré ($I$), par exemple, est
toujours $\half\,\half\,\half$, peu importe que la maille unitaire soit cubique,
tétragonale, orthorhombique ou peu importe sa taille. La coordonnée d'une
position ponctuelle, comme $\half\,\half\,\half$, peut également être vue comme
un opérateur qui, lorsqu'il est appliqué à un point à l'origine, le translatera à
la position $\half\,\half\,\half$, la position finale obtenue par simple addition
de l'opérateur $\half\half\half$ et la position originale $0\,0\,0$. Dans ce cas,
le veucteur entre $0\,0\,0$ et toutes les positions du centre dans la maille
cubique centrée, \ie $<\half\half\half>$ sont appelées \emph{translation de
réseau I}, car elles produisent les deux nœuds ponctuels caractéristiques
du réseau en étant appliquées à un point à l'origine. De façon similaire, les
quatre positions ponctuelles caractéristiques du système cubique faces-centrées
($F$), \ie $0\,0\,0$, $0\,\half\,\half$, $\half\,0\,\half$ et $\half\,\half\,0$,
sont reliées par la \emph{translation de réseau F} $<\half\half 0>$. Les
translations de réseau $A$, $B$ ou $C$  dépendent de la paire de faces opposées
sur laquelle elles s'appliquent. Si la maille est centrée sur la face $C$ par
exemple, alors les positions équivalentes sont $0\,0\,0$, $\half\,\half\,0$ et
les translations sont donc $[\half\half 0]$. Ainsi, on peut résumer ainsi :

\begin{itemize}
    \item translation de réseau $I$ : $<\half\half\half>$
    \item translation de réseau $F$ : $<\half \half 0>$
    \item translation de réseau $A$ : $[0 \half \half]$
    \item translation de réseau $B$ : $[\half 0 \half]$
    \item translation de réseau $C$ : $[\half \half 0]$
\end{itemize}

Pour les cristaux qui ne possèdent qu'un atome par nœud (\ie Nb, Ni, Cu, etc.),
on peut généralement écrire les positions comme \emph{$0\,0\,0$ + translation de
réseau $I$} par exemple. Si les mailles primitives ont plus d'un atome par nœud,
comme le silicium par exemple (qui a un réseau de Bravais cubique à
faces-centrées en $0\,0\,0$ et $\frac{1}{4}\,\frac{1}{4}\,\frac{1}{4}$ en plus
des translations de réseau $F$), ce qui fait un total de 8 atomes par maille. Des
cristaux moléculaires plus complexes, comme ceux trouvés dans les systèmes
biologiques, peuvent avoir un plus grand nombre d'atomes de différents types à
chaque nœud du réseau.

Il est important de noter que les incides d'un plan ou d'une direction n'ont
aucun sens si on ne définit pas préalablement l'orientation de la maille. Cela
signifie que les indices d'un plan réticulaire dépendent le la maille choisie.

Dans chaque système cristallin, il y a des ensembles de plans du réseau
équivalents, reliés par des symétries. Ceux-ci sont appelés \emph{famille de
plans}, et les indices de chacun de ces plans sont notés entre accolades
($\{hkl\}$) pour signifier la famille complète. En général, les plans d'une même
famille ont le même espacement mais des indices de Miller différents. Par
exemple, les faces d'un cube ($100$),($010$),($001$), ($\bar{1}00$),
($0\bar{1}0$) et ($00\bar{1}$) sont des plans de la famille $\{100\}$, car chacun
d'entre eux peut être généré par les autres, par l'opération de l'axe de rotation
d'ordre 4 perpendiculaire à la face du cube. Dans le système tétragonal,
cependant, seuls les plans ($100$),($010$), ($\bar{1}00$) et ($0\bar{1}0$) sont
équivalents (appartiennent à la même famille $\{100\}$), car l'axe $c$ a une
longueu différente. Les deux autres plans ($001$) et ($00\bar{1}$) appartiennent
à la famille $\{001\}$. Il est facile de voir que les quatre premiers sont reliés
entre eux par un axe de rotation d'ordre 4, et le troisième par un axe d'ordre 2.

Les plans d'une famille sont \emph{en zone} (ou aussi \emph{tautozonaux}) s'ils
sont tous parallèles à une même rangée, dite \emph{axe de zone}. L'ensemble des
plans est spécifié en donnant les indices de la rangée. De tels plans peuvent
avoir des indices et des espacements différents, la seule contrainte est qu'ils
soient parallèles à l'axe de zone.

\begin{marginfigure}
    \includegraphics{./images/part1/cullity56}
    \caption{Les plans grisésdu réseau cubique sont les plans en zone $\{001\}$}
    \label{fig:tautozonaux}
\end{marginfigure}

Prenons par exemple un axe de zone $[uvw]$. Alors chaque plan $(hkl)$ qui
appartient à cette zone vérifie la relation :

\begin{equation}
    hu + kv + lw = 0
\end{equation}

Chaque couple de plans non parallèles sont des plans de zone car ils sont
parallèles à la droite définie par leur intersection. Soient leurs indices :
$(h_1k_1l_1)$ et $(h_2k_2l_2)$, alors les indices de leur axe de zone $[uvw]$
sont définis par le produit tensoriel $[h_1k_1l_1] \times [h_2k_2l_2]$ :
\begin{eqnarray}
    u & = & k_1 l_2 - k_2 l_1\\
    v & = & l_1 h_2 - l_2 h_1\\
    w & = & h_1 k_2 - h_2 k_1
\end{eqnarray}

\subsection{distance interréticulaire}

La distance interréticulaire $d_{hkl}$ de la famille de plans $\{hkl\}$ dépend du
système cristallin dans lequel on se place. Le système cubique a la forme la plus
simple :


\begin{flalign}
    \text{\emph{cubique}} && d_{hkl} & = \frac{a}{\sqrt{h^2 + k^2 + l^2}} &
\end{flalign}

Dans le système cubique, il est important de se rappeler que $[hkl]$ est
orthogonal à $(hkl)$. Pour tous les autres systèmes cristallins, cela est
généralement faux.

Dans le système tétragonal, l'équation fait intervenir à la fois a et c, qui ne
sont généralement pas égaux :

\begin{flalign}
    \text{\emph{tétragonal}} && d_{hkl} &= \frac{a}{\sqrt{h^2 + k^2 + l^2
    \left(\frac{a^2}{c^2}\right) }} &
\end{flalign}

Cela se complique à mesure que la symétrie diminue :

\begin{flalign}
    \text{\emph{hexagonal}} && \frac{1}{d_{hkl}^2} &= \frac{4}{3} \left( \frac{h^2 + hk + k^2}{a^2} \right) + \frac{l^2}{c^2}&\\
    \text{\emph{trigonal}} && \frac{1}{d_{hkl}^2} &=\frac{(h^2 + k^2 + l^2) \sin^2 \alpha + 2(hk + kl + hl)(\cos^2 \alpha - \cos \alpha)}{a^2 (1 - 3 \cos^2 \alpha + 2 \cos^3 \alpha)}&\\
    \text{\emph{orthorhombique}} && \frac{1}{d_{hkl}^2} &= \frac{h^2}{a^2} + \frac{k^2}{b^2} + \frac{l^2}{c^2}&\\
    \text{\emph{monoclinique}} && \frac{1}{d_{hkl}^2} &= \frac{1}{\sin^2\beta} \left( \frac{h^2}{a^2} + \frac{k^2 \sin^2 \beta}{b^2} + \frac{l^2}{c^2} - \frac{2hl\cos \beta}{ac} \right)&\\
    \text{\emph{triclinique}} && \frac{1}{d_{hkl}^2} &= \frac{1}{V^2} \left( S_{11}h^2 + S_{22}k^2 + S_{33}l^2 + 2S_{12}hk + 2S_{23}kl + 2S_{13}hl \right) &
\end{flalign}

où les éléments de matrice $S$ sont donnés par :

\begin{eqnarray*}
    S_{11} & = & b^2c^2 \sin^2 \alpha\\
    S_{22} & = & a^2c^2 \sin^2 \beta\\
    S_{33} & = & a^2b^2 \sin^2 \gamma\\
    S_{12} & = & abc^2 (\cos\alpha\cos\beta - \cos\gamma)\\
    S_{23} & = & a^2bc (\cos\beta\cos\gamma - \cos\alpha)\\
    S_{13} & = & ab^2c (\cos\gamma\cos\alpha - \cos\beta)\\
\end{eqnarray*}

Les volumes des mailles sont donnés par :
\begin{flalign}
    \text{\emph{cubique}} && V & = a^3&\\
    \text{\emph{tétragonal}}l && V & = a^2 c&\\
    \text{\emph{hexagonal}} && V & =\frac{\sqrt{3}a^2 c}{2} = 0.866a^2c&\\
    \text{\emph{trigonal}} && V & = a^3 \sqrt{1-3\cos^2\alpha + 2\cos^3 \alpha}&\\
    \text{\emph{orthorhombique}} && V & = abc&\\
    \text{\emph{monoclinique}} && V & =abc \sin \beta&\\
    \text{\emph{triclinique}} && V & = abc\sqrt{1- \cos^2 \alpha - \cos^2 \beta - \cos^2 \gamma + 2 \cos\alpha \cos\beta \cos\gamma}&
\end{flalign}
    

%    \section{autres}
%
%
%    \TODO parler des opérations de symétrie microscopiques, au niveau de
%    l'arrangement des atomes, pas seulement visibles sur les cristaux.
%
%
%
%    \subsection{Opérations de symétrie compatibles avec la nature du réseau
%    cristallin}
%
%    La démonstration des propriétés évoquées ici est fournie dans le poly de N.
%    Lequeux. Pour cette raison, cela n'est pas présenté dans ce document.
%
%    \begin{enumerate}
%    \item Seuls les ordres de rotation 1 (identité), 2, 3, 4 et 6 sont permis. Les
%        ordres 5 et >6 sont interdits ;
%    \item tout axe de rotation est parallèle à une translation de réseau ;
%    \item tout axe de rotation est perpendiculaire à un plan réticulaire ;
%    \item une opération de symétrie directe est toujours une opération hélicoïdale.
%    \end{enumerate}
%
%\begin{table}[ht]
%\begin{tabularx}{\textwidth}{lRRR}
%\toprule
%ordre de rotation & translation $\omega_{//}$ & notation & symbole graphique \\
%\midrule
%1 & 0 & 1 & \\
%2 & 0 & $2$ & \cry{2}\\
%& $\frac{1}{2}\mathbf{a}$ & $2_1$ & \cry{21}\\
%3 & 0 & $3$ & \cry{3} \\
%& $\frac{1}{3}\mathbf{a}$ & $3_1$ & \cry{31} \\
%& $\frac{2}{3}\mathbf{a}$ & $3_2$ & \cry{32} \\
%4 & 0 & $4$ & \cry{4} \\
%& $\frac{1}{4}\mathbf{a}$ & $4_1$ & \cry{41} \\
%& $\frac{2}{4}\mathbf{a}$ & $4_2$ & \cry{42} \\
%& $\frac{3}{4}\mathbf{a}$ & $4_3$ & \cry{43} \\
%6 & 0 & $6$ & \cry{6} \\
%& $\frac{1}{6}\mathbf{a}$ & $6_1$ & \cry{61} \\
%& $\frac{2}{6}\mathbf{a}$ & $6_2$ & \cry{62} \\
%& $\frac{3}{6}\mathbf{a}$ & $6_3$ & \cry{63} \\
%& $\frac{4}{6}\mathbf{a}$ & $6_4$ & \cry{64} \\
%& $\frac{5}{6}\mathbf{a}$ & $6_5$ & \cry{65} \\
%\bottomrule
%\end{tabularx}
%\label{}
%\caption{Liste des symétries directes et symboles associés}
%\end{table}

%    \subsection{opérations de symétries inverses}
%    les miroirs purs se notent m
%
%    on note $\mathbf{a}$,$\mathbf{b}$,$\mathbf{c}$ les miroirs de glissement de
%    translation $\frac{\mathbf{a}}{2}$, $\frac{\mathbf{b}}{2}$,
%    $\frac{\mathbf{c}}{2}$.
%
%    les miroirs de translation $\frac{\mathbf{a\pm b}}{2}$, $\frac{\mathbf{a\pm
%    c}}{2}$ et $\frac{\mathbf{b \pm c}}{2}$ se notent $\mathbf{n}$.
%
%    il existe des miroirs de glissement \emph{diamant} notés $\mathbf{d}$ de
%    translations $\frac{\mathbf{a\pm b}}{4}$ et $\frac{\mathbf{a\pm b \pm c}}{4}$
%    autorisés dans certains réseaux de Bravais F et I, qui sont décrits par des
%    mailles non primitives et qui présentent des translations non entières.
%
%    \begin{table}[ht]
%        \begin{tabularx}{\textwidth}{lRR}
%            \toprule
%            ordre de rotation & notation & symbole graphique \\
%            \midrule
%            1 & $\bar{1}$ ou $I$ & \cry{10} \\
%            3 & $\bar{3}$ & \cry{30} \\
%            4 & $\bar{4}$ & \cry{24} \\
%            5 & $\bar{5}$ & \cry{60} \\
%            \bottomrule
%        \end{tabularx}
%        \label{}
%        \caption{Liste des symétries inverses et symboles associés}
%    \end{table}
%
%    Donc :
%    dans un cristal, les opérations de symétrie sont du type $(W,\mathbf{\omega})$ où
%    $W$ est une rotation ou une rotation-inversion d'ordre 1,2,3,4 ou 6.
%
%    $(E,\mathbf{t}_n)$ sont les translatinos de réseau, notées $\mathbf{t}_n$.
%
%    Toute opération directe d'ordre n supérieur à 1 est équivalente à une opération
%    hélicoïdale par rapport à un axe $(W, \omega_{//})$, où $\omega_{//}$ est une
%    translation de (m/n) fois la plus petite translation parallèle à l'axe de
%    rotation. On les note $\mathbf{n}$ pour les opérations pures, et $\mathbf{n}_m$
%    pour les opérations avec glissement.
%
%    Toute opération inverse d'ordre $n\neq 2$ est une opération inverse pure par
%    rapport à un point. On les note $\mathbf{\bar{n}} = \bar{1}, \bar{3}, \bar{4},
%    \bar{6}$
%
%    Les opérations inverses d'ordre $n=2$ sont des miroirs avec glissement $(\bar{2},
%    \omega_{\perp})$ où $\omega_\perp$ est soit nul (opération miroir pur), soit égal
%    à \half fois le plus petit vecteur de translation dans une rangée parallèle au
%    plan miroir. On les note $\mathbf{m}$ pour des miroirs purs et
%    $\mathbf{a}$,$\mathbf{b}$,$\mathbf{c}$,$\mathbf{n}$ et $\mathbf{d}$ pour des
%    miroirs avec glissement.
%
%    Lors du dénombrement des groupes ponctuels cristallographique, on peut alors
%    trouver :
%    \begin{itemize}
%    \item 11 groupes propres ;
%    \item 11 groupes impropres contenant l'inversion ;
%    \item 10 groupes impropres ne contenant pas l'inversion.
%    \end{itemize}
%    Sout au total 32 groupes ponctuels.
%
%
%    \section{32 groupes de symétrie}
%    On pourait redémontrer qu'il existe 32 groupes de symétrie d'orientation
%    cristalline à 3D qui résultent des combinaisons des rotations directes ou
%    inverses ponctuelles W d'ordre 1,2, 3, 4 et 6.
%
%    Chacun des 32 groupes ponctuels forme ce qu'on peut appeler une \emph{classe
%    cristalline}.
%
%
%    \section{Notations de Schonflies}
%
%    \section{7 systèmes cristallins}
%    Les 32 groupes de symétrie d'orientation sont regroupés en 7 systèmes
%    cristallins :
%    \begin{itemize}
%        \item triclinique
%        \item monoclinique
%        \item orthorhombique
%        \item quadratique
%        \item trigonal
%        \item hexagonal
%        \item cubique
%    \end{itemize}
%
%    Le système cubique est associé nécessairement à la présence dans un cristal de 4
%    axes ternaires orientés suivant les 4 diagonales du cube (angle de 
%    \SI{70.53}{\degree}).
%
%    Le système hexagonal est associé à la présence d'un axe sénaire (ordre 6).
%
%    Le système trigonal est associé à la présence d'un seul axe ternaire.
%
%    Le système quadratique est associé à la présence d'un axe quaternaire mais pas
%    d'axes ternaires.
%    \section{14 réseaux de bravais}
%    \section{230 groupes d'espace}
%    notation de Hermann-Mauguin

\section{Récapitulatif des notations utilisées}

Le tableau \ref{tab:rappel} récapitule l'ensemble des notations associées aux concepts que nous avons introduits dans ces trois premières sections.

\begin{table}[ht]
\begin{tabularx}{\textwidth}{Xl}
\toprule
notation & signification \\
\midrule
$\mathbf{R}$ & points du réseau de Bravais \\
$\mathbf{a}_i$ & vecteurs du réseau de Bravais \\
$\mathbf{K}$ & points du résau réciproque (vecteur d'onde) \\
$\mathbf{b}_i$ & vecteurs du réseau réciproque \\
$[u v w]$ & rangée du réseau direct\\
$(hkl)$ & plan du réseau direct \\
$[hkl]^*$ & rangée du réseau réciproque \\
$(uvw)^*$ & plan du réseau réciproque \\
$<hkl>$ & famille de rangées directes \\
$\{hkl\}$ & famille de plans équivalents\\
\bottomrule
\end{tabularx}
\caption[Notations utilisées en cristallographie]{Rappel des notations utilisées en cristallographie}
\label{tab:rappel}
\end{table}


\section{Exemples de structures ioniques simples}

Considérons des structures assez communes qui possèdent des mailles cubiques. Nous allons essayer de déterminer leurs réseau de Bravais, qui est sûrement la chose la plus utile pour comprendre comment se diffractent les rayons X sur ces structures.

\subsection{NaCl}
La structure \ch{NaCl} possède 8 ions, situés de la façon suivante :
\begin{itemize}
    \item 4 ions \ch{Na+} en positions $0\,0\,0$, $\half\,\half\,0$,$\half\,0\,\half$ et $0\,\half\,\half$ ;
    \item 4 ions \ch{Cl-} en positions $\half\,\half\,\half$, $0\,0\,\half$, $0\,\half\,0$ et $\half\,0\,0$.
\end{itemize}

\begin{marginfigure}
    \includegraphics{./images/part1/cullity63-02}
    \caption{La structure NaCl (qui est la même que pour KCl, CaSe, PbTe, etc.)}
    \label{fig:nacl}
\end{marginfigure}

Les ions sodium sont arrangés dans une maille cubique faces-centrées. Les translations de réseau $[000]$ et $<\half\half\half>$, lorsqu'elles sont appliquées à l'atome de chlore en $\half\,\half\,\half$,reproduisent toutes les positions des atomes de chlore. Par conséquent, le réseau de Bravais de \ch{NaCl} est cubique faces-centrées. Les positions ioniques peuvent être écrites comme :

\begin{itemize}
    \item 4\ch{Na+} en positions $0\,0\,0$ + translation de réseau F ;
    \item 4\ch{Cl-} en positions $\half\,\half\,\half$ + translations de réseau F.
\end{itemize}

Remarquons ici que les opérations de symétrie du réseau de Bravais cubique F doivent superposer les ions similaires entre eux : une rotation de \SI{90}{\degree} autour de l'axe $[010]$ amène l'ion \ch{Cl-} de $0\,1\,\half$ en $\half\,1\,1$, ce qui coïncide avec un autre ion \ch{Cl-}, l'ion \ch{Na+} en $0\,1\,1$ vient coincider avec l'ion \ch{Na+} en $1\,1\,1$, etc.

\subsection{CsCl}

\begin{marginfigure}
    \includegraphics{./images/part1/cullity63-01}
    \caption{Structure de CsCl (identique à CsBr, NiAl, CuPd ordonné, etc.)}
    \label{fig:cscl}
\end{marginfigure}

On peut constater sur la figure \ref{fig:cscl} que la maille primitive de \ch{CsCl} contient deux ions : un ion \ch{Cs+} en position $0\,0\,0$ et un ion \ch{Cl-} en position $\half\,\half\,\half$. On peut penser que la maille est cubique centrée, mais la translation de réseau $I$ vient superposer un ion césium avec un ion chlorure. Cette structure est donc cubique simble.

\subsection{Carbone diamant}
%cullity 64
\subsection{Zinc blende}
