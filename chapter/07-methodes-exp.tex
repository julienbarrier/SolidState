\chapter{Méthodes expérimentales}

référence : DB cullity, annexe 1., chap 6, 7 , 8, 9

Dans la section précédente, on a vu que l'intensité diffractée agrémente les nœuds du réseau réciproque. Cette propriété simple est suffisante pour comprendre la plupart des phénomènes de diffraction.
Nous n'avons cependant pas encore développé suffisemment de formalisme pour prédire correctement les intensités observées, mais nous pouvons prédire les positions des pics. Explorons maintenant différentes méthodes de diffraction.

Un cristal simple dans un faisceau de rayons X monochromatique sera rarement orienté de telle sorte à ce que la sphère d'Ewald intersecte les nœuds du réseau réciproque. En conséquence, les maxima de diffraction seront difficilement observable. Différentes méthodes expérimentales sont décrites ici pour produire la condition de diffraction. Chacune a ses avantages et ses inconvénients.

\section{Méthode du cristal mobile}
monocristal mobile, rayons X monochromatiques

\begin{marginfigure}
    \TODO
    \caption{Schéma montrant les axes d'études par la méthode du cristal mobile}
    \label{fig:cristalmobile}
\end{marginfigure}

Dans un diffractomètre à rayons X, un faisceau de rayons X monochromatiques est dirigé sur un échantillon, monté sur un goniomètre qui permet des rotations suivant plusieurs axes (figure \ref{fig:cristalmobile}. Le détecteur est monté sur un bras rotatif du goniomètre. Cet arrangement permet de contrôler la norme et l'orientation du vecteur de diffusion $\mathbf{q}$. L'expérience consiste en l'observation de l'intensité comme fonction de $\mathbf{q}$ (norme, orientation, ou les deux). Certains des chemins possibles dans le réseau réciproque sont présentés sur la figure \ref{fig:cristalmobilerecip}. Les diffractomètres modernes, avec contrôle numérique, permettent de contrôler des chemins très généraux le long de l'espace réciproque.
Un réseau réciproque peut être sélectionné, et l'intensité diffractée peut être observée comme fonction de la position dans le réseau réciproque. C'est souvent très pratique de balayer en $\mathbf{q}$ comme une fonction des variables $h,k,l$ :

\begin{equation}
    \mathbf{q} = 2\pi (h\mathbf{b}_1 + k\mathbf{b}_2 + l\mathbf{b}_3)
\end{equation}

\begin{figure}
    \TODO
    \caption{Schéma du réseau réciproque et des paramètres étudiés par la méthode du cristal mobile}
    \label{fig:cristalmobilerecip}
\end{figure}

Ainsi, par exemple, un balayage en $h$ peut diriger le vecteur de diffusion le long d'un chemin gardant constant $k$ et $l$. Ce type de balayage sera fait dans la direction $\mathbf{b}_1$. En observant l'intensité, la forme et la position de ces pics, l'information sur la structure atomique, la densité en défauts et la contrainte peut être obtenue.

\begin{figure}
    \TODO
    \caption{Tracé de contours logarithmiques de l'intersité dans le plan $(h,k,0)$ de l'espace réciproque pour des 10 mailles cubiques alignés selon la direction $x$, et 8 mailles selon la direction $y$, avec un facteur de diffusion atomique $f$ égal à 1. (a) : réseau cubique simple : intensités pour des valeurs entières de $h$ et $k$. (b) réseau cubique faces-centrées : les valeurs pour $h$ et $k$ impaires ne produisent pas de pic de diffraciton.}
    \label{fig:cristalmobilecubic}
\end{figure}

\section{Méthode de von Laue}
monocristal fixe, rayons X polychromatiques

\begin{marginfigure}
    \TODO
    \caption{Schémas de cônes de diffractions formant des ellipses en transmission et des hyperboles en réflexion}
    \label{fig:laue}
\end{marginfigure}


La méthode de von Laue est effectuée sur un échantillon monocristallin, en utilisant un faisceau de rayons X collimaté, à large spectre. Celui-ci est générallement produit par un tube à rayons X classique. Un film photographique est placé soit avant, soit après l'échantillon, suivant que la géométrie soit en réflexion ou en transmission. Les points de diffraction s'arrengent en ensembles selon des courbes elliptiques ou hyperboliques (figure \ref{fig:laue}). Ces courbes résultent de l'intersection d'un cône avec le film plan. Les plans cristallographiques \emph{en zone} produiront des taches de diffraction qui formeront un cône. L'axe du cône correspond à l'\emph{axe de zone}.

On peut résumer cela ainsi : les plans d'une même zone sont diffractés selon un cône. Cela est facilement visible dans l'espace réciproque. Pour cela, changons d'abord légèrement notre interprétation de l'espace réciproque, pour prendre en compte l'étalement en longueur d'ondes du réseau incident. Le réseau réciproque que nous étudions prend maintenant en compte l'étalement en longueur d'onde, qui est compris entr edeux longueur d'onde $\lambda_1$ et $\lambda_2$, où la plus petite des deux correspond à la longueur d'onde de coupure de la source à rayons X, et la plus grande est moins bien définie mais est généralement prise pour correspondre à la longueur d'absorption limite $K$ de l'argent( \SI{0.48}{\angstrom}).
La condition de diffraction :

\begin{equation}
    \mathbf{q}_B = (\mathbf{k'-k})_B = 2 \pi \mathbf{G}_{hkl}
\end{equation}

peut s'écrire comme :

\begin{equation}
    \label{eq:lauerecip}
    \mathbf{\hat{s}' - \hat{s}} = \lambda \mathbf{G}_{hkl}
\end{equation}

avec
\begin{equation}
    \mathbf{\hat{s}} = \frac{\mathbf{k}}{k} = \frac{\lambda}{2\pi}\mathbf{k}\quad\text{et}\quad
    \mathbf{\hat{s}'} = \frac{\mathbf{k'}}{k} = \frac{\lambda}{2\pi}\mathbf{k'}
\end{equation}
ce sont les vecteurs unitaires correspondants aux directions de $\mathbf{k}$ et $\mathbf{k'}$.

Construisont maintenant l'espace réciproque basé sur l'équation \ref{eq:lauerecip}, o\ la distance entre chaque point de l'espace réciproque dépend de la longueur d'onde $\lambda$. Par conséquent, chaque point de l'espace réciproque est étalé sur une ligne connectant les points $\lambda_1\mathbf{G}_{khl}$ et $\lambda_2\mathbf{G}_{hkl}$. La longueur de cette ligne dans l'espace réciproque augmente avec la distance depuis l'origine. Cela est montré schématiquement sur la figure \ref{fig:lauerecip}. Si maintenant nous traçons une sphère de rayon unitaire, centrée sur le centre du vecteur $\mathbf{\hat{s}}$, qui termine à l'origine du réseau réciproque, la condition de diffraction de l'équation \ref{eq:lauerecip} sera atetinte pour toute ligne du réseau réciproque qui s'intersecte avec cette sphère.

\begin{figure}
    \TODO
    \caption{Schéma de l'espace réciproque construit pour la méthode de von Laue. Les points de l'espace réciproque sont représentés come des lignes pour prendre en compte l'étalement en longueur d'onde du faisceau incidents. La condition de diffraction est satisfaite si la sphère de réflection intersecte ces lignes, comme montré dans plusieurs cas ici. Le vecteur $\mathbf{\hat{s}'}$ et l'angle de diffusion $2\theta$ sont représentés pour les réflexions (310) et (220)}
    \label{fig:lauerecip}
\end{figure}

Tous les plans d'une même zone sont représentés par des lignes sur un plan perpendiculaire à l'axe de zone. Ce plan intersecte la sphère de réflexion en un cercle, et le vecteur $\mathbf{\hat{s}'}$ pour tous les faisceaux incidents qui terminent sur ce cercle. Il lformera un cône correspondant à l'axe de zone.

Dans l'exemple de la figure \ref{fig:lauerecip}, le centre de la pshère de réflexion, qui est l'origine du vecteur $\mathbf{\hat{s}}$, est placé dans le plan de zone, de telle sorte à ce que tous les vecteurs $\mathbf{\hat{s}'}$ d'une même zone (la zone $[001]$) sont coplanaires avec le plan de zone. En outre, le cône est applati en un cercle, et sa trace sur le film sera une ligne. Cependant, si l'origine de la sphère de réflexion n'est pas dans le plan, alors les vecteurs $\mathbf{\hat{s}}$ et $\mathbf{\hat{s}'}$  forment un cône, comme montré sur la figure \ref{fig:lauecone}.

\begin{figure}
    \TODO
    \caption{Schéma des plans en zone formant un cône. Le plan de zone intersecte la sphère de réflexion en un cercle. Le vecteur direction de diffraciton $\mathbf{\hat{s}'}$ est situé sur ce cercle et forme un cône dont l'axe correspond à l'axe de zone.}
    \label{fig:lauecone}
\end{figure}

La direction du vecteur $\mathbf{\hat{s}}$ est dans le cône de diffraction, et l'axe de zone est l'axe du cône. Lorsque l'angle entre l'axe de zone et $\mathbf{\hat{s}}$ est inférieur à \SI{45}{\degree}, alors le cône intersecte le film avec un motif elliptique. Cela n'est posisble qu'en géométrie en transmission. Lorsque cet angle est supérieur à \SI{45}{\degree}, alors le cône intersecte le film avec un motif hyperbolique. Cela peut se produire soit en transmission, soit en réflexion. La symétrie du motif produit par la méthode de von Laue comporte des similitudes avec la symétrie du cristal, et la détermination des zones responsables d'ue certains motifs de diffraction est possible. Cette technique est largement utilisée pour orienter les échantillons en méthode du cristal mobile pour l'étude de certaines surfaces caractéristiques.


\section{Méthode des poudres}
polycristaux, rayons X monochromatiques

La méthode des poudres est la technique d'analyse structurelle la plus utilisée. Un échantillon réduit sous forme d'une fine poudre est exposé à un faisceau collimaté de rayons X monochromatiques. Idéallement, la poudre est faite de cristaux qui ont une orientation aléatoire en fonction de $\mathbf{k}$, la direction du faisceau incident. Cela est équivalent à la rotation d'un monocristal selon toutes les orientations possibles, en même temps. Cela signifie que haque vecteur du réseau réciproque peut prendre toutes les orientations possibles, en balayant une sphère de rayon $G_{hkl} = \frac{1}{d_{hkl}}$, centrée à l'origine du réseau réciproque. L'espace réciproque constiste alors en un ensemble de sphères concentriques de rayons $G_{hkl} = 1/ d_{hkl}$. L'intersection d'un de ces sphères du réseau réciproque avec la sphère d'Ewald forme un cercle. La condition de diffraction est alors satisfaite pour un faisceau diffracté $\mathbf{k'}/2\pi$ dont l'origine est à $\mathbf{k}/2\pi$ et termine à son intersection. Ce vecteur d'onde balaye un cône, comme présenté sur la figure \ref{fig:poudres}.

\begin{marginfigure}
    \TODO
    \caption{Schéma de la condition de diffraction pour un échantillon par la méthode des poudres, résultant en un cône de diffraction dans le réseau réciproque (a) et direct (b)}
    \label{fig:poudres}
\end{marginfigure}

Les diffractomètres pour la méthode des poudres utilisent férquemment la symétrie de réflexion, dans laquelle l'angle d'incidence et l'angle réfléchi sont égaux. Cela est accompagné par une rotation du détecteur de deux fois l'angle de l'échantillon, de telle sorte à ce que l'angle d'incidence et l'angle réfléchi soient toujours égaux. Le vecteur de diffusion $\mathbf{q}$ est maintenu perpendiculaire à l'échantillon, et le balayage consiste à suivre l'intensité au fur et à mesure que $\mathbf{q}$ varie. Dans l'espace réciproque, on peut voir que les pics apparaissent au fur et à mesure que $\mathbf{q}$ balaye les sphères concentriques successives de l'espace réciproque.
Générallement, les diffractomètres à rayons X utilisent une focale pour augmenter l'intensité diffractée.

Un exemple de motif de diffraciton, où l'intensité diffractée est tracée en fonction de l'angle de diffusion $2\theta$, est représenté sur la figure \ref{fig:poudrepic}. Généralement, une série de pics sera observée, et on donne généralement la liste des intensités et des positions angulaires. Comme la distance inter-réticulaire est donnée par :
\begin{equation}
    d = \frac{\lambda}{2\sin\theta}
\end{equation}
 alors la position des pics peut être convertie en distance inter-réticulaire, ce qui donne une liste d-I. Cette liste peut être utilisée pour l'analise de phase quantitative ou qualitative.

\begin{figure}
    \TODO
    \caption{Motif de diffraction pour une poudre de NaCl.}
    \label{fig:poudrepic}
\end{figure}


\subsection{Facteur d'intensité par la méthode des poudres}

POur mieux comprendre la diffraction expérimentale par la méthode des poudres, il faut regarder aux facteurs qui affectent l'intensité pour un pic de diffraction donné, pour une phase cristalline donnée. Nous avons déjà étudié certains de ces facteurs en étudiants des cristaux, et trouvé :

\begin{equation}
    I_{hkl} = I_0 \ 
        \left( \frac{r_e}{R} \right)^2 \ 
        |F|^2 \ 
        \prod_{j=1}^3 \left(
            \frac{ \sin \frac{N_j \mathbf{q \cdot a}_j}{2} }{ \sin \frac{\mathbf{q\cdot a}_j}{2} } \right)^2
\end{equation}

où $\mathbf{q}$ est le vecteur de diffusion, $\mathbf{a}_j$ le vecteur du réseau direct, $N_j$ le nombre de maille dans la direction $j$, $r_e$ le rayon d'un électron classique, $R$ la distance échantillon-détecteur et $I_0 = c\epsilon_0 E_0^2/2$ est l'intensité incidente, où $E_0$ est l'intensité du champ électrique incident, $c$ la célérité de la lumière, $\epsilon_0$ la permittivité du vide. Cette expression néglige cependant des facteurs importants. Discutons de ces facteurs, parfois d'une façon qualitative, pour déduire une expression de l'intensité diffractée par la méthode des poudres.

\paragraph{Facteur de structure}

Le facteur de structure $F$ est donné par la somme sur les atomes de la maille primitive :
\begin{equation}
    F = \sum_n^{N_b} f_n e^{i\mathbf{q\cdot r}_n}
\end{equation}

où $f_n$ et $\mathbf{r}_n$ sont respectivement les facteurs de diffusion atomique et la position dans la maille du n\ieme atome. Si nous écrivons :
\begin{equation}
    \mathbf{r}_n = x'_n \mathbf{a}_1 + y'_n \mathbf{a}_2 + z'_n \mathbf{a}_3
\end{equation}

où $x'_n, y'_n, z'_n$ sont les coordonnées du n\ieme atome dans la maille (nombre rationnels), alors le facteur de structure à la condition de Bragg ($\mathbf{q} = 2\pi \mathbf{G}$) devient :

\begin{equation}
    F_{hkl} = \sum_n^{N_b} f_n e^{2\pi i (h x'_n + k y'_n + l z'_n)}
\end{equation}

\paragraph{Facteur de polarization}

\paragraph{Facteur de Lorentz}

\paragraph{Intégration des pics}

\paragraph{Fraction orientée}

\paragraph{Segment de cône orienté}

\paragraph{Multiplicité}

\paragraph{Facteur d'absorption}

\paragraph{Facteur de température}

\paragraph{Bilan}

En combinant les résultats des paragraphes précédens, on peut maintenant écrire l'intensité résultant de la méthode des poudres comme :

\begin{equation}
    I = 
        \frac{I_0 r_e^2 \lambda^3 m_{hkl}}{16\pi R V_c^2}\ 
        \left(\frac{A_0}{2\mu} \right) \
        |F_{hkl}|^2 \ 
        \left( \frac{1+\cos^2 2\theta}{\sin\theta \sin 2\theta} \right)\ 
        e^{-2M}
\end{equation}

\section{Production des rayons X}
