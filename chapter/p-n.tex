\chapter{La jonction p-n}

une jonction p-n est faite d'un monocrystal modifié dans deux régions séparées.
Des impuretés accepteuses sont ajoutées dans une partie, pour produire une région
p dans laquelle la majorité des porteurs sont des trous. Des impuretés donneuses
sont ajoutées dans l'autre partie, pour produire une région n, dans laquelle les
porteurs majoritaires sont des électrons. La région à l'interface est moins
épaisse que $10^-4 cm$. Loin de l'interface, du côté p, il y a des impuretés
accepteuses ionisées (-) et une concentration égale en trous libres. Du côté n,
il y a des atomes donneurs ionisés (+), et une concentration égale en électrons
libres. Par conséquent, les porteurs majoritaires sont les trous du côté p et les
électrons du côté n.

Les trous sont concentrés du côté p voudront diffuser pour remplir le cristal de
façon uniforme. Les électrons voudront diffuser depuis le côté n. Mais la
diffusion va contrarier la neutralité électrique locale du système.

Un léger transfert de charge par diffusion laisse derrière, du côté p, un excès
d'accepteurs ionisés (-), et du côté n, un excès de donneurs ionisés (+). Cette
double couche de charges créé un champ électrique dirigé de la région n à p, qui
empêche la diffusion et ainsi maintient la séparation des deux types de porteurs.
À cause de cette double couche, le potentiel électrostatique dans le cristal fait
un saut en passant d'un côté à l'autre de la jonction.

À l'équilibre thermique, le potentiel chimique de chaque porteur est partout
constant dans le cristal, même à la jonction. Pour les trous :
\begin{equation}
    k_BT \ln p(\mathbf{r}) + e\phi(\mathbf{r}) = cste
\end{equation}
à travers le cristal, où $p$ est la concentration en trous et $\phi$ le potentiel
électrostatique. Par conséquent, $p$ est faible et $\phi$ est important. Pour les
électrons :
\begin{equation}
    k_BT \ln n(\mathbf{r}) - e\phi(\mathbf{r}) = cste
\end{equation}
et $n$ sera faible quand $\phi$ petit.

Le potentiel chimique total est constant partout dans le cristal. L'effet du
gradient de concentration annule parfaitement le potentiel électrostatique, et le
lux net de particules de chaque porteur est nul. Cependant, même à l'équilibre
thermique, il y a un léger flux d'électrons de $n$ vers $p$ où les électrons
finissent leur vie en se combinant avec des trous. Le courrant de recombination
$J_{nr}$ est contrebalancé par un courant $J_{ng}$ d'électrons qui sont générés
thermiquement dans la région $p$ et qui sont poussés par le champ auto-alimenté
dans la région $n$. Par conséquent, en l'absence de champ élecrtique appliqué :
\begin{equation}
    J_{nr}(0) + J_{ng}(0) = 0
\end{equation}

pour que sinon les électrons s'accumulent indéfiniment d'un côté de la barrière.

\section{Rectification}

Une jonction p-n peut agir comme un rectifieur. Un courant important apparaîtra
si on applique une tension le long de la jonction dans une direction, mais si la
tension est dans la direction opposée, juste un courrant faible découlera. Si une
tension alternative est appliquée le long de la jonction, le courrant s'écoulera
principalement dans une direction -- la jonction a rectifié le courrant.

\begin{marginfigure}
    \TODO
    \caption{caractéristique de la rectification d'une jonction p-n dans du
    germanium}
    \label{caracpn}
\end{marginfigure}

Pour des biais de tension inverse, une tension négative est appliquée dans la
région p et une tension positive dans la région n. Par conséquent, augmenter la
différence de potentiel entre les deux régions. Maintenant, pratiquement aucun
électron ne peut grimper la barrière d'énergie potentielle du côté bas en énergie
jusqu'en haut. Le courrant de recombinaison est réduit d'un facteur de Boltzmann
:
\begin{equation}
    J_{nr}(V back) = J_{nr}(0) \exp (-e|V|/k_BT)
\end{equation}

Le facteur de Boltzmann contrôle le nombre d'électrons avec suffisemment
d'énergie pour traverser la barrière.

La génération thermique d'un courant d'électrons n'est pas particulièrement
affectée par le courant inverse parce que la génération d'électron descend la
barrière (de p vers n). Par conséquent :
\begin{equation}
    J_{ng}(V back)=J_{ng}(0)
\end{equation}
On a vu précédemment que $J_{nr}(0) = -J_{ng}(0)$. Par conséquent, la génération
de courrant domine les courrants de recombinaison pour les biais inverses.

Lorsqu'une tension inverse est appliquée ,le courrant de recombinaison augmente
car la barrière d'énergie potentielle est abaissée, et par conséquent permet à
plus d'électrons de passer du côté n au côté p :
\begin{equation}
    J_{nr}(V forward) = J_{nr}(0) \exp (e|V|/k_BT)
\end{equation}
Encore une fois, le courrang de génération est inchangé :
\begin{equation}
    J_{ng}(V forward) = J_{ng}(0)
\end{equation}
Le courrant de trous traversant la jonction se comporte de façon similaire au
courant d'électrons. La tension appliquée, qui diminue la hauteur de la barrière
pour les électrons le fait également pour les trous, de sorte à ce qu'un grand
nombre d'électrons traverse de la région n àsous les mêmes conditions de tension
qui produisent un grand courrant de trous dans la direction opposée.

Les courrants électriques de trous et d'électrons sont additifs, et ainsi le
courrant total est :
\begin{equation}
    I = I_s [\exp(eV/k_BT)-1]
\end{equation}
où $I_s$ est la somm edes deux courrants générés. Cette équation est bien
satisfaite pour des jonctions p-n dans le germanium, mais pas trop pour d'autres
semiconducteurs.

\section{Cellules solaires et détecteurs photovoltaiques}

Si l'on illumine une jonction p-n, sans tension extérieure qui biaise le truc,
chaque photon absorbé créé un électron et un trou. Lorsque ces porteurs diffusent
dans la jonction, le champ électrique auto-alimenté de la jonction sépare les
charges à la barrière d'énergie. La séparation des poruteurs produit une tension
directe le long de la barrière. Directe, parce que le champ électrique des
porteurs photoexcités est opposé au champ intégré de la jonction.

L'apparition d'une tension inverse le long de la jonction illimunée est appelée
l'\emph{effet photovoltaique}. Une jonction illimunée peut délivrer de la
puissance à un circuit externe. Les jonctions p-n à grande surface en silicium
sont majoritairement utilisés dans les panneaux solaires, pour convertir des
photons d'origine solaire en énergie électrique.

\section{Barrière de Schottky}

Lorsqu'un semiconducteur est mis en contact avec un métal, on observe la
formation dans le semiconducteur d'une couche barrière à partir de laquelle les
porteurs de charge diminuent drastiquement. La couche barrière est aussi appelée
couche de déplétion (depletive layer ou exhaustion layer).

\begin{marginfigure}
    \TODO
    \caption{semiconducteur en contact avec un métal}
    \label{semicondmetal}
\end{marginfigure}

Sur la figure \ref{semicondnmetal}, un semiconducteur de type n est mis en
contact avec un métal. Les niveaux de Fermi sont coincident après le transfert
d'électrons à la bande de conduction du métal. Les donneurs chargés positivement
sont écartés de cette région qui est partiquement uniquement composée
d'électrons. Ici, l'équation de Poisson  donne :
\begin{equation}
    \div \mathbf{D} = ne/\epsilon_0
\end{equation}
où n est la concentration en donneur. Le potentiel électrostatique est déterminé
par :
\begin{equation}
    \frac{d^2\phi}{dx^2} = \frac{ne}{\epsilon \epsilon_0}
\end{equation}
Qui admet une solution de la forme :
\begin{equation}
    \phi = -(\frac{-ne}{2\epsilon \epsilon_0})x^2
\end{equation}

L'originedes x est prise, pour des raisons pratiques, au bord droit de la
barrirère. Le contact est à $-x_b$, et ici l'énergie potentielle relative au côté
droit est $-e\phi_0$, d'où une épaisseur de la barrière qui peut s'écrire sous la
forme :
\begin{equation}
    x_b = \left(\frac{2\epsilon \epsilon_0 |\phi_0|}{ne}\right)^{1/2}
\end{equation}

Avec $\epsilon = 16$, $e\phi_0 =$ \SI{0.5}{\electronvolt},
n=\SI{e16}{\per\cubic\centi\metre}, on trouve $x_b = $\SI{0.3}{\micro\metre}.
C'est une vision plutôt simplifiée.
