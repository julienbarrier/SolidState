\chapter{Diffusion des rayons X dans les cristaux}

Dans les solides, les mailles cristallines ont une distance caractéristique de l'ordre de l'angstrom (\SI{e-10}{\metre}). Si l'on veut produire des interférences constructives avec une onde électromagnétique diffusant dans un cristal, il faut que sa longueur d'onde soit au moins aussi petite que la distance carcactéristique du cristal. Cela correspond à une énergie de l'ordre de :

\begin{equation}
\hbar \omega \sim \frac{hc}{\lambda} = \frac{\SI{e-34}{}\SI{e8}{}}{\SI{e-10}{}} = \SI{e-16}{\joule} = \SI{e3}{\electronvolt}
\end{equation}

Les énergies de cet ordre de grandeur sont caractéristiques des rayons X. C'est pour cette raison que nous étudions la diffraction des rayons X pour sonder la matière à l'échelle de l'atome.

\section{Diffusion des rayons X}

Les rayons X ont été découverts par Röntgen en 1895. Leur nature ondulatoire a
été comprise en 1913 avec la réalisation des premières expériences de
diffraction suggérées par von Laue. Plus tard, Barkla a montré le caractère
transversal de ces ondes, ce qui a établit le fait qu'il s'agissait d'ondes
électromagnétiques.

Le domaine de longueur d'onde des rayons X va de 0.1 (limite des rayons
$\gamma$) à \SI{100}{\angstrom} (limite de l'UV lointain). En terme d'énergies,
cela correspond à la gamme 0.1 à \SI{100}{\kilo\electronvolt}. En
cristallographie, on utilise généralement des rayons X dont la longueur d'onde 
varie entre 0.5 et \SI{2.5}{\angstrom}.

Dans les solides, ce sont les électrons qui interagissent avec les radiations électromagnétiques (contrairement aux particules du noyau), et qui seront donc la source de diffusion. Écrivons donc l'amplitude de diffusion
$\epsilon(\mathbf{q})$ d'une densité de charge distribuée $\rho(\mathbf{r})$ :

\begin{equation}
    \epsilon(\mathbf{q}) = \frac{E_0 r_e}{R}\, \exp [i(\omega t - k R)]  \, f(\mathbf{q})
\end{equation}
Dans cette expression, $f$ est le facteur de diffusion ; il représente l'intensité de la diffusion, relativement à un électron libre. On peut écrire le facteur de diffusion atomique en sommant cette expression selon les électrons de l'atome :
\begin{equation}
    f(\mathbf{q}) = \int e^{i\mathbf{q}\cdot\mathbf{r}}\, \rho(\mathbf{r}) \,dV
\end{equation}
où $\mathbf{q = k' - k}$ est le \emph{vecteur de diffusion}, qui correspond au vecteur d'onde de l'espace réciproque entre
l'onde diffusée et l'onde incidente. On peut remarquer que cette expression du facteur de forme est simplement la transformée de Fourier de la densité de distribution électronique $\rho(\mathbf{r})$.

\subsection{Diffusion par groupe d'atomes}

Considérons à présent la diffusion par un arrangement périodique d'atomes. On montrera que cela provoque de la diffraction, qui est vue comme une somme des contributions cohérentes de diffusion par des atomes, en produisant des pics de diffraction de la radiation diffusée.

Historiquement, la diffraction des rayons X par un réseau cristallin a été la première preuve de deux concepts importants en physique :
\begin{enumerate}
    \item la nature ondulatoire des rayons X ;
    \item la nature cristalline des solides.
\end{enumerate}
Encore aujourd'hui, la diffraction des rayons X est encore très largement utilisée, et ce pour un très grand domaines d'applications, qui couvre l'identification de phase en physico-chimie, la mesure de contraintes, la détermination de la concentration en défauts, ou pour la détermination de paramètres structuraux des supers réseaux. Rappelons que récemment, la diffraction des rayons X  en incidence rasante (grazing incidence XRD) a été utilisée pour mesurer avec précision les constantes de réseaux de couches monoatomiques.

Pour traiter ces problèmes, on peut suivre deux méthodes équivalentes pour étudier la diffusion par un arrangement d'atomes.
D'une part, on peut représenter l'arrangement des atomes par une distribution de densité électronique de l'arrangement complet $\rho_c(\mathbf{r})$. Par conséquent,
$\rho_c(\mathbf{r})$ représente la densité totale électronique de tous les
électrons de tous les atomes du cristal. L'amplitude diffusée est simplement la
transformée de Fourier de cette distribution de densité électronique :
\begin{equation}
    \epsilon(\mathbf{q}) = \frac{E_0 r_e}{R} \exp [i(\omega t - k R)] \int e^{i \mathbf{q \cdot r}} \rho_c (\mathbf{r}) dV
    \label{fftdistrelec}
\end{equation}
Un moyen équivalent et totalement différent de trouver l'amplitude diffusée est
d'assigner chacun des électrons du solide à un atome. On calcule ensuite la
distribution de densité électronique pour chaque type d'atome. Par exemple, les
électrons assignés au p\ieme type d'atome seraient représentés par
$\rho_p(\mathbf{r})$ où $\mathbf{r}$ est la position relative au centre atomique.
On peut ainsi écrire la densité électronique du cristal comme :
\begin{equation}
    \rho_c(\mathbf{r}) = \sum_p \rho_p(\mathbf{r - R_p})
    \label{densiteelectroniquecristal}
\end{equation}

où la somme sur $p$ est effectuée sur tous les atomes du cristal. L'équation
\ref{densiteelectroniquecristal} place un atome $p$ en un site du cristal, au
bout du vecteur $\mathbf{R_p}$. Replacer cela dans l'équation \ref{fftdistrelec}
et échanger l'ondre de l'intégration et de la somme nous ramène à :

\begin{equation}
   \epsilon(\mathbf{q}) = \frac{E_0 r_e}{R} \exp [i(\omega t - k R)] 
   \sum_p \int e^{i \mathbf{q \cdot r}} \rho_p (\mathbf{r - R_p}) dV
    \label{fftdistrelecsum}
\end{equation}

En posant le changement de variables $\mathbf{r' = r - R_p}$, on se ramène alors
à :

\begin{equation}
    \epsilon(\mathbf{q}) = \frac{E_0 r_e}{R} \exp [i(\omega t - k R)] 
    \sum_p e^{i\mathbf{q \cdot R_p}} \int  \rho_p (\mathbf{r'}) dV'
    \label{fftdistrelecsumred}
\end{equation}

L'intégrale de l'équation \ref{fftdistrelecsumred} corresponsd au facteur de
diffusion $f$ de l'atome $p$ dans le solide. On obtient alors :


\begin{equation}
   \epsilon(\mathbf{q}) = \frac{E_0 r_e}{R} \exp [i(\omega t - k R)] \sum_p  f_p e^{i \mathbf{q \cdot R_p}}
    \label{fftdistrelecsumf}
\end{equation}

Dans cette équation, $f_p$ est le facteur de diffusion de l'atome $p$ ; $\mathbf{R_p}$ est la position relative à une position de référence dans le
cristal, et la somme sur $p$ est effectuée sur toutes les positions atomiques du 
cristal. Avec ce formalisme, on simplifie la tâche -- ardue -- de construire une
fonction de densité électronique pour le cristal, le calcul de $f_p$ pour chaque type d'atome jouant ce rôle de simplification.
Comme seuls les électrons périphériques entrent en jeu dans la cohésion d'un
cristal, $f$ ne dépend que faiblement de l'environnement dans lequel l'atome est
placé, et en pratique, les facteurs de diffusions des atomes libres sont
couremment utilisés \footnote{À l'exception des solides où les laisons ont un
    large caractère ionique, comme NaCl, où ce sont les facteurs de diffusion
des ions qui seront privilégiés}.

\subsection{Diffusion à partir d'un arrangement périodique d'atomes}

Pour les solides cristallins, on peut simplifier l'équation
\ref{fftdistrelecsumf} en écrivant les positions atomiques comme la somme des
positions $\mathbf{R}_m$ de la maille primitive dans laquelle l'atome est fixé
et d'une position $\mathbf{r}_n$ de l'atome dans la maille. Cela correspond à
écrire :
\begin{equation}
    \mathbf{R}_p = \mathbf{R}_m^n = \mathbf{R}_m + \mathbf{r}_n =
    \sum_{j=1}^3 m_j \mathbf{a}_j + \mathbf{r}_n
\end{equation}

La somme sur tous les atomes du solide se réduit ainsi à une somme sur tous
les atomes de la maille primitive et à la somme sur toutes les mailles primitives
du cristal :

\begin{eqnarray}
    \sum_p f_p e^{i\mathbf{q \cdot R_p}} & = & \sum_m^{N_c} \sum_n^{N_b}
    f_n e^{i\mathbf{q\cdot(R_m + r_n)}}\\
    & = & \left( \sum_m^{N_c} e^{i\mathbf{q\cdot R_m}} \right)
    \left( \sum_n^{N_b} f_n e^{i\mathbf{q\cdot r_n}} \right)
\end{eqnarray}

où $N_c$ est le nombre de mailles primitives dans le cristal et $N_b$ le nombre
d'atomes ans la maille primitive.

On définit le facteur de structure $F(\mathbf{q})$ comme la somme sur tous les
atomes de la maille :

\begin{equation}
    F(\mathbf{q}) = \sum_n^{N_b} f_n e^{i \mathbf{q \cdot r_n}}
\end{equation}

Ce facteur de structure contient toute l'information sur les positions atomiques
de la maille, et s'affranchit de toutes les complications liées à la distribution
élecrtonique des atomes, cachée dans les facteurs de diffusions.

En combinant toutes ces équations, on se ramène à l'amplitude diffractée par le
cristal :

\begin{equation}
    \epsilon = \frac{E_0 r_e}{R} \exp[i(\omega t - k R)] F(\mathbf{q}) \sum_m^{N_c} e^{i\mathbf{q\cdot R_m}}
    \label{ampdiffract}
\end{equation}

\section{Diffraction des rayons X}

\subsection{Réseau réciproque}

L'équation \ref{ampdiffract} donne le formalisme pour calculer l'amplitude de
diffusion élastique pour un arrangement périodique d'atomes. Concentrons-nous,
dans un premier temps, sur la somme sur une maille primitive:

\begin{equation}
    \sum_m^{N_c} e^{i\mathbf{q\cdot R}_m}
\end{equation}

On considère un cristal avec les vecterus de translation de réseau $\mathbf{a}_1$
, $\mathbf{a}_2$ et $\mathbf{a}_3$, de telle sorte que les positions atomiques
dans la maille primitive soient données par :
\begin{equation}
    \mathbf{R}_m = m_1 \mathbf{a}_1 + m_2 \mathbf{a}_2 + m_3 \mathbf{a}_3
\end{equation}
où $m_1$, $m_2$ et $m_3$ sont des entiers. Le produit scalaire 
$\mathbf{q\cdot R}_m$ devient alors :

\begin{equation}
    \mathbf{q\cdot R}_m = m_1 \mathbf{q \cdot a}_1 + m_2 \mathbf{q \cdot a}_2 + m_3 \mathbf{q \cdot a}_3
\end{equation}

En outre, si l'on considère un cristal qui est un parallélépipède avec $N_1$
mailles selon la direction $\mathbf{a}_1$, $N_2$ selon la direction $\mathbf{a}_2$
et $N_3$ selon $\mathbf{a}_3$, alors la somme sur toutes les mailles du cristal
devient :

\begin{eqnarray}
    \sum_m^{N_c} e^{i \mathbf{q\cdot R_m}} & = &
        \sum_{m_1 = 0}^{N_1 - 1} \sum_{m_2 = 0}^{N_2 - 1} \sum_{m_3 = 0}^{N_3 - 1}
        \exp [i(m_1 \mathbf{q \cdot a}_1 + m_2 \mathbf{q \cdot a}_2 + m_3 \mathbf{q \cdot a}_3)]\\
        & = & \left( \sum_{m_1 = 0}^{N_1 - 1} e^{i m_1 \mathbf{q\cdot a}_1} \right) 
        \left( \sum_{m_2 = 0}^{N_2-1} e^{i m_2 \mathbf{q\cdot a}_2} \right)
        \left( \sum_{m_3 = 0}^{N_3-1} e^{i m_3 \mathbf{q\cdot a}_3} \right)\\
        & = & \prod_{j=1}^3 \left( \sum_{m_j = 0}^{N_j -1} e^{i m_j \mathbf{q\cdot a}_j} \right)
    \end{eqnarray}

Il s'agit d'une série géométrique de raison $e^{i \mathbf{q\cdot a}_j}$, d'où :
\begin{eqnarray}
    \sum_{m_j = 0}^{N_j - 1} e^{i m_j \mathbf{q\cdot a}_j} & = &
    \frac{1 - e^{i N_j \mathbf{q\cdot a}_j}}{1 - e^{i \mathbf{q \cdot a}_j}} \\
    & = & e^{i \phi_j} \frac{\sin \left( N_j \frac{\mathbf{q\cdot a}_j}{2} \right) }{\sin \left( \frac{\mathbf{q\cdot a}_j}{2} \right)}
\end{eqnarray}
où le terme de phase $\phi_j$ est donné par : $\frac{\mathbf{q\cdot a}_j}{2} (N_j
-1)$. Cela donne la somme sur toutes les mailles du cristal :

\begin{equation}
    \sum_m^{N_c} e^{i\mathbf{q\cdot R}_m} = \prod_{j=1}^3 \left\{ e^{i\phi_j} \left[ \frac{\sin \left(N_j \frac{\mathbf{q\cdot a}_j}{2}
    \right)}{\sin \left(\frac{\mathbf{q\cdot a}_j}{2} \right)} \right] \right\}
    \label{sommecristal}
\end{equation}

Ce résultat peut être injecté dans l'équation \ref{ampdiffract}, pour trouver
l'amplitude diffusée pur un cristal. Lorsque l'on effectue de la diffraction, on
mesure l'intensité, qui est en fait le carré du module de l'amplitude
$\epsilon \cdot \epsilon^* = |\epsilon|^2$, multipliée par la constante
$c\epsilon_0$. En prenant le carré complexe, le terme de phase disparaît et l'on
obtient alors :

\begin{equation}
    I = c\epsilon_0 \left( \frac{E_0 r_e}{R} \right)^2 |F(\mathbf{q})|^2
    \prod_{j=1}^3 \left[ \frac{\sin \left(N_j \frac{\mathbf{q\cdot a}_j}{2} \right)}{\sin \left( \frac{\mathbf{q\cdot a}_j}{2}\right)} \right]^2
\end{equation}

L'intensité diffractée contient le facteur de diffraction de Lorentz:

\begin{equation}
    \left[ \frac{\sin \left(N_j \frac{\mathbf{q\cdot a}_j}{2} \right)}{\sin \left( \frac{\mathbf{q\cdot a}_j}{2}\right)} \right]^2
\end{equation}

\TODO: cf ouvrage diffraction.

Ce facteur de Lorentz présente des pics lorsque :
\begin{equation}
    \frac{\mathbf{q \cdot a}_j}{2} = n \pi
\end{equation}

où n est un entier. Pour observer un pic d'intensité diffractée, il faut donc
que cela soit vrai dans chacune des trois directions, c'est à dire pour
$j = 1,2,3$. Le vecteur de diffusion $\mathbf{q}_B$ qui satisfait cette condition
est défini par :

\begin{eqnarray}
    \frac{\mathbf{q}_B \cdot \mathbf{a}_1}{2} & = & h \pi \\
    \frac{\mathbf{q}_B \cdot \mathbf{a}_2}{2} & = & k \pi \\
    \frac{\mathbf{q}_B \cdot \mathbf{a}_3}{2} & = & l \pi
\end{eqnarray}
où $h,k,l$ sont des entiers ; ce sont les indices de Miller définis précedemment.
Cette condition est appelée condition de Laue. Cela revient à écrire :

\begin{eqnarray}
    \mathbf{G}_{hkl} \cdot \mathbf{a}_1 = h \\
    \mathbf{G}_{hkl} \cdot \mathbf{a}_2 = k \\
    \mathbf{G}_{hkl} \cdot \mathbf{a}_3 = l
\end{eqnarray}
De telle sorte que les conditions de Laue soient satisfaites si :

\begin{equation}
    \mathbf{q}_B = 2\pi \mathbf{G}_{hkl}
\end{equation}

Cette équation représente le fait que pour un pic du spectre de diffraction d'un
cristal, le vecteur de diffusion est $2 \pi$ fois le vecteur du réseau réciproque
. Cela est la relation la plus importante à retenir dans ce chapitre. On peut
immédiatement voir que pour un pic, le vecteur de diffusion est perpendiculaire
aux plans de diffraction. En comparant les amplitudes des vecteurs, on peut aussi
remarquer que :

\begin{equation}
    q_B = \frac{4\pi}{\lambda} \sin \theta_B = 2\pi |\mathbf{G}_{hkl} | = \frac{2\pi}{d_{hkl}}
\end{equation}

Ce qui nous amène à :
\begin{equation}
    \lambda = 2 d_{hkl} \sin \theta_B
\end{equation}

et
\begin{equation}
    q_B = \frac{2\pi}{d_{hkl}}
\end{equation}
Ces deux dernières équations sont des formulations équivalentes de ce qui est
connu sous le nom de loi de Bragg.

\TODO: on dit von laue, pas laue.

\subsection{Loi de Bragg}

La diffraction est un concept très très important en physique du solide et en
science des matériaux, non seulement poru les techniques d'analyses qui en sont
rendues possibles, mais aussi pour la théorie des bandes d'énergie dans les
cristaux. Le phénomène de diffraction apparaît dans de nombreuses situations dans
les solides cristallins, et les chercheurs de différents domaines ont pu
construire différents moyens de représenter la diffraction. Nous nous intéressons
nous à certaines représentations graphiques de différents aspects de la
condition de la diffraction.

Nous avons vu précedemment qu'un moyen d'établir la condition de diffraction pour
trouver un maximum de difraction d'un ensemble de plans espacés d'une distance
$d$, illuminés par des rayons X de longueur d'onde $\lambda$ est :
\begin{equation}
    n \lambda = 2d \sin \theta_B
\end{equation}

Cette expression a été formulée pour la première fois par Bragg et est connue
sous le nom de loi de Bragg. Cette relation peut être déduite en considérent la
différence de phase entre les rayons X diffusés et les plans adjacents. Comme
présenté sur la figure \TODO, la différence de chemin entre les rayons X diffusés
à partir d'un plan et ceux qui sont diffusés par le plan prédédent est 
$2d sin \theta$, ce qui nous ramène à une différence de phase de
$\frac{2\pi}{\lambda} 2d \sin \theta$. Si cette différence de phase est égale à
un entier $n$ fois $2\pi$, alors les ondes diffusées à partir de plans succesifs
seront en phase et interféreront de façon constructive.

En appliquant cette condition, on voit :
\begin{equation}
    n 2 \pi = \frac{2 \pi}{\lambda} 2d\sin \theta_B \rightarrow
    n\lambda = 2d \sin \theta_B
\end{equation}

\TODO: construction de Bragg

Cette construction peut être utilisée pour visualiser l'expérience de diffraction
commune, où l'angle $\theta$ peut être varié en tournant le cristal sous un
faisceau de rayons X monochromatiques. Un détecteur est tourné à une fréquence
angulaire deux fois plus importante, suivant le meme axe, de sorte à maintenir la
symmétrie entre les rayons X incidents et diffusés, relativement aux plans
cristallins. En même temps que l'angle $\theta$ varie, dès que la condition de
Bragg est atteinte, un pic d'intensité apparaît. Cela est montré schématiquement
sur la figure \TODO.

La géométrie symmétrique entre $\mathbf{k}$ et $\mathbf{k'}$ relativement aux
plans de diffraction, maintient la condition que le vecteur de diffusion est
perpendiculaire aux plans, à la condition de Bragg. Dans de nombreux cas, la
géométrie symmétrique est également maintenue, relativement à la surface de
l'échantillon, de telle sorte à ce que les plans étudiés soient parallèles à la
surface de l'échantillon. Cependant, ce n'est pas toujours le cas ; nous verrons
plusieurs géométries de diffraction plus tard.

\subsection{Sphère d'Ewald}

La loi de Bragg est un postulat de la condition de la diffraction correct, mais
incomplet, du fait qu'il ne containet qu'une information scalaire, et ne
représente pas les aspects plus généraux, directionnels ou vectoriels, que nous
avons vu comme :
\begin{equation}
    (\mathbf{k'-k})_B = \mathbf{q}_B = 2\pi \mathbf{G}_{hkl}
\end{equation}

où $\mathbf{G}_{hk}$ est un vecteur de l'espace réciproque.

Un moyen facile de voir cette relation est sa représentation dans l'espace
réciproque, aussi connue sous le nom de sphère d'Ewald. On construit d'abord le
réseau réciproque pour le cristal qui nous intéresse. Ensuite, on place
l'extrémité du vecteur $\frac{\mathbf{k}}{2\pi}$ sur un site du réseau réciproque
. Ensuite, une sphère de rayon $\frac{k}{2\pi} = \frac{1}{\lambda}$ est tracé,
ave son centre à l'origine de $\frac{\mathbf{k}}{2\pi}$
\footnote{On peut noter que l'origine du vecteur $\frac{\mathbf{k}}{2\pi}$ n'est
pas nécessairement sur un nœud du réseau réciproque.}
Comme deux nœuds du réseau réciproque peuvent être connectés entre eux par un
vecteur du réseau réciproque $\mathbf{G}_{hkl}$, tout nœud du réseau réciproque
qui apparaît sur cette pshère (autre que celui tracé au début, qui termine à
$\frac{\mathbf{k}}{2\pi}$) sera à l'extrémité d'un vecteur
$\frac{\mathbf{k'}}{2\pi}$, qui satisfait la condition de la diffraction :

\begin{equation}
    \frac{1}{2\pi} (\mathbf{k'-k})_B = \mathbf{G}_{hkl} \rightarrow
    (\mathbf{k'-k})_B = \mathbf{q}_B = 2 \pi \mathbf{G}_{hkl}
\end{equation}

Ce qui peut être illustré sur la figure \TODO.

Sur les représentations dans l'espace réciproque, les expériences de diffraction 
sont des observations de l'intensité diffractée en fonction de l'orientation
ou de la longueur du vecteur de diffusion $\mathbf{q}$. Si l'expérience est faite
à une énergie constante (faisceau monochromatique), alors le diamètre de la
sphère d'Ewald est onstant, et son orientation est changée pendant l'expérience,
ce qui apporte un autre set de points du réseau réciporque en contact avec la
sphère d'Ewald.

La technique de la diffraction des électrons dans un microscope électronique à
transmission (TEM) utilise la nature ondulatoir edes électrons pour faire de la
diffraction à partir de cristaux. La longueur d'onde des électrons à hautes
énergies (\SI{100}{\kilo\volt} à \SI{1}{\mega\electronvolt}) utilisés est bien
plus courte que celle des rayons X typiques, ramenant la sphère d'Ewald à un
rayon très large, de telle sorte à ce que sa surface soit quasi planne.
En outre, l'échantillon cristallin est fin (nécessaire pour assurer la
transparence électronique), ce qui résulte en une élongation de la région de
diffraction dans la direction parallèle à $\mathbf{k}$.
Par conséquent, même s'il y a quelques courbures de la sphère d'Ewald, les nœuds
du réseau réciproque intersectent toujours la sphère d'Ewald. Cela signifie que
l'alignement d'une zone de l'axe avec le vecteur d'onde de l'électron
incident donnera des spots de diffraction pour quasiment tous les plans dans la
zone. Cela est présenté schématiquement sur la figure \TODO.



\section{Zones de Brillouin et condition de diffraction}

Les zones de Brillouin forment l'énoncé de la condition de diffraction la plus
utilisée en physique du solide : elle permet de retrouver la théorie des bandes
d'énergie électronique.

Nous avons vu précédement que la 1\iere zone de Brillouin comme la cellule de
Wigner-Seitz dans le réseau réciproque. Cela permet de former une interprétation
géométrique de la condition de la diffraction :
\begin{equation}
    2\mathbf{k} \cdot \mathbf{G} = G^2
\end{equation}
Que l'on peut aussi écrire :
\begin{equation}
    \mathbf{k} \cdot \left( \frac{1}{2} \mathbf{G} \right) = \left( \frac{1}{2} G \right)^2
\end{equation}

En travaillant dans l'espace réciproque, $\mathbf{k},\mathbf{G})$, on choisit un
vecteur $\mathbf{G}$ à l'origine d'un point du réseau réciproque. Le plan normal
à ce vecteur $\mathbf{G}$, au milieu du vecteur. Ce plan forme une partie de la
frontière de la zone de Brillouin.

Si une onde, de vecteur d'onde $\mathbf{k}$, est diffractée sur le réseau, le
faisceau diffracté aura la direction $\mathbf{k}-\mathbf{G}$, soit $\Delta\mathbf{k} = \Delta\mathbf{k} = -\mathbf{G}$.
La construction de Brillouin présente tous les vecteurs d'onde $\mathbf{k}$
réfléchis par le cristal en suivant la loi de Bragg.

Les plans qui sont les bissectrices des vecteurs du réseau réciproque sont d'une
grande importance en ce qui'il s'agit de la propagation d'ondes dans les
cristaux : une onde dont le vecteur d'onde tracé de l'origine termine sur un 
de ces plans respectera la condition de diffraction. Ces plans divisent l'espace
de Fourier du cristal en fragments. Le fragment central est la cellule de
Wigner-Seitz du réseau réciproque.

On peut former une définition plus précise des zones de Brillouin : \emph{
    La première zone de Brillouin est le plus petit volume entièrement entouré
    par les plans qui sont les bissectrices des vecteurs du réseau réciproque
tracés depuis l'origine.}




L'étude de réseau dans l'espace réel (K), généralement il est pratique et
intéressant de considérer un polyhèdre connu sous le nom de cellule de 
Wigner-Seitz, décrite précedement.

L'analogie entre la construction dans l'espace réciproque résulte en ce qui est
connu en tant que zore de Brillouin. Rappelons que la première zone de Brillouin 
est la cellule de Wigner-Seitz du réseau réciproque.

Généralement ,la construction de la zone de Brillouin est utiisée poru décrire
les électrons dans un solide périodique ; les plans bisecteurs et les
zones ont une signification pour la diffraction. Cela peut être remarqué en 
(ré)écrivant les conditions e la diffraction :
\begin{equation}
    \mathbf{q}_B = (\mathbf{k'-k})_B = 2\pi\mathbf{G} \rightarrow \mathbf{k} + 2\pi \mathbf{G} = \mathbf{k'}
\end{equation}

Lorsque $k=k'$, on peut écrire :
\begin{equation}
    (\mathbf{k} + 2\pi \mathbf{G})^2 = k^2
\end{equation}

mais :
\begin{equation}
    2 \mathbf{k\cdot G} = 2\pi G^2
\end{equation}

Soit le résultat :
\begin{equation}
    \mathbf{k}\cdot\frac{\mathbf{G}}{G} = 2\pi \left( \frac{G}{2} \right)
\end{equation}

où on a utiisé le fait que $-\mathbf{G}$ esta ussi un vecteur du réseau
réciproque.

Cette dernière équation pose le fait que l acondition de la diffraction est
satisfaite lorsque la composante de $\mathbf{k}$ selon $\mathbf{G}$ est égale à
$2\pi$ fois la demin longueur de $\mathbf{G}$. En fait, cette condition est
vérifiée pour tous les vecteurs $\frac{\mathbf{k}}{2\pi}$ qui ont leur origine
en un nœud du réseau réciproque, et terminent sur les plans bissecteurs
perpendiculaires au vecteur entre l'origine et un autre nœud du réseau réciproque. \TODO figure.


 
\subsection{Conclusions sur les conditions de la diffraction}

Les conditions de Laue, la realtion de Bragg et la construction d'Ewald sont des
représentations équivalentes du même phénomène : les directions de diffraction 
d'un réseau sont déterminées par son réseau réciproque.

La nature du motif influe uniquement sur l'intensité diffractée et pas sur les
directions de diffraction. La mesure des angles de diffraction des rayons X par
un cristal donne seulement des informations sur le réseau translatoire du cristal
. Pour obtenir la position des atomes dans la maille, il faut aussi utiliser les
intensités des figures de diffraction. Suivant la nature du problème étudié et
les techniques de diffraction employées, on utilisera poru déterminer les
directions de diffraction, l'une de ces trois méthodes.

\section{Exemples}

\subsection{Facteur de structure}

Aussi importante qu'elle puisse être, la somme de l'équation \ref{sommecristal}
ne donne pas une idée générale. Le facteur de structure $F(\mathbf{q})$ donne une
échelle de l'amplitude diffractée, et l'intensité diffractée est contrôlée
par le carré du facteur de structure. Comme le facteur de structure est une
fonction variant lentement par rapport au vecteur de diffusion, il est commun de
considérer sa valeur à la condition de Bragg exacte. On définit :

\begin{equation}
    F_{hkl} = F(\mathbf{q}_B) = \sum_n^{N_B} f_n e^{i2\pi \mathbf{G}_{hkl}\cdot\mathbf{r}_n}
\end{equation}

Ce facteur de structure reflète l'information de la distribution électronique
dans la maille. On verra q'il implique d'éliminer certains pics de diffraction
dus aux interférences entre les radiations diffusée des différents atomes dans
la maille.
Regardons plus en détail certaines structures caractéristiques.

\subsection{Structure cubique simple}

Dans un réseau cubique simple,

\begin{eqnarray}
    \mathbf{a}_1 & = & a \mathbf{\hat x} \\
    \mathbf{a}_2 & = & a \mathbf{\hat y} \\
    \mathbf{a}_3 & = & a \mathbf{\hat z}
\end{eqnarray}

Les vecteurs du réseau réciproque sont :

\begin{eqnarray}
    \mathbf{b}_1 & = & \frac{1}{a} \mathbf{\hat x} \\
    \mathbf{b}_2 & = & \frac{1}{a} \mathbf{\hat y} \\
    \mathbf{b}_3 & = & \frac{1}{a} \mathbf{\hat z}
\end{eqnarray}

Il y a un atome par maille primitive et il est situé sur la coordonnée (0,0,0).
Par conséquent, on trouve :

\begin{equation}
    F_{hkl} = \sum_n f(2\pi \mathbf{G}_{hkl}) e^{i 2\pi \mathbf{G}_{hkl}\cdot\mathbf{r}_n } = f(2\pi\mathbf{G}_{hkl})
\end{equation}
L'intensité diffusée présentera des pcis à chaque vecteur du réseau réciproque,
et ils seront pondérés par $|f(\mathbf{q})|^2$ évalué en $\mathbf{q = q}_B$. On a
vu préceddement que $f(\mathbf{q})$ est une fonction monotone décroissante,
de telle sorte à ce que les pics correspondant à de plus grands vecteur du
réseau réciproque, qui provionnent de plans espacés d'une distance $d$ plus
courte, auront une intensité décroissante.
\subsection{Structure cubique centré}

À des fins illustratives, calculons le facteur de structure en utilisant la
maille conventionnelle (et non pas la maille primitive). On a les mêmes vecteurs
du réseau direct et du réseau réciproque que dans le système cubique simple, mais
maintenant il y a un atome positionné à $(0,0,0)$ et un autre en
$\frac{a}{2}(1,1,1)$. On trouve par conséquent :

\begin{eqnarray}
    F_{hkl} & = & \sum_n f(2\pi\mathbf{G}_{hkl}) e^{i 2\pi \mathbf{G}_{hkl}\cdot\mathbf{r}_n} \\
    & = & f(2\pi \mathbf{G}_{hkl}) [1 + e^{i\pi(h+k+l)}] \\
    & = & f(2\pi \mathbf{G}_{hkl}) \times \begin{cases} 2 \text{ si } h+k+l \text{ est pair}\\ 0 \text{ si } h+k+l \text{ est impair}\end{cases}
\end{eqnarray}

Par conséquent, pour une structure cubique centrée dans l'espace réel, le réseau 
réciproque du cubique simple a une intensité nulle pour les points pour lesquels
$h + k + l$ est impair. Les pics avec un facteur de structure nul sont dits
interdits, alors que ceux qui ont un facteur de structure autorisé sont dits
autorisés (ou permis). Par exemple, les quatre premiers pis permis sont :
$(110)$,$(200)$,$(211)$ et $(220)$. Les indices $(100)$,$(111)$ et $(211)$ ont un
facteur de structure nul, qui correspond à des pics interdits. Si l'on consiidère
les pics autorisés pour un réseau cubique centré, et qu'on associe chacun d'entre
eux aux vecteurs du réseau réciproque corrrespondant, les points constitués par
les terminaisons de ces veucteurs produisent un réseau cubique faces-centrées,
avec des côtés de longueur $\frac{2}{a}$. Par conséquent, un réseau BCC dans
l'espace réel est considéré comme ayant une réseau réciproque FCC. 

\subsection{Cubique face centrée}

Ici, nous utilisons encore une fois la maille conventionnelle, mais cette fois-ci
nous avons un atome principal aux positions $(0,0,0)$,$\frac{a}{2}(1,1,0)$,
$\frac{a}{2}(1,0,1)$ et $\frac{a}{2}(0,1,1)$. On trouve, par conséquent un
facteur de structure:

\begin{eqnarray}
    F_{hkl} & = & f(2\pi \mathbf{G}_{hkl}) [ 1 + e^{i\pi(h+l)} + e^{i\pi(k+l)} + e^{i\pi(h+k)}]\\
    & = & f(2\pi\mathbf{G}_{hkl}) \times \begin{cases} 4 \text{ si } h,k,l \text{ sont de même parités}\\
    0\text{ sinon.} \end{cases}
\end{eqnarray}

Par conséquent, pour un réseau FCC dans l'espace réel, le réseau cubique simple a
des points manquants lorsque les indices $(hkl)$ ne sont pas de même parité. Les
quatre premiers pics autorisés sont $(111)$, $(200)$, $(220)$ et $(311)$. Les
pics correspondants aux plans $(100)$, $(110)$, $(210)$ et $(310)$ sont interdits
. Si encore une fois, on associe chaque pic avec un vecteur du réseau réciproque,
on produit un réseau FCC dans le réseau réciproque, avec un côté de longueur
$\frac{2}{a}$. Par conséquent, un réseau FCC dans l'espace réel donne un réseau
BCC dans l'espace réciproque.



\subsection{Structure diamant}
\subsection{NaCl}
